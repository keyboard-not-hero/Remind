\documentclass[dvisvgm]{beamer}
\usepackage{tikz}
\usepackage{animate}
\usepackage{ifthen}
\definecolor{darkgreen}{RGB}{10,90,10}
% 制作svg格式animate,需要满足以下条件:
% 1.使用beamer类的documentclass,并且携带dvisvgm参数
% 2.使用animate包
% 3.使用texlive编译命令: latexmk -pdflatex -dvi <tex_file>
% 4.dvi文件转化为svg命令: dvisvgm <dvi_file>
% 5.ubuntu下需要使用浏览器查看svg格式图片

% 制作pdf格式animate连续片段,需要满足一下条件:
% 1.使用用standalone类的documentclass,并且携带export参数,切记不能包含在frame环境内(begin{frame} ... \end{frame})
% 2.使用animate包
% 3.使用texlive编译命令: latexmk -pdflatex -pv <tex_file>
% 4.可使用以下站点进行pdf到gif的转化: https://ezgif.com/pdf-to-gif
% 示例参考: https://tex.stackexchange.com/questions/558850/how-can-i-animate-a-tikz-drawing-in-beamer
\begin{document}
\begin{frame}
% animateinline格式: \begin{animateinline}[<options>]{<frame_per_second>}...\end{animateinline}
% options为可选参数. 列表如下:
%   1)loop - 动画结束后进行循环
%   2)poster - 使用第几帧作为默认显示帧. 可选列表:
%     [1]first - 第一帧,默认值
%     [2]<num> - 第num帧,从0计数
%     [3]last - 最后一帧
%     [4]none - 没有帧作为默认显示
%   3)every - 每第n帧显示,其他帧不显示
%   4)autopause - 关闭页面时,暂停而非返回默认帧
%   5)autoplay - 打开页面时,自动开始动画(从上次暂停页面)
%   6)autoresume - 打开页面时,停留在上次暂停页面
%   7)palindrome - 动画结束后,向前或向后继续运行
%   8)step - 每点击一次,动画过一帧,该配置使frame_per_second被忽略
%   9)scale - 对动画组件进行放大/缩小
%   10)controls - 控制按钮的显示. 列表如下:
%     [1]all/true/on - 显示所有按钮
%     [2]none/false/off - 不显示按钮
%     [3]play/step/stop/speed - 分别显示播放/单步/停止/速度按钮,可同时指定该组内多个,使用','分隔
%   11)buttonsize - 指定按钮大小,必须为TeX单位,默认为1.44em
%   12)nomouse - 动画不响应鼠标操作
% frame_per_second代表动画中,每秒的帧数
    \begin{animateinline}[controls,loop]{50}
    % \multiframe用于指定连续帧,格式为:\multiframe{<count_of_frame>}{[<variable>]}{...}
    % count_of_frame代表连续帧的迭代次数
    % variable为可选变量,格式为<variable_name>=<init_val>+<increment>,每一次迭代变量值的增量为increment,定义变量名称不带\符号(引用时需要带\符号). 变量类型有三种, 由第一个字母确定, 列表如下:
    %   1)i/I - 整数
    %   2)n/N/r/R - 实数
    %   3)d/D - 带LaTeX长度单位的值
	\multiframe{180}{rt=-45+1}{%
	    \begin{tikzpicture}
	    \ifthenelse{\rt < 45}
	    {
		\draw[rounded corners,fill=cyan,rotate around={180-\rt:(2,0.2)}] (0,0) rectangle (4,0.4);
		\draw [fill=white] (2,0.2) circle (1mm);
		\draw [fill=darkgreen] (2,0.2) circle (0.5mm);
	    }
	    {
		\draw[rounded corners,fill=darkgreen,rotate around={90+\rt:(2,0.2)}] (0,0) rectangle (4,0.4);
		\draw [fill=white] (2,0.2) circle (1mm);
		\draw [fill=cyan] (2,0.2) circle (0.5mm);
	    };
	    \draw [>=stealth,->,very thick] ([shift=(175:2.15)]2,0.2) arc (175:135:2.15) node[xshift=-5pt,left] {$+45^\circ$};
	    \draw [>=stealth,->,very thick] ([shift=(185:2.15)]2,0.2) arc (185:225:2.15) node[xshift=-5pt,left] {$-45^\circ$};
	    \node at (4.0,4.0) {}; %phantom node
	    \node at (-4.0,-4.0) {}; %phantom node
	    \end{tikzpicture}}%
    \end{animateinline}
\end{frame}
\end{document}

% 拼接两段动画,第一段动画结束后,点击界面,再开始第二段动画
\begin{frame}
    \begin{animateinline}{50}
        \multiframe{101}{rt=0+1.732/100}{
            \begin{tikzpicture}
                \draw (0,0) -- (2,0);
                \draw (0,0) -- (60:2);
                \draw[red] (0,0) -- (30:\rt);
            \end{tikzpicture}}
        % 插入一个帧,需要点击界面才能跳过该帧
        \newframe*
        \multiframe{101}{rt=0+-1/100,rq=0+1.732/100}{
            \begin{tikzpicture}
                \draw (0,0) -- (2,0);
                \draw (0,0) -- (60:2);
                \draw[red] (0,0) -- (30:1.732);
                \draw[blue] (2,0) -- ++(\rt,\rq);
            \end{tikzpicture}}
    \end{animateinline}
\end{frame}


% 将.png格式帧导入pdf文件(可以由pdf格式动画帧,转化为png图片. 转化地址: https://ezgif.com/pdf-to-gif)
%   1.使用article等常规文档类型
%   2.使用animate和graphicx宏包
%   3.编译指令: latexmk -xelatex <tex_fil>
%   4.使用Okular PDF软件
\documentclass{ctexart}
\usepackage{tikz}
\usepackage{float}
\usepackage{animate}
\usepackage{graphicx}
\setlength{\parindent}{0mm}
\begin{document}
    \animategraphics[autoplay]{90}{picture/circle/circle-}{001}{362}
\end{document}

