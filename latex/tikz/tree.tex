\documentclass[UTF8,fontset=ubuntu]{ctexart}
\usepackage{tikz}
\begin{document}
% 树形结构
% \path ... child [<options>] <for_state> {<child_path>}
% options为可选参数. 列表如下:
%   1.grow - 将child按逆时针方向排列,可选参数列表:
%     [1]down/up/left/right - 分别为向下/上/左/右方向延伸
%     [2]north/south/west/east/north west/north east/south west/south east - 分别为北/南/西/东/西北/东北/西南/东南方向延伸
%     [3]角度值 - 向指定角度方向延伸. x轴正向为0度
%   2.grow' - 类似于grow,将child按顺时针方向排列
%   3.<color> - 给连线配置颜色
%   4.missing - 当前child被编号,但是内容被忽略(包含子节点)
\begin{tikzpicture}
    \coordinate
      child {}
      child {
        child {}
	child {}
      };
\end{tikzpicture}\\\vspace{1cm}

\begin{tikzpicture}
    \node {begin}[grow=12]
      child[red] {node {1}}
      child {node {2}
	child {node {1}}
	child {node {2}}
      };
\end{tikzpicture}\\\vspace{1cm}

\begin{tikzpicture}
    % 配置全局属性
    %   1.level distance - 父节点与子节点之间的距离
    %   2.sibling distance - 兄弟节点之间的距离. 通常在不同等级节点上分别配置
    %   3.every node/.style - 配置所有node的属性
    %   4.level <num>/.style - 配置指定等级子节点的属性
    [level distance=25mm,
     every node/.style={fill=red!20!green!40},
     level 1/.style={sibling distance=24mm,nodes={fill=red!35}},
     level 2/.style={sibling distance=6mm,nodes={fill=red!20}}]
    \node {开始}[grow=100]
        child {node {A}
          child {node {B}}
          child {node {C}}
          child {node {D}}
        }
	child {node {B}
          child {node {A}}
          child {node {C}}
          child {node {D}}
        }
        child {node {C}
          child {node {A}}
          child {node {B}}
          child {node {D}}
        }
        child {node {D}
          child {node {A}}
          child {node {B}}
          child {node {C}}
        };
\end{tikzpicture}

\begin{tikzpicture}
    [level distance=25mm,
     every node/.style={fill=red!8!green!40},
     level 1/.style={sibling distance=36mm,nodes={fill=red!45}},
     level 2/.style={sibling distance=6mm,nodes={fill=red!30}}]
    \node {开始}[grow'=north east]
      child foreach \x in {1,2,3,4,5,6} {node {\x}
	child foreach \y in {1,2,3,4,5,6} {node {\y}}
      };
\end{tikzpicture}
\end{document}
