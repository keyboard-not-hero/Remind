\documentclass[UTF8,fontset=ubuntu]{ctexart}
\usepackage{tikz}
\usetikzlibrary{trees}
\begin{document}
% pic, short picture
%   可以在picture内进行重复引用. 参考section 18

% 语法
%   \path pic <foreach_statements> [<options>] (<prefix>) at (<coordinate>) {<pic_type>}
%     所有pic与{之间的选项都是可选的
%     如果pic跟在坐标后,则at为非必选,否则at必须指定
%     如果带foreach选项,必须紧跟node关键字,其他可选选项位置不固定



% 定义pic type
%   <name>/.pic={
%     ...
%   }
\begin{tikzpicture}[line/.pic={\draw (-2,0) -- (2,0);}]
    \path pic[draw]{line};
\end{tikzpicture}


% 相关参数:
%   1.pic type
%     指定pic,并不再读取{<pic_type>}

%   2.pic text
%     pic注解文本


% angles库
%   帮助画角度的pic类型库
%   预定义pic type:
%     1)angle=<A>--<B>--<C>
%       作ABC角的标注
%     2)right angle=<A>--<B>--<C>
%       作ABC直角的标注
%     ** A/B/C必须为coordinate坐标
%   相关选项:
%     1)angle radius=<dimension>
%       该角度标注圆弧的半径
%     2)angle eccentricity=<factor>
%       注解文本的位置,从顶点到标注圆弧的比值系数. 初始为0.6
%       固定于起始线段到结束线段中间,
%     3)draw
%       作标注圆弧边框. 可指定颜色
%     4)fill
%       填充标准圆弧与夹角形成的闭合区域. 可指定颜色
%     5)line width
%       指定圆弧边框线条宽度
\begin{tikzpicture}
    \coordinate (A) at (2,0);
    \coordinate (B) at (0,0);
    \coordinate (C) at (50:2);
    \path pic[fill=blue!20,draw=red!40,line width=0.6pt,angle radius=3mm,angle eccentricity=1.6,pic text=1]{angle=A--B--C};
    \draw (A) -- (B) -- (C);
\end{tikzpicture}
\end{document}
