\documentclass{article}
\usepackage{tikz}
\begin{document}
% 标记角度
\usetikzlibrary{angles,quotes}
\begin{tikzpicture}
    % AB和BC交于点B, 标记由AB到BC的夹角,需要使用angles库
    % 参考section 41,第568页
    \coordinate (A) at (3,0);
    \coordinate (B) at (0,0);
    \coordinate (C) at (30:3);

    \draw[->] (-6,0) -- (6,0) node[right] {$x$};
    \draw[->] (0,-6) -- (0,6) node[above] {$y$};
    \draw (A) -- (B) -- (C);
    % 下列为弧度标记. A/B/C必须为node或coordinate类型,不能直接代入坐标点
    % pic可选参数列表:
    %   1.draw - 圆弧的颜色
    %   2.fill - 角度填充的颜色
    %   3.angle radius - 标记与夹角形成圆弧的半径
    %   4.angle eccentricity - 标度标记label的偏心距. 默认为0.6
    \draw pic[line width=2pt,draw=green!50,fill=green!20,angle radius=9mm,"$\alpha$",angle eccentricity=0.6] {angle=A--B--C};
\end{tikzpicture}\\\vspace{1cm}

\begin{tikzpicture}
    % 标记直角
    \coordinate (A) at (0,0);
    \coordinate (B) at (4,0);
    \coordinate (C) at (0,4);

    \draw (A) -- (B);
    \draw (A) -- (C);
    \draw pic[draw,angle radius=4mm]{right angle=B--A--C};
\end{tikzpicture}
\end{document}
