\documentclass{article}
\usepackage{tikz}
\begin{document}
% update 2025-04-16
% 1.主要作图环境
%   \begin{tikzpicture}[<option>] ... \end{tikzpicture}
%   所有tikz作图语句,必须放置在tikzpicture环境内,且需要导入tikz宏包
\begin{tikzpicture}
    \draw (-2,0) -- (2,0);
\end{tikzpicture}\vspace{1cm}


% 2.单语句或少量语句作图环境
%   \tikz[<options>]{...}
\tikz{\draw (-2,0) -- (2,0);}


% 3.成组指定属性
%   \begin{scope}[<option>] ... \end{scope}
%   给多个组内语句设置参数
\begin{tikzpicture}
    \begin{scope}[>=stealth,blue]
        \draw[->] (-2,0) -- (2,0) node[below]{$x$};
    \end{scope}
    \begin{scope}[>=latex,red]
        \draw[->] (-2,-1) -- (2,-1) node[below]{$x$};
    \end{scope}
\end{tikzpicture}\vspace{1cm}


% 4.打包参数. section 12.4.2
%   <name>/.style={<option>}
%     指定变量name为参数合集
%   <name>/.append style={<option>}
%     在已定义变量后append参数
%   <name>/.prefix style={<option>}
%     在已定义变量前prepend参数
%   <name>/.style={color=#1}
%     带参数调用格式
% 预定义style:
%   1)需调用
%   help lines/.style - 灰色度小,线条细的辅助线
%   2)无需调用,直接生效
%   every node/.style - 所有node的属性
%   every <shape> node/.style - 指定shape node的属性
%   every <part> node part/.style - 指定node part的属性
%   every label/.style - 每个label node的属性
%   every pin/.style - 每个pin node的属性
%   every pin edge/.style - 每个pin node连线的属性
%   every child/.style - 每个子节点的属性
%   every child node/.style - 每个子节点node的属性
%   level <num>/.style - 指定层级
%   level/.style - 所有层级,使用#1参数传递当前层级num,使得每层属性进行分别配置
\begin{tikzpicture}[my line/.style={color=#1}]
    \draw[help lines] (0,0) -- (1,0);
    \draw[my line=blue] (0,-2) -- (1,-2);
\end{tikzpicture}


% 相关参数:
% 5.图片与文字的对齐方式
% 默认图片的底部放在文字的基线上
%   1)baseline
%   将图片的x轴放在文字的基线上 
Hello
\begin{tikzpicture}[baseline]
    \draw (0,-2) circle [radius=2];
\end{tikzpicture}
World\\[2cm]

%   2)baseline=<dimension>
%   将图片的y=<dimension>放在文字的基线上 
Hello
\begin{tikzpicture}[baseline=-1cm]
    \draw (0,-2) circle [radius=2];
\end{tikzpicture}
World\\[2cm]

%   3)baseline=<point>
%   取坐标的y值n,将图片的y=n放在文字的基线上
Strike
\begin{tikzpicture}[baseline=(X.base)] 
    \node [name=X,strike out,draw] {me};
\end{tikzpicture}
out!

% 相关参数:
% 6.指定x/y轴默认单位长度
%   x=<dimension>
%   y=<dimension>
%   默认为1cm
\begin{tikzpicture}[x=0.5cm,y=0.5cm]
    \draw[->] (-2cm,0) -- (2cm,0) node[below]{$x$};
    \draw[->] (0,-2cm) -- (0,2cm) node[left]{$y$};
    \draw[red] (-2,0) -- (2,0);
\end{tikzpicture}
\end{document}
