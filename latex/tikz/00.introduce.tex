\documentclass{article}
\usepackage{tikz}
\begin{document}
% update 2025-04-16
% 1.主要作图环境
%   \begin{tikzpicture}[<option>] ... \end{tikzpicture}
%   所有tikz作图语句,必须放置在tikzpicture环境内,且需要导入tikz宏包
\begin{tikzpicture}
    \draw (-2,0) -- (2,0);
\end{tikzpicture}\vspace{1cm}


% 2.单语句或少量语句作图环境
%   \tikz[<options>]{...}
\tikz{\draw (-2,0) -- (2,0);}


% 3.成组指定属性
%   \begin{scope}[<option>] ... \end{scope}
%   给多个组内语句设置参数
\begin{tikzpicture}
    \begin{scope}[>=stealth,blue]
        \draw[->] (-2,0) -- (2,0) node[below]{$x$};
    \end{scope}
    \begin{scope}[>=latex,red]
        \draw[->] (-2,-1) -- (2,-1) node[below]{$x$};
    \end{scope}
\end{tikzpicture}\vspace{1cm}


% 4.打包参数. section 12.4.2
%   <name>/.style={<option>}
%     指定变量name为参数合集
%   <name>/.append style={<option>}
%     在已定义变量后append参数
%   <name>/.prefix style={<option>}
%     在已定义变量前prepend参数
\begin{tikzpicture}[help lines/.prefix style={red,line width=0.6pt}]
    \draw[help lines] (-2,0) -- (2,0);
\end{tikzpicture}


% 相关参数:
% 5.图片与文字的对齐方式
% 默认图片的底部放在文字的基线上
%   1)baseline
%   将图片的x轴放在文字的基线上 
Hello
\begin{tikzpicture}[baseline]
    \draw (0,-2) circle [radius=2];
\end{tikzpicture}
World\\[2cm]

%   2)baseline=<dimension>
%   将图片的y=<dimension>放在文字的基线上 
Hello
\begin{tikzpicture}[baseline=-1cm]
    \draw (0,-2) circle [radius=2];
\end{tikzpicture}
World\\[2cm]

%   3)baseline=<point>
%   取坐标的y值n,将图片的y=n放在文字的基线上
Strike
\begin{tikzpicture}[baseline=(X.base)] 
    \node [name=X,strike out,draw] {me};
\end{tikzpicture}
out!
\end{document}
