\documentclass[titlepage]{report}
\usepackage[toc]{multitoc}
\title{first paper}
\author{steven liu\thanks{compter/article}\\
electroc\\
lt19891121@live.cn \and
lucy\\
MBA\\
85xxxxx@qq.com
}
\date{\today}
\begin{document}
\maketitle
\renewcommand{\abstractname}{summary}
\begin{abstract}
this is abstract
\end{abstract}
\setcounter{tocdepth}{3}
\tableofcontents
\pagenumbering{Roman}
\part{part}
\pagenumbering{arabic}
\section{section}
\subsection{subsection}
\subsubsection{subsubsection}
\paragraph{paragraph}
\subparagraph{subparagraph}
This is my first document.
Welcome to \LaTeX !!
Steven Liu
\end{document}

% \documentclass[arg1,arg2,...]{doc_type}用于标示文档类型。
% 可选参数列表:
% twocolumn - 文档为双栏
% titlepage - 配置标题自成一页,摘要自成一页。默认article标题/摘要/正文融合为一页;report标题/摘要各自一页;book标题自成一页,无摘要环境
% notitlepage - 配置标题/摘要/正文融合为一页
% doc_type代表文档类型,类型列表:article/report/book/beamer

% 在双栏文档中,有公式:\columnwidth=(\textwidth-\columnsep)/2。其中,\columnwidth为单栏宽度;\textwidth为版心宽度;\columnsep为栏距;\columnseprule为双栏之间的垂直分割线,默认为0pt

% \begin{document}...\end{document}用于指定文档正文环境

% \title{}配置标题;\author{}配置作者;\and用于配置多个作者;\thanks{}用于配置标题或作者的脚注;\date{}用于配置论文生成日期;\today用于日期中,生成当前日期;\maketitle用于显示以上配置信息

% \renewcommand{\abstractname}{Summary}为修改摘要名称

% \begin{abstract}...\end{abstract}为摘要环境

% report/book标题层次:part/chapter/section/subsection/subsubsection/paragraph/subparagraph

% article标题层次:part/section/subsection/subsubsection/paragraph/subparagraph

% \setcounter用于配置计数器,\setlength用于配置长度
% \setcounter{tocdepth}{3}为配置目录深度,article默认为3,report默认为2,book默认为2

% \tableofcontents用于生成目录

% \listoffigures用于生成图片目录

% \listoftables用于生成表格目录

% \usepackage[toc]{multitoc}用于为双栏目录宏包。可选参数列表:
% 1.toc - 双栏目录
% 2.lof - 双栏图像目录
% 3.lot - 双栏表格目录
% 栏间距离为\columnsep,默认为10pt

% \pagenumbering{number_style}为配置页码计数方式,可为目录和正文分别配置一次,以区分页码。计数方式列表:
% 1.alph - 小写字母
% 2.Alph - 大写字母
% 3.arabic - 阿拉伯数字
% 4.roman - 小写罗马数字
% 5.Roman - 大写罗马数字



