\documentclass{article}
\begin{document} 
	This is my first document.
	Welcome to \LaTeX!!
	\newcommand{\myname}{Steven Liu}
	\myname
\end{document}

% 符号'%'用于注释。可使用\begin{comment}...\end{comment}进行块注释,该功能包含在comment宏包中

% \usepackage{comment}为使用宏包comment

% 命令格式如下:
% \command - 无参
% \command{args} - {}内为多字符必选参数
% \command[args] - []内为可选参数

% \documentclass{article}代表文档类型,文档类型列表:article/report/book/beamer
% article - 10页以内,不设置目录
% report/book - 10页~几百页,设置目录

% \documentclass到\begin{document}之间的为导言,导言指令对全文产生影响

% \begin{document} ... \end{document}标识文档正文范围

% 命令的形式:
% 1.声明 - 作用于指令后所有的内容。如粗体命令\bfseries 
% 2.参数 - 只作用于指令所带的参数。如倾斜体命令\textsl{Asia}
% 3.组合 - 把声明形式和需作用的内容至于花括号中,从而将作用范围限定在花括号内。如粗体命令{\bfseries this is bracket}
% 4.环境 - 由环境限定的作用范围。如公式限定\begin{equation}...\end{equation}

% \newcommand{<new_command>}[arg_number][first_arg_default]{<command_content>}为新建指令,解释如下:
% <new_command> - 新指令名称
% arg_number - 新指令可包含参数个数,默认为0
% first_arg_defualt - 当arg_number不为0时,该参数指定第一个可选参数的默认值,其他剩余参数必须为必选参数
% <command_content> - 新指令指代内容

% \renewcommand{<new_command>}[arg_number][first_arg_default]{<command_content>}为修改已有指令,解释见\newcommand
