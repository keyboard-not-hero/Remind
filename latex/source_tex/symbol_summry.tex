\documentclass[UTF8,fontset=ubuntu]{ctexart}
\usepackage{amsmath}
\usepackage{amssymb}
\usepackage{graphicx}
\begin{document}
\begin{table}
\begin{tabular}{c c c c c c c c}
	\hline
	符号 & 代码 & 符号 & 代码 & 符号 & 代码 & 符号 & 代码\\
	\hline
	$\backslash$ & $\backslash$backslash & $\{$ & $\backslash\{$ & $\}$ & $\backslash\}$ & $\sim$ & $\backslash$sim\\
	\$ & $\backslash$\$ & \% & $\backslash$\% & \^{} & $\backslash$\^{}$\{\}$ & \# & $\backslash$\#\\
	\& & $\backslash$\& & \_ & $\backslash$\_\\
	\hline
\end{tabular}\par
**$\backslash \sim \{ \}$是数学公式类符号\\
**也可使用$\backslash$verb$<$sep$><$content$><$sep$>$来抄录特殊字符,$\backslash$begin\{verbatim\}...$\backslash$end\{verbatim\}用于环境抄录\\
\caption{特殊符号}
\end{table}
\begin{table}
\begin{tabular}{l l l l l l l l}
	\hline
	符号 & 代码 & 符号 & 代码 & 符号 & 代码 & 符号 & 代码\\
	\hline
	$\alpha$ & $\backslash$alpha & $\beta$ & $\backslash$beta & $\gamma$ & $\backslash$gamma & $\delta$ & $\backslash$delta\\
	$\epsilon$ & $\backslash$epsilon & $\zeta$ & $\backslash$zeta & $\eta$ & $\backslash$eta & $\theta$ & $\backslash$theta\\
	$\iota$ & $\backslash$iota & $\kappa$ & $\backslash$kappa & $\lambda$ & $\backslash$lambda & $\mu$ & $\backslash$mu\\
	$\nu$ & $\backslash$nu & $\xi$ & $\backslash$xi & $\pi$ & $\backslash$pi & $rho$ & $\backslash$rho\\
	$\sigma$ & $\backslash$sigma & $\tau$ & $backslash$tau & $\upsilon$ & $\backslash$upsilon & $\phi$ & $\backslash$phi\\
	$\chi$ & $\backslash$chi & $\psi$ & $\backslash$psi & $\omega$ & $\backslash$omega & $\varepsilon$ & $\backslash$varepsilon\\
	$\vartheta$ & $\backslash$vartheta & $\varkappa$ & $\backslash$varkappa\footnotemark[1] & $\varpi$ & $\backslash$varpi & $\varrho$ & $\backslash$varrho\footnotemark[1]\\
	$\varsigma$ & $\backslash$varsigma & $\varphi$ & $\backslash$varphi & $\digamma$ & $\backslash$digamma\footnotemark[1]\\
	\hline
\end{tabular}
\footnotetext[1]{ams符号,包含在amssymb宏包中}
\caption{公式-小写希腊字母}
\end{table}
\begin{table}
\begin{tabular}{l l l l l l l l}
	\hline
	符号 & 代码 & 符号 & 代码 & 符号 & 代码 & 符号 & 代码\\
	\hline
	$\Gamma$ & $\backslash$Gamma & $\Delta$ & $\backslash$Delta & $\Theta$ & $\backslash$Theta & $\Lambda$ & $\backslash$Lambda\\
	$\Xi$ & $\backslash$Xi & $\Pi$ & $\backslash$Pi & $\Sigma$ & $\backslash$Sigma & $\Upsilon$ & $\backslash$Upsilon\\
	$\Phi$ & $\backslash$Phi & $\Psi$ & $\backslash$Psi & $\Omega$ & $\backslash$Omega & $\varGamma$ & $\backslash$varGamma\\
	$\varDelta$ & $\backslash$varDelta & $\varTheta$ & $\backslash$varTheta & $\varLambda$ & $\backslash$varLambda & $\varXi$ & $\backslash$varXi\\
	$\varPi$ & $\backslash$varPi & $\varSigma$ & $\backslash$varSigma & $\varUpsilon$ & $\backslash$varUpsilon & $\varPhi$ & $\backslash$varPhi\\
	$\varPsi$ & $\backslash$varPsi & $\varOmega$ & $\backslash$varOmega\\
\end{tabular}
**$\backslash$var格式的代码由amsmath宏包提供
\caption{公式-大写希腊字母}
\end{table}
\begin{table}
\begin{tabular}{l l l l l l l l}
	\hline
	符号 & 代码 & 符号 & 代码 & 符号 & 代码 & 符号 & 代码\\
	\hline
	$\sum$ & $\backslash$sum & $\prod$ & $\backslash$prod & $\coprod$ & $\backslash$coprod & $\int$ & $\backslash$int\\
	$\oint$ & $\backslash$oint & $\bigcup$ & $\backslash$bigcup & $\biguplus$ & $\backslash$biguplus & $\bigsqcup$ & $\backslash$bigsqcup\\
	$\bigvee$ & $\backslash$bigvee & $\bigwedge$ & $\backslash$bigwedge & $\bigcap$ & $\backslash$bigcap & $\bigodot$ & $\backslash$bigodot\\
	$\bigoplus$ & $\backslash$bigoplus & $\bigotimes$ & $\backslash$bigotimes & $\iint$ & $\backslash$iint & $\iiint$ & $\backslash$iiint\\
	$\iiiint$ & $\backslash$iiiint & $\idotsint$ & $\backslash$idotsint\\
\end{tabular}
**最后四个积分符号需要amsmath宏包
\caption{公式-大小可变的运算符}
\end{table}
\begin{table}
\begin{tabular}{l l l l l l l l}
	\hline
	符号 & 代码 & 符号 & 代码 & 符号 & 代码 & 符号 & 代码\\
	\hline
	$($ & $($ & $[$ & $[$ & $\{$ & $\backslash\{$ & $\langle$ & $\backslash$langle\\
	$)$ & $)$ & $]$ & $]$ & $\}$ & $\backslash\}$ & $\rangle$ & $\backslash$rangle\\
	$\lfloor$ & $\backslash$lfloor & $\lceil$ & $\backslash$lceil\\
	$\rfloor$ & $\backslash$rfloor & $\rceil$ & $\backslash$rceil\\
\end{tabular}\par
**在左/右括号前使用$\backslash$left/right可使定界符随视情况改变大小,left/right必须在同一行配对,但不需要匹配对应括号,可使用$\backslash<direction>$.来匹配,无可视单元。还有$\backslash$middle调节中间的定界符
**也可手动调节大小,位置:$\backslash$big $\backslash$bigl $\backslash$bigm $\backslash$bigr,规格:$\backslash$big $\backslash$Big $\backslash$bigg $\backslash$Bigg
\caption{公式-括号定界符}
\end{table}
\begin{table}
\begin{tabular}{l l l l l l l l}
	\hline
	符号 & 代码 & 符号 & 代码 & 符号 & 代码 & 符号 & 代码\\
	\hline
	$\ldots$ & $\backslash$ldots & $\cdots$ & $\backslash$cdots & $\vdots$ & $\backslash$vdots & $\iddots$ & $\backslash$iddots\\
\end{tabular}
\caption{公式-省略号}
\end{table}
\end{document}
