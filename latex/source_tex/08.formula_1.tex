\documentclass[UTF8, fontset=ubuntu]{ctexart}
%\usepackage{mathtools}
%\usepackage{tensor}
\usepackage{amsmath}
%\usepackage{amsmath}
%\usepackage{framed}
%\usepackage{color}
%\usepackage{pstricks}
%\usepackage[framed]{ntheorem}
%\usepackage{fancybox}
%\usepackage{xcolor}
%\usepackage{amscd}
\begin{document}
%	这是一个行内公式:$a+b=b+a$\par
%	这也是一个行内公式:\( a=b+1\)\par
%	这又是一个行内公式:
%	\begin{math}
%	y=x^2
%	\end{math}\par
	
%	这是一个行间公式:$$f(x)=mx+b$$
%	这还是一个行间公式:
%	\[a+b+c+d+e+f+g+h+i+j+k+l+m+n+o+p+q+r+s+t=t+s+r+q+p+o+n+m+l+k+j+i+h+g+f+e+d+c+b+a\]
%	这仍旧是一个行间公式:
%	\begin{displaymath}
%	a+b+c=c+b+a
%	\end{displaymath}

%	equation环境
%	\begin{equation}
%	a+b+c+d=d+c+b+a\tag*{2}
%	\end{equation}

%	\begin{equation*}
% 	y=\sin(x)
%	\end{equation*}	
	
%	$\int\limits_0^n x$\\
%	$\int\nolimits_0^n y$\\
%	$\prescript{n}{m}{H}^i_j$\\
%	$\tensor[_a^b_c^d]{M}{^a_b^c_d}$\\
%	$\overline{a+b}$\\
%	$\underline{a+b}$\\
%	$\overleftarrow{a+b}$\\
%	$\overrightarrow{a+b}$\\
%	$\overleftrightarrow{a+b}$\\
%	$\underleftarrow{a+b}$\\
%	$\underrightarrow{a+b}$\\
%	$\underleftrightarrow{a+b}$\\
%	$\overbrace{a+b+c}$\\
%	$\underbrace{a+b+c}$\\
%	$a+\rlap{$\overbrace{\phantom{b+c+d}}^{m}$}b+\underbrace{c+d+e}_{n}+f$\\
    \begin{flalign*}
        &\lim_{x\to0}\frac{\sin(x)}{x}=1&\\
        &\lim_{x\to0}\frac{\tan(x)}{x}=1&\\
        &\lim_{x\to0}\frac{1-\cos^2(x)}{x^2}=1&\\
        &\lim_{x\to0}\frac{1-\cos(x)}{x}=0&
    \end{flalign*}

%\begin{align}
%&\mathrel{\phantom{=}}(a+b)(a^2-ab+b^2) \notag\\
%&=a^3-a^2b+ab^2+a^2b-ab^2+b^3\notag\\
%&=a^3+b^3
%\end{align}

%\begin{align}
%	a_{11} &=b_{11} & a_{12} &=b_{12}\notag\\
%	a_{21} &=b_{21} & a_{22} &=b_{22}
%\end{align}

%\begin{align*}
% \lim_{x\to 5}\frac{\sqrt{x^2-9}-4}{x-5}&=\lim_{x\to 5}\frac{\sqrt{x^2-9}-4}{x-5}\times\frac{\sqrt{x^2-9}+4}{\sqrt{x^2-9}+4} \\
% & =\lim_{x\to 5}\frac{x^2-25}{(x-5)(\sqrt(x^2-9)+4)}\\
% & =\lim_{x\to 5}\frac{(x+5)(x-5)}{(x-5)(\sqrt{x^2-9}+4)}\\
% & =\lim_{x\to 5}\frac{x+5}{\sqrt{x^2-9}+4}\\
% & =\frac{5}{4}
%\end{align*}

%\begin{equation}
%f(x) = \begin{cases}
%1, & \text{if } x > 1;\\
%0, & \text{if } x < 1.
%\end{cases}
%\end{equation}

%\begin{equation}
%	a+b+c+d=d+c+b+a
%\end{equation}

%\begin{equation}
%\left.\begin{gathered}
%	S \subseteq T\\
%	S \subseteq T
%	\end{gathered} \right\}
%	\implies S = T
%\end{equation}

% 边框环境
%\begin{equation}
%\mathsurround=10pt
%\fbox{$\int_a^bf(x)\,dx=-\int_b^af(x)\,dx$}
%\end{equation}
%
%\begin{equation}
%\boxed{\int_a^bf(x)\,dx=-\int_b^af(x)\,dx}
%\end{equation}
%
%\begin{equation}
%\mathsurround=10pt
%\shadowbox{$\int_a^bf(x)\,dx=-\int_b^af(x)\,dx$}
%\end{equation}
%
%\begin{equation}
%\mathsurround=10pt
%\doublebox{$\int_a^bf(x)\,dx=-\int_b^af(x)\,dx$}
%\end{equation}
%
%\begin{equation}
%\mathsurround=10pt
%\colorbox{blue}{$\int_a^bf(x)\,dx=-\int_b^af(x)\,dx$}
%\end{equation}
%
%\begin{equation}
%\mathsurround=10pt
%\fcolorbox{blue}{pink}{$\int_a^bf(x)\,dx=-\int_b^af(x)\,dx$}
%\end{equation}

%\begin{equation*}
%\left.
%\begin{array}[c]{l r@{}l@{}l}
%\text{常数} & y & = & c\\
%\text{直线} & y & = & ax+b\\
%\text{抛物线} & y & = & ax^2+bx+c\\
%\end{array}
%\right\} \text{多项式}
%\end{equation*}

%\begin{equation}
%	\lim_{\substack{x\to\infty\\y\to\infty}}f(x)=8
%\end{equation}

%\theoremseparator{.}
%\theoremindent40pt
%\theoremsymbol{证毕}
%\theoremstyle{nonumberplain}
%\newframedtheorem{theorem1}{theorem}
%\begin{theorem1}
%if three line satisfy $3^2+4^2=5^2$,mean the Pythagorean theorem
%\end{theorem1}

%\begin{equation}
%\begin{CD}
%	A @>>> B\\
%	@| @VjVkV\\
%	D @<j<k< C\\
%\end{CD}
%\end{equation}

%\setlength{\multlinegap}{0pt}
%\begin{multline}
%	a^2+b^2=c^2\\
%	\shoveleft{y=f(x)}\\
%	g=2x^2+9
%\end{multline}

\end{document}

% $...$用于行内公式环境。其余还有\(...\)与\begin{math}...\end{math}。行内公式的上下角标普遍位于右上下角

% $$...$$用于行间公式环境。其余还有\[...\]与\begin{displaymath}...\end{displaymath}。以上都是单行行间公式

% amsmath宏包可选参数列表:
% centertags - 用于equation中包含split环境,此时编号垂直居中。默认选项
% tbtags - 用于equation中包含split环境,当编号位于右边时,编号位于底部;当编号位于左边时,编号位于顶部
% sumlimits - 在多行公式模式中,汇总符号的上下角标位于头部和顶部。如\sum、\prod。默认选项
% nosumlimits - 在多行公式模式中,汇总符号的上下角标位于右上下角
% intlimits - 类似于sumlimits,但作用于积分符号。如\int。
% nointlimits - 与intlimites相反。默认选项
% namelimits - 类似于sumlimits,但作用于det、inf、lim、max、min符号。默认选项
% nonamelimits - 与namelimits相反
% leqno - 编号位于公式左边
% reqno - 编号位于公式右边
% fleqn - 编号位于公式右边,但公式缩进

% \begin{equation}...\end{equation}为单行行间公式环境,该环境自动配置一个公式号码,并且换行\\无效。行间公式的上下角标普遍位于正上下方。包含在amsmath宏包中

% \left<>和\right<>用于公式起始和结束时自动调整定界符大小,如:\left.为起始限定符为空,\left\{为起始限定符为{

% \begin{equation*}...\end{equation*}与equation环境类似,但与equation不同,它不配置公式号码。包含在amsmath宏包中

% \indent用于当前行行首缩进,\noindent用于当前行行首不缩进

% \limits用于限定后续上下角标位于正上下方,\nolimits限制上下角标位于右方上下角。该指令包含在amsmath宏包中

% \notag取消行参与编号,在\\之前插入;\tag{label}使用指定公式编号,格式为(number),在\\之前插入;\tag*{label}使用指定公式编号,与\tag不同,格式不包含(),在\\之前插入。包含在amsmath宏包中

% \prescript{super_script}{sub_script}{variable}用于显示左上下角标。该指令包含在mathtools宏包中

% \tensor[left_script]{variable}{right_script}用于显示左右方的上下角标,并且上下角标错次排列。该指令包含在tensor宏包中

% 关于化学式的上下角标,参考mhchem宏包

% \overline用于上横线,\underline用于下横线

% \overleftarrow用于上方向左箭头,类似列表:overleftarrow/overrightarrow/overleftrightarrow/underleftarrow/underrightarrow/underleftrightarrow。该指令集合包含在amsmath宏包中

% \overbrace{}和\underbrace{}用于上下大括号

% \rlap{}将环境内的内容与环境右边的内容进行左对齐重叠

% \phantom{}将环境内的内容进行占位,但不显示

% \begin{multline}...\end{multline}用于多行行间公式,可使用\\分行,只有末行参数编号,默认第一行左对齐,中间行居中对齐,最后一行右对齐。\shoveleft{}强制某一中间行左对齐,\shoveright{}强制某一中间行右对齐。\setlength{\multlinegap}{0pt}用于配置首行与左边界的距离,默认为10pt。包含在amsmath宏包

% \begin{split}...\end{split}用于多行行间公式,可使用\\分行。与multline区别:
% 1.可使用&分列,但最多只能分为两列,第一列右对齐,第二列左对齐
% 2.必须置于除multline的其他数学环境中
% 3.不生成公式编号,由外部数学环境生成,并垂直居中

% \begin{gather}...\end{gather}用于多行行间公式,可以使用\\来分行(最后一行不能使用\\),每行都参与公式编号,每行在垂直方向上居中对齐,并且使用\notag来使当前行不使用公式序号。该环境内的公式应该使用split环境限定。\begin{gather*}...\end{gather*}是无编号版本。包含在amsmath宏包中

% \begin{align}...\end{align}用于多行行间公式,可以使用\\分行,每行分别参与编号,并且在关系运算符之前使用&来表明对齐位置,并且以&分列,奇数列右对齐,偶数列左对齐。\begin{align*}..,\end{align*}是无编号版本。包含在amsmath宏中

% \begin{alignat}{equa_col}...\end{alignat}类似于align,但同一行的公式之间无空格,需显式指定,equa_col指定一行的公式数量,该数字等于(&数量+1)/2。\begin{alignat*}为无编号版本。包含在amsmath中

% \begin{flalign}...\end{flalign}类似于align,但同一行的公式之间空格尽量大。\begin{flalign*}为无编号版本。包含在amsmath中
% 可使用flalign来创造公式在文档中靠近左边缘,公式之间的空格尽量大...

% \begin{gathered}...\end{gathered}类似于gather,但与gather不同,它可以使用行的剩余空间,需放置于数学环境内。包含在amsmath宏包中

% \phantom{}参与占位,但不显示内容;\mathrel可保证占位空间

% \begin{cases}...\end{cases}用于在数学公式环境中构造条件公式,可使用&来区分公式内容和条件。包含在amsmath宏包中。另外,mathtools宏包的dcases环境可以保证公式为行间公式大小

% \text{}用于在公式中插入文本

% \overbracket[line_width][bracket_height]{math_formula}用于上方括号。包含在mathtools

% \boldmath可将希腊字母更改为粗体模式,该指令用在数学公式范围之外

% \fbox{inline_formula}用于给行内公式增加边框,\mathsurround用于增加公式与边框左右位置的间距,\,用于增加公式内的间隔空间

% \boxed{text}用于给行内公式增加边框。包含在amsmath中

% \shadowbox{inline_formula}和\doublebox{inline_formula}分别用于阴影边框和双层边框,类似于\fbox使用方法。包含在fancybox宏包中

% \colorbox{bg_color}{inline_formula}和\fcolorbox{box_color}{bg_color}{inline_formula}用于配置背景色边框,类似于\fbox。包含在xcolor宏包中

% \begin{array}[position]{col_format1 col_format2 ...}...\end{array}用于数学环境的表格,参数形式与tabular完全相同

% 在文类book和report中,行间公式是以章为排序单位的,跨越章节时计数器重置;在文类article中,行间公式以全文为排序单位。可使用\numberwithin{equation}{section}将节作为排序单位

% \substack可在上下角标内换行,包含在amsmath宏包中

% \newtheorem{env_name}[reuse_counter]{title}[sort_base]用于定义定理等环境,env_name为定理计数器名称;reuse_counter为当前定时器与之共用的已定义计数器;title为定理名称,全称为定理名称+计数器;sort_base为排序单位,默认为全文,可指定chapter或section等指定以章或者节为排序单位

% \begin{env_name}[sub_title]...\end{env_name}可指定特定计数器定理环境,sub_title用于副标题


% 更深入的定理环境参考宏包ntheorem,宏包可选参数列表:
% framed - 配合framed宏包使用

%该宏包指令集合:
% \newtheorem{env_name}[reuse_counter]{title}[sort_base],类似于系统提供的newtheorem环境

% \renewtheorem{env_name}[reuse_counter]{title}[sort_base],重定义定理环境,但计数器不重新初始化

% \theoremstyle{style}表明定理风格,可用列表如下:
% plain - 原始LaTeX风格。默认选项
% break - 定理头部与定理正文分别占一行
% change - 定理号码在定理头部起始部位
% changebreak - 类似于change,并且定理头部后跟line break
% margin - 定理号码在定理起始部位,并且位于左边缘
% marginbreak - 类似于margin,定理头部后跟line break
% nonumberplain - 类似于plain,但没有号码。类似于使用\newtheorem*{env_name}{theorem_head}
% nonumberbreak - 类似于break,但没有号码
% empty - 没有定理头部

% \theoremheaderfont{font}为定理头部字体

% \theorembodyfont{font}为定理内容字体

% \theoremnumbering{style}为设置定理数字格式,可选列表如下:
% arabic - 阿拉伯数字
% alph - 小写字母
% Alph - 大写字母
% roman - 小写罗马数字
% Roman - 大写罗马数字
% greek - 小写希腊字母
% Greek - 大写希腊字母

% \theoremseparator{separator}定理头部与定理正文的分隔符

% \theorempreskip{skip}定理之前的垂直距离

% \theorempostskip{skip}定理之后的垂直距离

% \theoremindent<dimen>定理缩进距离

% \theoremsymbol{str}定理结尾语,配合\usepackage[thmmarks]{ntheorem}使用

% \newframedtheorem{env_name}[reuse_counter]{title}[sort_base],提供带边框的定理环境,需要framed宏包和ntheorem宏包的framed可选参数

% \begin{CD}...\end{CD}为交换图环境,包含在amscd宏包。图标列表:
% @>>> - 右箭头,@>j>>代表注解放在箭头上方,@>>j>代表注解放在箭头下方
% @<<< - 左箭头,注解方向同右箭头
% @AAA - 上箭头,@AjAA代表注解放在箭头左边,@AAjA代表注解放在箭头右边
% @VVV - 下箭头,注解方向同上箭头
% @= - 水平双实线
% @| - 垂直双实线