\documentclass[a4paper, titlepage]{article}
%\usepackage{syntonly}
%\syntaxonly
\title{Demo}
\author{steven}
\date{\today}
\begin{document}
\maketitle
\tableofcontents
\section{section}
this is section content
\subsection{subsection}
this is subsection content
%\include{chapter_02}
%\input{chapter_03}
\end{document}

% \documentclass[a4paper, titlepage]{article}的a4paper代表页为A4纸大小,titlepage代表标题独立占一页

% \begin{document}之前的内容为引言

% \title表示标题,\author表示作者,\date表示日期

% \maketitle使引言的标题部分显示生效

% \tableofcontents用于配置目录索引。其中,常用结构层级:chapter/section/subsection/subsubsection。report/book文档层级chapter/section/subsection,article文档层级section/subsection/subsubsection。目录需要经过两次编译,第一次编译将目录结构输出到.toc文件,第二次编译将.toc文件目录结构输出到目标文档

% \include{file_name}用于将指定文件内的内容复制过来,导入后会在新页开始。通常用于整个章节导入

% \input{file_name}用于将指定文件内的内容复制过来,在导入的地方开始,通常用于小文件导入

% \syntaxonly用于注明 - 文档只进行文档检查,不输出dvi/pdf文档,可加快编译速度。该指令包含在syntonly宏包中