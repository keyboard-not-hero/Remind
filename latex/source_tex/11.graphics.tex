\documentclass{article}
\usepackage[final]{graphicx}
\usepackage{xcolor}
\usepackage{picinpar}
\usepackage{wallpaper}
\begin{document}
%\includegraphics[height=6cm,angle=45]{snuggle.jpg}

%\rotatebox{90}{i love you}

%Hello\qquad\scalebox{-1}[1]{\color[gray]{0.6}Hello}\\
%Hello\\\scalebox{1}[-1]{\color[gray]{0.6}Hello}

%\begin{figwindow}[1,r,{\includegraphics[scale=0.2]{snuggle.jpg}},{dog}]
%After the rebels have been brutally overpowered by the Empire, Luke Skywalker takes advanced Jedi training with Master Yoda, while his friends are pursued by Darth Vader as part of his plan to capture Luke.
%\end{figwindow}

%\TileSquareWallPaper{2}{snuggle}
%\ThisTileSquareWallPaper{2}{snuggle}
%\TileWallPaper{100pt}{150pt}{snuggle}
\ThisTileWallPaper{100pt}{150pt}{snuggle}
this is a page
\newpage
this is another page

\end{document}

% 图片类型;
% 1.位图 - 由像素栅栏组成,旋转伸缩会造成失真
% 2.向量图 - 由线条组成,旋转伸缩不会失真
% 优缺点:位图存储空间较大;向量图打开时间较慢,因为向量图由向量算法构成

% latex支持图形文件格式列表:
% 1.EPS - Encapsulated PostScript文件,可记录位图图形和向量图形
% 2.PS - PostScript,一种页面描述和编程语言,保存的图形为向量图
% 3.JPG - Joint Photographc Experts Group,有损压缩图形
% 4.PNG - Portable Network Graphics,无损压缩图形
% 5.PDF - Portable Document Format,可移植文档格式,较之前的PS更强大且压缩率更高
% 图形格式转换工具ImageMagick,该工具为指令式,指令为convert sour_file tar_file


% \includegraphics[height,totleheight,width,scale,origin,angle]{<file_name>}用于插入图片
% scale为缩放比例,也可以使用width/height来指定缩放宽度或高度
% origin为旋转固定位置,水平方向可选列表l/c/r;垂直方向列表/t/c/B/b,其中B代表Baseline;水平方向和垂直方向组合数为3*4=12
% angle为逆时针旋转角度
% file_name为文件名称,包含或不包含后缀名都可,可使用文件格式jpg/png/pdf;一般为源文件目录,也可以使用相对路径和绝对路径
% 该功能包含在graphicx宏包中。在使用宏包的过程中,可使用可选参数draft来将图片插入图片位置留空,以此来加快编译速度;与draft相对的是final选项

% \graphicspath{{path1}{path2}...}可用于指定图片搜索路径,路径可以为绝对路径和相对路径。包含在graphicx宏包中

% \rotatebox[arg1=val1,arg2=val2,...]{angle}{object}用于旋转文本、图形、表格等,由graphicx宏包提供。可选参数列表:
% 1.origin - 旋转点,参考\includegraphics相同参数
% 2.x,y - 以基准点作为原点,给出指定坐标为旋转点
% 3.units - 设置旋转角度的单位,默认为度并逆时针旋转。-360设置单位为度并顺时针旋转;6.283185设置单位为弧度并且逆时针旋转

% \scalebox{horizontal_factor}[vertical_factor]{object}用于内容缩放,horizontal_factor为水平缩放系数,为负数时反转180度再缩放;vertical_factor为垂直缩放系数,不提供时等于水平系数

% \begin{figwindow}[line_upon_pic,position,{pic},{title}]...\end{figwindow}用于文字绕排,line_upon_pic代表位于图片上方的行数;position代表图片位于绕排文字的位置,可选l/c/r;pic为图片;title为图片标题

% \TileSquareWallPaper{tile_number}{pic}用于配置墙纸背景,tile_number代表用多少张图片平铺满水平方向;pic为图片文件名称。\ThisTileSquareWallPaper为只作用于当前页的版本。包含在wallpaper宏包中

% \TileWallPaper{pic_width}{pic_height}{pic}用于配置墙纸背景,pic_width配置图片宽度;pic_height配置图片高度;pic为图片文件名称。\ThisTileWallPaper为只用于与当前页的版本。包含在wallpaper宏包中
