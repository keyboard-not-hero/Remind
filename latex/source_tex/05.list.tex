\documentclass{article}
\usepackage{bbding}
\begin{document}
%  \begin{enumerate}
%	\itemsep=0pt
%	\parskip=0pt
%    \item number one
%    \item number two
%    \item number three
%  \end{enumerate}
  
  \begin{itemize}
  	\renewcommand{\labelitemi}{\PencilRight}
	\itemsep=0pt
	\parskip=0pt
	\item item one
	\item item two
	\item item three
  \end{itemize}
  
%  \begin{description}
%    \item[high] high light
%    \item[low] low light
%  \end{description}
\end{document}

% \begin{enumerate}...\end{enumerate}用于指定编号列表环境,列表entry使用\item[arg]指定,可选参数可用于修改默认item标识符

% \begin{itemize}...\end{itemize}用于指定无编号列表,其他特性与编号列表类似

% \begin{description}...\end{description}用于指定自定义编号,列表entry格式\item[]中的可选参数一般都会使用

% 所有列表皆可进行嵌套(最多四层)

%编号列表嵌套计数器为enumi/enumii/enumiii/enumiv,可通过\the<counter_name>进行引用(如1),通过\label<counter_name>进行标签引用(如1.)。编号列表嵌套格式:默认第一层为数字,第二层为小写字母,第三层为罗马数字,第四层为大写字母

% \renewcommand{\labelitemi}{\PencilRight}用于配置无编号列表的前置标记符号,嵌套的各层级标记符号\labelitemi、\labelitemii、\labelitemiii、\labelitemiv。外层标记默认为\textbullet。\PencilRight标记包含在bbding宏包中

% 所有列表的item与item之间有两个叠加额外空白,分别使用\itemsep和\parskip指定,默认值都为4pt plus 2pt minus 1pt
