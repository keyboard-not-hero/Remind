%\documentclass[UTF8,fontset=ubuntu]{ctexart}
\documentclass{article}
\usepackage[UTF8,fontset=ubuntu,space]{ctex}
\usepackage{graphicx}
\linespread{2}
\begin{document}
\parindent=0pt
这是一个中文文~档!!\\
\songti 我是宋体\par
%\heiti 我是黑体\par
\kaishu 我是楷书\par
\zihao{0}我是初号\par
\zihao{-0}我是小初号\par
\zihao{1}我是一号\par
\zihao{-1}我是小一号\par
\resizebox{!}{20mm}{China  中国}\par
  有时候我觉得自己像一只小小鸟,想要飞却怎么样也飞不高,也许有一天我栖上了枝头,却成为猎人的目标,我飞上了青天才发现自己从此无依无靠,每次到了夜深人静的时候,我总是睡不着,我怀疑是不是只有我的明天没有变得更好\par
  \renewcommand{\CJKglue}{\hskip 1pt}
  有时候我觉得自己像一只小小鸟,想要飞却怎么样也飞不高,也许有一天我栖上了枝头,却成为猎人的目标,我飞上了青天才发现自己从此无依无靠,每次到了夜深人静的时候,我总是睡不着,我怀疑是不是只有我的明天没有变得更好
\end{document}

% \documentclass[UTF8,fontset=ubuntu]{ctexart}中,ctexart代表中文文档类型;UTF8代表中文编码方式;fontset=ubuntu代表字体定义文件,Linux系统必须额外该选项,Windows/MacOS系统没有必要

% \usepackage[UTF8,fontset=ubuntu]{ctex}为另一种使用ctex包的方式

% 在CJK中文宏包中,字与字之间空格被省略,可使用~符号生成空格。或者使用usepackage的space可选参数,来保留字与字之间的空格

% 中文字体默认使用宋体,ubuntu提供字体宋体/黑体/楷书,windows提供字体宋体/黑体/仿宋/楷书/隶书/幼圆

% \zihao{number}提供字号,字号列表 - 初号/小初号/一号/小一号/二号/小二号/三号/小三号/四号/小四号/五号/小五号/六号/小六号/七号/八号。小开头对应-符号,如小初号-0

% \resizebox{width}{height}{text}用于指定文字盒子的宽度和高度,width和height为!时,可使该参数指定为根据原高宽比来缩放或扩展

% 查看当前中文字体:fc-list :lang=zh-cn

% \renewcommand{\CJKglue}{\hskip 1pt}用于配置汉字之间的间距

% \linespread{factor}用于在导言中修改全文中文行距,该指令需要在ctex导入之后使用
