\documentclass{article}
\usepackage{xfrac}
\usepackage{mathtools}
\usepackage{amsmath}
\usepackage{bm}
\begin{document}
	\noindent $\frac{1}{2}$\\
	$\sfrac{1}{2}$\\
	$\binom{4}{3}$\\
	$\sqrt[3]{8}$\\
\[
\begin{vmatrix}
 	a_11 & a_12 & a_13\\
 	a_21 & a_22 & a_23\\
 	\vdots & \ddots & a_33\\
 	a_n1 & \dots & a_n3\\
\end{vmatrix}
\]
\[
\begin{vmatrix*}[r]
	9 & 100 & 2\\
	-1001 & 3 & 90\\
	100 & 29 & 11\\
\end{vmatrix*}
\]
$\hm\int > \bm\int = \pmb\int > \int$
\begin{equation}
	e^{\pi i}+1=0
\end{equation}
\end{document}

% \frac{}{}用于分数形式

% \sfrac{}{}用于斜杠分数形式。包含在xfrac宏包中

% \binom{}{}用于二项式形式。包含在amsmath宏包中

% \sqrt[]{}用于开方,可选参数代表开方的次数,必选参数代表被开方的数字。可使用\mathstrut表示较统一的开方框架大小,放置于被开方数字之前

% \begin{vmatrix}...\end{vmatrix}用于配置矩阵环境。行元素之间使用&分隔,行之间使用\\分隔

% \dots为横向省略号,\vdots为纵向省略号,\ddots为斜向省略号

% \begin{vmatrix*}[<align>]...\end{vmatrix*}为可指定对齐方式的矩阵。由mathtools宏包提供

% 矩阵环境默认最多只有10列,可使用\setcounter{MaxMatrixCols}{<number>}来配置

% \hm用于公式字符使用加重粗体,\bm用于公式字符使用粗体。由bm宏包提供

% \pmb用于公式字符使用伪粗体(由字体稍稍错位连续输出得到,当无法使用粗体时,可使用该方法)。由amsmath宏包提供

