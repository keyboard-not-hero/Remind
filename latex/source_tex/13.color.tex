\documentclass{article}
%\usepackage{color}
\usepackage{colortbl}
\usepackage{xcolor}
\begin{document}
%	\fcolorbox{green}{yellow}{this is box}
%	\color[cmyk]{0,0.5,1.0,0}i love you
%	\textcolor{blue}{this is comment }do you love me

\begin{tabular}{>{\columncolor{red}}c c c}
	\arrayrulecolor{black}
	\hline
	\rowcolor{green} name & sex & score \\
	lucy & female & 89 \\
	lily & \cellcolor{blue}female & 90 \\
	leijun & \textcolor{green}{male} & 98 \\
	\hline
\end{tabular}
\definecolorseries{mycolors}{rgb}{last}{red}{yellow}
\resetcolorseries[5]{mycolors}
\color{mycolors!!+}i like 
\color{mycolors!!+}this beautiful 
\color{mycolors!!+}girl, but
\color{mycolors!!+}you don't
\color{mycolors!!+}know this.
\end{document}

% \color{color}用于指定声明式字体颜色;\textcolor{color}{text}用于指定局部字体颜色;\pagecolor{color}用于指定页面背景颜色;\colorbox{box_color}{text}用于指定盒子背景颜色;\fcolorbox{border_color}{box_color}{text}用于指定盒子的边框和背景颜色,边框大小为\fboxrule。包含在color宏包中
% 颜色列表:
% black/white/red/green/blue,以及印刷三原色cyan/magenta/yellow
% 颜色模式列表:
% gray - 灰度模式。color[gray]{0.5}
% rgb - 光谱三基色模式,red/green/blue。color[rgb]{0.6,0.6,0}
% cmyk - 印刷四分色模式,cyan/magenta/yellow/black。color[cmyk]{0,0.5,1,0}
 
% >{columncolor{color}}用于配置表格列的背景色,\rowcolor{color}用于配置表格行的背景色,\cellcolor{color}用于配置表格单元的背景色,\arrayrulecolor{color}用于配置表格框颜色,包含在colortbl宏包中

% \definecolor{color_name}{mode}{define}用于定义颜色,color_name为定义的颜色名称;mode为颜色模式;define为基色占比,通常为0~1。包含在xcolor宏包中

% 色系是一系列光谱相近的颜色排列

% \definecolorseries{color_ser_name}{mode}{last}{first_color}{last_color}用于定义色系,color_ser_name为色系名称;mode为使用的颜色模式;last为色差值计算方法;first_color为该色系首个颜色名称;last_color为该色系末尾颜色名称。首颜色和末尾颜色可选颜色:red/orange/yellow/green/cyan/blue/purple/white/black,且两者不宜间隔超过两个颜色,如red和yellow间隔一个颜色

% \resetcolorseries[level_number]{color_ser_name}用于初始化色系分配的级数,用于首颜色到末尾颜色的逐级过渡,level_number为级数分隔次数,如level_number为5时,0~5为色系6个颜色级别;color_ser_name为色系名称

% \color{color_ser_name!!+}用于从色系中的index=0开始使用,!!代表色系,+代表使用当前级别后,进行index +1。额外的,++代表index +2,+++代表index +3

% \color{color_ser_name!![index]}代表从色系中取用指定级别索引
