\documentclass[UTF8,fontset=ubuntu]{ctexart}
\usepackage{amsmath}
\usepackage{amssymb}
\usepackage{graphicx}
\usepackage{mathdots}
\usepackage{ragged2e}
\begin{document}
\begin{table}
\begin{tabular}{c c c c c c c c}
	\hline
	符号 & 代码 & 符号 & 代码 & 符号 & 代码 & 符号 & 代码\\
	\hline
	\textbackslash & \textbackslash textbackslash & \{ & \textbackslash\{ & \} & \textbackslash\} & \~{} & \verb|\~{}|\\
	\$ & \textbackslash\$ & \% & \textbackslash\% & \^{} & \verb|\^{}| & \# & \textbackslash\#\\
	\& & \textbackslash\& & \_ & \textbackslash\_\\
	\hline
\end{tabular}\par
\RaggedRight
**也可使用\textbackslash verb\verb'<'sep\verb'>'\verb'<'content\verb'>'\verb'<'sep\verb'>'来抄录单行特殊字符,$\backslash$begin\{verbatim\}...$\backslash$end\{verbatim\}用于环境式多行抄录\\
\caption{专用符号}
\end{table}
\begin{table}
\begin{tabular}{l l l l l l}
	\hline
	符号 & 代码 & 符号 & 代码 & 符号 & 代码\\
	\hline
	\textbar & \textbackslash textbar & \textless & \textbackslash textless & \textgreater & \textbackslash textgreater\\
	\hline
\end{tabular}\par
\RaggedRight
**也可使用\textbackslash verb\verb'<'sep\verb'>'\verb'<'content\verb'>'\verb'<'sep\verb'>'来抄录单行特殊字符,$\backslash$begin\{verbatim\}...$\backslash$end\{verbatim\}用于环境式多行抄录\\
\caption{键盘符号}
\end{table}
\begin{table}
\begin{tabular}{l l l l l l l l}
	\hline
	符号 & 代码 & 符号 & 代码 & 符号 & 代码 & 符号 & 代码\\
	\hline
	$\alpha$ & $\backslash$alpha & $\beta$ & $\backslash$beta & $\gamma$ & $\backslash$gamma & $\delta$ & $\backslash$delta\\
	$\epsilon$ & $\backslash$epsilon & $\zeta$ & $\backslash$zeta & $\eta$ & $\backslash$eta & $\theta$ & $\backslash$theta\\
	$\iota$ & $\backslash$iota & $\kappa$ & $\backslash$kappa & $\lambda$ & $\backslash$lambda & $\mu$ & $\backslash$mu\\
	$\nu$ & $\backslash$nu & $\xi$ & $\backslash$xi & $\pi$ & $\backslash$pi & $\rho$ & $\backslash$rho\\
	$\sigma$ & $\backslash$sigma & $\tau$ & $\backslash$tau & $\upsilon$ & $\backslash$upsilon & $\phi$ & $\backslash$phi\\
	$\chi$ & $\backslash$chi & $\psi$ & $\backslash$psi & $\omega$ & $\backslash$omega & $\varepsilon$ & $\backslash$varepsilon\\
	$\vartheta$ & $\backslash$vartheta & $\varkappa$ & $\backslash$varkappa${}^1$ & $\varpi$ & $\backslash$varpi & $\varrho$ & $\backslash$varrho${}^1$\\
	$\varsigma$ & $\backslash$varsigma & $\varphi$ & $\backslash$varphi & $\digamma$ & $\backslash$digamma${}^1$\\
	\hline
\end{tabular}
**$\backslash$var格式的代码由amsmath宏包提供\par
${}^1$\hspace{1mm}ams符号,包含在amssymb宏包中
\caption{公式-小写希腊字母}
\end{table}
\begin{table}
\begin{tabular}{l l l l l l l l}
	\hline
	符号 & 代码 & 符号 & 代码 & 符号 & 代码 & 符号 & 代码\\
	\hline
	$\Gamma$ & $\backslash$Gamma & $\Delta$ & $\backslash$Delta & $\Theta$ & $\backslash$Theta & $\Lambda$ & $\backslash$Lambda\\
	$\Xi$ & $\backslash$Xi & $\Pi$ & $\backslash$Pi & $\Sigma$ & $\backslash$Sigma & $\Upsilon$ & $\backslash$Upsilon\\
	$\Phi$ & $\backslash$Phi & $\Psi$ & $\backslash$Psi & $\Omega$ & $\backslash$Omega & $\varGamma$ & $\backslash$varGamma\\
	$\varDelta$ & $\backslash$varDelta & $\varTheta$ & $\backslash$varTheta & $\varLambda$ & $\backslash$varLambda & $\varXi$ & $\backslash$varXi\\
	$\varPi$ & $\backslash$varPi & $\varSigma$ & $\backslash$varSigma & $\varUpsilon$ & $\backslash$varUpsilon & $\varPhi$ & $\backslash$varPhi\\
	$\varPsi$ & $\backslash$varPsi & $\varOmega$ & $\backslash$varOmega\\
	\hline
\end{tabular}
**$\backslash$var格式的代码由amsmath宏包提供
\caption{公式-大写希腊字母}
\end{table}
\begin{table}
\begin{tabular}{l l l l l l l l}
	\hline
	符号 & 代码 & 符号 & 代码 & 符号 & 代码 & 符号 & 代码\\
	\hline
	$\sum$ & $\backslash$sum & $\prod$ & $\backslash$prod & $\coprod$ & $\backslash$coprod & $\int$ & $\backslash$int\\
	$\oint$ & $\backslash$oint & $\bigcup$ & $\backslash$bigcup & $\biguplus$ & $\backslash$biguplus & $\bigsqcup$ & $\backslash$bigsqcup\\
	$\bigvee$ & $\backslash$bigvee & $\bigwedge$ & $\backslash$bigwedge & $\bigcap$ & $\backslash$bigcap & $\bigodot$ & $\backslash$bigodot\\
	$\bigoplus$ & $\backslash$bigoplus & $\bigotimes$ & $\backslash$bigotimes & $\iint$ & $\backslash$iint & $\iiint$ & $\backslash$iiint\\
	$\iiiint$ & $\backslash$iiiint & $\idotsint$ & $\backslash$idotsint\\
	\hline
\end{tabular}
**最后四个积分符号需要amsmath宏包
\caption{公式-大小可变的运算符}
\end{table}
\begin{table}
\begin{tabular}{l l l l l l l l l l}
	\hline
	符号 & 代码 & 符号 & 代码 & 符号 & 代码 & 符号 & 代码 & 符号 & 代码\\
	\hline
	$\log$ & $\backslash$log & $\lg$ & $\backslash$lg & $\ln$ & $\backslash$ln & $\sin$ & $\backslash$sin & $\arcsin$ & $\backslash$arcsin\\
	$\cos$ & $\backslash$cos & $\arccos$ & $\backslash$arccos & $\tan$ & $\backslash$tan & $\arctan$ & $\backslash$arctan & $\cot$ & $\backslash$cot\\
	$\sinh$ & $\backslash$sinh & $\cosh$ & $\backslash$cosh & $\tanh$ & $\backslash$tanh & $\coth$ & $\backslash$coth & $\sec$ & $\backslash$sec\\
	$\csc$ & $\backslash$csc & $\arg$ & $\backslash$arg & $\ker$ & $\backslash$ker & $\dim$ & $\backslash$dim & $\hom$ & $\backslash$hom\\
	$\exp$ & $\backslash$exp & $\deg$ & $\backslash$deg\\
	\hline
\end{tabular}
\caption{不带上下限的数学运算符}
\end{table}
\begin{table}
\begin{tabular}{l l l l l l l l}
	\hline
	符号 & 代码 & 符号 & 代码 & 符号 & 代码 & 符号 & 代码\\
	\hline
	$\lim$ & $\backslash$lim & $\limsup$ & $\backslash$limsup & $\liminf$ & $\backslash$liminf & $\max$ & $\backslash$max\\
	$\min$ & $\backslash$min & $\sup$ & $\backslash$sup & $\inf$ & $\backslash$inf & $\det$ & $\backslash$det\\
	$\Pr$ & $\backslash$Pr & $\gcd$ & $\backslash$gcd & $\varliminf$ & $\backslash$varliminf & $\varlimsup$ & $\backslash$varlimsup\\
	$\injlim$ & $\backslash$injlim & $\projlim$ & $\backslash$projlim & $\varinjlim$ & $\backslash$varinjlim & $\varprojlim$ & $\backslash$varprojlim\\
	\hline
\end{tabular}
**$\backslash$var类型需要amsmath宏包
\caption{带上下限的数学运算符}
\end{table}
\begin{table}
\begin{tabular}{l l l l l l l l}
	\hline
	符号 & 代码 & 符号 & 代码 & 符号 & 代码 & 符号 & 代码\\
	\hline
	$hbar$ & \textbackslash hbar & $\imath$ & \textbackslash imath & $\jmath$ & \textbackslash jmath & $\ell$ & \textbackslash ell\\
	$\wp$ & \textbackslash wp & $\Re$ & \textbackslash Re & $\Im$ & \textbackslash Im & $\partial$ & \textbackslash partial\\
	$\infty$ & \textbackslash infty & $\prime$ & \textbackslash prime & $\emptyset$ & \textbackslash emptyset & $\nabla$ & \textbackslash nabla\\
	$\surd$ & \textbackslash surd & $\top$ & \textbackslash top & $\bot$ & \textbackslash bot & $\angle$ & \textbackslash angle\\
	$\triangle$ & \textbackslash triangle & $\forall$ & \textbackslash forall & $\exists$ & \textbackslash exists & $\neg$ & \textbackslash neg\\
	$\flat$ & \textbackslash flat & $\natural$ & \textbackslash natural & $\sharp$ & \textbackslash sharp & $\clubsuit$ & \textbackslash clubsuit\\
	$\diamondsuit$ & \textbackslash diamondsuit & $\heartsuit$ & \textbackslash heartsuit & $\spadesuit$ & \textbackslash spadesuit & $\backslash$ & \textbackslash $backslash^1$\\
	$\backprime$ & \textbackslash backprime & $\hslash$ & \textbackslash hslash & $\varnothing$ & \textbackslash varnothing & $\vartriangle$ & \textbackslash vartriangle\\
	$\blacktriangle$ & \textbackslash blacktriangel & $\triangledown$ & \textbackslash triangledown & $\blacktriangledown$ & \textbackslash blacktriangledown & $\square$ & \textbackslash square\\
	$\blacksquare$ & \textbackslash blacksquare & $\lozenge$ & \textbackslash lozenge & $\blacklozenge$ & \textbackslash blacklozenge & $\circledS$ & \textbackslash circledS\\
	$\bigstar$ & \textbackslash bigstar & $\sphericalangle$ & \textbackslash sphericalangle & $\measuredangle$ & \textbackslash measuredangle & $\nexists$ & \textbackslash nexists\\
	$\complement$ & \textbackslash complement & $\mho$ & \textbackslash mbo & $\eth$ & \textbackslash eth & $\Finv$ & \textbackslash Finv\\
	$\diagup$ & \textbackslash diagup & $\Game$ & \textbackslash Game & $\diagdown$ & \textbackslash diagdown & $\Bbbk$ & \textbackslash Bbbk\\
	\hline
\end{tabular}
**从\textbackslash backprime开始是ams符号\par
${}^1$ \hspace{1pt}\textbackslash backslash同时也是长度可变的定界符,并有一个同形的二元运算符\textbackslash setminus
\caption{数学普通符号}
\end{table}
\begin{table}
\begin{tabular}{l l l l l l l l}
	\hline
	符号 & 代码 & 符号 & 代码 & 符号 & 代码 & 符号 & 代码\\
	\hline
	$\mp$ & \textbackslash mp & $\pm$ & \textbackslash pm & $\ast$ & \textbackslash ast & $\times$ & \textbackslash times\\
	$\div$ & \textbackslash div & $\circ$ & \textbackslash circ & $\bigcirc$ & \textbackslash bigcirc & $\setminus$ & \textbackslash setminus\\
	$\cdot$ & \textbackslash cdot & $\star$ & \textbackslash star & $\cap$ & \textbackslash cap & $\cup$ & \textbackslash cup\\
	$\triangleleft$ & \textbackslash triangleleft & $\triangleright$ & \textbackslash triangleright & $\bigtriangleup$ & \textbackslash bigtriangleup & $\bigtriangledown$ & \textbackslash bigtriangledown\\
	$\wedge$ & \textbackslash wedge & $\vee$ & \textbackslash vee & $\ddagger$ & \textbackslash ddagger & $\dagger$ & \textbackslash dagger\\
	$\sqcap$ & \textbackslash sqcap & $\sqcup$ & \textbackslash sqcup & $\uplus$ & \textbackslash uplus & $\amalg$ & \textbackslash amalg\\
	$\diamond$ & \textbackslash diamond & $\bullet$ & \textbackslash & $\wr$ & \textbackslash wr & $\odot$ & \textbackslash odot\\
	$\oslash$ & \textbackslash & $\otimes$ & \textbackslash otimes & $\oplus$ & \textbackslash oplus\\
	\hline
\end{tabular}
\caption{二元运算符}
\end{table}
\begin{table}
\begin{tabular}{l l l l}
	\hline
	符号 & 代码 & 符号 & 代码\\
	\hline
	$\gets$ & $\backslash$leftarrow或$\backslash$gets & $\nleftarrow$ & $\backslash nleftarrow^2$\\
	$\to$ & $\backslash$rightarrow或$\backslash$to & $\nrightarrow$ & $\backslash nrightarrow^2$\\
	$\Leftarrow$ & $\backslash$Leftarrow & $\nLeftarrow$ & $\backslash nLeftarrow^2$\\
	$\Rightarrow$ & $\backslash$Rightarrow & $\nRightarrow$ & $\backslash$nRightarrow\\
	$\leftrightarrow$ & $\backslash$leftrightarrow & $\nleftrightarrow$ & $\backslash nleftrightarrow^2$\\
	$\Leftrightarrow$ & $\backslash$Leftrightarrow & $\nLeftrightarrow$ & $\backslash nLeftrightarrow^2$\\
	$\longleftarrow$ & $\backslash$longleftarrow & $\longrightarrow$ & $\backslash$longrightarrow\\
	$\Longleftarrow$ & $\backslash$Longleftarrow & $\Longrightarrow$ & $\backslash$Longrightarrow\\
	$\longleftrightarrow$ & $\backslash$longleftrightarrow & $\Longleftrightarrow$ & $\backslash$Longleftrightarrow\\
	$\mapsto$ & $\backslash$mapsto & $\longmapsto$ & $\backslash$longmapsto\\
	$\hookleftarrow$ & $\backslash$hookleftarrow & $\hookrightarrow$ & $\backslash$hookrightarrow\\
	$\leftharpoonup$ & $\backslash$leftharpoonup & $\rightharpoonup$ & $\backslash$rightharpoonup\\
	$\leftharpoondown$ & $\backslash$leftharpoondown & $\rightharpoondown$ & $\backslash$rightharpoondown\\
	$\rightleftharpoons$ & $\backslash$rightleftharpoons\\
	$\nearrow$ & $\backslash$nearrow & $\searrow$ & $\backslash$searrow\\
	$\swarrow$ & $\backslash$swarrow & $\nwarrow$ & $\backslash$nwarrow\\
	$\uparrow$ & $\backslash$uparrow & $\Uparrow$ & $\backslash$Uparrow\\
	$\downarrow$ & $\backslash$downarrow & $\Downarrow$ & $\backslash$Downarrow\\
	$\updownarrow$ & $\backslash$updownarrow & $\Updownarrow$ & $\backslash$Updownarrow\\
	\hline
\end{tabular}\par
**最后三行的垂直箭头同时也是可延长的定界符\par
${}^2$\hspace{2pt}ams否定箭头
\caption{\LaTeX 箭头符号}
\end{table}
\begin{table}
\begin{tabular}{l l l l l l l l}
	\hline
	符号 & 代码 & 符号 & 代码 & 符号 & 代码 & 符号 & 代码\\
	\hline
	$($ & $($ & $[$ & $[$ & $\{$ & $\backslash\{$ & $\langle$ & $\backslash$langle\\
	$)$ & $)$ & $]$ & $]$ & $\}$ & $\backslash\}$ & $\rangle$ & $\backslash$rangle\\
	$\lfloor$ & $\backslash$lfloor & $\lceil$ & $\backslash$lceil\\
	$\rfloor$ & $\backslash$rfloor & $\rceil$ & $\backslash$rceil\\
	\hline
\end{tabular}\par
**在左/右括号前使用$\backslash$left/right可使定界符随视情况改变大小,left/right必须在同一行配对,但不需要匹配对应括号,可使用$\backslash<direction>$.来匹配,无可视单元。还有$\backslash$middle调节中间的定界符\par
**也可手动调节大小,位置:$\backslash$big $\backslash$bigl $\backslash$bigm $\backslash$bigr,规格:$\backslash$big $\backslash$Big $\backslash$bigg $\backslash$Bigg
\caption{公式-括号定界符}
\end{table}
\begin{table}
\begin{tabular}{l l l l l l l l}
	\hline
	符号 & 代码 & 符号 & 代码 & 符号 & 代码 & 符号 & 代码\\
	\hline
	$\ldots$ & $\backslash$ldots & $\cdots$ & $\backslash$cdots & $\vdots$ & $\backslash$vdots & $\ddots$ & $\backslash$ddots\\
	$\iddots$ & $\backslash$iddots & $\dotsc$ & $\backslash$dotsc & $\dotsb$ & $\backslash$dotsb & $\dotsm$ & $\backslash$dotsm\\
	$\dotsi$ & $\backslash$dotsi & $\dotso$ & $\backslash$dotso\\
	\hline
\end{tabular}
**$\backslash$iddots需要mathdots宏包
\caption{公式-省略号}
\end{table}
\begin{table}
\begin{tabular}{l l}
	\hline
	单位 & 说明\\
	\hline
	mm & 1 mm=2.845 pt\\
	pt & 1 pt=0.351 mm\\
	bp & 1 bp=0.353 mm$\approx$ 1 pt\\
	dd & 1 dd=0.376 mm=1.07 pt\\
	pc & 1 pc=4.218 mm=12 pt\\
	sp & 65536 sp=1 pt\\
	cm & 1 cm=10 mm=28.453 pt\\
	cc & 1 cc =4.513 mm=12 dd=12.84 pt\\
	in & 1 in=25.4 mm=72.27 pt\\
	ex & 1 ex=当前字体中x的高度\\
	em & 1 em=当前字体尺寸$\approx$ M的宽度\\
	\hline
\end{tabular}
\caption{通用长度单位}
\end{table}
\end{document}
