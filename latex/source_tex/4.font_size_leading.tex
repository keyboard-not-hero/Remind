\documentclass[10pt]{article}
\begin{document}
    \tiny this is first para\par
    this is normalsize text\par
    \small this is second para\par
    \normalsize this is third para\par
    \large this is fourth para\par
    \huge  this is fifth para\par
    \linespread{2}\selectfont
    this is leading test\par
    what this deep\par
    this is 1\,cm\par
    here is a\hspace{1cm}big space
    \mbox{it is a no broken line.if you don't stop,it will continue all the time.you are my friend,i am your gay friend,i want to see what is you}
    \makebox[10pt][l]{it is a no broken line.if you don't stop,it will continue all the time.you are my friend,i am your gay friend,i want to see what is you}i love you
    \setlength{\parskip}{10pt}
    \raggedright
    this is first paragraphics for test parskip\\
    this is second paragraphics for test parskip\par
    this is third paragraphics for test parskip
    \begin{flushright}
	    this is para for envir align

	    this is second para for align
    \end{flushright}
\end{document}

% latex含两种作用范围:1.文本限定 - \command{text};2.当前环境 - 由最内层\begin...\end确定,并作用于该环境内\command之后所有文本,但可以使用{\command text}来限定环境

% 文字大小单位:
% pt - 磅
% pc - 1pc=12pt
% in - 英寸,1in=72.27pt
% cm - 厘米,2.54cm=1in
% mm - 毫米
% em - 当前字体大小


% \documentclass[12pt]中的12pt代表全局所有文字基准(normalsize)大小。默认为10pt,限10~12pt

% \tiny用于限制局部文字大小,基于全局磅数。可选列表:tiny/scriptsize/footnotesize/small/normalsize/large/Large/LARGE/huge/Huge。该集合为环境指令

% \linespread{2}用于指定行间距,公式:行间距=系数x基本行距(默认字体高度*1.2),西文默认系数为1(1.2),汉字系数默认为1.3(1.3x1.2),\selectfont用于作用生效。也可使用\setstretch{系数},并默认生效,该指令属于setspace宏包。属于环境指令

% \,用于小间距(0.1667em)。\hspacei{}用于指定水平间距,距离为负数时,相当于backspace,可使用\hspace*{}在行首或行尾强制留空白

% 盒子模式:每个文字都是一个小盒子,每一行文字为小盒子组成的中盒子,每一页文字为中盒子组合成的大盒子

% \mbox{text}用于不断行内容。所有文字属于一个中盒子

% \makebox[width][align]{text}也用于不断行内容,并且指定宽度和对齐方式。非限定文字跟随在指定宽度后,可能与盒子内文字重叠

% \setlength{\parskip}{}用于配置段落与段落之间的距离。属于环境指令

% \raggedright为左对齐,后续缩进失效,即便使用\indent。对齐方式列表:raggedright/centering/raggedleft。属于环境指令

% \begin{flushright}...\end{flushright}用于对指定环境使用右对齐方式,不同于\raggedright,它含有段首尾额外间隔。对齐方式列表:flushleft/center/flushright
