\documentclass{article}
\usepackage{xspace}
\usepackage{microtype}
%\linespread{2}
\begin{document}
  \parindent=2em
  Hello    world!!	\textless tab\textgreater disappear!!
  and newline\xspace here
  This is my first document\\
  Happy \TeX ing!\\
  this is another \qquad paragraphics!

  welcome to new paragraphics\\
  \newpage
  \indent this is also fourth para\par\addvspace{2em}
  \noindent this is\hspace{2ex} fifth para\\
  \phantom{i love you }i think i'm a grass\\
  i love you i think i'm a grass\\
  i believe i can fly,i believe i can in the sky,look in my eyes,you will honorificabilit\-udinitatibus\\
%  \showhyphens{honorificabilitudinitatibus}
  \textls[200]{i believe i can fly,i believe i can in the sky,look in my eyes,you will honorificabilitudinitatibus}\par
  \spaceskip=2mm
  \setlength{\parskip}{10pt}
  i believe i can f{\it l}y. i believe i can in the sky. look in my eyes. you will honorificabilitudinitatibus\par
  \xspaceskip=10mm
  i believe i can f{\it l}\/y. i believe  \vadjust{\vspace{10pt}} i can in the sky. look in my eyes. you will honorificabilit\-udinitatibus\par
\end{document}

% 多个<space>与<enter>/<tab>都被视为单个空格

% \xspace为单个空格,该功能包含在xspace宏包中

% \\用于在当前位置当前换行;\\*代表在当前位置换行,但不能换页;\\[height]用于在当前位置换行,并指定当前行与下一行的行间距;\newline用于换行,但不适用于数学模式

% <enter><enter>开启新段落,\par效果相同

% \quad代表1em的水平空白,\qquad代表2em的水平空白

% \parindent=0pt用于配置后续所有段落缩进距离

% \indent强制当前行进行缩进;\noindent强制当前行取消缩进

% \newpage强制进行分页显示。分页原理:首先使用空白填充当前剩余空间,然后使用\pagebreak进行换页。使用\newpage \mbox{} \newpage生成空白页

% \addvspace{length}用于设置段落之间或者图表与上下文之间的垂直间距,必须在\par或者空行之后

% \vspace{length}用于设置段落之间或者图表与上下文之间的垂直间距,与\addvspace不同,该指令无使用位置限制。不可用于页首或页尾,该情况可使用\vspace*{length}

% \hspace{length}用于设置水平空白间距。不可用于行首或行尾,该情况可使用\hspace*{length}

% \phantom{string}生成一块高度和宽度分别等于所示字符串的空白

% \-用于配置可断词点

% \hyphenation{word1 word2 ...}用于配置后续所有相同单词断词规则。如\hyphenation{str-ing expon-ent}用于配置后续所有string与exponent断词规则

% \uchyph=0可禁止大写单词被断词,默认为\uchyph=1

% \showhyphens{word1 word2 ...}可用于在console中显示指定单词的默认断词方式

% \textls[alpha_distance_factor]{text}用于调整字距,alpha_distance_factor为字母距离系数,默认为100,可使用范围-1000~1000;text用于指代调整字距的文本。功能包含在microtype宏包中。该宏包只适用于pdfLaTeX编译

% \spaceskip用于修改单词间距

% 当包含倾斜体时,右侧单词与倾斜体单词的单词间距变小,可在倾斜体后使用\/词距补偿指令,来增大词距。该功能针对声明式,参数式自动补偿,如\textit{}含词距自动补偿,而\itfamily不包含自动补偿

% \xspaceskip用于调整句子间距

% 句子划分规则:大写字母后跟.<space>,代表缩写单词,不是句子结束;小写字母后跟.<space>,代表句子结束

% 当大写字母后跟.<space>为句子结束时,在句末大写字母与.之间插入\@,可表明为句子结束;当小写字母后跟.<space>不是句子结束时,在.之后插入\<space>,可表明为单词结束

% \linespread{factor}用于在导言中配置行距

% \vadjust{\vspace{dimension}}用于修改单行行距。该指令句子中插入

% \fontsize{font_size}{line_distance}\selectfont用于同时配置字体大小和行距。如\fontsize{10pt}{1.25\baselineskip}\selectfont用于指定10pt字体大小,1.25*1.2=1.5倍行距

% 段落间距除了行间距之外,额外包含\parskip指定的垂直距离

% 段落形状修改列表:
% \leftskip用于修改后续所有段落左侧移动距离,正值代表向右移动,负值代表向左移动。默认为0pt
% \rightskip用于修改后续所有段落右侧移动距离,正值代表向左移动,负值代表向右移动。默认为0pt
% \hangafter段落悬挂缩进的段落,正值代表前n行不缩进,之后全部缩进;负值代表前n行缩进,之后全部不缩进。默认为1
% \hangindent代表下一段落缩进策略,正值代表左侧向右移动,负值代表右侧向左移动。默认为0pt

% \parshape n i_1 l_1 i_2 l_2 ... i_n l_n text用于指定段落特定形状,n指定段落行数,i用于指定行的缩进宽度,l用于指定该行长度
