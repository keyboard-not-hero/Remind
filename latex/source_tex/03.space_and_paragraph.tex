\documentclass{article}
\usepackage{xspace}
\usepackage{microtype}
\usepackage{graphicx}
\usepackage{lettrine}
\usepackage{contour}
\usepackage{parallel}
%\linespread{2}
\begin{document}
%  \parindent=2em
%  Hello    world!!	\textless tab\textgreater disappear!!
%  and newline\xspace here
%  This is my first document\\
%  Happy \TeX ing!\\
%  this is another \qquad paragraphics!
%
%  welcome to new paragraphics\\
%  \newpage
%  \indent this is also fourth para\par\addvspace{2em}
%  \noindent this is\hspace{2ex} fifth para\\
%  \phantom{i love you }i think i'm a grass\\
%  i love you i think i'm a grass\\
%  i believe i can fly,i believe i can in the sky,look in my eyes,you will honorificabilit\-udinitatibus\\
%%  \showhyphens{honorificabilitudinitatibus}
%  \textls[200]{i believe i can fly,i believe i can in the sky,look in my eyes,you will honorificabilitudinitatibus}\par
%  \spaceskip=2mm
%  \setlength{\parskip}{10pt}
%  i believe i can f{\it l}y. i believe i can in the sky. look in my eyes. you will honorificabilitudinitatibus\par
%  \xspaceskip=10mm
%  i believe i can f{\it l}\/y. i believe  \vadjust{\vspace{10pt}} i can in the sky. look in my eyes. you will honorificabilit\-udinitatibus\par
%  \leftskip=2em
%  \rightskip=2em  
%  If we can only encounter each other rather than stay with each other,then I wish we had never encountered.\par
%  In the end, it’s not the years in your life that count. It’s the life in your years.\par
%  No one indebted for others,while many people don't know how to cherish others.\par
%  \parindent=0pt
%  \includegraphics[scale=0.1]{snuggle.jpg}\par
%  \vspace{-6em}
%  \hangafter=-4 \hangindent=7em
%  Life is full of confusing and disordering Particular time,a particular location,Do the arranged thing of ten million time in the brain,Step by step ,the life is hard to avoid delicacy and stiffness No enthusiasm forever,No unexpected happening of surprising and pleasing So,only silently ask myself in mind Next happiness,when will come?\par
%  \parshape=3 8mm 92mm 5mm 95mm 0mm 100mm hen the wandering soul wild crane stands still the memory river Listen to whistle play tightly ring slowly,Water rises a ship to go medium long things of the past.Wait for a ship’s person Wait for one and other,But hesitate always should ascend which ship Missed Had to consign the hope to next time,Finally what to wait for until has no boats and ships to come and go,Sunset west.\par
%  \vspace{-10.2em}
%  \mbox{\fontsize{160pt}{180pt}\selectfont\bf V}\par
%  \newpage
%  \lettrine[lines=2,lhang=0,lraise=0.2]{S}{}wim, let nature cleanse your heart; read, let the words warmyour stillness; keep a raise, let free conditioning your exhaustion; put out of your mind, let things. Fun, let laughteraround your heart; sing, let the song express your memories; let a let, let pass to retain your tolerance; say, let you unlock your doubts; look, let the distance from your hope.\par
%  \contour{black}{\color{red}LaTeX source file}
%  \begin{Parallel}[p]{50mm}{50mm}
%  \ParallelLText{Childhood is a cup of coffee, drink a people lead a person to endless aftertastes; childhood is a book, each page is a record of our hours of the passions; childhood is a cup of tea, after drinking, the mouth is also revealing the sweet taste; childhood is a painting, we have a colorful life picture. An insect, a toy, a discovery, a quarrel. Not worth mentioning are full ofhappiness, the pursuit of dreams and. Childhood innocence,unforgettable years.}
%  \ParallelRText{Sometimes you dream to be a kind of happiness, sometimes the dream is also a kind of happiness; sometimes is a kind of happiness, sometimes the loss is also a kind of happiness;sometimes success is a kind of happiness, sometimes failure is also a kind of happiness. Sometimes the rich is a kind of happiness, sometimes poverty is also a kind of happiness. "Not happy" today, now can not be "happy", while it may be tomorrow or later become "happiness"!}
%  \end{Parallel}
i love you\xspace\xspace\xspace\xspace but you don't
\end{document}

% 多个<space>与<enter>/<tab>都被视为单个空格

% \quad代表1em的水平空白,\qquad代表2em的水平空白。跟在\\之后无效

% \xspace为单个空格,该功能包含在xspace宏包中。跟在\\或\par之后无效

% \\用于在当前位置当前换行;\\*代表在当前位置换行,但不能换页;\\[height]用于在当前位置换行,并指定当前行与下一行的行间距;\newline用于换行,但不适用于数学模式

% <enter><enter>开启新段落,\par效果相同

% \begin{center}...\end{center}为居中环境,用于指定该环境的内容居中。对齐方式列表:
% center - 居中对齐
% flushleft - 左对齐
% flushright - 右对齐

% 声明形式的对齐方式,对齐方式列表:
% centering - 居中对齐
% raggedright - 左对齐
% raggedleft - 右对齐

% 上述两种系统自带的对齐方式不含断词,文本会由于对齐方式显得单词之间差距过大,可使用ragged2e宏包
% 对应环境:Center/FlushLeft/FlushRight
% 对应声明形式:Centering/RaggedRight/RaggedLeft

% \parindent=0pt用于配置后续所有段落缩进距离

% \indent强制当前行进行缩进;\noindent强制当前行取消缩进

% \newpage强制进行分页显示。分页原理:首先使用空白填充当前剩余空间,然后使用\pagebreak进行换页。使用\newpage \mbox{} \newpage生成空白页

% \addvspace{length}用于设置段落之间或者图表与上下文之间的垂直间距,必须在\par或者空行之后

% \vspace{length}用于设置段落之间或者图表与上下文之间的垂直间距,与\addvspace不同,该指令无使用位置限制。不可用于页首或页尾,该情况可使用\vspace*{length}

% \hspace{length}用于设置水平空白间距。不可用于行首或行尾,该情况可使用\hspace*{length}

% \phantom{string}生成一块高度和宽度分别等于所示字符串的空白

% \textls[alpha_distance_factor]{text}用于调整字距,alpha_distance_factor为字母距离系数,默认为100,可使用范围-1000~1000;text用于指代调整字距的文本。功能包含在microtype宏包中。该宏包只适用于pdfLaTeX编译

% \spaceskip用于修改单词间距

% 当包含倾斜体时,右侧单词与倾斜体单词的单词间距变小,可在倾斜体后使用\/词距补偿指令,来增大词距。该功能针对声明式,参数式自动补偿,如\textit{}含词距自动补偿,而\itfamily不包含自动补偿

% \xspaceskip用于调整句子间距

% 句子划分规则:大写字母后跟.<space>,代表缩写单词,不是句子结束;小写字母后跟.<space>,代表句子结束

% 当大写字母后跟.<space>为句子结束时,在句末大写字母与.之间插入\@,可表明为句子结束;当小写字母后跟.<space>不是句子结束时,在.之后插入\<space>,可表明为单词结束

% \linespread{factor}用于在导言中配置行距

% \vadjust{\vspace{dimension}}用于修改单行行距。该指令句子中插入

% \fontsize{font_size}{line_distance}\selectfont用于同时配置字体大小和行距。如\fontsize{10pt}{1.25\baselineskip}\selectfont用于指定10pt字体大小,1.25*1.2=1.5倍行距

% 段落间距除了行间距之外,额外包含\parskip指定的垂直距离

% 段落形状修改列表:
% \leftskip用于修改后续所有段落左侧移动距离,正值代表向右移动,负值代表向左移动。默认为0pt
% \rightskip用于修改后续所有段落右侧移动距离,正值代表向左移动,负值代表向右移动。默认为0pt
% \hangafter段落悬挂缩进的段落,正值代表前n行不缩进,之后全部缩进;负值代表前n行缩进,之后全部不缩进。默认为1
% \hangindent代表下一段落缩进策略,正值代表左侧向右移动,负值代表右侧向左移动。默认为0pt

% \parshape=n i_1 l_1 i_2 l_2 ... i_n l_n text用于指定段落特定形状,n指定段落行数,i用于指定行的缩进宽度,l用于指定该行长度。当段落行数小于n时,多余行参数被忽略;当段落行数大于n时,n行的应用于段落n行之后的剩余行

% \lettrine[ary1=val1,arg2=val2,...]{first_alpha}{text}用于首字母下沉或上浮,first_alpha为首字母;text为单行大写字符串;通常字符串在限定范围之后。可选参数列表如下:
% lines - 下沉占据行数,默认为2;当值为1时,代表上浮一行,总计占据两行
% lhang - 首字母向左侧凸进的宽度与首字母宽度的比值,取值范围为[-1,1]。默认为0
% loversize - 当取值[-1,0]时,代表缩放尺寸占行字体大小的百分比,取值范围为(-1,1]。默认为0
% lraise - 首字母上移距离,为正时向上移,为负时向下移。默认为0
% 功能包含在lettrine宏包中

% \contour{color}{text}为文本配置边缘轮廓颜色,color为配色;text为配置颜色的字符串。包含在contour宏包中

% \begin{Parallel}[format]{left_width}{right_width}
% \ParallelLText{l_text}
% \ParallelRText{r_text}
% \end{Parallel}
% 用于文本并列显示,format为显示格式,格式列表:
% c - 使用空格分隔
% v - 使用竖线分隔
% p - 使用分页分隔
% 功能包含在parallel宏包中

% \ver<sep><content><sep>为抄录指令

% badness为排版劣质参数,取值范围为[0,10000],0代表单词间距完全符合自然宽度,10000代表单词间距偏离其自然宽度

% \tolerance为容许度,即容许单词间距可达到的最大值,取值范围为[0,10000],默认为200

% \hfuzz为溢出警告值,当溢出宽度超过该值时,产生overfull warning信息,默认为0.1pt

% \hbadness为单词间距警告阀值,当单词间距过大,超过该阀值时,产生underfull warning信息。阀值默认为1000

% 孤行:在某一页中只有段落的其中一行。种类区分:
% 1.将第一行遗弃在上一页,称为孤儿。使用\clubpenalty指定大数值来抑制该孤行,默认为150。取值范围为[0,10000]
% 2.将最后一行流落到下一页,称为寡妇。使用\widowpenalty指定大数值来抑制该孤行,默认为150。取值范围为[0,10000]

% \-用于配置可断词点。指定希望断词的地方,非强制性指令

% \hyphenation{word1 word2 ...}用于配置后续所有相同单词断词规则,指定可断点。如\hyphenation{str-ing expon-ent}用于配置后续所有string与exponent单词的断词规则

% \uchyph=0可禁止大写单词被断词。默认为\uchyph=1

% \showhyphens{word1 word2 ...}用于显示指定单词的默认断词方式
