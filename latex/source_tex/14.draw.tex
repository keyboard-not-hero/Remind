\documentclass{article}
\usepackage[all]{xy}
\begin{document}
%	\xymatrix{
%		a\ar"1,2" & b\ar@<1ex>[l] & a+b\ar@/_3pc/[l]^{on}_{under} \\
%		1 & 2 & 3
%	}
	\xymatrix{
			A'\ar[rr]\ar[dr] & & B'\ar[dd]|\hole\ar[dr]\\
		   & A\ar[rr] & & B\ar[dd]\\
			C'\ar[dr] \ar[uu]& & D'\ar[dr]\ar[ll]|\hole\\
					  & C\ar[uu] & & D\ar[ll]\\
	}
\end{document}

% \ar@<distance>@/^/@{arr_style}[direction]^{super_script}使用[direction]从当前单元配置指定相对位置的箭头,并使用@{arr_style}指定箭头样式,使用^/_/|在箭头上/下方或内部配置注解,使用@/^/或@/_/使用箭头上/下弧度,@<distance>用于指定偏移竖直方向中心的距离。
% 1.[direction]相对位置参数如下:
% l - 左边
% r - 右边 
% t - 上边
% b - 下边
% 也可以使用[r, c]的相对位置,当前位置为[0,0],向下为r增加方向,向右为c增加方向
% 可以使用"r,c"的绝对位置,左上角为"1,1"
% 可使用f;t指定箭头起始和结束位置
% 2.箭头样式参数如下:
% ->/-->/=>/.>/:>/~>/-/<space>
% 3.箭头注解 - 在箭头方向为右的基础上,^为箭头上方,_为箭头下方,|为箭头内部镶嵌(镶嵌空内容使用|\hole)。在^/_/|与实际注解内容之间,<代表箭头尾部,>代表箭头首部(箭头端);<<代表箭头尾部范围内,>>代表箭头首部范围内;()内可指定0~1的分数,指定从尾端到头部的位置。根据箭头方向进行旋转
% 4.箭头弧度 - 在箭头方向为右的基础上,@/^/为上弧度,@/_/为下弧度,还可以使用@/^<n>pc/或@/_<n>pc/指定高弧曲线
% 5.偏离位置 - 在箭头方向为右的基础上,使用@<distance>向上偏离竖直方向中心位置
% 以上所有功能包含在xy-pic宏包中,使用usepackage[all]{xy}导入包




