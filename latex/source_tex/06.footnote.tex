\documentclass[UTF8,fontset=ubuntu]{ctexart} 
\usepackage[marginal]{footmisc}
\begin{document}
	勾股定理,参考于欧几里得\footnote{前325年-前265年}\\
	
	\renewcommand\thefootnote{}	
	
	脚注带斜杠字符串应用\footnote{\texttt{\string\verb}}\\
	
	\renewcommand{\thefootnote}{\Alph{footnote}}
	
	鸭子一只\footnote{量词},小鸡一只\footnotemark[\value{footnote}],燕子一只\footnotemark[\value{footnote}]
	
	\begin{tabular}{c c c}
		姓名 & 身高 & 体重\\
		李磊 & 169 & 62\\
		张三 & 175 & 89\footnotemark[2]\\
		王麻子 & 162 & 70\footnotemark[2]\\
	\end{tabular}
	\footnotetext[2]{严重偏胖}

\end{document}

% 脚注:位于版心底部,紧接页脚。由于脚注会打乱阅读顺序,应尽量减少其使用,尤其避免在一页中出现多个脚注

% latex在注解处生成一个上角标序号,并且在版心底部生成对应序号和对应注解内容。第一个注解之前会生成0.4*/columnwidth宽度和0.4pt高度的水平脚注线。不可嵌套

% \footnote[number]{comment}用于脚注,number为脚注编号,未提供时,使用系统自带脚注计数器;comment为对应脚注内容。
% 注意事项:
% 1.不能用于浮动环境、左右盒子
% 2.可使用数学式、表格和插图
% 3.不可包含抄录指令
% 4.可使用等宽字体指令+字符串指令的方式处理带反斜杠的字符串\texttt{\string<content>}
% 5.可使用书签指令/label
% 6.可使用\interfootnotelinepenalty=<number>指定当前页可显示的书签数量
% 7.多个注释对象使用同一个脚注 - 第一个注释对象使用\footnote{comment},第二个以及之后对象使用\footnotemark[\value{footnote}]
% 8.可使用\renewcommand{\thefootnote}{\alpha{footnote}}修改脚注计数器的计数形式。默认为阿拉伯数字,\alph为小写字母,\Alph为大写字母
% 9.可使用\renewcommand\thefootnote{}删除脚注排序

% \footnotemark[number]用于脚注位置标记,可选参数指定脚注编号;footnotetext[number]{comment}用于给指定编号脚注配置注解内容
% 注意事项:
% 1.可用于浮动环境、左右盒子
% 2.可使用\renewcommand{\thefootnote}{\alpha{footnote}}修改脚注计数器的计数形式。默认为阿拉伯数字
% 3.可使用\renewcommand\thefootnote{}删除脚注排序

% 以下内容包含在footmisc宏包中。footmisc - 用于脚注格式
% usepackage[args]{footmisc},args可选参数列表:
% flushmargin - 脚注序号靠近脚注文本,脚注首行不缩进
% marginal - 脚注序号凸进左边,脚注首行不缩进
% multiple - 当注释对象包含两个脚注时,默认两个序号紧密连接,如12;采用该选项后,序号排列为1,2
% norule - 取消脚注线
% para - 将一页中所有脚注合为一个段落
% perpage - 将脚注以页为排序单位,需经过两次编译
% ragged - 脚注文本左对齐
% side - 将脚注修改为边注
% symbol - 将默认的阿拉伯脚注序号,修改为各种符号,符号共九种,超过提示出错
