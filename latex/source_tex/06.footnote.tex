\documentclass[UTF8,fontset=ubuntu]{ctexart} 
\usepackage[marginal]{footmisc}
\usepackage{threeparttable}
\usepackage{endnotes}
\begin{document}
%	\renewcommand{\footnoterule}{\vspace*{-3pt} \hrule width \columnwidth height 3pt \vspace*{10pt}}
%	勾股定理,参考于欧几里得\footnote{前325年-前265年}\\
%	
%	\renewcommand{\thefootnote}{}	
%	
%	脚注带斜杠字符串应用\footnote{\texttt{\string\verb}}\\
%	
%	\renewcommand{\thefootnote}{\Alph{footnote}}
%	
%	鸭子一只\footnote{量词},小鸡一只\footnotemark[\value{footnote}],燕子一只\footnotemark[\value{footnote}]
%	
%	\begin{tabular}{c c c}
%		姓名 & 身高 & 体重\\
%		李磊 & 169 & 62\\
%		张三 & 175 & 89\footnotemark[2]\\
%		王麻子 & 162 & 70\footnotemark[2]\\
%	\end{tabular}
%	\footnotetext[2]{严重偏胖}

%\begin{threeparttable}
%\caption{table label}
%\begin{tabular}{ll}
%\hline
%AMPS\tnote{1} & 移动电话服务系统\\
%GSM\tnote{2} & 全球移动通讯系统\\
%CDMA\tnote{2} & 码分多址通信系统\\
%NMT\tnote{1} & 北欧移动电话系统\\
%\hline
%\end{tabular}
%\begin{tablenotes}
%\item[1] 贝尔实验室于1969年开始研究,1983年投入使用。
%\item[2] 数字技术。
%\end{tablenotes}
%\end{threeparttable}

南北朝时著名数学家祖冲之\endnote{公元429-500}发明了一种计算方法,从而得到了更为准确的圆周率\endnote{圆周长与直径之比}值。
\renewcommand{\notesname}{本章注释}
\theendnotes
\setcounter{endnote}{0}

\end{document}

% 脚注:位于版心底部,紧接页脚。由于脚注会打乱阅读顺序,应尽量减少其使用,尤其避免在一页中出现多个脚注

% latex在注解处生成一个上角标序号,并且在版心底部生成对应序号和对应注解内容。第一个注解之前会生成0.4*/columnwidth宽度和0.4pt高度的水平脚注线。不可嵌套

% 脚注在article中以全文为排序单位,report/book中以章为排序单位

% 脚注与脚注之间的距离为\footnotesep,当\footnotesep小于默认值时,距离为\baselineskip;脚注与正文之间的距离由\skip\footins控制,默认为10pt plus 4pt minus 2pt;脚注线由\footnoterule定义。
% book文类的\footnoterule默认定义如下:
% \renewcommand{\footnoterule}{\vspace*{-3pt} \hrule width 0.4\columnwidth height 0.4pt \vspace*{2.6pt}}
% \vspace*{-3pt} - 脚注线与脚注的距离
% \vspace*{2.6} - 脚注线与正文的距离
% \hrule为TeX水平画线命令。width为画线宽度,height为画线高度

% \footnote[number]{comment}用于脚注,number为脚注编号,未提供时,使用系统自带脚注计数器,提供number时,不会引起脚注计数器增长;comment为对应脚注内容。
% 注意事项:
% 1.不能用于浮动环境、左右盒子
% 2.可使用数学式、表格和插图
% 3.不可包含抄录指令
% 4.可使用等宽字体指令+字符串指令的方式处理带反斜杠的字符串\texttt{\string<content>}
% 5.可使用书签指令/label
% 6.可使用\interfootnotelinepenalty=<number>指定当前页可显示的书签数量
% 7.多个注释对象使用同一个脚注 - 第一个注释对象使用\footnote{comment},第二个以及之后对象使用\footnotemark[\value{footnote}]
% 8.可使用\renewcommand{\thefootnote}{\alpha{footnote}}修改脚注计数器的计数形式。默认为阿拉伯数字,\alph为小写字母,\Alph为大写字母
% 9.可使用\renewcommand\thefootnote{}删除脚注排序

% \footnotemark[number]用于脚注位置标记,可选参数指定脚注编号;footnotetext[number]{comment}用于给指定编号脚注配置注解内容
% 注意事项:
% 1.可用于浮动环境、左右盒子
% 2.可使用\renewcommand{\thefootnote}{\alpha{footnote}}修改脚注计数器的计数形式。默认为阿拉伯数字
% 3.可使用\renewcommand\thefootnote{}删除脚注排序

% 以下内容包含在footmisc宏包中。footmisc - 用于脚注格式
% usepackage[args]{footmisc},args可选参数列表:
% flushmargin - 脚注序号靠近脚注文本,脚注首行不缩进
% marginal - 脚注序号凸进左边,脚注首行不缩进
% multiple - 当注释对象包含两个脚注时,默认两个序号紧密连接,如12;采用该选项后,序号排列为1,2
% norule - 取消脚注线
% para - 将一页中所有脚注合为一个段落
% perpage - 将脚注以页为排序单位,需经过两次编译
% ragged - 脚注文本左对齐
% side - 将脚注修改为边注
% symbol - 将默认的阿拉伯脚注序号,修改为各种符号,符号共9种,超过提示出错
% symbol* - 类似于symbol,但有16种符号及组合,超过16使用阿拉伯数字

% \begin{threeparttable}...\end{threeparttable}用于指定表格标注环境,可看作不可浮动table环境;\begin{tablenotes}...\end{tablenotes}用于脚注环境;\tnote{}用于脚注位置和编号标注,其脚注文字格式可使用\TPTtagStyle修改;\item[]用于对应编号的脚注详细内容。包含在threeparttable宏包中

% \endnote用于标注尾注位置,\notesname用于配置修改尾注标题名,\theendnotes用于生成尾注,\setcounter{endnote}{0}用于重置尾注计数器,以便下一章尾注计数从新开始。包含在宏包endnotes中

% \marginpar[left_marnote]{right_marnote}用于配置边注,left_marnote用于双页或双栏的左侧边注;right_marnote用于双页或双栏的右侧边注,或单页的标注
