\documentclass[UTF8, fontset=ubuntu]{ctexart}
\usepackage{mathtools}
\usepackage{tensor}
\usepackage{amsmath}
\begin{document}
	这是一个行内公式:$a+b=b+a$\par
	这也是一个行内公式:\( a=b+1\)\par
	这又是一个行内公式:\begin{math}y=x^2\end{math}\par
	这是一个行间公式:$$f(x)=mx+b$$
	\indent 这还是一个行间公式:\[E=mc^2\]
	\indent 这仍旧是一个行间公式:\begin{displaymath}Ft=mv_1-mv_2\end{displaymath}
	\indent 最后一个行间公式:\begin{equation}a+b=b+a\\b+c=c+b\end{equation}
	$\int\limits_0^n x$\\
	$\int\nolimits_0^n y$\\
	$\prescript{n}{m}{H}^i_j$\\
	$\tensor[_a^b_c^d]{M}{^a_b^c_d}$\\
	$\overline{a+b}$\\
	$\underline{a+b}$\\
	$\overleftarrow{a+b}$\\
	$\overrightarrow{a+b}$\\
	$\overleftrightarrow{a+b}$\\
	$\underleftarrow{a+b}$\\
	$\underrightarrow{a+b}$\\
	$\underleftrightarrow{a+b}$\\
	$\overbrace{a+b+c}$\\
	$\underbrace{a+b+c}$\\
	$a+\rlap{$\overbrace{\phantom{b+c+d}}^{m}$}b+\underbrace{c+d+e}_{n}+f$\\
\begin{gather}
	a+b=b+a\notag \\
	b+c=c+a
\end{gather}
\begin{align}
	f(x)&=1*2+100/2 \notag \\
	       &=2+50\notag \\
	       &=52
\end{align}
\begin{equation}
f(x) = \begin{cases}
1, & \text{if } x > 1;\\
0, & \text{if } x < 1.
\end{cases}
\end{equation}
\begin{equation}
\left.\begin{gathered}
	S \subseteq T\\
	S \subseteq T
	\end{gathered} \right\}
	\implies S = T
\end{equation}
\end{document}

% $...$用于限定行内公式区域。其余还有\(...\)与\begin{math}...\end{math}。行内公式的上下角标普遍位于右上下角

% $$...$$用于限定行间公式区域。其余还有\[...\]与\begin{displaymath}...\end{displaymath}。此外还有\begin{equation}...\end{equation},该环境自动配置一个公式号码,并且\\无效。行间公式的上下角标普标位于正向下方

% \limits用于限定后续上下角标位于正上下方。该指令包含在amsmath宏包中
 
% \nolimits类似于\limits,不过是限制上下角标位于右上下角

% \prescript{super_script}{sub_script}{variable}用于显示左上下角标。该指令包含在mathtools宏包中

% \tensor[left_script]{variable}{right_script}用于显示左右方的上下角标,并且上下角标错次排列。该指令包含在tensor宏包中

% 关于化学式的上下角标,参考mhchem宏包

% \overline用于上横线,\underline用于下横线

% \overleftarrow用于上方向左箭头,类似列表:overleftarrow/overrightarrow/overleftrightarrow/underleftarrow/underrightarrow/underleftrightarrow。该指令集合包含在amsmath宏包中

% \overbrace{}和\underbrace{}用于上下大括号

% \rlap{}将环境内的内容与环境右边的内容进行左对齐重叠

% \phantom{}将环境内的内容进行占位,但不显示

% \begin{gather}...\end{gather}用于多行行间公式,可以使用\\来分行,并且使用\notag来使当前行不使用公式序号。包含在amsmath宏包中

% \begin{align}...\end{align}用于多行行间公式,可以使用\\分行,使用\notag不编号,并且在关系运算符之前使用&来表明对齐位置。包含在amsmath宏中

% \begin{cases}...\end{cases}用于在数学公式环境中配置条件公式,可使用&来区分公式内容和条件。包含在amsmath宏包中。另外,mathtools宏包的dcases环境可以保证公式为行间公式大小

% \begin{gathered}...\end{gathered}可以用于将环境内的多行公式组合成一个整体。包含在amsmath宏包中