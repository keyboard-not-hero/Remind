\documentclass[10pt]{article}
\begin{document}
%    \tiny this is first para\par
%    this is normalsize text\par
%    \small this is second para\par
%    \normalsize this is third para\par
%    \large this is fourth para\par
%    \huge  this is fifth para\par
%    \linespread{2}\selectfont
%    this is leading test\par
%    what this deep\par
%    this is 1\,cm\par
%    here is a\hspace{1cm}big space\\
%    \mbox{it is a no broken line.if you don't stop,it will continue all the time.you are my friend,i am your gay friend,i want to see what is you}
%    \setlength{\fboxrule}{3pt}
%    \setlength{\fboxsep}{10pt}
%    \fbox{it's is a line,if you want to know what,please see this.continue...}
%    \makebox[20em][s]{it is }\par
%    \framebox[10em][s]{this is me}\par
%    \setlength{\parskip}{10pt}
%    \raggedright
%    this is first paragraphics for test\raisebox{2mm}{this is raisebox }parskip\\
%    this is second paragraphics for test parskip\par
%    this is third paragraphics for test parskip
%    \begin{flushright}
%	    this is para for envir align
%
%	    this is second para for align
%    \end{flushright}

%	字体簇
%	\textrm{This is Roman Family}\par
%	\textsf{This is San Serif Family}\par
%	\texttt{This is TypeWrite Family}\par

%	\rmfamily This is Roman Family\par
%	\sffamily This is San Serif Family\par
%	\ttfamily This is TypeWrite Family\par

%	\rm This is Roman Family\par
%	\sf This is San Serif Family\par
%	\tt This is TypeWrite Family\par

%	粗细序列
%	\textbf{This is bold series}\par
%	\textmd{This is medium series}\par

%	\bfseries This is bold series\par
%	\mdseries This is medium series\par

%	\bf This is bold series\par

%	形状
%	\textit{This is Italic Shape}\par
%	\textsc{This is Small Caps Shape}\par
%	\textsl{This is Slanted Shape}\par
%	\textup{This is Upright Shape}\par
%
%	\itshape This is Italic Shape\par
%	\scshape This is Small Caps Shape\par
%	\slshape This is Slanted Shape\par
%	\upshape This is Upright Shape\par
%
%	\it This is Italic Shape\par
%	\sc This is Smalll Caps Shape\par
%	\sl This is Slannted Shape\par

%	文本
%	\textnormal{This is Normal style}\par
%	\emph{This is emphasized text}\par
%
%	\normalfont This is Normal style\par
%	\em This is emphasized text\par

%	LaTeX默认使用罗马字体/常规序列/直立形状
%	This is Roman Family\par
%	\textrm{\textmd{\textup{This is Roman Family}}}\par
\end{document}


% 文字大小单位:
%mm - 1 mm=2.845 pt
%pt - 1 pt=0.351 mm
%bp - 1 bp=0.353 mm≈ 1 pt
%dd - 1 dd=0.376 mm=1.07 pt
%pc - 1 pc=4.218 mm=12 pt
%sp - 65536 sp=1 pt
%cm - 1 cm=10 mm=28.453 pt
%cc - 1 cc =4.513 mm=12 dd=12.84 pt
%in - 1 in=25.4 mm=72.27 pt
%ex - 1 ex=当前字体中x的高度
%em - 1 em=当前字体尺寸≈ M的宽度

% \documentclass[12pt]中的12pt代表全局所有文字基准(normalsize)大小。默认为10pt,限10~12pt

% \tiny用于限制局部文字大小,基于全局磅数。可选列表:tiny/scriptsize/footnotesize/small/normalsize/large/Large/LARGE/huge/Huge

% 盒子模式:基准线将总高度分为高度(基准线之上)和深度(基准线之下)

% 每个文字都是一个小盒子,每一行文字为小盒子组成的中盒子,每一页文字为中盒子组合成的大盒子

% \linespread{2}用于指定行间距,行间距 - 两排文字基准线的距离。公式:行间距=系数x基本行距(默认字体高度*1.2),西文默认系数为1(1.2),汉字系数默认为1.3(1.3x1.2),\selectfont用于作用生效。也可使用\setstretch{系数},并默认生效,该指令属于setspace宏包。

% \,用于小间距(0.1667em)。\hspacei{}用于指定水平间距,距离为负数时,相当于backspace,可使用\hspace*{}在行首或行尾强制留空白

% \mbox{text}用于不断行内容。所有文字属于一个行盒子。用于保证不被断词,或者将普通文本插入数学公式中

% \fbox{text}类似于mbox,但行盒子带边框。边框与文本距离为3pt,边框线的宽度为0.4pt。\setlength{\fboxrule}{3pt}用于配置边框线宽度;\setlength{\fboxsep}{10pt}用于配置边框线与文本的距离

% \makebox[width][align]{text}也用于不断行内容,width指定盒子宽度,align用于限定文本text与盒子的对其方式。对齐参数如下:
% l - 盒子与文本左对齐
% c - 盒子与文本中间对齐
% r - 盒子与文本右对齐
% s - 类似于l,但间隔均匀

% \framebox类似于\makebox,但带有边框。边框与文本距离为3pt,边框线的宽度为0.4pt。

% \raisebox{offset}[height][deepth]{text}用于配置盒子与当前文本行的垂直位移,offset代表垂直位移量,正数为向上移动;height为高度;deepth为深度

% \setlength{\parskip}{}用于配置段落与段落之间的距离。属于环境指令

% \raggedright为左对齐,后续缩进失效,即便使用\indent。对齐方式列表:raggedright/centering/raggedleft

% \RaggedRight为左对齐,与\raggedright不同,它包含断词功能。包含在ragged2e宏包中

% \begin{flushright}...\end{flushright}用于对指定环境使用右对齐方式,不同于\raggedright,它含有段首尾额外间隔。对齐方式列表:flushleft/center/flushright

% 字体类型列表:
% 1.位图字体 - 由像素栅栏组成,放大或缩小会失真
% 2.向量字体 - 在字形上分割出若干个关键点,相邻关键点之间由一条光滑曲线连接。向量字体可以进行无级缩放而不会产生失真

% 向量字体类型:
% 1.TrueType - 使用二次贝塞尔曲线来描述字形,扩展名为.ttf
% 2.Typel - 使用三次贝塞尔曲线来描述字形,比TrueType更精美,扩展名为.pfb
% 3.OpenType - 集合前两种优点并兼容于前两者,拓展名为.otf

% 手写体宏包:pbsi/calligra/aurical/chancery/emerald/oesch/suetterl/tgchorus
