\documentclass{article}
%\documentclass[fleqn]{article}
%\usepackage{mathtools}
\usepackage{amsmath}
%\usepackage{bbm}
%\usepackage{amssymb}
%\usepackage{texnames}
%\usepackage{mathrsfs}
%\usepackage{eucal}
%\usepackage{eufrak}
\begin{document}
%	$\sqrt[3]{8}$\\

% 矩阵
%\setcounter{MaxMatrixCols}{12}
%\begin{gather*}
%\begin{matrix}
%1 & 2 & 3 & 4 & 5 & 6 & 7 & 8 & 9 & 10 & 11\\
%1 & 2 & 3 & 4 & 5 & 6 & 7 & 8 & 9 & 10 & 11\\
%\end{matrix}\\
%\begin{pmatrix}
%5 & 6\\
%7 & 8\\
%\end{pmatrix}\\
%\begin{bmatrix}
%9 & 10\\
%11 & 12\\
%\end{bmatrix}\\
%\begin{Bmatrix}
%13 & 14\\
%15 & 16\\
%\end{Bmatrix}\\
%\begin{vmatrix}
%17 & 18\\
%19 & 20\\
%\end{vmatrix}\\
%\begin{Vmatrix}
%21 & 22\\
%23 & 24\\
%\end{Vmatrix}
%\end{gather*}

%$\sum\limits_{\substack{i=1\\j=1}}(i+j)$

%$F=\big(\begin{smallmatrix}A & B\\C & D\end{smallmatrix}\big)$

%\[
%\begin{vmatrix*}[r]
%	9 & 100 & 2\\
%	-1001 & 3 & 90\\
%	100 & 29 & 11\\
%\end{vmatrix*}
%\]

%\[
%\bordermatrix{
%    & 1 & 2 & 3 \cr
%1 & A & B & C \cr
%2 & D & E & F \cr
%}
%\]

%\parindent=0pt
%$\mathfrak{A+B=B+A}$\\
%$\mathfrak{a+b=c+d}$\\
%$\mathfrak{1+3=2+1}$

%$\hm\int > \bm\int = \pmb\int > \int$

%\thinmuskip=0mu
%\medmuskip=4mu plus 2mu minus 4mu
%\thickmuskip=5mu plus 5mu
%$f(x)=x+\sin x-\sqrt{2}$
%
%\thinmuskip=3mu
%\medmuskip=0mu
%\thickmuskip=5mu plus 5mu
%$f(x)=x+\sin x-\sqrt{2}$
%
%\thinmuskip=3mu
%\medmuskip=4mu plus 2mu minus 4mu
%\thickmuskip=0mu
%$f(x)=x+\sin x-\sqrt{2}$

%\noindent I\negthickspace I\\
%I\negmedspace I\\
%I\negthinspace I\\
%I\thinspace I\\
%I\medspace I\\
%I\thickspace I\\
%I\enskip I\\
%I\quad I\\
%I\qquad I\\

%I$\mspace{-3mu}$I

%\abovedisplayshortskip=0pt plus 3pt
%\abovedisplayskip=11pt plus 3pt minus 6pt
%\belowdisplayshortskip=6.5pt plus 3.5pt minus 3pt
%\belowdisplayskip=11pt plus 3pt minus 6pt
%this is short text
%\begin{equation}
%f(x)=x^2+10
%\end{equation}
%If we can only encounter each other rather than stay with each other,then I wish we had never encountered.\par
%this is a very long text,you can see
%\begin{equation}
%f(x)=2x^2+10x+9
%\end{equation}
%If we can only encounter each other rather than stay with each other,then I wish we had never encountered.\par

%\begin{gather}
%f(x)=x^2+10x+21\\[10pt]
%g(x)=x^3+2x^2+6x+9
%\end{gather}

%$\overline{x} \otimes \overline{t}$\\
%$\overline{x\mathstrut} \otimes \overline{t\mathstrut}$\\

%\parindent=0pt
%$\displaystyle\int_0^1\frac{x^{p-1}p}{1-x^n}^{\frac{p}{n}}dx,\sum_{n=1}^\infty,\lim_{n\rightarrow\infty}$\\
%$\textstyle\int_0^1\frac{x^{p-1}p}{1-x^n}^{\frac{p}{n}}dx,\sum_{n=1}^\infty,\lim_{n\rightarrow\infty}$\\
%$\scriptstyle\int_0^1\frac{x^{p-1}p}{1-x^n}^{\frac{p}{n}}dx,\sum_{n=1}^\infty,\lim_{n\rightarrow\infty}$\\
%$\scriptscriptstyle\int_0^1\frac{x^{p-1}p}{1-x^n}^{\frac{p}{n}}dx,\sum_{n=1}^\infty,\lim_{n\rightarrow\infty}$\\

%$n+1 \choose k$
%\begin{gather}
%\binom{n+1}{k}\\
%\tbinom{n+1}{k}\\
%\dbinom{n+1}{k}\\
%\end{gather}

%\sfrac{1}{2}

%\begin{gather}
%\sqrt{a} \qquad \sqrt{f} \qquad \sqrt{a^3_i}
%\end{gather}
%\begin{gather}
%\sqrt{a\vphantom{a^3_i}} \qquad \sqrt{f\vphantom{a^3_i}} \qquad \sqrt{a^3_i}
%\end{gather}

%\begin{gather}
%\sqrt[k]{b} \qquad \sqrt[\leftroot{-2}\uproot{2}k]{b}
%\end{gather}

$f(x)\stackrel{\text{d}}{=}ax^2+bx+c$

\end{document}

% \sqrt[]{}用于开方,可选参数代表开方的次数,必选参数代表被开方的数字

% 可使用\uproot和\leftroot来修改开方次数的位置。包含在amsmath宏包中

% 也可使用\mathstrut表示圆括号高度和深度的支架

% \vphantom可用于调整垂直方向占位,并因此调整根号高度

% \begin{matrix}...\end{matrix}为原始矩阵环境,无限定框。行元素之间使用&分隔,行之间使用\\分隔。矩阵列表:
% pmatrix - 小括号限定矩阵环境
% bmatrix - 中括号限定矩阵环境
% Bmatrix - 大括号限定矩阵环境
% vmatrix - 单垂直线限定矩阵环境
% Vmatrix - 双垂直线限定矩阵环境

% \begin{smallmatrix}...\end{smallmatrix}用于限定行内矩阵环境

% \dots为横向省略号,\vdots为纵向省略号,\ddots为斜向省略号。具体参考symbol_summary文档-表12

% \hdotsfor{cols}用于指定占据指定列数的横向省略号。包含在amsmath宏包

% \begin{vmatrix*}[<align>]...\end{vmatrix*}为可指定对齐方式的矩阵,其他矩阵也有对应的可对齐方式环境。由mathtools宏包提供

% 矩阵环境默认最多只有10列,可使用\setcounter{MaxMatrixCols}{<number>}来配置

% \substack{}可在用于限定上下角标,该上下角标内可换行。包含在amsmath宏包中

% \bordermatrix{}用于限定带矩阵行列注释的矩阵,\cr为特有换行标志

% \hm用于公式字符使用加重粗体,\bm用于公式字符使用粗体。由bm宏包提供

% \pmb用于公式字符使用伪粗体(由字体稍稍错位连续输出得到,当无法使用粗体时,可使用该方法)。由amsmath宏包提供

% 统一水平距离调整
% \thinmuskip为三角函数与弧度的距离;\medmuskip为二元运算符与运算数的距离;\thickmuskip为赋值运算符与运算数的距离

% 单一水平距离调整
% thinspace等九条指令都是水平间隔指令。详情参考汇总
% \mspace{length}为统一替代方案,只能用在数学模式中。包含在amsmath中

% \abovedisplayshortskip、\abovedisplayskip、\belowdisplayshortskip、\belowdisplayskip分别为单行公式的上间距与下间距,短公式定义为公式在上行文本结束的右侧,长公式定义为公式在上行文本结束的左侧

% 多行公式环境内,公式之间的间距可使用\\[height]来指定

% \mathstrut用于创建零水平宽度盒子,其高度和深度与圆括号相同

% \displaystyle、\textstyle、\scriptstyle、\scriptscriptstyle用于定制公式字体档次。行间公式默认使用\displaystyle,行内公式默认使用\textstyle

% \everymath{\displaystyle}使用在导言中,可使所有数学公式使用同一种公式字体档次

% \documentclass[fleqn]{article}中的fleqn可选参数,可使所有行间公式左对齐,缩进宽度为\mathindent=2.5em。行间公式默认为居中对齐

% \choose用于二项式系数

% \raisetag{height}用于调整公式序号垂直位置

% \stackrel{comment}{symbol}用于在符号上添加注释,comment为注释内容,symbol为符号
