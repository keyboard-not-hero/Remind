\documentclass{article}
\usepackage{multirow}
\usepackage{diagbox}
\usepackage{array}
\usepackage{rotating}
\usepackage{longtable}
\begin{document}
%\begin{tabular}[t]{l|c|r}
%	1 & 2 & 3\\
%	\hline
%	4 & 50 & 6\\
%	\hline
%	7 & 10000 & 9\\
%\end{tabular}\par
%\begin{tabular}{|c|r|r|}
%	\hline
%	\multirow{2}*{姓名} & \multicolumn{2}{c|}{成绩}\\
%	\cline{2-3}
%	& 语文 & 数学\\
%	\hline
%	张三 & 87 & 100 \\
%	\hline
%\end{tabular}\par
%\begin{tabular}{|c|c|c|}
%	\hline
%	\diagbox{姓名}{成绩}{性别} & 男 & 女\\
%	\hline
%	琳达 & & 100\\
%	\hline
%	李磊 & 94 &\\
%	\hline
%	露西 & & 79\\
%	\hline
%\end{tabular}\par

%\arrayrulewidth=2pt
%\tabcolsep=10pt
%\begin{sidewaystable}
%\begin{tabular}[b]{|l>{\$}l|r|}
%	\firsthline
%	姓名 & 年龄 & 成绩\\
%	\hline
%	汤普森 & 45\vline23 & 100\\
%	\hline
%	约翰布兰妮 & 31 &97\\
%	\hline
%	威尔 & 12 & 67\\
%	\lasthline
%\end{tabular}
%\end{sidewaystable}
%this is document

\begin{longtable}{l r r}
\endfirsthead
\multicolumn{2}{l}{continue...}\\
\hline
\endhead
\hline
\multicolumn{2}{l}{next tabular...}\\
\endfoot
\endlastfoot
name & age & score\\\hline
thompson & male & 89 \\
peter & male & 97\\
lucy & female & 78\\
lily & female & 87\\
david & male & 69\\
steven & male & 100\\
lilei & male & 74\\
hanmei & female & 72\\
Barry & male & 92\\
Arien & male & 83\\
Andrew &male & 39\\
Allen & male & 69\\
Alger & male & 82\\
Albert & female & 65\\
Abel & male & 29\\
John & male & 85\\
Aaron & male & 79\\
Alan & male & 82\\
Baron & male & 84\\
Barton & male & 79\\
Ben & male & 73\\
Berton & male & 71\\
Bob & male & 91\\
Brian & male & 58\\
Carl & male & 99\\
Colin & male & 78\\
Daniel & male & 87\\
Dean & male & 90\\
Dennis & male & 80\\
Devin & male & 79\\
peter & male & 97\\
lucy & female & 78\\
lily & female & 87\\
david & male & 69\\
steven & male & 100\\
lilei & male & 74\\
hanmei & female & 72\\
Barry & male & 92\\
Arien & male & 83\\
Andrew &male & 39\\
Allen & male & 69\\
Alger & male & 82\\
Albert & female & 65\\
Abel & male & 29\\
John & male & 85\\
Aaron & male & 79\\
Alan & male & 82\\
Baron & male & 84\\
Barton & male & 79\\
Ben & male & 73\\
Berton & male & 71\\
Bob & male & 91\\
Brian & male & 58\\
Carl & male & 99\\
Colin & male & 78\\
Daniel & male & 87\\
Dean & male & 90\\
Dennis & male & 80\\
Devin & male & 79\\
thompson & male & 89 \\
peter & male & 97\\
lucy & female & 78\\
lily & female & 87\\
david & male & 69\\
steven & male & 100\\
lilei & male & 74\\
hanmei & female & 72\\
Barry & male & 92\\
Arien & male & 83\\
Andrew &male & 39\\
Allen & male & 69\\
Alger & male & 82\\
Albert & female & 65\\
Abel & male & 29\\
John & male & 85\\
Aaron & male & 79\\
Alan & male & 82\\
Baron & male & 84\\
Barton & male & 79\\
Ben & male & 73\\
Berton & male & 71\\
Bob & male & 91\\
Brian & male & 58\\
Carl & male & 99\\
Colin & male & 78\\
Daniel & male & 87\\
Dean & male & 90\\
Dennis & male & 80\\
Devin & male & 79\\
peter & male & 97\\
lucy & female & 78\\
lily & female & 87\\
david & male & 69\\
steven & male & 100\\
lilei & male & 74\\
hanmei & female & 72\\
Barry & male & 92\\
Arien & male & 83\\
Andrew &male & 39\\
Allen & male & 69\\
Alger & male & 82\\
Albert & female & 65\\
Abel & male & 29\\
John & male & 85\\
Aaron & male & 79\\
Alan & male & 82\\
Baron & male & 84\\
Barton & male & 79\\
Ben & male & 73\\
Berton & male & 71\\
Bob & male & 91\\
Brian & male & 58\\
Carl & male & 99\\
Colin & male & 78\\
Daniel & male & 87\\
Dean & male & 90\\
Dennis & male & 80\\
Devin & male & 79\\
\hline
\caption{score table}
\end{longtable}
\end{document}

% \begin{tabular}[position]{col_format1 col_format2 ...}...\end{tabular}代表表格环境,可选参数为表格内内容与外部内容的对齐方式;必选参数{}代表列数与分隔方式

% 对齐方式列表:
% t - 表格顶线与外部文本对齐。可使用array宏包的\firsthline和\lasthline改善,分别替换第一个和最后一个\hline
% c - 表格中部与外部文本对齐
% b - 表格底线与外部文本对齐。可使用array宏包的\firsthline和\lasthline改善,分别替换第一个和最后一个\hline

% 必选参数详解,col_format为列格式,详细格式参数如下:
% l - 左对齐
% c - 中央对齐
% r - 右对齐
% p{<width>} - 指定列的宽度,行内垂直方向上对齐
% | - 以竖线作为列分隔符,不占用列
% || - 以双竖线作为列分隔符,使用\doublerulesep作为双线距离,默认为2pt
% @{<content>} - 以指定内容为列分隔符,不占用列
% *{<counter>}{<col_des>} - col_des代表以上各种模式,并重复指定counter次
% **-** 使用array宏包后的额外选项:
% m{<width>} - 类似于p{},但行内垂直方向居中对齐
% b{<width>} - 类似于p{},但行内垂直方向下对齐
% >{<content>} - 将content声明到后续列所有数据之前,不占用列
% <{<content>} - 将content声明到前方列所有数据之后,不占用列
% !{<content>} - 类似于@{},但左右会有额外距离

% \multicolumn{<col_num>}{<align>}{<content>}用于表明合并多列。col_num合并列数;align代表合并列的对齐方式,不能使用*{<conter>}{<col_des>}格式;content代表列内容

% \multirow{<row_num>}{<col_width>}[<vertical_offset>]{<content>}用于合并多行。row_num合并行数;col_width为列宽度,超过自动换行,也可以将{<width>}换为*,取最长那一行的宽度为自然宽度;vertical_offset为数据的垂直位位移量,正值为向上移动;content为单元的内容。包含在multirow宏包中

% \hline为占据所有列的横线,紧跟在\\换行之后

% \cline{<col_range>}为占据指定列范围的横线,紧跟在\\换行之后

% \vline用于在所在列侧边画一条垂直线

% \diagbox{<left>}{<center>}{<right>}用于创建斜线分割,其中{<center>}可以省略。包含在diagbox宏包中

% 表格额外参数列表:
% \arrayrulewidth - 表格线粗细,默认为0.4pt
% \arraystretch - 行与行之间的距离系数,默认是1
% \extrarowheight - 每行的附加额外高度
% \tabcolsep - 列与列之间空白的一半,默认为6pt

% \begin{sidewaystable}...\end{sidewaystable}用于将表格逆时针旋转90度,该指令开启新的一页,并且可在其中放置\caption指令

% \begin{longtable}[position]{col_format1 col_format2 ...}...\end{longtable}用于可跨页的长表格,可在其中包含\caption指令。包含在longtable宏包中
% position可选列表:
% 1.c - 表格垂直方向居中。默认选项
% 2.l -  表格垂直方向向左
% 3.r - 表格垂直方向向右

% longtable环境内的指令列表:
% 1.endfirsthead - 第一页的表格头部,指定该指令之前的部分。如果没有指定,默认被endhead覆盖
% 2.endhead - 每一页的表格头部,指定该指令之前的部分
% 3.endfoot - 每一页的表格尾部,指定该指令之前的部分
% 4.endlastfoot - 最后一页的表格尾部,指定该指令之前的部分。如果没有指定,默认被endfoot覆盖
