\documentclass[UTF8, fontset=ubuntu]{ctexart}
\usepackage{multirow}
\usepackage{diagbox}
\usepackage{array}
\begin{document}
%\begin{tabular}[t]{l|c|r}
%	1 & 2 & 3\\
%	\hline
%	4 & 50 & 6\\
%	\hline
%	7 & 10000 & 9\\
%\end{tabular}\par
%\begin{tabular}{|c|r|r|}
%	\hline
%	\multirow{2}*{姓名} & \multicolumn{2}{c|}{成绩}\\
%	\cline{2-3}
%	& 语文 & 数学\\
%	\hline
%	张三 & 87 & 100 \\
%	\hline
%\end{tabular}\par
%\begin{tabular}{|c|c|c|}
%	\hline
%	\diagbox{姓名}{成绩}{性别} & 男 & 女\\
%	\hline
%	琳达 & & 100\\
%	\hline
%	李磊 & 94 &\\
%	\hline
%	露西 & & 79\\
%	\hline
%\end{tabular}\par
\begin{tabular}{|l>{\$}l|r|}
	\hline
	姓名 & 年龄 & 成绩\\
	\hline
	汤普森 & 45 & 100\\
	\hline
	约翰布兰妮 & 31 &97\\
	\hline
	威尔 & 12 & 67\\
	\hline
\end{tabular}
\end{document}

% \begin{tabular}[t]{<align><align>...}...\end{tabular}代表表格环境,可选参数[]为当前表格的顶部/中部/底部与当前行文字对齐,使用t/c/b表示,默认为c;必选参数{}代表列数与分隔方式

% 必选参数详解 - <align>为列格式,详细格式参数如下:
% l - 左对齐
% c - 中央对齐
% r - 右对齐
% p{<width>} - 指定列的宽度,行内垂直方向上对齐
% | - 以竖线作为列分隔符,不占用列
% @{<content>} - 以指定内容为列分隔符,不占用列
% *{<counter>}{<col_des>} - col_des代表以上各种模式,并重复指定counter次
% **-** 使用array宏包后
% m{<width>} - 类似于p{},但行内垂直方向中对齐
% b{<width>} - 类似于p{},但行内垂直方向下对齐
% >{<content>} - 将content添加到后续列的开头,不占用列
% <{<content>} - 将content添加到前方列的末尾,不占用列
% !{<content>} - 类似于@{},但左右会有额外距离

% \multicolumn{<col_num>}{<align>}{<content>}用于表明合并多列。col_num合并列数;align代表列格式,不能使用*{<conter>}{<col_des>}格式;content代表列内容

% \multirow{<row_num>}{<width>}{<content>}用于合并多行。row_num合并行数;width为单元宽度,超过自动换行,也可以将{<width>}换为*,取最长那一行的宽度;content为单元的内容。包含在multirow宏包中

% \cline{<col_range>}为占据指定列范围的横线

% \diagbox{<left>}{<center>}{<right>}用于创建斜线分割,其中{<center>}可以省略。包含在diagbox宏包中