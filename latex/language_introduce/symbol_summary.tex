\documentclass[UTF8,fontset=ubuntu]{ctexart}
\usepackage{amsmath}
\usepackage{amssymb}
\usepackage{bbm}
\usepackage{eucal}
\usepackage{float}
\usepackage{graphicx}
\usepackage{latexsym}
\usepackage{mathdots}
\usepackage{mathrsfs}
\usepackage{metalogo}
\usepackage{mflogo}
\usepackage{ragged2e}
\usepackage{texnames}
\usepackage{threeparttable}
\begin{document}

\begin{table}[H]
\begin{tabular}{c c c c c c c c}
	\hline
	符号 & 代码 & 符号 & 代码 & 符号 & 代码 & 符号 & 代码\\
	\hline
	\textbackslash & \textbackslash textbackslash & \{ & \textbackslash\{ & \} & \textbackslash\} & \~{} & \verb|\~{}|\\
	\$ & \textbackslash\$ & \% & \textbackslash\% & \^{} & \verb|\^{}| & \# & \textbackslash\#\\
	\& & \textbackslash\& & \_ & \textbackslash\_\\
	\hline
\end{tabular}\\[2mm]
\RaggedRight
**也可使用\textbackslash verb\textless sep\textgreater\textless content\textgreater\textless sep\textgreater 来抄录单行特殊字符,\textbackslash begin\{verbatim\}...\textbackslash end\{verbatim\}用于环境式多行抄录\\
\caption{专用符号}
\end{table}

\begin{table}[H]
\begin{tabular}{l l l l l l}
	\hline
	符号 & 代码 & 符号 & 代码 & 符号 & 代码\\
	\hline
	\textbar & \textbackslash textbar & \textless & \textbackslash textless & \textgreater & \textbackslash textgreater\\
	\hline
\end{tabular}\\[2mm]
\RaggedRight
**也可使用\textbackslash verb\textless sep\textgreater\textless content\textgreater\textless sep\textgreater 来抄录单行特殊字符,\textbackslash begin\{verbatim\}...\textbackslash end\{verbatim\}用于环境式多行抄录\\
\caption{键盘符号}
\end{table}

\begin{table}[H]
\begin{tabular}{l l l l l l l l}
	\hline
	符号 & 代码 & 符号 & 代码 & 符号 & 代码 & 符号 & 代码\\
	\hline
	\'{o} & \textbackslash'\{o\} & \`{o} & \textbackslash`\{o\} & \^{o} & \textbackslash\^{}\{o\} & \"{o} & \textbackslash"\{o\}\\
	\~{o} & \textbackslash\~{}\{o\} & \={o} & \textbackslash=\{o\} & \.{o} & \textbackslash.\{o\} & \u{o} & \textbackslash u\{o\}\\
	\v{o} & \textbackslash v\{o\} & \H{o} & \textbackslash H\{o\} & \t{oo} & \textbackslash t\{oo\} & \c{o} & \textbackslash c\{o\}\\
	\d{o} & \textbackslash d\{o\} & \b{o} & \textbackslash b\{o\}\\
	\hline
\end{tabular}
\caption{重音符}
\end{table}

\begin{table}[H]
\begin{tabular}{l l l l l l l l}
	\hline
	符号 & 代码 & 符号 & 代码 & 符号 & 代码 & 符号 & 代码\\
	\hline
	$\hat{a}$ & \textbackslash hat\{a\} & $\check{a}$ & \textbackslash check\{a\} & $\breve{a}$ & \textbackslash breve\{a\} & $\acute{a}$ & \textbackslash acute\{a\}\\
	$\grave{a}$ & \textbackslash grave\{a\} & $\tilde{a}$ & \textbackslash tilde\{a\} & $\bar{a}$ & \textbackslash bar\{a\} & $\vec{a}$ & \textbackslash vec\{a\}\\
	$\dot{a}$ & \textbackslash dot\{a\} & $\ddot{a}$ & \textbackslash ddot\{a\} & $\widehat{aa}$ & \textbackslash widehat\{aa\} & $\widetilde{aa}$ & \textbackslash widetilde\{aa\}\\
	\hline
\end{tabular}\\[2mm]
** 对i/j进行重音注解时, 需要先消除'.', 格式为\textbackslash imath或\textbackslash jmath
\caption{公式-重音符}
\end{table}

\begin{table}[H]
\begin{tabular}{l l l l l l l l}
	\hline
	符号 & 代码 & 符号 & 代码 & 符号 & 代码 & 符号 & 代码\\
	\hline
	\oe & \textbackslash oe & \OE & \textbackslash OE & \ae & \textbackslash ae & \AE & \textbackslash AE\\
	\aa & \textbackslash aa & \AA & \textbackslash AA & \o & \textbackslash o & \O & \textbackslash O\\
	\l & \textbackslash l & \L & \textbackslash L & \ss & \textbackslash ss & ?` & ?\,`\\
	!` & !\,`\\
	\hline
\end{tabular}
\caption{非英文标记符号}
\end{table}

\begin{threeparttable}
\begin{tabular}{l l l l l l l l}
	\hline
	符号 & 代码 & 符号 & 代码 & 符号 & 代码 & 符号 & 代码\\
	\hline
	$\alpha$ & $\backslash$alpha & $\beta$ & $\backslash$beta & $\gamma$ & $\backslash$gamma & $\delta$ & $\backslash$delta\\
	$\epsilon$ & $\backslash$epsilon & $\zeta$ & $\backslash$zeta & $\eta$ & $\backslash$eta & $\theta$ & $\backslash$theta\\
	$\iota$ & $\backslash$iota & $\kappa$ & $\backslash$kappa & $\lambda$ & $\backslash$lambda & $\mu$ & $\backslash$mu\\
	$\nu$ & $\backslash$nu & $\xi$ & $\backslash$xi & $\pi$ & $\backslash$pi & $\rho$ & $\backslash$rho\\
	$\sigma$ & $\backslash$sigma & $\tau$ & $\backslash$tau & $\upsilon$ & $\backslash$upsilon & $\phi$ & $\backslash$phi\\
	$\chi$ & $\backslash$chi & $\psi$ & $\backslash$psi & $\omega$ & $\backslash$omega & $\varepsilon$ & $\backslash$varepsilon\\
	$\vartheta$ & $\backslash$vartheta & $\varkappa$ & $\backslash$varkappa\tnote{1} & $\varpi$ & $\backslash$varpi & $\varrho$ & $\backslash$varrho\tnote{1}\\
	$\varsigma$ & $\backslash$varsigma & $\varphi$ & $\backslash$varphi & $\digamma$ & $\backslash$digamma\tnote{1}\\
	\hline
\end{tabular}
**$\backslash$var格式的代码由amsmath宏包提供
\begin{tablenotes}
	\item[1] \AMS 符号,包含在amssymb宏包中
\end{tablenotes}
\caption{公式-小写希腊字母}
\end{threeparttable}

\begin{table}[H]
\begin{tabular}{l l l l l l l l}
	\hline
	符号 & 代码 & 符号 & 代码 & 符号 & 代码 & 符号 & 代码\\
	\hline
	$\Gamma$ & $\backslash$Gamma & $\Delta$ & $\backslash$Delta & $\Theta$ & $\backslash$Theta & $\Lambda$ & $\backslash$Lambda\\
	$\Xi$ & $\backslash$Xi & $\Pi$ & $\backslash$Pi & $\Sigma$ & $\backslash$Sigma & $\Upsilon$ & $\backslash$Upsilon\\
	$\Phi$ & $\backslash$Phi & $\Psi$ & $\backslash$Psi & $\Omega$ & $\backslash$Omega & $\varGamma$ & $\backslash$varGamma\\
	$\varDelta$ & $\backslash$varDelta & $\varTheta$ & $\backslash$varTheta & $\varLambda$ & $\backslash$varLambda & $\varXi$ & $\backslash$varXi\\
	$\varPi$ & $\backslash$varPi & $\varSigma$ & $\backslash$varSigma & $\varUpsilon$ & $\backslash$varUpsilon & $\varPhi$ & $\backslash$varPhi\\
	$\varPsi$ & $\backslash$varPsi & $\varOmega$ & $\backslash$varOmega\\
	\hline
\end{tabular}\\[2mm]
**$\backslash$var格式的代码由amsmath宏包提供
\caption{公式-大写希腊字母}
\end{table}

\begin{table}[H]
\begin{tabular}{l l l l l l l l l}
    \hline
    inline & display & 代码 & inline & display & 代码 & inline & display & 代码\\
    \hline
    $\sum$ & $\displaystyle\sum$ & \textbackslash sum & $\prod$ & $\displaystyle\prod$ & \textbackslash prod & $\coprod$ & $\displaystyle\coprod$ & \textbackslash coprod\\
    $\int$ & $\displaystyle\int$ & \textbackslash int & $\oint$ & $\displaystyle\oint$ & \textbackslash oint & $\bigcap$ & $\displaystyle\bigcap$ & \textbackslash bigcap\\
    $\bigcup$ & $\displaystyle\bigcup$ & \textbackslash bigcup & $\bigsqcup$ & $\displaystyle\bigsqcup$ & \textbackslash bigsqcup & $\bigvee$ & $\displaystyle\bigvee$ & \textbackslash bigvee\\
    $\bigwedge$ & $\displaystyle\bigwedge$ & \textbackslash bigwedge & $\bigodot$ & $\displaystyle\bigodot$ & \textbackslash bigodot & $\bigoplus$ & $\displaystyle\bigoplus$ & \textbackslash bigoplus\\
    $\bigotimes$ & $\displaystyle\bigotimes$ & \textbackslash bigotimes & $\biguplus$ & $\displaystyle\biguplus$ & \textbackslash biguplus & $\iint$ & $\displaystyle\iint$ & \textbackslash iint\\
    $\iiint$ & $\displaystyle\iiint$ & \textbackslash iiint & $\iiiint$ & $\displaystyle\iiiint$ & \textbackslash iiiint & $\idotsint$ & $\displaystyle\idotsint$ & \textbackslash idotsint\\
	\hline
\end{tabular}\\[2mm]
**最后四个积分符号需要amsmath宏包
\caption{公式-大小可变的运算符}
\end{table}

\begin{table}[H]
\begin{tabular}{l l l l l l l l l l}
    \hline
    符号 & 代码 & 符号 & 代码 & 符号 & 代码 & 符号 & 代码 & 符号 & 代码\\
    \hline
    $\arccos$ & \textbackslash arccos & $\arcsin$ & \textbackslash arcsin & $\arctan$ & \textbackslash arctan & $\arg$ & \textbackslash arg & $\cos$ & \textbackslash cos\\
    $\cosh$ & \textbackslash cosh & $\cot$ & \textbackslash cot & $\coth$ & \textbackslash coth & $\csc$ & \textbackslash csc & $\deg$ & \textbackslash deg\\
    $\det$ & \textbackslash det & $\dim$ & \textbackslash dim & $\exp$ & \textbackslash exp & $\gcd$ & \textbackslash gcd & $\hom$ & \textbackslash hom\\
    $\inf$ & \textbackslash inf & $\ker$ & \textbackslash ker & $\lg$ & \textbackslash lg & $\lim$ & \textbackslash lim & $\liminf$ & \textbackslash liminf\\
    $\limsup$ & \textbackslash limsup & $\ln$ & \textbackslash ln & $\log$ & \textbackslash log & $\max$ & \textbackslash max & $\min$ & \textbackslash min\\
    $\Pr$ & \textbackslash Pr & $\sec$ & \textbackslash sec & $\sin$ & \textbackslash sin & $\sinh$ & \textbackslash sinh & $\sup$ & \textbackslash sup\\
    $\tan$ & \textbackslash tan & $\tanh$ & \textbackslash tanh\\
    \hline
\end{tabular}\\[2mm]
**可在导言区使用\textbackslash DeclareMathOperator\{\textbackslash \textless command\textgreater\}\{\textless str\textgreater\}来定义新数学符号.如\textbackslash DeclareMathOperator\{\textbackslash sech\}\{sech\}
\caption{不带上下限的数学运算符}
\end{table}

\begin{table}[H]
\begin{tabular}{l l l l l l l l}
	\hline
	符号 & 代码 & 符号 & 代码 & 符号 & 代码 & 符号 & 代码\\
	\hline
	$\lim$ & $\backslash$lim & $\limsup$ & $\backslash$limsup & $\liminf$ & $\backslash$liminf & $\max$ & $\backslash$max\\
	$\min$ & $\backslash$min & $\sup$ & $\backslash$sup & $\inf$ & $\backslash$inf & $\det$ & $\backslash$det\\
	$\Pr$ & $\backslash$Pr & $\gcd$ & $\backslash$gcd & $\varliminf$ & $\backslash$varliminf & $\varlimsup$ & $\backslash$varlimsup\\
	$\injlim$ & $\backslash$injlim & $\projlim$ & $\backslash$projlim & $\varinjlim$ & $\backslash$varinjlim & $\varprojlim$ & $\backslash$varprojlim\\
	\hline
\end{tabular}\\[2mm]
**$\backslash$var类型需要amsmath宏包
\caption{带上下限的数学运算符}
\end{table}

\begin{threeparttable}
\begin{tabular}{l l l l l l l l}
	\hline
	符号 & 代码 & 符号 & 代码 & 符号 & 代码 & 符号 & 代码\\
	\hline
	$\hbar$ & \textbackslash hbar & $\imath$ & \textbackslash imath & $\jmath$ & \textbackslash jmath & $\ell$ & \textbackslash ell\\
	$\wp$ & \textbackslash wp & $\Re$ & \textbackslash Re & $\Im$ & \textbackslash Im & $\partial$ & \textbackslash partial\\
	$\infty$ & \textbackslash infty & $\prime$ & \textbackslash prime & $\emptyset$ & \textbackslash emptyset & $\nabla$ & \textbackslash nabla\\
	$\surd$ & \textbackslash surd & $\top$ & \textbackslash top & $\bot$ & \textbackslash bot & $\angle$ & \textbackslash angle\\
	$\triangle$ & \textbackslash triangle & $\forall$ & \textbackslash forall & $\exists$ & \textbackslash exists & $\neg$ & \textbackslash neg\\
	$\flat$ & \textbackslash flat & $\natural$ & \textbackslash natural & $\sharp$ & \textbackslash sharp & $\clubsuit$ & \textbackslash clubsuit\\
	$\diamondsuit$ & \textbackslash diamondsuit & $\heartsuit$ & \textbackslash heartsuit & $\spadesuit$ & \textbackslash spadesuit & $\backslash$ & \textbackslash backslash\tnote{1}\\
	$\backprime$ & \textbackslash backprime & $\hslash$ & \textbackslash hslash & $\varnothing$ & \textbackslash varnothing & $\vartriangle$ & \textbackslash vartriangle\\
	$\blacktriangle$ & \textbackslash blacktriangel & $\triangledown$ & \textbackslash triangledown & $\blacktriangledown$ & \textbackslash blacktriangledown & $\square$ & \textbackslash square\\
	$\blacksquare$ & \textbackslash blacksquare & $\lozenge$ & \textbackslash lozenge & $\blacklozenge$ & \textbackslash blacklozenge & $\circledS$ & \textbackslash circledS\\
	$\bigstar$ & \textbackslash bigstar & $\sphericalangle$ & \textbackslash sphericalangle & $\measuredangle$ & \textbackslash measuredangle & $\nexists$ & \textbackslash nexists\\
	$\complement$ & \textbackslash complement & $\mho$ & \textbackslash mbo & $\eth$ & \textbackslash eth & $\Finv$ & \textbackslash Finv\\
	$\diagup$ & \textbackslash diagup & $\Game$ & \textbackslash Game & $\diagdown$ & \textbackslash diagdown & $\Bbbk$ & \textbackslash Bbbk\\
	\hline
\end{tabular}
**从\textbackslash backprime开始是\AMS 符号
\begin{tablenotes}
	\item[1] \textbackslash backslash同时也是长度可变的定界符,并有一个同形的二元运算符\textbackslash setminus
\end{tablenotes}
\caption{数学普通符号}
\end{threeparttable}

\begin{table}[H]
\begin{tabular}{l l l l l l l l}
	\hline
	符号 & 代码 & 符号 & 代码 & 符号 & 代码 & 符号 & 代码\\
	\hline
	$\mp$ & \textbackslash mp & $\pm$ & \textbackslash pm & $\ast$ & \textbackslash ast & $\times$ & \textbackslash times\\
	$\div$ & \textbackslash div & $\circ$ & \textbackslash circ & $\bigcirc$ & \textbackslash bigcirc & $\setminus$ & \textbackslash setminus\\
	$\cdot$ & \textbackslash cdot & $\star$ & \textbackslash star & $\cap$ & \textbackslash cap & $\cup$ & \textbackslash cup\\
	$\triangleleft$ & \textbackslash triangleleft & $\triangleright$ & \textbackslash triangleright & $\bigtriangleup$ & \textbackslash bigtriangleup & $\bigtriangledown$ & \textbackslash bigtriangledown\\
	$\wedge$ & \textbackslash wedge & $\vee$ & \textbackslash vee & $\ddagger$ & \textbackslash ddagger & $\dagger$ & \textbackslash dagger\\
	$\sqcap$ & \textbackslash sqcap & $\sqcup$ & \textbackslash sqcup & $\uplus$ & \textbackslash uplus & $\amalg$ & \textbackslash amalg\\
	$\diamond$ & \textbackslash diamond & $\bullet$ & \textbackslash bullet & $\wr$ & \textbackslash wr & $\odot$ & \textbackslash odot\\
	$\oslash$ & \textbackslash oslash & $\otimes$ & \textbackslash otimes & $\oplus$ & \textbackslash oplus & $\ominus$ & \textbackslash ominus\\
	$\lhd$ & \textbackslash lhd & $\rhd$ & \textbackslash rhd & $\unlhd$ & \textbackslash unlhd & $\unrhd$ & \textbackslash unrhd\\
	\hline
\end{tabular}
\RaggedRight
**最后一排指令包含在latexsym宏包中
\caption{二元运算符}
\end{table}

\begin{table}[H]
\begin{tabular}{l l l l l l l l}
    \hline
    符号 & 代码 & 符号 & 代码 & 符号 & 代码 & 符号 & 代码\\\hline
	$\leq$ & \textbackslash leq & $\geq$ & \textbackslash geq & $\leqslant$ & \textbackslash leqslant & $\geqslant$ & \textbackslash geqslant\\
    $\equiv$ & \textbackslash equiv & $\models$ & \textbackslash models & $\prec$ & \textbackslash prec & $\succ$ & \textbackslash succ\\
    $\sim$ & \textbackslash sim & $\perp$ & \textbackslash perp & $\preceq$ & \textbackslash preceq & $\succeq$ & \textbackslash succeq\\
    $\simeq$ & \textbackslash simeq & $\mid$ & \textbackslash mid & $\ll$ & \textbackslash ll & $\gg$ & \textbackslash gg\\
    $\asymp$ & \textbackslash asymp & $\parallel$ & \textbackslash parallel & $\subset$ & \textbackslash subset & $\supset$ & \textbackslash supset\\
    $\approx$ & \textbackslash approx & $\bowtie$ & \textbackslash bowtie & $\subseteq$ & \textbackslash subseteq & $\supseteq$ & \textbackslash supseteq\\
    $\cong$ & \textbackslash cong & $\neq$ & \textbackslash neq & $\smile$ & \textbackslash smile & $\sqsubseteq$ & \textbackslash sqsubseteq\\
    $\sqsupseteq$ & \textbackslash sqsupseteq & $\doteq$ & \textbackslash doteq & $\frown$ & \textbackslash frown & $\in$ & \textbackslash in\\
    $\ni$ & \textbackslash ni & $\notin$ & \textbackslash notin & $\propto$ & \textbackslash propto & $\vdash$ & \textbackslash vdash\\
    $\dashv$ & \textbackslash dashv & $\Join$ & \textbackslash Join & $\sqsubset$ & \textbackslash sqsubset & $\sqsupset$ & \textbackslash sqsupset\\
    \hline
\end{tabular}
** leqslant和geqslant包含在amssymb宏包中\\
** 最后三个符号包含在latexsym宏包中
\caption{二元关系符}
\end{table}

\begin{table}[H]
\begin{tabular}{l l l l l l}
    \hline
    符号 & 代码 & 符号 & 代码 & 符号 & 代码\\
    \hline
    $\leftarrow$ & \textbackslash leftarrow & $\longleftarrow$ & \textbackslash longleftarrow & $\uparrow$ & \textbackslash uparrow\\
    $\Leftarrow$ & \textbackslash Leftarrow & $\Longleftarrow$ & \textbackslash Longleftarrow & $\Uparrow$ & \textbackslash Uparrow\\
    $\rightarrow$ & \textbackslash rightarrow & $\longrightarrow$ & \textbackslash longrightarrow & $\downarrow$ & \textbackslash downarrow\\
    $\Rightarrow$ & \textbackslash Rightarrow & $\Longrightarrow$ & \textbackslash Longrightarrow & $\Downarrow$ & \textbackslash Downarrow\\
    $\leftrightarrow$ & \textbackslash leftrightarrow & $\longleftrightarrow$ & \textbackslash longleftrightarrow & $\updownarrow$ & \textbackslash updownarrow\\
    $\Leftrightarrow$ & \textbackslash Leftrightarrow & $\Longleftrightarrow$ & \textbackslash Longleftarrow & $\Updownarrow$ & \textbackslash Updownarrow\\
    $\mapsto$ & \textbackslash mapsto & $\longmapsto$ & \textbackslash longmapsto & $\nearrow$ & \textbackslash nearrow\\
    $\hookleftarrow$ & \textbackslash hookleftarrow & $\hookrightarrow$ & \textbackslash hookrightarrow & $\searrow$ & \textbackslash searrow\\
   $\leftharpoonup$ & \textbackslash leftharpoonup & $\rightharpoonup$ & \textbackslash rightharpoonup & $\swarrow$ & \textbackslash swarrow\\
    $\leftharpoondown$ & \textbackslash leftharpoondown & $\rightharpoondown$ & \textbackslash rightharpoondown & $\nwarrow$ & \textbackslash nwarrow\\
    $\rightleftharpoons$ & \textbackslash rightleftharpoons & $\leadsto$ & \textbackslash leadsto\\
    \hline
\end{tabular}
**\textbackslash leadsto包含在latexsym宏包中
\caption{\LaTeX 箭头符号}
\end{table}

\begin{table}[H]
\begin{tabular}{l l l l l l l l}
	\hline
	符号 & 代码 & 符号 & 代码 & 符号 & 代码 & 符号 & 代码\\
	\hline
	$($ & ( & $)$ & ) & $[$ & [ & $]$ & ]\\
	$\{$ & \textbackslash \{ & $\}$ & \textbackslash \} & $\lfloor$ & \textbackslash lfloor & $\rfloor$ & \textbackslash rfloor\\
	$\lceil$ & \textbackslash lceil & $\rceil$ & \textbackslash rceil & $\langle$ & \textbackslash langle & $\rangle$ & \textbackslash rangle\\
	$/$ & / & $\backslash$ & \textbackslash backslash & $|$ & \textbar & $\|$ & \textbackslash\textbar\\
	$\uparrow$ & \textbackslash uparrow & $\downarrow$ & \textbackslash downarrow & $\updownarrow$ & \textbackslash updownarrow & $\Uparrow$ & \textbackslash Uparrow\\
	$\Downarrow$ & \textbackslash Downarrow & $\Updownarrow$ & \textbackslash Updownarrow\\
	\hline
\end{tabular}\\[2mm]
** 在左/右括号前使用\textbackslash left或\textbackslash right可使限定符视情况改变大小\par
** \textbackslash left与\textbackslash right必须成对匹配, 但限定符类型可从集合中任意选取两个\par
** 当只包含左限定符时, 使用\textbackslash right.来关闭. 只包含右限定符时,原理类似\par
** 也可手动调节大小,位置:\textbackslash big \textbackslash bigl \textbackslash bigm \textbackslash bigr,规格:\textbackslash big \textbackslash Big \textbackslash bigg \textbackslash Bigg
\caption{公式-括号限定符}
\end{table}

\begin{table}[H]
\begin{tabular}{l l l l l l l l}
	\hline
	符号 & 代码 & 符号 & 代码 & 符号 & 代码 & 符号 & 代码\\
	\hline
	$\ldots$ & $\backslash$ldots & $\cdots$ & $\backslash$cdots & $\vdots$ & $\backslash$vdots & $\ddots$ & $\backslash$ddots\\
	$\iddots$ & $\backslash$iddots & $\dotsc$ & $\backslash$dotsc & $\dotsb$ & $\backslash$dotsb & $\dotsm$ & $\backslash$dotsm\\
	$\dotsi$ & $\backslash$dotsi & $\dotso$ & $\backslash$dotso\\
	\hline
\end{tabular}\\[2mm]
\RaggedRight
**除\textbackslash ldots、\textbackslash vdots、\textbackslash dotso外,其他只能用于math mode\par
**\textbackslash iddots在mathdots宏包中\par
**\textbackslash dots*在amsmath宏包中
\caption{公式-省略号}
\end{table}

\begin{table}[H]
\begin{tabular}{l l}
	\hline
	单位 & 说明\\
	\hline
	mm & 1 mm=2.845 pt\\
	pt & 1 pt=0.351 mm\\
	bp & 1 bp=0.353 mm$\approx$ 1 pt\\
	dd & 1 dd=0.376 mm=1.07 pt\\
	pc & 1 pc=4.218 mm=12 pt\\
	sp & 65536 sp=1 pt\\
	cm & 1 cm=10 mm=28.453 pt\\
	cc & 1 cc =4.513 mm=12 dd=12.84 pt\\
	in & 1 in=25.4 mm=72.27 pt\\
	ex & 1 ex=当前字体中x的高度\\
	em & 1 em=当前字体尺寸$\approx$ M的宽度\\
	\hline
\end{tabular}
\caption{通用长度单位}
\end{table}

\begin{threeparttable}
\begin{tabular}{l l l}
	\hline
	类别 & 字体命令 & 输出效果\\\hline
	数学环境的默认字体 & \textbackslash mathnormal & $\mathnormal{ABCHIJXYZabchijxyz12345}$\\
	斜体 & \textbackslash mathit & $\mathit{ABCHIJXYZabchijxyz12345}$\\
	粗体 & \textbackslash mathbf & $\mathbf{ABCHIJXYZabchijxyz12345}$\\
	罗马体 & \textbackslash mathrm & $\mathrm{ABCHIJXYZabchijxyz12345}$\\
	无衬线体 & \textbackslash mathsf & $\mathsf{ABCHIJXYZabchijxyz12345}$\\
	打字机体 & \textbackslash mathtt & $\mathtt{ABCHIJXYZabchijxyz12345}$\\
	手写体(花体)\tnote{1} & \textbackslash mathcal & $\mathcal{ABCHIJXYZ}$\\\hline
\end{tabular}
\begin{tablenotes}
	\item[1] LaTeX默认只支持大写字母,使用专业字体包可支持小写字母
\end{tablenotes}
\caption{LaTeX默认提供的数学字体}
\end{threeparttable}

\begin{table}[H]
	\begin{tabular}{l l l l}
		\hline
		类别 & 字体命令 & 输出效果 & 宏包及说明\\\hline
		黑板粗体 & \textbackslash mathbb & $\mathbb{ABCXYZ}$ & amssymb,仅大写字母\\
				 & \textbackslash mathbbm & $\mathbbm{ABCXYZabcxyz12}$ & bbm,数字仅有1和2\\
		花体 & \textbackslash mathscr & $\mathscr{ABCXYZ}$ & mathrsfs,仅大写字母\\
			 & \textbackslash mathcal & $\mathcal{ABCXYZ}$ & eucal,仅大写字母\\
		哥特体 & \textbackslash mathfrak & $\mathfrak{ABCXYZabcxyz123890}$ & amssymb或eufrak\\\hline
	\end{tabular}
	\caption{其他宏包字体}
\end{table}

\begin{table}[H]
	\begin{tabular}{l l l}
		\hline
		标识符 & 符号指令 & 所需宏包\\\hline
		\TeX & \textbackslash TeX &\\
		\LaTeX & \textbackslash LaTeX &\\
		\LaTeXe & \textbackslash LaTeXe &\\
		\AMS & \textbackslash AMS & texnames\\
		\AMSTeX & \textbackslash AMSTeX & texnames\\
		\BibTeX & \textbackslash BibTeX & texnames\\
		\XeTeX & \textbackslash XeTeX & metalogo\\
		\XeLaTeX & \textbackslash XeLaTeX & metalogo\\
		\LuaTeX & \textbackslash LuaTeX & metalogo\\
		\LuaLaTeX & \textbackslash LuaLaTeX & metalogo\\
		\hline
	\end{tabular}
	\caption{TeX家族标识符}
\end{table}
\end{document}
