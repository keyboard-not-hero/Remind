\documentclass[UTF8,fontset=ubuntu]{ctexart}
\usepackage{amsmath}
\usepackage{amssymb}
\usepackage{bbding}
\usepackage{bbm}
\usepackage{float}
\usepackage{graphicx}
\usepackage{latexsym}
\usepackage{mathdots}
\usepackage{mathrsfs}
\usepackage{euscript}
\usepackage{metalogo}
\usepackage{mflogo}
\usepackage{texnames}
\usepackage{pifont}
\usepackage{longtable}
\let\mathdollar\undefined
\renewcommand{\footnoterule}{\vspace*{-2mm}\rule{\textwidth}{0pt}}
\renewcommand{\thempfootnote}{\arabic{mpfootnote}}
\begin{document}
\begin{table}[H]
\begin{minipage}{\textwidth}
\begin{tabular}{c c c c c c c c}
	\hline
	符号 & 代码 & 符号 & 代码 & 符号 & 代码 & 符号 & 代码\\
	\hline
	\textbackslash  & \textbackslash  textbackslash & \{ & \textbackslash \{ & \} & \textbackslash \} & \~{} & \verb|\~{}|\\
	\$ & \textbackslash \$ & \% & \textbackslash \% & \^{} & \verb|\^{}| & \# & \textbackslash \#\\
	\& & \textbackslash \& & \_ & \textbackslash \_\\
	\hline
\end{tabular}
\footnotetext[99]{也可使用\textbackslash  verb\textless sep\textgreater\textless content\textgreater\textless sep\textgreater 来抄录单行特殊字符,\textbackslash  begin\{verbatim\}...\\\textbackslash end\{verbatim\}用于环境式多行抄录}
\end{minipage}
\caption{专用符号}
\end{table}

\begin{table}[H]
\begin{minipage}{\textwidth}
\begin{tabular}{l l l l l l}
	\hline
	符号 & 代码 & 符号 & 代码 & 符号 & 代码\\
	\hline
	\textbar & \textbackslash  textbar & \textless & \textbackslash  textless & \textgreater & \textbackslash  textgreater\\
	\hline
\end{tabular}
\footnotetext[99]{也可使用\textbackslash  verb\textless sep\textgreater\textless content\textgreater\textless sep\textgreater 来抄录单行特殊字符,\textbackslash  begin\\\{verbatim\}...\textbackslash  end\{verbatim\}用于环境式多行抄录}
\end{minipage}
\caption{键盘符号}
\end{table}

\begin{table}[H]
\begin{tabular}{l l l l l l l l}
	\hline
	符号 & 代码 & 符号 & 代码 & 符号 & 代码 & 符号 & 代码\\
	\hline
	\'{o} & \textbackslash '\{o\} & \`{o} & \textbackslash `\{o\} & \^{o} & \textbackslash \^{}\{o\} & \"{o} & \textbackslash "\{o\}\\
	\~{o} & \textbackslash \~{}\{o\} & \={o} & \textbackslash =\{o\} & \.{o} & \textbackslash .\{o\} & \u{o} & \textbackslash  u\{o\}\\
	\v{o} & \textbackslash  v\{o\} & \H{o} & \textbackslash  H\{o\} & \t{oo} & \textbackslash  t\{oo\} & \c{o} & \textbackslash  c\{o\}\\
	\d{o} & \textbackslash  d\{o\} & \b{o} & \textbackslash  b\{o\}\\
	\hline
\end{tabular}
\caption{重音符}
\end{table}

\begin{table}[H]
\begin{minipage}{\textwidth}
\begin{tabular}{l l l l l l}
	\hline
	符号 & 代码 & 符号 & 代码 & 符号 & 代码\\
	\hline
	$\hat{a}$ & \textbackslash  hat\{a\} & $\check{a}$ & \textbackslash  check\{a\} & $\breve{a}$ & \textbackslash  breve\{a\}\\
	$\acute{a}$ & \textbackslash  acute\{a\} & $\grave{a}$ & \textbackslash  grave\{a\} & $\tilde{a}$ & \textbackslash  tilde\{a\}\\
	$\bar{a}$ & \textbackslash  bar\{a\} & $\vec{a}$ & \textbackslash  vec\{a\} & $\dot{a}$ & \textbackslash  dot\{a\}\\
	 $\ddot{a}$ & \textbackslash  ddot\{a\} & $\widehat{aa}$ & \textbackslash  widehat\{aa\} & $\widetilde{aa}$ & \textbackslash  widetilde\{aa\}\\
	\hline
\end{tabular}
\footnotetext[99]{对i/j进行重音注解时,需要先消除'.',格式为\textbackslash  imath或\textbackslash  jmath}
\end{minipage}
\caption{公式-重音符}
\end{table}

\begin{table}[H]
\begin{tabular}{l l l l l l l l}
	\hline
	符号 & 代码 & 符号 & 代码 & 符号 & 代码 & 符号 & 代码\\
	\hline
	\oe & \textbackslash  oe & \OE & \textbackslash  OE & \ae & \textbackslash  ae & \AE & \textbackslash  AE\\
	\aa & \textbackslash  aa & \AA & \textbackslash  AA & \o & \textbackslash  o & \O & \textbackslash  O\\
	\l & \textbackslash  l & \L & \textbackslash  L & \ss & \textbackslash  ss & ?` & ?\,`\\
	!` & !\,`\\
	\hline
\end{tabular}
\caption{非英文标记符号}
\end{table}

\begin{table}[H]
\begin{tabular}{l l l l l l}
	\hline
	符号 & 代码 & 符号 & 代码 & 符号 & 代码\\
	\hline
	\dag & \textbackslash  dag & \S & \textbackslash  S & \copyright & \textbackslash  copyright\\
	\ddag & \textbackslash  ddag & \P & \textbackslash  P & \pounds & \textbackslash  pounds\\
	\hline
\end{tabular}
\caption{适用于所有模式的符号}
\end{table}

\begin{table}[H]
\begin{minipage}{\textwidth}
\begin{tabular}{l l l l l l}
	\hline
	符号 & 代码 & 符号 & 代码 & 符号 & 代码\\
	\hline
	$\alpha$ & \textbackslash alpha & $\beta$ & \textbackslash beta & $\gamma$ & \textbackslash gamma\\
	$\delta$ & \textbackslash delta & $\epsilon$ & \textbackslash epsilon & $\zeta$ & \textbackslash zeta\\
	$\eta$ & \textbackslash eta & $\theta$ & \textbackslash theta & $\iota$ & \textbackslash iota\\
	$\kappa$ & \textbackslash kappa & $\lambda$ & \textbackslash lambda & $\mu$ & \textbackslash mu\\
	$\nu$ & \textbackslash nu & $\xi$ & \textbackslash xi & $\pi$ & \textbackslash pi\\
	$\rho$ & \textbackslash rho & $\sigma$ & \textbackslash sigma & $\tau$ & \textbackslash tau\\
	$\upsilon$ & \textbackslash upsilon & $\phi$ & \textbackslash phi & $\chi$ & \textbackslash chi\\
	$\psi$ & \textbackslash psi & $\omega$ & \textbackslash omega & $\varepsilon$ & \textbackslash varepsilon\\
	$\vartheta$ & \textbackslash vartheta & $\varkappa$ & \textbackslash varkappa\footnotemark[1] & $\varpi$ & \textbackslash varpi\\
	$\varrho$ & \textbackslash varrho\footnotemark[1] & $\varsigma$ & \textbackslash varsigma & $\varphi$ & \textbackslash varphi\\
	$\digamma$ & \textbackslash digamma\footnotemark[1]\\
	\hline
\end{tabular}
\footnotetext[1]{\AMS 符号,包含在amssymb宏包中}
\footnotetext[99]{\textbackslash var格式的代码由amsmath宏包提供}
\end{minipage}
\caption{公式-小写希腊字母}
\end{table}

\begin{table}[H]
\begin{minipage}{\textwidth}
\begin{tabular}{l l l l l l}
	\hline
	符号 & 代码 & 符号 & 代码 & 符号 & 代码\\
	\hline
	$\Gamma$ & \textbackslash Gamma & $\Delta$ & \textbackslash Delta & $\Theta$ & \textbackslash Theta\\
	$\Lambda$ & \textbackslash Lambda & $\Xi$ & \textbackslash Xi & $\Pi$ & \textbackslash Pi\\
	$\Sigma$ & \textbackslash Sigma & $\Upsilon$ & \textbackslash Upsilon & $\Phi$ & \textbackslash Phi\\
	$\Psi$ & \textbackslash Psi & $\Omega$ & \textbackslash Omega & $\varGamma$ & \textbackslash varGamma\\
	$\varDelta$ & \textbackslash varDelta & $\varTheta$ & \textbackslash varTheta & $\varLambda$ & \textbackslash varLambda\\
	$\varXi$ & \textbackslash varXi & $\varPi$ & \textbackslash varPi & $\varSigma$ & \textbackslash varSigma\\
	$\varUpsilon$ & \textbackslash varUpsilon & $\varPhi$ & \textbackslash varPhi & $\varPsi$ & \textbackslash varPsi\\
	$\varOmega$ & \textbackslash varOmega\\
	\hline
\end{tabular}
\footnotetext[99]{\textbackslash var格式的代码由amsmath宏包提供}
\end{minipage}
\caption{公式-大写希腊字母}
\end{table}

\begin{table}[H]
\begin{minipage}{\textwidth}
\begin{tabular}{l l l l l l}
    \hline
    inline & display & 代码 & inline & display & 代码\\
    \hline
    $\sum$ & $\displaystyle\sum$ & \textbackslash sum & $\prod$ & $\displaystyle\prod$ & \textbackslash prod\\
	$\coprod$ & $\displaystyle\coprod$ & \textbackslash coprod & $\int$ & $\displaystyle\int$ & \textbackslash  int\\
	$\oint$ & $\displaystyle\oint$ & \textbackslash  oint & $\bigcap$ & $\displaystyle\bigcap$ & \textbackslash  bigcap\\
    $\bigcup$ & $\displaystyle\bigcup$ & \textbackslash  bigcup & $\bigsqcup$ & $\displaystyle\bigsqcup$ & \textbackslash  bigsqcup\\
	$\bigvee$ & $\displaystyle\bigvee$ & \textbackslash  bigvee & $\bigwedge$ & $\displaystyle\bigwedge$ & \textbackslash  bigwedge\\
	$\bigodot$ & $\displaystyle\bigodot$ & \textbackslash  bigodot & $\bigoplus$ & $\displaystyle\bigoplus$ & \textbackslash  bigoplus\\
    $\bigotimes$ & $\displaystyle\bigotimes$ & \textbackslash  bigotimes & $\biguplus$ & $\displaystyle\biguplus$ & \textbackslash  biguplus\\
	$\iint$ & $\displaystyle\iint$ & \textbackslash  iint & $\iiint$ & $\displaystyle\iiint$ & \textbackslash  iiint\\
	$\iiiint$ & $\displaystyle\iiiint$ & \textbackslash  iiiint & $\idotsint$ & $\displaystyle\idotsint$ & \textbackslash  idotsint\\
	\hline
\end{tabular}
\footnotetext[99]{最后四个积分符号需要amsmath宏包}
\footnotetext[99]{微分符号的实现: \textbackslash newcommand\{\textbackslash dif\}\{\textbackslash mathop\{\}\textbackslash !\textbackslash mathrm\{d\}\}}
\end{minipage}
\caption{公式-大小可变的运算符}
\end{table}

\begin{table}[H]
\begin{minipage}{\textwidth}
\begin{tabular}{l@{\hspace{5ex}}l@{\hspace{5ex}}l@{\hspace{5ex}}l@{\hspace{5ex}}l@{\hspace{5ex}}l}
    \hline
    符号 & 代码 & 符号 & 代码 & 符号 & 代码\\
    \hline
    $\arccos$ & \textbackslash arccos & $\arcsin$ & \textbackslash arcsin & $\arctan$ & \textbackslash arctan\\
	$\arg$ & \textbackslash arg & $\cos$ & \textbackslash cos & $\cosh$ & \textbackslash cosh\\
	$\cot$ & \textbackslash cot & $\coth$ & \textbackslash coth & $\csc$ & \textbackslash csc\\
	$\deg$ & \textbackslash deg & $\det$ & \textbackslash det & $\dim$ & \textbackslash dim\\
	$\exp$ & \textbackslash exp & $\gcd$ & \textbackslash gcd & $\hom$ & \textbackslash hom\\
	$\inf$ & \textbackslash inf & $\ker$ & \textbackslash ker & $\lg$ & \textbackslash lg\\
	$\lim$ & \textbackslash lim & $\liminf$ & \textbackslash liminf & $\limsup$ & \textbackslash limsup\\
	$\ln$ & \textbackslash ln & $\log$ & \textbackslash  log & $\max$ & \textbackslash max\\
	$\min$ & \textbackslash  min & $\Pr$ & \textbackslash Pr & $\sec$ & \textbackslash sec\\
	$\sin$ & \textbackslash sin & $\sinh$ & \textbackslash sinh & $\sup$ & \textbackslash sup\\
	$\tan$ & \textbackslash tan & $\tanh$ & \textbackslash tanh\\
    \hline
\end{tabular}
\footnotetext[99]{可在导言区使用\textbackslash  DeclareMathOperator\{\textbackslash  \textless command\textgreater\}\{\textless str\textgreater\}来定义新数学符号.\\如\textbackslash  DeclareMathOperator\{\textbackslash  sech\}\{sech\}, 包含在amsmath宏包中}
\end{minipage}
\caption{不带上下限的数学运算符}
\end{table}

\begin{table}[H]
\begin{minipage}{\textwidth}
\begin{tabular}{l l l l l l}
	\hline
	符号 & 代码 & 符号 & 代码 & 符号 & 代码\\
	\hline
	$\lim$ & \textbackslash lim & $\limsup$ & \textbackslash limsup & $\liminf$ & \textbackslash liminf\\
	$\max$ & \textbackslash max & $\min$ & \textbackslash min & $\sup$ & \textbackslash sup\\
	$\inf$ & \textbackslash inf & $\det$ & \textbackslash det & $\Pr$ & \textbackslash Pr\\
	$\gcd$ & \textbackslash gcd & $\varliminf$ & \textbackslash varliminf & $\varlimsup$ & \textbackslash varlimsup\\
	$\injlim$ & \textbackslash injlim & $\projlim$ & \textbackslash projlim & $\varinjlim$ & \textbackslash varinjlim\\
	$\varprojlim$ & \textbackslash varprojlim\\
	\hline
\end{tabular}
\footnotetext[99]{\textbackslash  var类型需要amsmath宏包}
\end{minipage}
\caption{带上下限的数学运算符}
\end{table}

\begin{longtable}{p{8mm}@{\hspace{1ex}}l@{\hspace{1ex}}l@{\hspace{1ex}}l@{\hspace{1ex}}l@{\hspace{1ex}}l}
	\caption{数学普通符号}\\
	\hline
	符号 & 代码 & 符号 & 代码 & 符号 & 代码\\
	\hline
	$\hbar$ & \textbackslash hbar & $\imath$ & \textbackslash imath & $\jmath$ & \textbackslash jmath\\
	$\ell$ & \textbackslash ell & $\wp$ & \textbackslash wp & $\Re$ & \textbackslash Re\\
	$\Im$ & \textbackslash Im & $\partial$ & \textbackslash partial & $\infty$ & \textbackslash infty\\
 	$\prime$ & \textbackslash prime & $\emptyset$ & \textbackslash emptyset & $\nabla$ & \textbackslash nabla\\
	$\surd$ & \textbackslash surd & $\top$ & \textbackslash top & $\bot$ & \textbackslash bot \\
	$\angle$ & \textbackslash angle & $\triangle$ & \textbackslash triangle & $\forall$ & \textbackslash forall \\
	$\exists$ & \textbackslash exists & $\neg$ & \textbackslash neg & $\flat$ & \textbackslash flat \\
	$\natural$ & \textbackslash natural & $\sharp$ & \textbackslash sharp & $\clubsuit$ & \textbackslash clubsuit\\
	$\diamondsuit$ & \textbackslash diamondsuit & $\heartsuit$ & \textbackslash heartsuit & $\spadesuit$ & \textbackslash spadesuit \\
	\textbackslash & \textbackslash backslash\footnotemark[1] & $\backprime$ & \textbackslash backprime & $\hslash$ & \textbackslash hslash \\
	$\varnothing$ & \textbackslash varnothing & $\vartriangle$ & \textbackslash vartriangle & $\blacktriangle$ & \textbackslash blacktriangel\\
	$\triangledown$ & \textbackslash triangledown & $\blacktriangledown$ & \textbackslash blacktriangledown & $\square$ & \textbackslash square\\
	$\blacksquare$ & \textbackslash blacksquare & $\lozenge$ & \textbackslash lozenge & $\blacklozenge$ & \textbackslash blacklozenge\\
	$\circledS$ & \textbackslash circledS & $\bigstar$ & \textbackslash bigstar & $\sphericalangle$ & \textbackslash sphericalangle \\
	$\measuredangle$ & \textbackslash measuredangle & $\nexists$ & \textbackslash nexists & $\complement$ & \textbackslash complement \\
	$\mho$ & \textbackslash mbo & $\eth$ & \textbackslash eth & $\Finv$ & \textbackslash Finv\\
	$\diagup$ & \textbackslash diagup & $\Game$ & \textbackslash Game & $\diagdown$ & \textbackslash diagdown \\
	$\Bbbk$ & \textbackslash Bbbk & $\because$ & \textbackslash because & $\therefore$ & \textbackslash therefore\\
	\hline
\begin{minipage}{\textwidth}
\footnotetext[1]{\textbackslash  backslash同时也是长度可变的定界符,并有一个同形的二元运算符\textbackslash  setminus}
\footnotetext[99]{从\textbackslash  backprime开始是\AMS 符号, 包含在amssymb宏包中}
\end{minipage}
\end{longtable}

\begin{table}[H]
\begin{minipage}{\textwidth}
\begin{tabular}{l@{\hspace{2ex}}l l@{\hspace{2ex}}l l@{\hspace{2ex}}l}
	\hline
	符号 & 代码 & 符号 & 代码 & 符号 & 代码\\
	\hline
	$\mp$ & \textbackslash  mp & $\pm$ & \textbackslash  pm & $\ast$ & \textbackslash  ast \\
	$\times$ & \textbackslash  times & $\div$ & \textbackslash  div & $\circ$ & \textbackslash  circ \\
	$\bigcirc$ & \textbackslash  bigcirc & $\setminus$ & \textbackslash  setminus & $\cdot$ & \textbackslash  cdot\\
	$\star$ & \textbackslash  star & $\cap$ & \textbackslash  cap & $\cup$ & \textbackslash  cup\\
	$\triangleleft$ & \textbackslash  triangleleft & $\triangleright$ & \textbackslash  triangleright & $\bigtriangleup$ & \textbackslash  bigtriangleup\\
	$\bigtriangledown$ & \textbackslash  bigtriangledown & $\wedge$ & \textbackslash  wedge & $\vee$ & \textbackslash  vee\\
	$\ddagger$ & \textbackslash  ddagger & $\dagger$ & \textbackslash  dagger & $\sqcap$ & \textbackslash  sqcap \\
	$\sqcup$ & \textbackslash  sqcup & $\uplus$ & \textbackslash  uplus & $\amalg$ & \textbackslash  amalg\\
	$\diamond$ & \textbackslash  diamond & $\bullet$ & \textbackslash  bullet & $\wr$ & \textbackslash  wr \\
	$\odot$ & \textbackslash  odot & $\oslash$ & \textbackslash  oslash & $\otimes$ & \textbackslash  otimes \\
	$\oplus$ & \textbackslash  oplus & $\ominus$ & \textbackslash  ominus & $\lhd$ & \textbackslash  lhd\footnotemark[1]\\
	$\rhd$ & \textbackslash  rhd\footnotemark[1] & $\unlhd$ & \textbackslash  unlhd\footnotemark[1] & $\unrhd$ & \textbackslash  unrhd\footnotemark[1]\\
	\hline
\end{tabular}
\footnotetext[1]{包含在latexsym宏包中}
\end{minipage}
\caption{二元运算符}
\end{table}

\begin{table}[H]
\begin{minipage}{\textwidth}
\begin{tabular}{l l l l l l}
	\hline
	符号 & 代码 & 符号 & 代码 & 符号 & 代码\\
	\hline
	$\leq$ & \textbackslash leq & $\geq$ & \textbackslash  geq & $\leqslant$ & \textbackslash  leqslant\footnotemark[1]\\
	$\geqslant$ & \textbackslash  geqslant\footnotemark[1] & $\equiv$ & \textbackslash  equiv & $\models$ & \textbackslash  models\\
	$\prec$ & \textbackslash  prec & $\succ$ & \textbackslash  succ & $\sim$ & \textbackslash  sim\\
	$\backsim$ & \textbackslash  backsim\footnotemark[1] & $\perp$ & \textbackslash  perp & $\preceq$ & \textbackslash  preceq\\
	$\succeq$ & \textbackslash  succeq & $\simeq$ & \textbackslash  simeq & $\mid$ & \textbackslash  mid\\
	$\ll$ & \textbackslash  ll & $\gg$ & \textbackslash  gg & $\asymp$ & \textbackslash  asymp\\
	$\parallel$ & \textbackslash  parallel & $\subset$ & \textbackslash  subset & $\supset$ & \textbackslash  supset\\
	$\approx$ & \textbackslash  approx & $\bowtie$ & \textbackslash  bowtie & $\subseteq$ & \textbackslash  subseteq\\
	$\supseteq$ & \textbackslash  supseteq & $\cong$ & \textbackslash  cong & $\neq$ & \textbackslash  neq\\
	$\smile$ & \textbackslash  smile & $\sqsubseteq$ & \textbackslash  sqsubseteq & $\sqsupseteq$ & \textbackslash  sqsupseteq\\
	$\doteq$ & \textbackslash  doteq & $\frown$ & \textbackslash  frown & $\in$ & \textbackslash  in\\
	$\ni$ & \textbackslash  ni & $\notin$ & \textbackslash  notin & $\propto$ & \textbackslash  propto\\
	$\vdash$ & \textbackslash  vdash & $\dashv$ & \textbackslash  dashv & $\Join$ & \textbackslash  Join\footnotemark[2]\\
	$\sqsubset$ & \textbackslash sqsubset\footnotemark[2] & $\sqsupset$ & \textbackslash  sqsupset\footnotemark[2]\\
	\hline
\end{tabular}
\footnotetext[1]{包含在amssymb宏包中}
\footnotetext[2]{包含在latexsym宏包中}
\footnotetext[99]{可在符号指令前加上\textbackslash  not, 使符号持相反意义}
\end{minipage}
\caption{二元关系符}
\end{table}

\begin{longtable}{p{8mm}@{\hspace{1ex}}l@{\hspace{1ex}}l@{\hspace{1ex}}l@{\hspace{1ex}}l@{\hspace{1ex}}l}
	\caption{\LaTeX{}箭头符号}\\
	\hline
	符号 & 代码 & 符号 & 代码 & 符号 & 代码\\
	\hline
	$\leftarrow$ & \textbackslash  leftarrow & $\longleftarrow$ & \textbackslash  longleftarrow & $\uparrow$ & \textbackslash  uparrow\\
	$\Leftarrow$ & \textbackslash  Leftarrow & $\Longleftarrow$ & \textbackslash  Longleftarrow & $\Uparrow$ & \textbackslash  Uparrow\\
	$\rightarrow$ & \textbackslash  rightarrow & $\longrightarrow$ & \textbackslash  longrightarrow & $\downarrow$ & \textbackslash  downarrow\\
	$\Rightarrow$ & \textbackslash  Rightarrow & $\Longrightarrow$ & \textbackslash  Longrightarrow & $\Downarrow$ & \textbackslash  Downarrow\\
	$\leftrightarrow$ & \textbackslash  leftrightarrow & $\longleftrightarrow$ & \textbackslash  longleftrightarrow & $\updownarrow$ & \textbackslash  updownarrow\\
	$\Leftrightarrow$ & \textbackslash  Leftrightarrow & $\Longleftrightarrow$ & \textbackslash  Longleftrightarrow & $\Updownarrow$ & \textbackslash  Updownarrow\\
	$\mapsto$ & \textbackslash  mapsto & $\longmapsto$ & \textbackslash  longmapsto & $\nearrow$ & \textbackslash  nearrow\\
	$\hookleftarrow$ & \textbackslash  hookleftarrow & $\hookrightarrow$ & \textbackslash  hookrightarrow & $\searrow$ & \textbackslash  searrow\\
	$\leftharpoonup$ & \textbackslash  leftharpoonup & $\rightharpoonup$ & \textbackslash  rightharpoonup & $\swarrow$ & \textbackslash  swarrow\\
	$\leftharpoondown$ & \textbackslash  leftharpoondown & $\rightharpoondown$ & \textbackslash  rightharpoondown & $\nwarrow$ & \textbackslash  nwarrow\\
	$\rightleftharpoons$ & \textbackslash  rightleftharpoons & $\leadsto$ & \textbackslash  leadsto\footnotemark[1]\\
	\multicolumn{3}{l}{$\xleftarrow{\hspace*{2cm}}$} & \multicolumn{3}{l}{\textbackslash xleftarrow[<downscript>]\{<upscript>\}}\footnotemark[2]\\
	\multicolumn{3}{l}{$\xrightarrow{\hspace*{2cm}}$} & \multicolumn{3}{l}{\textbackslash xrightarrow[<downscript>]\{<upscript>\}}\footnotemark[2]\\
    \hline
\begin{minipage}{\textwidth}
\footnotetext[1]{包含在latexsym宏包中}
\footnotetext[2]{包含在amsmath宏包中}
\end{minipage}
\end{longtable}

\begin{table}[H]
\begin{minipage}{\textwidth}
\begin{tabular}{l l l l l l}
	\hline
	符号 & 代码 & 符号 & 代码 & 符号 & 代码\\
	\hline
	( & ( & ) & ) & [ & [\\
	] & ] & $\{$ & \textbackslash \{ & $\}$ & \textbackslash \}\\
	$\lfloor$ & \textbackslash lfloor & $\rfloor$ & \textbackslash rfloor & $\lceil$ & \textbackslash lceil\\
	$\rceil$ & \textbackslash rceil & $\langle$ & \textbackslash langle & $\rangle$ & \textbackslash rangle\\
	$/$ & / & \textbackslash & \textbackslash backslash & $|$ & \textbar\\
	$\|$ & \textbackslash\textbar & $\uparrow$ & \textbackslash uparrow & $\downarrow$ & \textbackslash downarrow\\
	$\updownarrow$ & \textbackslash updownarrow & $\Uparrow$ & \textbackslash Uparrow & $\Downarrow$ & \textbackslash Downarrow\\
	$\Updownarrow$ & \textbackslash  Updownarrow\\
	\hline
\end{tabular}
\footnotetext[99]{在左/右括号前使用\textbackslash  left或\textbackslash  right可使限定符视情况改变大小}
\footnotetext[99]{\textbackslash  left与\textbackslash  right必须成对匹配, 但限定符类型可从集合中任意选取两个}
\footnotetext[99]{当只包含左限定符时, 使用\textbackslash  right.来关闭. 只包含右限定符时,原理类似}
\footnotetext[99]{也可手动调节大小,位置:\textbackslash  big \textbackslash  bigl \textbackslash  bigm \textbackslash  bigr,规格:\textbackslash  big \textbackslash  Big \textbackslash  bigg \textbackslash  Bigg}
\end{minipage}
\caption{公式-括号限定符}
\end{table}

\begin{table}[H]
\begin{minipage}{\textwidth}
\begin{tabular}{l l l l l l l l}
	\hline
	符号 & 代码 & 符号 & 代码 & 符号 & 代码 & 符号 & 代码\\
	\hline
	$\cdot$ & \textbackslash  cdot & $\ldots$ & \textbackslash  ldots\footnotemark[1] & $\cdots$ & \textbackslash  cdots & $\vdots$ & \textbackslash  vdots\footnotemark[1]\\
	$\ddots$ & \textbackslash  ddots\footnotemark[1] & $\iddots$ & \textbackslash  iddots\footnotemark[2] & $\dotsc$ & \textbackslash  dotsc\footnotemark[3] & $\dotsb$ & \textbackslash  dotsb\footnotemark[3]\\
	$\dotsm$ & \textbackslash  dotsm\footnotemark[3] & $\dotsi$ & \textbackslash  dotsi\footnotemark[3] & $\dotso$ & \textbackslash  dotso\footnotemark[3]\\
	\hline
\end{tabular}
\footnotetext[1]{除标注外,其他只能用于math mode}
\footnotetext[2]{包含在mathdots宏包中}
\footnotetext[3]{包含在amsmath宏包中}
\end{minipage}
\caption{公式-省略号}
\end{table}

\begin{table}[H]
\begin{tabular}{l l}
	\hline
	单位 & 说明\\
	\hline
	sp & 65536 sp=1 pt\\
	pt & 1 pt=0.351 mm\\
	bp & 1 bp=0.353 mm$\approx$ 1 pt\\
	dd & 1 dd=0.376 mm=1.07 pt\\
	mm & 1 mm=2.845 pt\\
	ex & 1 ex=当前字体中x的高度\\
	em & 1 em=当前字体尺寸$\approx$ M的宽度\\
	pc & 1 pc=4.218 mm=12 pt\\
	cc & 1 cc =4.513 mm=12 dd=12.84 pt\\
	cm & 1 cm=10 mm=28.453 pt\\
	in & 1 in=25.4 mm=72.27 pt\\
	\hline
\end{tabular}
\caption{通用长度单位}
\end{table}

\begin{table}[H]
\begin{minipage}{\textwidth}
\begin{tabular}{l l l}
	\hline
	类别 & 字体命令 & 输出效果\\\hline
	数学环境的默认字体 & \textbackslash  mathnormal & $\mathnormal{ABCHIJXYZabchijxyz12345}$\\
	斜体 & \textbackslash  mathit & $\mathit{ABCHIJXYZabchijxyz12345}$\\
	粗体 & \textbackslash  mathbf & $\mathbf{ABCHIJXYZabchijxyz12345}$\\
	罗马体 & \textbackslash  mathrm & $\mathrm{ABCHIJXYZabchijxyz12345}$\\
	无衬线体 & \textbackslash  mathsf & $\mathsf{ABCHIJXYZabchijxyz12345}$\\
	打字机体 & \textbackslash  mathtt & $\mathtt{ABCHIJXYZabchijxyz12345}$\\
	手写体(花体)\footnotemark[1] & \textbackslash mathcal & $\mathcal{ABCHIJXYZ}$\\\hline
\end{tabular}
\footnotetext[1]{\LaTeX{}默认只支持大写字母,使用专业字体包可支持小写字母}
\end{minipage}
\caption{LaTeX默认提供的数学字体}
\end{table}

\begin{table}[H]
\begin{minipage}{\textwidth}
\begin{tabular}{l l l}
	\hline
	字体命令 & 输出效果 & 宏包及说明\\\hline
	\textbackslash mathbb & $\mathbb{ABCXYZ}$ & amssymb,仅大写字母\\
	\textbackslash  mathbbm & $\mathbbm{ABCXYZabcxyz12}$ & bbm,数字仅有1和2\\
	\textbackslash  mathscr & $\mathscr{ABCXYZ}$ & mathrsfs,仅大写字母\\
	\textbackslash  EuScript & $\EuScript{ABCXYZ}$ & euscript,仅大写字母\footnotemark[1]\\
	\textbackslash  mathfrak & $\mathfrak{ABCXYZabcxyz123890}$ & amssymb或eufrak\\\hline
\end{tabular}
\footnotetext[1]{已废弃,但使用eucal宏包会覆盖原有的\textbackslash mathcal指令,参考链接:\\https://www.maths.usyd.edu.au/u/SMS/texdoc/euscript.pdf}
\end{minipage}
\caption{其他宏包字体}
\end{table}

\begin{table}[H]
\begin{tabular}{l l l}
	\hline
	标识符 & 符号指令 & 所需宏包\\\hline
	\TeX & \textbackslash  TeX &\\
	\LaTeX & \textbackslash  LaTeX &\\
	\LaTeXe & \textbackslash  LaTeXe &\\
	\AMS & \textbackslash  AMS & texnames\\
	\AMSTeX & \textbackslash  AMSTeX & texnames\\
	\BibTeX & \textbackslash  BibTeX & texnames\\
	\XeTeX & \textbackslash  XeTeX & metalogo\\
	\XeLaTeX & \textbackslash  XeLaTeX & metalogo\\
	\LuaTeX & \textbackslash  LuaTeX & metalogo\\
	\LuaLaTeX & \textbackslash  LuaLaTeX & metalogo\\
	\hline
\end{tabular}
\caption{TeX家族标识符}
\end{table}

\begin{table}[H]
\begin{minipage}{\textwidth}
\begin{tabular}{|r@{\hspace{1ex}}c|r@{\hspace{1ex}}c|r@{\hspace{1ex}}c|r@{\hspace{1ex}}c|r@{\hspace{1ex}}c|r@{\hspace{1ex}}c|r@{\hspace{1ex}}c|r@{\hspace{1ex}}c|}
\hline
32 & \ding{32} & 33 & \ding{33} & 34 & \ding{34} & 35 & \ding{35} & 36 & \ding{36} & 37 & \ding{37} & 38 & \ding{38} & 39 & \ding{39}\\
\hline
40 & \ding{40} & 41 & \ding{41} & 42 & \ding{42} & 43 & \ding{43} & 44 & \ding{44} & 45 & \ding{45} & 46 & \ding{46} & 47 & \ding{47}\\
\hline
48 & \ding{48} & 49 & \ding{49} & 50 & \ding{50} & 51 & \ding{51} & 52 & \ding{52} & 53 & \ding{53} & 54 & \ding{54} & 55 & \ding{55}\\
\hline
56 & \ding{56} & 57 & \ding{57} & 58 & \ding{58} & 59 & \ding{59} & 60 & \ding{60} & 61 & \ding{61} & 62 & \ding{62} & 63 & \ding{63}\\
\hline
64 & \ding{64} & 65 & \ding{65} & 66 & \ding{66} & 67 & \ding{67} & 68 & \ding{68} & 69 & \ding{69} & 70 & \ding{70} & 71 & \ding{71}\\
\hline
72 & \ding{72} & 73 & \ding{73} & 74 & \ding{74} & 75 & \ding{75} & 76 & \ding{76} & 77 & \ding{77} & 78 & \ding{78} & 79 & \ding{79}\\
\hline
80 & \ding{80} & 81 & \ding{81} & 82 & \ding{82} & 83 & \ding{83} & 84 & \ding{84} & 85 & \ding{85} & 86 & \ding{86} & 87 & \ding{87}\\
\hline
88 & \ding{88} & 89 & \ding{89} & 90 & \ding{90} & 91 & \ding{91} & 92 & \ding{92} & 93 & \ding{93} & 94 & \ding{94} & 95 & \ding{95}\\
\hline
96 & \ding{96} & 97 & \ding{97} & 98 & \ding{98} & 99 & \ding{99} & 100 & \ding{100} & 101 & \ding{101} & 102 & \ding{102} & 103 & \ding{103}\\
\hline
104 & \ding{104} & 105 & \ding{105} & 106 & \ding{106} & 107 & \ding{107} & 108 & \ding{108} & 109 & \ding{109} & 110 & \ding{110} & 111 & \ding{111}\\
\hline
112 & \ding{112} & 113 & \ding{113} & 114 & \ding{114} & 115 & \ding{115} & 116 & \ding{116} & 117 & \ding{117} & 118 & \ding{118} & 119 & \ding{119}\\
\hline
120 & \ding{120} & 121 & \ding{121} & 122 & \ding{122} & 123 & \ding{123} & 124 & \ding{124} & 125 & \ding{125} & 126 & \ding{126} & &\\
\hline
& & 161 & \ding{161} & 162 & \ding{162} & 163 & \ding{163} & 164 & \ding{164} & 165 & \ding{165} & 166 & \ding{166} & 167 & \ding{167}\\
\hline
168 & \ding{168} & 169 & \ding{169} & 170 & \ding{170} & 171 & \ding{171} & 172 & \ding{172} & 173 & \ding{173} & 174 & \ding{174} & 175 & \ding{175}\\
\hline
176 & \ding{176} & 177 & \ding{177} & 178 & \ding{178} & 179 & \ding{179} & 180 & \ding{180} & 181 & \ding{181} & 182 & \ding{182} & 183 & \ding{183}\\
\hline
184 & \ding{184} & 185 & \ding{185} & 186 & \ding{186} & 187 & \ding{187} & 188 & \ding{188} & 189 & \ding{189} & 190 & \ding{190} & 191 & \ding{191}\\
\hline
192 & \ding{192} & 193 & \ding{193} & 194 & \ding{194} & 195 & \ding{195} & 196 & \ding{196} & 197 & \ding{197} & 198 & \ding{198} & 199 & \ding{199}\\
\hline
200 & \ding{200} & 201 & \ding{201} & 202 & \ding{202} & 203 & \ding{203} & 204 & \ding{204} & 205 & \ding{205} & 206 & \ding{206} & 207 & \ding{207}\\
\hline
208 & \ding{208} & 209 & \ding{209} & 210 & \ding{210} & 211 & \ding{211} & 212 & \ding{212} & 213 & \ding{213} & 214 & \ding{214} & 215 & \ding{215}\\
\hline
216 & \ding{216} & 217 & \ding{217} & 218 & \ding{218} & 219 & \ding{219} & 220 & \ding{220} & 221 & \ding{221} & 222 & \ding{222} & 223 & \ding{223}\\
\hline
224 & \ding{224} & 225 & \ding{225} & 226 & \ding{226} & 227 & \ding{227} & 228 & \ding{228} & 229 & \ding{229} & 230 & \ding{230} & 231 & \ding{231}\\
\hline
232 & \ding{232} & 233 & \ding{233} & 234 & \ding{234} & 235 & \ding{235} & 236 & \ding{236} & 237 & \ding{237} & 238 & \ding{238} & 239 & \ding{239}\\
\hline
240 & \ding{240} & 241 & \ding{241} & 242 & \ding{242} & 243 & \ding{243} & 244 & \ding{244} & 245 & \ding{245} & 246 & \ding{246} & 247 & \ding{247}\\
\hline
248 & \ding{248} & 249 & \ding{249} & 250 & \ding{250} & 251 & \ding{251} & 252 & \ding{252} & 253 & \ding{253} & 254 & \ding{254} & &\\
\hline
\end{tabular}
\footnotetext[101]{单个字符指令:\quad\textbackslash ding\{num\}}
\footnotetext[102]{特殊label列表环境:\quad\textbackslash begin\{dinglist\}...\textbackslash end\{dinglist\}}
\end{minipage}
\caption{pifont宏包}
\end{table}

\begin{longtable}{c@{\hspace{1ex}}l|c@{\hspace{1ex}}l}
\caption{bbding宏包}\\
\hline
	\ScissorRight & \textbackslash ScissorRight & \ScissorLeft & \textbackslash ScissorLeft\\
	\ScissorRightBrokenTop & \textbackslash ScissorRightBrokenTop & \ScissorLeftBrokenTop & \textbackslash ScissorLeftBrokenTop\\
	\ScissorRightBrokenBottom & \textbackslash ScissorRightBrokenBottom & \ScissorLeftBrokenBottom & \textbackslash ScissorLefttBrokenBottom\\
	\ScissorHollowRight & \textbackslash ScissorHollowRight & \ScissorHollowLeft & \textbackslash ScissorHollowLeft\\
	\hline
	\HandRight & \textbackslash HandRight & \HandLeft & \textbackslash HandLeft\\
	\HandRightUp & \textbackslash HandRightUp & \HandLeftUp & \textbackslash HandLeftUp\\
	\HandCuffRight & \textbackslash HandCuffRight & \HandCuffLeft & \textbackslash HandCuffLeft\\
	\HandCuffRightUp & \textbackslash HandCuffRightUp & \HandCuffLeftUp & \textbackslash HandCuffLeftUp\\
	\HandPencilLeft & \textbackslash HandPencilLeft\\
	\hline
	\PencilRight & \textbackslash PencilRight & \PencilLeft & \textbackslash PencilLeft\\ 
	\PencilRightUp & \textbackslash PencilRightUp & \PencilLeftUp & \textbackslash PencilLeftUp\\
	\PencilRightDown & \textbackslash PencilRightDown & \PencilLeftDown & \textbackslash PencilLeftDown\\
	\NibRight & \textbackslash NibRight & \NibLeft & \textbackslash NibLeft\\
	\NibSolidRight & \textbackslash NibSolidRight & \NibSolidLeft & \textbackslash NibSolidLeft\\
	\hline
	\XSolid & \textbackslash XSolid & \XSolidBold & \textbackslash XSolidBold\\
	\XSolidBrush & \textbackslash XSolidBrush & \Plus & \textbackslash Plus\\
	\PlusOutline & \textbackslash PlusOutline & \PlusCenterOpen & \textbackslash PlusCenterOpen\\
	\PlusThinCenterOpen & \textbackslash PlusThinCenterOpen & \Cross & \textbackslash Cross\\
	\CrossOpenShadow & \textbackslash CrossOpenShadow & \CrossOutline & \textbackslash CrossOutline\\
	\CrossBoldOutline & \textbackslash CrossBoldOutline & \CrossClowerTips & \textbackslash CrossClowerTips\\
	\CrossMaltese & \textbackslash CrossMaltese\\
	\hline
	\DavidStar & \textbackslash DavidStar & \DavidStarSolid & \textbackslash DavidStarSolid\\
	\JackStar & \textbackslash JackStar & \JackStarBold & \textbackslash JackStarBold\\
	\FourStar & \textbackslash FourStar & \FourStarOpen & \textbackslash FourStarOpen\\
	\FiveStar & \textbackslash FiveStar & \FiveStarLines & \textbackslash FiveStarLines\\
	\FiveStarOpen & \textbackslash FiveStarOpen & \FiveStarOpenCircled & \textbackslash FiveStarOpenCircled\\
	\FiveStarCenterOpen & \textbackslash FiveStarCenterOpen & \FiveStarOpenDotted & \textbackslash FiveStarOpenDotted\\
	\FiveStarOutline & \textbackslash FiveStarOutline & \FiveStarOutlineHeavy & \textbackslash FiveStarOutlineHeavy\\
	\FiveStarConvex & \textbackslash FiveStarConvex & \FiveStarShadow & \textbackslash FiveStarShadow\\
	\SixStar & \textbackslash SixStar & \EightStar & \textbackslash EightStar\\
	\EightStarBold & \textbackslash EightStarBold & \EightStarTaper & \textbackslash EightStarTaper\\
	\EightStarConvex & \textbackslash EightStarConvex & \TwelweStar & \textbackslash TwelweStar\\
	\SixteenStarLight & \textbackslash SixteenStarLight & \Asterisk & \textbackslash Asterisk\\
	\AsteriskBold & \textbackslash AsteriskBold & \AsteriskCenterOpen & \textbackslash AsteriskCenterOpen\\
	\AsteriskThin & \textbackslash AsteriskThin & \AsteriskThinCenterOpen & \textbackslash AsteriskThinCenterOpen\\
	\AsteriskRoundedEnds & \textbackslash AsteriskRoundedEnds & \FourAsterisk & \textbackslash FourAsterisk\\
	\EightAsterisk & \textbackslash EightAsterisk\\
	\hline
	\FiveFlowerOpen & \textbackslash FiveFlowerOpen & \FiveFlowerPetal & \textbackslash FiveFlowerPetal\\
	\SixFlowerOpenCenter & \textbackslash SixFlowerOpenCenter & \SixFlowerRemovedOpenPetal & \textbackslash SixFlowerRemovedOpenPetal\\
	\SixFlowerAlternate & \textbackslash SixFlowerAlternate & \SixFlowerAltPetal & \textbackslash SixFlowerAltPetal\\
	\SixFlowerPetalDotted & \textbackslash SixFlowerPetalDotted & \SixFlowerPetalRemoved & \textbackslash SixFlowerPetalRemoved\\
	\EightFlowerPetalRemoved & \textbackslash EightFlowerPetalRemoved & \EightFlowerPetal & \textbackslash EightFlowerPetal\\
	\FourClowerOpen & \textbackslash FourClowerOpen & \FourClowerSolid & \textbackslash FourClowerSolid\\
	\Sparkle & \textbackslash Sparkle & \SparkleBold & \textbackslash SparkleBold\\
	\SnowflakeChevron & \textbackslash SnowflakeChevron & \SnowflakeChevronBold & \textbackslash SnowflakeChevronBold\\
	\Snowflake & \textbackslash Snowflake\\
	\hline
	\CircleSolid & \textbackslash CircleSolid & \CircleShadow & \textbackslash CircleShadow\\
	\HalfCircleRight & \textbackslash HalfCircleRight & \HalfCircleLeft & \textbackslash HalfCircleLeft\\
	\Ellipse & \textbackslash Ellipse & \EllipseSolid & \textbackslash EllipseSolid\\
	\EllipseShadow & \textbackslash EllipseShadow & \Square & \textbackslash Square\\
	\SquareSolid & \textbackslash SquareSolid & \SquareShadowBottomRight & \textbackslash SquareShadowBottomRight\\
	\SquareShadowTopRight & \textbackslash SquareShadowTopRight & \SquareShadowTopLeft & \textbackslash SquareShadowTopLeft\\
	\SquareCastShadowBottomRight & \textbackslash SquareCastShadowBottomRight & \SquareCastShadowTopRight & \textbackslash SquareCastShadowTopRight\\
	\SquareCastShadowTopLeft & \textbackslash SquareCastShadowTopLeft & \TriangleUp & \textbackslash TriangleUp\\
	\TriangleDown & \textbackslash TriangleDown & \DiamondSolid & \textbackslash DiamondSolid\\
	\OrnamentDiamondSolid & \textbackslash OrnamentDiamondSolid & \RectangleThin & \textbackslash RectangleThin\\
	\Rectangle & \textbackslash Rectangle & \RectangleBold & \textbackslash RectangleBold\\
	\hline
	\Phone & \textbackslash Phone & \PhoneHandset & \textbackslash PhoneHandset\\
	\Tape & \textbackslash Tape & \Plane & \textbackslash Plane\\
	\Envelope & \textbackslash Envelope & \Peace & \textbackslash Peace\\
	\Checkmark & \textbackslash Checkmark & \CheckmarkBold & \textbackslash CheckmarkBold\\
	\SunshineOpenCircled & \textbackslash SunshineOpenCircled & \ArrowBoldRightStrobe & \textbackslash ArrowBoldRightStrobe\\
	\ArrowBoldUpRight & \textbackslash ArrowBoldUpRight & \ArrowBoldDownRight & \textbackslash ArrowBoldDownRight\\
	\ArrowBoldRightShort & \textbackslash ArrowBoldRightShort & \ArrowBoldRightCircled & \textbackslash ArrowBoldRightCircled\\
\hline
\end{longtable}
\end{document}

%参考链接: http://tug.ctan.org/info/symbols/comprehensive/symbols-a4.pdf
