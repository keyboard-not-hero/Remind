\documentclass[UTF8,fontset=ubuntu]{ctexart}
\usepackage{chemfig}
\begin{document}
%1.不同连线类型
%  atom sep参数用于调整atom之间中心的间隔
%  double bond sep参数用于调整double\triple线条之间的空隙(默认2pt)
%  bond offset参数用于调整连线端点与atom边缘的距离(默认2pt)
%  #(<dim1>, <dim2>)用于调整当前连线端点与atom的距离, dim1为与前atom的距离, dim2为与后atom的距离
\chemfig{A-B}\qquad \chemfig[atom sep=1pt]{A=B}\qquad \chemfig[double bond sep=5pt]{A~B}\qquad \chemfig[bond offset=1pt]{A-B}\qquad \chemfig{A-#(5pt)B-#(,5pt)C}\\

%2.atom处于math mode
\chemfig{A_1B^2-C^4_3}\\

%3.预定义角度, [n]代表连线逆时针n*45度角(初始为水平x轴)
\chemfig{A-[1]B-[2]C-[-1]D}\\

%4.绝对角度, [:N]代表连线逆时针N度角(初始为水平x轴)
\chemfig{H-[:38]O-[:-38]H}\\

%5.相对角度, [::M]代表相对于前一个连线角度, 再逆时针旋转M度角
\chemfig{A-[:45]B-[::-30]C-[::-15]D}

%6.调整atom中心之间的间隔
%  fixed length=true参数用于指定使用atom左右边缘之间的间隔
%  [<factor_with_degree>, <factor_with_distance]中, factor_with_distance用于指定当前连线两端, atom中心之间距离的系数, 即factor_with_distance * length_of_center_in_atom
\chemfig{A^{++++}-B}\qquad \chemfig[fixed length=true]{A^{++++}-b}\qquad \chemfig{A-[,3]B}

%7.[,,<int1>,<int2>]指定连接线在起始atom的第int1个字符位置, 在结束atom的第int2个字符位置
%  默认(-90,90), 起始atom最后一个字符, 结束atom第一个字符; [90,270], 起始atome第一个字符, 结束atom最后一个字符
\chemfig{ABCD-[:90]EFGHI}\qquad \chemfig{ABCD-[:90,,2,3]EFGHI}\\

%8.调整连接线属性
%  bond styple参数调整后续所有连接线属性
%  [,,,,<tikz line style>]设置当前连接线属性
\chemfig[bond style={line width=2pt,red}]{A-B}\qquad \chemfig{ABC-DEF-[,,,,red,line width=2pt]GHI}\\

%9.bond join=true参数可填补连接线与连接线之间的间隙
\chemfig[bond style={line width=3pt}]{-[1]-[7]}\qquad \chemfig[bond join=true,bond style={line width=3pt}]{-[1]-[7]}

%10.[]放置在第一个atom之前, 则代表后续所有atom的默认属性
\chemfig{[:20,1.5]A-B-C-[:70]D-F-G}\\

%11.分支模式, atom后紧跟()指定分支, ()内第一个[]指定分支内所有atom的默认属性
\chemfig{A-B(-[:90]D-[:20]E)(-[:-90]F-G)-C}\qquad \chemfig{A-B([:30]-D-E)([:-30]-F-G)-C}\\

%12.分支嵌套
\chemfig{A-B([1]-X([2]-Z)-Y)(-[-1]D)-C}\\

%13.复杂分子式步骤
%(1)找出最长的主干
\chemfig{R-C-[::-60]O-[::-60]C-[::-60]R}\qquad
%(2)在主干上添加分支
\chemfig{R-C(=[::+60]O)-[::-60]O-[::-60]C(=[::+60]O)-[::-60]R}\qquad
%(3)由于全部使用相对角度, 可以添加默认值, 进入分子式旋转
\chemfig{[:75]R-C(=[::+60]O)-[::-60]O-[::-60]C(=[::+60]O)-[::-60]R}\\

%14.连接任意两个atom(利用hook符号?)
%(1)不带参数的?, 后续的多个带?的atom, 自动与第一个带?的atom连接
\chemfig{A-B(-[:60]W-X)(-[:-60]Y-Z)-C}\qquad \chemfig{A-B(-[:60]W-X?)(-[:-60]Y-Z?)-C?}\\
%(2)带参数的?, ?[<name>,<bond_type>,<tike_line_style>]
\chemfig{A?[a]-B(-[:60]W?[a,2,{line width=2pt},red]-X?[b])(-[:-60]Y-Z?[b,1,blue])-C?[b,3,green]}\\
%(3)后续atom出现多条连接线
\chemfig{A-[:-72]B-C-[:72]D-[:144]E}\qquad \chemfig{A?[a]-[:-72]B-C?[a]?[b]-[:72]D-[:144]E?[a]?[b]}\\

%15.多边形, <atom>*<n>(<code>), n代表多边形的边数量
\chemfig{A*3(-B-C-)}\qquad \chemfig{A*4(-B-C-D-)}\qquad \chemfig{A*5(-B=C-D=E-)}\qquad \chemfig{*5(-=-=-)}\qquad \chemfig{A*5(-B=C-D)}\\

%16.多边形内的圆弧, <atom>**[<start_dgress>,<end_dgress>,<tike_line_style>]<n>(<code>), start_angle/end_angle分别代表圆弧的起始角度和结束角度
\chemfig{**[0,360,dash pattern=on 2pt off 2pt]6(------)}\\

%17.旋转多边形
\chemfig{A*6(------)}\qquad \chemfig{[:30]A*6(------)}\qquad \chemfig{[:60]A*6(------)}\qquad \chemfig{A=[:-30]*6(=-=-=-)}\\

%18.多边形顶点的分支, 默认角度不为0, 为外角的平分线
\chemfig{X*6(-=-(-A-B=C)=-=-)}\qquad \chemfig{*5((-A=B-C)-(-D-E)-(=)-(-F)-(-G=)-)}\qquad \chemfig{*5(---([:90]-A-B)--)}\qquad \chemfig{*5(---([::0]-A-B)--)}\\

%19.连接多边形与粘合多边形的边
\chemfig{*6(-(-*5(-----))-----)}\qquad \chemfig{*6(-*5(----)-----)}

%20.粘合多边形的优化
\chemfig{A-B*5(-C-D*5(-X-Y-Z-)-E-F-)}\qquad \chemfig{A-B*5(-C-D*5(-X-Y-Z?)-E?-F-)}\qquad \chemfig{A-B*5(-C-D*5(-X-Y-Z-\phantom{E})-E-F-)}\\

%21.多边形的顶点atom包含多个字母的优化
\chemfig{AB*5(-CDE-F-GH-I-)}\qquad \chemfig{AB*5(-CDE-[,,1]F-[,,,1]GH-I-)}\\
\end{document}
