\documentclass[UTF8,fontset=ubuntu]{ctexart}
\usepackage{makecell}
\usepackage{multirow}
\begin{document}
\begin{table}
\centering
\begin{tabular}{|c|c|c|}
    \hline
    \makecell{Channel ID\\(20 MHz)} & $F_c$(MHz) & \makecell{Frequency\\ range\\ (MHz)}\\
    \hline
    32 & 5160 & 5150-5170 \\
    \hline
    36 & 5180 & 5170-5190 \\
    \hline
    40 & 5200 & 5190-5210 \\
    \hline
    44 & 5220 & 5210-5230 \\
    \hline
\end{tabular}
\caption{手动单元格内容断行}
\end{table}

\begin{table}
\centering
\begin{tabular}{|l|l|l|l|}
    \hline
    Adopted & \makecell[t]{IEEE\\ standard\_name} & \makecell{Maximum\\ link rate\\(Mbit/s)} & \makecell[b]{Radio\\ frequency\\ (GHz)} \\
    \hline
    1999 & \makecell[l]{802.11b} & \makecell{1-11} & \makecell[r]{2.4} \\
    \hline
\end{tabular}
\caption{makecell的水平和垂直对齐方式}
\end{table}

\begin{table}
\centering
\begin{tabular}{|l|l|}
\hline
1 & \multirowcell{3}[-2ex][r]{1.率性而为, 身随心动\\ 2.无为而治}\\
\cline{1-1}
2 & \\
\cline{1-1}
3 & \\
\hline
\end{tabular}
\caption{示例}
\end{table}
\end{document}
