\documentclass[UTF8,fontset=ubuntu]{ctexart}
\usepackage{multirow}
\usepackage{bigstrut}
\usepackage{array}
\renewcommand{\multirowsetup}{\centering}
\begin{document}
\begin{table}
\centering
\begin{tabular}{|c|c|c|c|}
    \hline
    Code & Subcode\\
    \hline
    \multirow{3}{*}{0} & 0\\
    \cline{2-2}
     & 1\\
    \cline{2-2}
     & 2\\
    \hline
     & 0\\
    \cline{2-2}
    \multirow{-2}*{1} & 1\\
    \hline
\end{tabular}
\caption{合并行}
\end{table}

\begin{table}
\centering
\begin{tabular}{|c|c|>{\centering\arraybackslash}p{1.6cm}|>{\centering}p{1.6cm}|c|c|c|c|c|c|}
\hline
\multirow{2}[b 8]{*}{CHNL\_ID} & \multirow{2}{*}{$F_c$(MHz)} & DSSS & OFDM & \multirow{2}{*}{FCC} & \multirow{2}{*}{IC} & \multirow{2}{*}{ETSI} & \multirow{2}{*}{Spain} & \multirow{2}{*}{France} & \multirow{2}{*}{MKK}\\
\cline{3-4}
    & & Frequency (MHz) & Frequency (MHz) & & & & & &\\
\hline
\end{tabular}
\caption{合并行的垂直位置}
\end{table}

\begin{table}
\centering
\begin{tabular}{|c|c|c|c|c|}
\hline
\multicolumn{2}{|c|}{\multirow{2}*{各年级不同科目平均分}} & \multicolumn{3}{c|}{科目}\\
\cline{3-5}
 \multicolumn{2}{|c|}{}  & 语文 & 数学 & 英语\\
\hline
\multirow{3}*{年级} & 一年级 & 85 & 93 & 96\\
 & 二年级 & 83 & 95 & 86\\
 & 三年级 & 87 & 78 & 92\\
\hline
\end{tabular}
\caption{合并行与列}
\end{table}
\end{document}
