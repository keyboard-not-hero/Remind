\documentclass[UTF8,fontset=ubuntu]{ctexart}
\usepackage{multirow}
\usepackage{bigstrut}
\usepackage{array}
\usepackage{makecell}
\usepackage{color}
\usepackage{colortbl}
\renewcommand{\multirowsetup}{\centering}
\begin{document}
\begin{table}
\centering
\begin{tabular}{|c|c|c|c|}
    \hline
    Code & Subcode\\
    \hline
    \multirow{3}{*}{0} & 0\\
    \cline{2-2}
     & 1\\
    \cline{2-2}
     & 2\\
    \hline
     & 0\\
    \cline{2-2}
    \multirow{-2}*{1} & 1\\
    \hline
\end{tabular}
\caption{合并行}
\end{table}

\begin{table}
\centering
\begin{tabular}{|c|c|>{\centering\arraybackslash}p{1.6cm}|>{\centering}p{1.6cm}|c|c|c|c|c|c|}
\hline
\multirow{2}[b 8]{*}{CHNL\_ID} & \multirow{2}{*}{$F_c$(MHz)} & DSSS & OFDM & \multirow{2}{*}{FCC} & \multirow{2}{*}{IC} & \multirow{2}{*}{ETSI} & \multirow{2}{*}{Spain} & \multirow{2}{*}{France} & \multirow{2}{*}{MKK}\\
\cline{3-4}
    & & Frequency (MHz) & Frequency (MHz) & & & & & &\\
\hline
\end{tabular}
\caption{合并行的垂直位置}
\end{table}

\begin{table}
\centering
\begin{tabular}{|c|c|c|c|c|}
\hline
\multicolumn{2}{|c|}{\multirow{2}*{各年级不同科目平均分}} & \multicolumn{3}{c|}{科目}\\
\cline{3-5}
 \multicolumn{2}{|c|}{}  & 语文 & 数学 & 英语\\
\hline
\multirow{3}*{年级} & 一年级 & 85 & 93 & 96\\
 & 二年级 & 83 & 95 & 86\\
 & 三年级 & 87 & 78 & 92\\
\hline
\end{tabular}
\caption{合并行与列}
\end{table}

\begin{table}
\centering
\begin{tabular}{|c|c|c|c|c|}
    \hline
    \makecell{IEEE\\ standard} & Adopted & Generation & \makecell{Maximum\\ link rate\\(Mbit/s)} & \makecell{Radio\\ frequency\\ (GHz)}\\
    \hline
    802.11b & 1999 & Wi-Fi 1 & 1-11 & 2.4\\
    \hline
    802.11a & 1999 & Wi-Fi 2 & 6-54 & 5\\
    \hline
    802.11g & 2003 & Wi-Fi 3 & 6-54 & 2.4\\
    \hline
    802.11n & 2008 & Wi-Fi 4 & 72-600 & 2.4, 5\\
    \hline
    802.11ac & 2014 & Wi-Fi 5 & 433-6933 & 5\\
    \hline
    \multirow{2}{*}{802.11ax} & 2019 & Wi-Fi 6 & \multirow{2}{*}{574-9608} & 2.4, 5\\
    \cline{2-3}
    \cline{5-5}
    & 2020 & Wi-Fi 6E & & 6\\
    \hline
    802.11be & \cellcolor[gray]{0.8}2024 & Wi-Fi 7 & 1376-46120 & 2.4, 5, 6\\
    \hline
\end{tabular}
\caption{\\cline的应用}
\end{table}

\begin{table}[H]
\begin{tabular}{|c|c|l|}
\hline
    反应物类别 & 反应物条件 & 生成物条件\\
\hline
    盐+盐 & 二者皆可溶 & \multirowcell{5}{至少具备以下条件:\\ 1.有沉淀生成\\ 2.>有气体生成\\ 3.有水生成}\\
    \cline{1-2}
    盐+碱 & 二者皆可溶 &\\
    \cline{1-2}
    盐+酸 & 盐可溶或难溶, 酸可溶 &\\
    \cline{1-2}
    酸+碱 & 其中一种可溶 &\\
    \cline{1-2}
    酸+金属氧化物 & 酸可溶 &\\
\hline
\end{tabular}
\caption{合并行内指定位置换行}
\end{table}
\end{document}
