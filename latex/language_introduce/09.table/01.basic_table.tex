\documentclass[UTF8,fontset=ubuntu]{ctexart}
\usepackage{array}
\begin{document}
\begin{table}
\centering
\begin{tabular}{|l|c|r||c|}
    \hline
    name & sex & item & score\\
    \hline
    peter & male & math & 99\\
    \hline
    david & male & english & 76\\
    \hline
    steven & male & english & 59\\
    \hline
\end{tabular}
\caption{列对齐方式与基础分隔符}
\end{table}

\begin{table}
\centering
\begin{tabular}{|@{i}c|!{i}c|@{\hspace{2pt}}c|!{\hspace{2pt}}c|}
    \hline
    name & sex & item & score\\
    \hline
    peter & male & math & 99\\
    \hline
    david & male & english & 76\\
    \hline
    steven & male & english & 59\\
    \hline
\end{tabular}

\begin{tabular}[t]{|@{\extracolsep{1cm}}c|c|!{\extracolsep{1cm}}c|c|}
    \hline
    name & sex & item & score\\
    \hline
    peter & male & math & 99\\
    \hline
    david & male & english & 76\\
    \hline
    steven & male & english & 59\\
    \hline
\end{tabular}
\caption{在表格中填充额外内容}
\end{table}

\begin{table}
\setlength{\tabcolsep}{1cm}
\centering
\begin{tabular}{|*{4}{c|}}
    \hline
    name & sex & item & score\\
    \hline
    peter & male & math & 99\\
    \hline
    david & male & english & 76\\
    \hline
    steven & male & english & 59\\
    \hline
\end{tabular}
\caption{格式重复的列与分隔符}
\end{table}

\begin{table}
\centering
\begin{tabular}{|p{1.2cm}|m{2cm}|b{2cm}|c|}
    \hline
    channel id & frequency (center)& Frequency range (MHz) & Frequency range (MHz)\\
    \hline
    1 & 2412 & 2401–2423 & 2402–2422\\
    \hline
    2 & 2417 & 2406–2428 & 2407–2427\\
    \hline
    3 & 2422 & 2411–2433 & 2412–2432\\
    \hline
\end{tabular}
\caption{指定列宽度, 并自动换行}
\end{table}

\begin{table}
\centering
\begin{tabular}{|>{\centering\arraybackslash}m{2cm}|m{2cm}|m{2cm}|m{2cm}<{\raggedleft\arraybackslash}|}
    \hline
    channel id & frequency (center)& Frequency range (MHz) & Frequency range (MHz)\\
    \hline
    1 & 2412 & 2401–2423 & 2402–2422\\
    \hline
    2 & 2417 & 2406–2428 & 2407–2427\\
    \hline
    3 & 2422 & 2411–2433 & 2412–2432\\
    \hline
\end{tabular}
\caption{使用指定列宽后, 修改列内容水平对齐方向}
\end{table}

\begin{table}
\centering
\begin{tabular}{|c|c|c|}
\hline
    Hello World
    \begin{tabular}[c]{|c|c|}
    \hline
        A & B\\
        \hline
        C & D\\
    \hline
    \end{tabular} &
    Hello World
    \begin{tabular}[t]{|c|c|}
    \hline
        A & B\\
    \hline
        C & D\\
    \hline
    \end{tabular} &
        Hello World
    \begin{tabular}[b]{|c|c|}
    \hline
        A & B\\
    \hline
        C & D\\
    \hline
    \end{tabular}\\
\hline
     &
    Hello World
    \begin{tabular}[t]{|c|c|}
    \firsthline
        A & B\\
    \hline
        C & D\\
    \lasthline
    \end{tabular} &
        Hello World
    \begin{tabular}[b]{|c|c|}
    \hline
        A & B\\
    \hline
        C & D\\
    \lasthline
    \end{tabular}\\
\hline
\end{tabular}
\caption{列表与外部行的垂直对齐方式}
\end{table}
\end{document}

