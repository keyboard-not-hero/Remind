\chapter{组合分析}
\section{计数基本法则}
{\par\centering
\framebox{\begin{minipage}{\textwidth}
{\par\centering\textbf{计数基本法则}\par}
假设有两个试验,其中试验1有$m$种可能的结果,对应于试验1的每一个结果,试验2有$n$种可能的结果,则这两个试验一共有$mn$种可能的结果
\end{minipage}}
\par}\vspace{4ex}

{\par\centering
\framebox{\begin{minipage}{\textwidth}
{\par\centering\textbf{推广的计数基本法则}\par}
假设一共有$r$个试验. 试验1有$n_1$种可能的结果; 对应于试验1的每一种可能的结果, 试验2有$n_2$种可能的结果; 对应于前两个试验的每一种可能的结果, 试验3有$n_3$种可能的结果$\cdots$那么这$r$个试验一共有$n_1n_2\cdots n_r$种可能的结果
\end{minipage}}
\par}\vspace{1ex}

例.\ 一个大学计划委员会由3名新生、4名二年级学生、5名三年级学生、2名毕业班学生组成,现在要从中选4个人组成一个分委员会,要求来自不同的年级,一共有多少种选择方式?\\
解:\ 每个年级选取一个学生为一个试验单位,所以,共有$3\times 4\times 5\times 2$种选择方式\\[2ex]

\section{排列}
\textbf{排列}:将不同的物件或符号根据不同顺序进行安置,每个顺序都称为一个排列\\[2ex]

{\par\centering
\framebox{\begin{minipage}{\textwidth}
假设有$n$个不同元素,将其进行排列,一共有
\[n(n-1)(n-2)\cdots 3\cdot 2\cdot 1=n!\]
种不同排列方式
\end{minipage}}
\par}\vspace{2ex}

例1.\ 某班级共有6名男生、4名女生,有次测验是根据他们的表现来排名次,假设没有两个学生成绩一样.\\
(a)\ 一共有多少种排名方式?\\
(b)\ 如限定男生、女生分开排名,一共有多少种排名的方式?\\
解:\\
(a)\ 将所有不同成绩进行排名,一共有$10!=3 628 800$种排列方式\\
(b)\ 将男生和女生进行分开排名,男生有$6!=720$种排列方式,女生有$4!=24$种排列方式,所以一共有$6!\times 4!=720\times 24=17280$种排列方式\\[1ex]

例2.\ 把10本书放到书架上,其中有4本数学书、3本化学书、2本历史书和1本语文书. 现在要求相同类别的书必须紧挨着放,问一共有多少种方法?\\
解:\ 四种书籍其内部的排列为$4!\times 3!\times 2!\times 1!=288$种排列方式,不同书籍之间的排列为$4!=24$中排列方式,所以,一共有$4!\times 4!\times 3!\times 2!\times 1!=24\times 288=6912$种排列方式\\[2ex]

{\par\centering
\framebox{\begin{minipage}{\textwidth}
假设有$n$个元素,如果其中$n_1$个元素彼此相同,另$n_2$个彼此相同,$\cdots$,$n_r$个也彼此相同,那么一共有
\[\dfrac{n!}{n_1!n_2!\cdots n_r!}\]
种不同的排列方式
\end{minipage}}
\par}\vspace{2ex}

例.\ 有9面小旗排列在一条直线上吗,其中4面白色、3面红色和2面蓝色,颜色相同的旗是一样的. 如果不同的排列方式代表不同的信号,那么一共有多少可能的信号?\\
解:\ 一共有$\dfrac{9!}{4!3!2!}=1260$种不同的信号\\[2ex]

\section{组合}
\textbf{组合}:从$n$个元素中抽取$m$个元素,并且不考虑抽取顺序\\[2ex]

{\par\centering
\framebox{\begin{minipage}{\textwidth}
{\par\centering\textbf{记号与术语}\par}
对$r\leqslant n$,我们定义$\binom{n}{r}$如下:
\[\binom{n}{r}=\frac{n!}{(n-r)!r!}\]
并且说$\binom{n}{r}$表示了从$n$个元素中一次取$r$个的可能组合数
\end{minipage}}
\par}\vspace{2ex}

例1.\ 从20人当中选择3人组成委员会,一共有多少中选法?\\
解:\ 一共有$\binom{20}{3}=d\frac{20\times 19\times 18}{3\times 2\times 1}=1140$种选法\\[1ex]

例2.\ 有个12人组成的团体,其中5位女士,7位男士,现从中选取2位女士,3位男士组成一个委员会.
(1)\ 问有多少种取法?
(2)\ 另外,如果其中有2位男士之间有矛盾,并且坚决拒绝一起工作,那又有多少中取法?\\
解:\\
(1)\ 有$\binom{5}{2}\binom{7}{3}=\dfrac{5\times 4}{2\times 1}\times\dfrac{7\times 6\times 5}{3\times 2\times 1}=350$种取法\\
(2)\ 男士一共有$\binom{7}{3}=35$种取法,选中两个有矛盾男士有$\binom{2}{2}\binom{5}{1}=5$种取法,所以排除同时选中两位有矛盾男士的取法: $35-5=30$; 另外,选取女士的方法为$\binom{5}{2}=10$种,所以,一共有$30\times 10=300$种取法\\[1ex]

例3.\ 假设在一排$n$个天线中,有$m$个是失效的,另$n-m$个是有效的,并且假设所有有效的天线之间不可区分,同样,所有失效的天线之间也不可区分. 问有多少种线性排列方式,使得任何两个失效的天线都不相邻?\\
解:\ 先将$n-m$个有效天线放置好,在两个有效天线之间(或最左/右侧有效天线的左/右边)的$n-m+1$个位置上,每个位置只能放置一个失效天线,即从$n-m+1$位置上,选择$m$个放置失效天线,所以,有$\binom{n-m+1}{m}$种排列方式\\[2ex]

组合恒等式
\[\binom{n}{r}=\binom{n-1}{r-1}+\binom{n-1}{r}\qquad 1\leqslant r\leqslant n\]
原理:\\
括号左边:\\
从$n$个元素中抽取$r$个元素的组合
\[\binom{n}{r}\]
括号右边:\\
将其中一个元素视为特殊元素,包含以下两种情况:\\
(1)包含该特殊元素,从余下的$n-1$个元素中再抽取$r-1$个元素
\[\binom{n-1}{r-1}\]
(2)不包含特殊元素,从余下的$n-1$个元素中抽取$r$个元素
\[\binom{n-1}{r}\]\\[2ex]

{\par\centering
\framebox{\begin{minipage}{\textwidth}
{\par\centering\textbf{二项式定理}\par}
\[(x+y)^n=\sum_{k=0}^n\binom{n}{k}x^ky^{n-k}\]
\end{minipage}}
\par}\vspace{2ex}

例.\ 展开$(x+y)^3$\\
解:\\
$\begin{array}{r l}
(x+y)^3 & =\binom{3}{0}x^0y^3+\binom{3}{1}x^1y^2+\binom{3}{2}x^2y^1+\binom{3}{3}x^3y^0\\
& =y^3+3xy^2+3x^2y+x^31
\end{array}$

\section{多项式系数}
{\par\centering
\framebox{\begin{minipage}{\textwidth}
{\par\centering\textbf{记号}\par}
如果$n_1+n_2+\cdots+n_r=n$,则定义$\binom{n}{n_1,n_2,\cdots,n_r}$为
\[\binom{n}{n_1,n_2,\cdots,n_r}=\frac{n!}{n_1!n_2!\cdots n_r!}\]
因此,$\binom{n}{n_1,n_2,\cdots,n_r}$表示把$n$个不同的元素分成大小分别为$n_1,n_2,\cdots,n_r$的$r$个不同组的组合数
\end{minipage}}
\par}\vspace{2ex}

例1.\ 将10个小孩平均分成A,B两队分别去参加两场不同的比赛,一共有多少种分法?\\
解:\ 一共有$\displaystyle\binom{10}{5,5}=\frac{10!}{5!5!}=252$种分法\\[1ex]

例2.\ 把10个孩子平均分成两组进行篮球比赛,一共有多少种分法?\\
解:\ 相对于例1,本例不注重组间次序,所以,一共有$\displaystyle\frac{\frac{10!}{5!5!}}{2!}=126$种分法\\[2ex]

{\par\centering
\framebox{\begin{minipage}{\textwidth}
{\par\centering\textbf{多项式定理}\par}
\[(x_1+x_2+\cdots+x_r)^n=\sum_{\substack{(n_1,\cdots,n_r):\\n_1+\cdots+n_r=n}}\binom{n}{n_1,n_2,\cdots,n_r}x_1^{n_1}x_2^{n_2}\cdots x_r^{n_r}\]
上式的求和号是对满足$n_1+n_2+\cdots+n_r=n$的所有非负整数向量$(n_1,n_2,\cdots,n_r)$求和
\end{minipage}}
\par}\vspace{2ex}

例.\ 在第一轮淘汰赛中有$n=2^m$名选手,这$n$名选手被分成$n/2$组,每组内都要进行比赛. 每一场比赛的败者将被淘汰而胜者将晋级下一轮,这个过程持续到只有一名选手留下. 假设我们有一场淘汰赛,其中有8名选手.\\
(a)\ 第一轮之后有多少种可能的结果?(如一种结果是1赢了2,3赢了4,5赢了6,7赢了8)\\
(b)\ 这场淘汰赛有多少种可能的结果,其中每个结果是否包含了所有轮次的完整信息?\\
解:\\
(a)\ 先将8名选手分为4组,分法为$\binom{8}{2,2,2,2}=\frac{8!}{2^4}$;\\
当前我们不将分组编组序,有$\binom{8}{2,2,2,2}/4!=8!/(2^4\times 4!)$种可能;\\
最后,进行比赛时,每组有2种可能,所以,总的有$8!\times 2^4/(2^4\times 4!)=8!/4!$种可能\\
(b)\ 以此类推,4进2半决赛有$4!/2!$种可能,2进1决赛有$2!/1!$种可能,所以,最终结果有$(8!/4!)\times(4!/2!)\times(2!/1!)=8!$种可能\\[2ex]

\section{方程的整数解个数}
\begin{Theorem}
共有$\dbinom{n-1}{r-1}$个不同的正整数向量$(x_1,x_2,\cdots,x_r)$满足
\[x_1+x_2+\cdots+x_r=n\qquad x_i>0,i=1,\cdots,r\]
\end{Theorem}\vspace{1ex}
原理:如下\\
{\par\centering
\framebox{\begin{minipage}{4cm}
{\par\centering$\underbrace{1|1|1|\cdots |1|1}_{n\text{个}1}$\par}
\end{minipage}}
\par}
1.\ $n$个$1$代表总和为$n$;\\
2.\ $|$代表切割数字的位置,使得$x_1$等于第一个切割位置之前的数字总和,$x_2$等于第一个切割位置和第二个切割位置之间的数字总和,$\cdots$,$x_r$等于最后一个切割位置之后的数字总和;\\
3.\ 从$n-1$个间隔中,插入$r-1$切割符,使数字总共划分为$r$份.\\[2ex]

\begin{Theorem}\label{chap-one:01}
共有$\dbinom{n+r-1}{r-1}$个不同的非负整数向量$(x_1,x_2,\cdots,x_r)$满足
\[x_1+x_2+\cdots+x_r=n\]
\end{Theorem}\vspace{4ex}
原理:\\
因为$x_1+x_2+\cdots+x_r=n$的非负整数解个数与$y_1+y_2+\cdots+y_r=n+r$的正整数解个数是相同的(令$y_i=x_i+1,i=1,\cdots,r$)\\[1ex]

例1.\ 一位投资者有2万美元可投资到4个项目上,且每种投资必须是1000美元的整数倍.\\
(1)如果要求将2万美元全部投资,一共有多少种可行的投资方法?\\
(2)如果不要求将钱全部投资呢?\\
解:\\
(1)根据已知和问题,可得如下方程:\\
$x_1+x_2+x_3+x_4=20$\\
根据命题\ref{chap-one:01},得:\\[1ex]
$\dbinom{20+4-1}{4-1}=1771$\\
所以,一共有1771种可能的投资方式\\
(2)可以将未投资的钱视为第五个"项目",可得如下方程:\\
$x_1+x_2+x_3+x_4+x_5=20$\\
根据命题\ref{chap-one:01},得:\\[1ex]
$\dbinom{20+5-1}{5-1}=10626$\\
所以,一共有10626种可能的投资方式\\[1ex]

例2.\ 在$(x_1+x_2+\cdots+x_r)^n$的展开式中,一共有多少项?\\
解:
\[(x_1+x_2+\cdots+x_r)^n=\sum\binom{n}{n_1,\cdots,n_r}x_1^{n_1}\cdots x_r^{n_r}\]
满足方程$n_1+n_2+\cdots+n_r=n$\\
根据命题\ref{chap-one:01},得:\\[1ex]
$\dbinom{n+r-1}{r-1}$\\
所以,一共有$\dbinom{n+r-1}{r-1}$项\\[2ex]

如下图.\\
\begin{tikzpicture}
\draw (0,0) grid[step=1] (4,3);
\draw[fill=black] (0,0) circle[radius=0.6mm] node[below left]{$A$};
\draw[fill=black] (4,3) circle[radius=0.6mm] node[above right]{$B$};
\end{tikzpicture}
从位置A到达位置B的方式有多少种.\\
假设从起始位置到目标位置需要水平方向移动$n$个单元,垂直方向移动$m$个单元,路径到达方式满足以下公式:
\[\binom{n+m}{n}\]
所以,当前解为:\\
\[\binom{4+3}{4}=7\times 5=35\]
并且,由于组合满足以下公式:\\
\[\binom{n}{r}=\binom{n-1}{r-1}+\binom{n-1}{r}\qquad 1\leqslant r\leqslant n\]
所以,该移动路径满足如下:\\
\[R_nC_m=R_{n-1}C_m+R_nC_{m-1}\]
