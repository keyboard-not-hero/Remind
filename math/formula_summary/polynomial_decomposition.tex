\documentclass[UTF8,fontset=ubuntu]{ctexart}
\usepackage{parskip}
\usepackage{array}
\usepackage{amsmath}
\begin{document}
1.二项式定理:
\[\begin{array}{>{\displaystyle}r >{\displaystyle}l}
(x+y)^n & =\binom{n}{0}x^0y^n+\binom{n}{1}x^1y^{n-1}+\cdots+\binom{n}{n-1}x^{n-1}y^1+\binom{n}{n}x^ny^0\\
& =\sum_{k=0}^{n}\binom{n}{k}x^ky^{n-k}
\end{array}\]\\[2ex]

2.多项式因式分解:\\
(1)将多项式组合为$ax^n+bx^{n-1}+\cdots+cx+d=0$格式方程\\
(2)找出常数项的因子\\
(3)找出符合(1)中方程的因子(包含因子相反数)\\
(4)根据该因子进行拆分匹配(多个因子符合时,任选其一)\\[1ex]

例. 对$x^3+3x^2-4$进行因式分解\\
(1)组合方程\\
$x^3+3x^2-4=0$\\
(2)找出常数项的因子\\
1/2/4\\
(3)找出符合方程$x^3+3x^2-4$的因子(包含因子相反数)\\
$x=1\ \Rightarrow\ 1^3+3\times 1^2-4=0$\\
$x=-2\ \Rightarrow\ (-2)^3+3\times(-2)^2-4=0$\\
(4)根据该因子进行拆分匹配\\
$\begin{array}{r l}
x^3+3x^2-4 & =x^3+2x^2+x^2+2x-2x-4\\
& =x^2(x+2)+x(x+2)-2(x+2)\\
& =(x^2+x-2)(x+2)\\
& =(x-1)(x+2)^2
\end{array}$
\end{document}
