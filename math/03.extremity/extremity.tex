\documentclass[UTF8,fontset=ubuntu]{ctexart}
\usepackage{ntheorem}
\begin{document}
\noindent名词解析\par
左极限:当定义域$x$从左边非常接近于a,但不等于a时,函数$f(x)$接近的数。与$f(a)$的值无关。公式如下:$\lim_{x\to a^-}f(x)=L$

右极限:当定义域$x$从右边非常接近于a,但不等于a时,函数$f(x)$接近的数。与$f(a)$的值无关。公式如下:$\lim_{x\to a^+}f(x)=L$

双侧极限:当$f(x)$在x轴上关于某具体点的左右极限同时存在,并且相等,则该点的极限存在。公式如下:$\lim_{x\to a}f(x)=L$

\theoremstyle{nonumberplain}
\newtheorem{theorem}{夹逼定理}
\begin{theorem}
	如果一个函数$f$被夹在函数$g$和$h$之间,当$x\rightarrow a$时,这两个函数$g$和$h$都收敛于同一个极限L,那么当$x\rightarrow a$时,$f$也收敛于极限L
\end{theorem}
\end{document}
