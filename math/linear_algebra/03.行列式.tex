\chapter{行列式}
\section{行列式介绍}
有$3\times 3$矩阵$A$
\[\left[\begin{array}{l l l}
a_{11} & a_{12} & a_{13}\\
a_{21} & a_{22} & a_{23}\\
a_{31} & a_{32} & a_{33}
\end{array}\right]\]
其中
\[\Delta=a_{11}a_{22}a_{33}+a_{12}a_{23}a_{31}+a_{13}a_{21}a_{32}-a_{11}a_{23}a_{32}-a_{12}a_{21}a_{33}-a_{13}a_{22}a_{31}\]
$\Delta$称为$3\times 3$矩阵$A$的\textbf{行列式}, 也可以写成\\
$\Delta=(a_{11}a_{22}a_{33}-a_{11}a_{23}a_{32})-(a_{12}a_{21}a_{33}-a_{12}a_{23}a_{31})+(a_{13}a_{21}a_{32}-a_{13}a_{22}a_{31})$\\[1ex]
$\phantom{\Delta}=a_{11}\cdot\det\left[\begin{array}{l l}
a_{22} & a_{23}\\
a_{32} & a_{33}
\end{array}\right]-a_{12}\cdot\det\left[\begin{array}{l l}
a_{21} & a_{23}\\
a_{31} & a_{33}
\end{array}\right]+a_{13}\cdot\det\left[\begin{array}{l l}
a_{21} & a_{22}\\
a_{31} & a_{32}
\end{array}\right]$\\[1ex]
$\phantom{\Delta}=a_{11}\cdot\det A_{11}-a_{12}\cdot\det A_{12}+a_{13}\cdot\det A_{13}$\\
其中, $A_{ij}$表示去除矩阵第$i$行和第$j$列元素后的内容.\\
例. $A_{11}$表示如下:\\
\[\left[\begin{array}{l l l}
\cancel{a_{11}} & \cancel{a_{12}} & \cancel{a_{13}}\\
\cancel{a_{21}} & a_{22} & a_{23}\\
\cancel{a_{31}} & a_{32} & a_{33}
\end{array}\right]\]
即
\[\left[\begin{array}{l l}
a_{22} & a_{23}\\
a_{32} & a_{33}
\end{array}\right]\]\\[2ex]

\begin{definition}
当$n\geqslant 2$, $n\times n$矩阵$A=[a_{ij}]$的行列式是形如$\pm a_{1j}\det A_{1j}$的$n$个项的和, 其中加号和减号交替出现, 这里元素$a_{11}$,$a_{12}$,$\cdots$,$a_{1n}$来自$A$的第一行, 即
\[\begin{array}{l}
\det A=a_{11}\cdot\det A_{11}-a_{12}\cdot\det A_{12}+\cdots+(-1)^{1+n}a_{1n}\cdot\det A_{1n}\\
\phantom{\det A}=\displaystyle\sum_{j=1}^n(-1)^{1+j}a_{1j}\det A_{1j}
\end{array}\]
\end{definition}\vspace{4ex}

给定$A=[a_{ij}]$, $A$的$(i,j)$\textbf{余因子}$C_{ij}$由下式给出
\[C_{ij}=(-1)^{i+j}\det A_{ij}\]
则
\[\det A=a_{11}\cdot C_{11}+a_{12}\cdot C_{12}+\cdots+a_{1n}\cdot C_{1n}\]
上述公式称为按$A$的\textbf{第一行的余因子展开式}\\[2ex]

\begin{TheoremOne}
$n\times n$矩阵A的行列式可按任意行或列的余因子展开式来计算. 按第i行的余因子展开式为:
\[\det A=a_{i1}C_{i1}+a_{i2}C_{i2}+\cdots+a_{in}C_{in}\]
按第j列的余因子展开式为:
\[\det A=a_{1j}C_{1j}+a_{2j}C_{2j}+\cdots+a_{nj}C_{nj}\]
\end{TheoremOne}\vspace{4ex}

\begin{TheoremOne}
若A为三角阵, 则$\det A$等于A的主对角线上元素的乘积
\end{TheoremOne}\vspace{8ex}

\section{行列式的性质}
\begin{TheoremTwo}[行变换]
令A是一个方阵.\\
a.若A的某一行的倍数加到另一行得矩阵B, 则$\det B=\det A$\\
b.若A的两行互换得矩阵B, 则$\det B=-\det A$\\
c.若A的某行乘以k倍得到矩阵B, 则$\det B=k\det A$
\end{TheoremTwo}\vspace{4ex}

\begin{TheoremOne}
方阵A是可逆的当且仅当$\det A\neq0$
\end{TheoremOne}\vspace{4ex}

\begin{TheoremOne}
若A为一个$n\times n$矩阵, 则$\det A^T=\det A$
\end{TheoremOne}\vspace{4ex}

\begin{TheoremTwo}[乘法的性质]
若A和B均为$n\times n$矩阵, 则$\det AB=(\det A)(\det V)$
\end{TheoremTwo}\vspace{8ex}

\section{克拉默法则、体积和线性变换}
对任意$n\times n$矩阵A和任意的$\mathbb{R}^n$中向量b, 令$A_i(b)$表示A中第i列由向量$\mathbf{b}$替换得到的矩阵
\[A_i(b)=[\mathbf{a}_1\hspace{1ex}\cdots\hspace{1ex}\mathbf{b}\hspace{1ex}\cdots\hspace{1ex}\mathbf{a}_n]\]\\[-1ex]

\begin{TheoremTwo}[克拉默法则]
设A是一个可逆的$n\times n$矩阵, 对$\mathbb{R}^n$中任意向量b, 方程Ax=b的唯一解可由下式给出
\[x_i=\frac{\det A_i(b)}{\det A},i=1,2,\cdots,n\]
\end{TheoremTwo}\vspace{4ex}

\[A^{-1}=\frac{1}{\det A}\left[\begin{array}{c c c c}
C_{11} & C_{21} & \cdots & C_{n1}\\
C_{12} & C_{22} & \cdots & C_{n2}\\
\vdots & \vdots & & \vdots\\
C_{1n} & C_{2n} & \cdots & C_{nn}
\end{array}\right]\]
其中, 余因子组成的矩阵称为A的\textbf{伴随矩阵}, 记为$adj\,A$\\[2ex]

\begin{TheoremTwo}[逆矩阵公式]
设A是一个可逆的$n\times n$矩阵, 则$A^{-1}=\dfrac{1}{\det A}adj\,A$
\end{TheoremTwo}\vspace{4ex}

\begin{TheoremOne}
若A是一个$2\times 2$矩阵, 则由A的列确定的平行四边形的面积为$|\det A|$, 若A是一个$3\times 3$矩阵, 则由A的列确定的平行六面体的体积为$|\det A|$
\end{TheoremOne}\vspace{4ex}

\framebox{
\begin{minipage}{\linewidth}
设$\mathbf{a}_1$和$\mathbf{a}_2$为非零向量, 则对任意数$c$, 由$\mathbf{a}_1$和$\mathbf{a}_2$确定的平行四边形的面积等于由$\mathbf{a}_1$和$\mathbf{a}_2+c\mathbf{a}_1$确定的平行四边形的面积
\end{minipage}}\\[2ex]

\begin{TheoremOne}
设$T:\mathbb{R}^2\rightarrow\mathbb{R}^2$是由一个$2\times 2$矩阵A确定的线性变换, 若S是$\mathbb{R}^2$中一个平行四边形, 则
\[\{T(S)\text{的面积}\}=|\det A|\cdot\{\text{S的面积}\}\]
若T是一个由$3\times 3$矩阵A确定的线性变换, 而S是$R^3$中的一个平行六面体, 则
\[\{T(S)\text{的体积}\}=|\det A|\cdot\{\text{S的体积}\}\]
\end{TheoremOne}
