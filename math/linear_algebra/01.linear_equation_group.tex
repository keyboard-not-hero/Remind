\documentclass[UTF8,fontset=ubuntu]{ctexart}
\usepackage{amsmath}
\usepackage{parskip}
\usepackage{pifont}
\usepackage{amssymb}
\usepackage{framed}
\usepackage[framed]{ntheorem}
\theorembodyfont{\normalfont}
\theoremstyle{nonumberplain}
\newtheorem{definition}{定义}
\theoremstyle{break}
\newtheorem{theorem}{定理}
\theoremstyle{nonumberbreak}
\newframedtheorem{law}{$\mathbb{R}^n$中向量的代数性质}
\begin{document}
线性方程:
	\[ a_{1}x_{1}+a_{2}x_{2}+\cdots +a_{n}x_{n}=b \]
其中, $b$与系数$a_{1}$,$a_{2}$,$\cdots$,$a_{n}$是实数或复数, 通常为已知数. n为任意正整数

线性方程组 -- 由一个或几个包含相同变量的线性方程组成, 如:
\[\left\{\begin{array}{r l r l r l}
	2x_1 & - & x_2 & + & 1.5x_3 & =8\\
	x_1  &   &     & - & 4x_3   & =-7		
\end{array}\right.\]
若线性方程组的方程个数少于未知数个数, 称之为\textbf{欠定方程组}\\
若线性方程组的方程个数多余未知数个数, 称之为\textbf{超定方程组}\\
方程组所有可能的解的集合称为线性方程组的\textbf{解集}\\
若两个方程组有相同的解集, 则这两个方程组称为\textbf{等价}的

线性方程组解集情况:\\
1.无解;\\
2.有唯一解;\\
3.有无穷多解.\\
当方程组无解时, 称线性方程组\textbf{不相容}\\
当方程组有解时, 称线性方程组\textbf{相容}

线性方程组
\[\left\{\begin{array}{r@{}l@{}r@{}l@{}r@{}l}
x_1 & - & 2x_2 & + &  x_3 & =0\\
	& 	& 2x_2 & - & 8x_3 & =8\\
-4x_1 & + & 5x_2 & + & 9x_3 & =-9
\end{array}\right.\]
线性方程组的\textbf{系数矩阵}
\[\left[\begin{array}{r r r}
	1 & -2 & 1\\
	0 & 2  & -8\\
	-4 & 5 & 9
\end{array}\right]\]
线性方程组的\textbf{增广矩阵}
\[\left[\begin{array}{r r r r}
	1 & -2 & 1 & 0\\
	0 & 2  & -8 & 8\\
	-4 & 5 & 9 & -9
\end{array}\right]\]

解线性方程组\\
思路: 将方程组用一个更容易求解的等价方程式替代\\
初等行变换:\\
1.(倍加变换)将某方程替换为它与另一方程倍数的和;\\
2.(对换变换)交换两个方程的位置;\\
3.(倍乘变换)方程的所有系数乘以一个非0实数.\\
例. 简化如下方程组
\[\left[\begin{array}{r r r r}
	1 & -2 & 1 & 0\\
	0 & 2 & -8 & 8\\
	-4 & 5 & 9 & -9
\end{array}\right]\]
步骤1 -- \ding{174}+4\ding{172}
\[\left[\begin{array}{r r r r}
    1 & -2 & 1 & 0\\
    0 & 2 & -8 & 8\\
    -4 & 5 & 9 & -9
\end{array}\right] \Rightarrow \left[\begin{array}{r r r r}
    1 & -2 & 1 & 0\\
    0 & 2 & -8 & 8\\
    0 & -3 & 13 & -9
\end{array}\right]\]
步骤2 -- $\frac{1}{2}$\ding{173}
\[\left[\begin{array}{r r r r}
    1 & -2 & 1 & 0\\
    0 & 2 & -8 & 8\\
    0 & -3 & 13 & -9
\end{array}\right] \Rightarrow \left[\begin{array}{r r r r}
    1 & -2 & 1 & 0\\
    0 & 1 & -4 & 4\\
    0 & -3 & 13 & -9
\end{array}\right]\]
步骤3 -- \ding{174}+3\ding{173}
\[\left[\begin{array}{r r r r}
    1 & -2 & 1 & 0\\
    0 & 1 & -4 & 4\\
    0 & -3 & 13 & -9
\end{array}\right] \Longrightarrow \left[\begin{array}{r r r r}
	1 & -2 & 1 & 0\\
	0 & 1 & -4 & 4\\
	0 & 0 & 1 & 3
\end{array}\right]\]
步骤4 -- \ding{173}+4\ding{174}
\[\left[\begin{array}{r r r r}
    1 & -2 & 1 & 0\\
    0 & 1 & -4 & 4\\
    0 & 0 & 1 & 3
\end{array}\right] \Longrightarrow \left[\begin{array}{r r r r}
	1 & -2 & 1 & 0\\
	0 & 1 & 0 & 16\\
	0 & 0 & 1 & 3
\end{array}\right]\]
步骤5 -- \ding{172}+(-\ding{174})
\[\left[\begin{array}{r r r r}
    1 & -2 & 1 & 0\\
    0 & 1 & 0 & 16\\
    0 & 0 & 1 & 3
\end{array}\right] \Longrightarrow \left[\begin{array}{r r r r}
	1 & -2 & 0 & -3\\
	0 & 1 & 0 & 16\\
	0 & 0 & 1 & 3
\end{array}\right]\]
步骤6 -- \ding{172}+2\ding{173}
\[\left[\begin{array}{r r r r}
    1 & -2 & 0 & -3\\
    0 & 1 & 0 & 16\\
    0 & 0 & 1 & 3
\end{array}\right] \Longrightarrow \left[\begin{array}{r r r r}
	1 & 0 & 0 & 29\\
	0 & 1 & 0 & 16\\
	0 & 0 & 1 & 3
\end{array}\right]\]

非零行(列): 矩阵中至少包含一个非零元素的行(列)\\
先导元素: 该行最左边的非零元素\\
\begin{definition}
一个矩阵称为阶梯形, 若它由以下三个性质:\\
1.所有非零行在零行之上;\\
2.某一行先导元素的列位于上一行先导元素的右边;\\
3.某一先导元素所在列下方元素都是零;\\
若还满足以下性质, 则称为简化阶梯形:\\
4.每一非零行的先导元素是1;\\
5.每一先导元素是该元素所在列的唯一非零元素.\\[1ex]
\end{definition}

\begin{theorem}[简化阶梯形矩阵的唯一性]
每个矩阵行等价于唯一的简化阶梯形矩阵.\\[1ex]
\end{theorem}

\begin{definition}
矩阵A中的{\heiti 主元位置}是A中对应于它的阶梯形中先导元素的位置.{\heiti 主元列}是A中含有主元位置的列.\\[1ex]
\end{definition}

基本变量: 位于主元列的变量\\
自由变量: 位于非主元列的变量\\
\begin{theorem}[存在与唯一性定理]
线性方程组相容的充要条件是增广矩阵的最右列不是主元列. 也就是说, 增广矩阵的阶梯形没有形如
	\[[0\ \cdots\ 0\ b], b\neq 0\]
的行. 若线性方程组相容, 则它的解集可能有两种情形:\\
(i)当没有自由变量时, 有唯一解;\\
(ii)若至少有一个自由变量, 则有无穷多解.\\[1ex]
\end{theorem}
应用行化简算法解线性方程组:\\
1.写出方程组的增广矩阵\\
2.应用行化简算法把增广矩阵化为阶梯形, 确定方程组是否相容, 如果没有解则停止; 否则进行下一步\\
3.继续行化简算法得到它的简化阶梯形\\
4.写出简化阶梯形矩阵对应的方程组\\
5.将每个非零方程改写为使用自由变量表示基本变量的形式

列向量: 仅含一列的矩阵, 简称为向量. 如:
\[u=\left[\begin{array}{c}
	3\\
	-1
\end{array}\right] \text{\quad or\quad} u=(3, -1)\]
行向量: 仅含一行的矩阵. 如:
\[v=\left[\begin{array}{c c}
	2 & 5
\end{array}\right]\]
所有两个元素的向量表示为$\mathbb{R}^2$, $\mathbb{R}$表示元素为实数, 2表示向量包含两个元素\\
向量加法:
\begin{equation*}
\left[\begin{array}{c}
	1\\
	-2
\end{array}\right]
+
\left[\begin{array}{c}
	2\\
	5
\end{array}\right]
=
\left[\begin{array}{c}
	1+2\\
	-2+5
\end{array}\right]
=
\left[\begin{array}{c}
	3\\
	3
\end{array}\right]
\end{equation*}
标量乘法:\\
\indent 若$u=\left[\begin{array}{c}3\\-1\end{array}\right]$, $c=5$, 则:
\[cu=5\left[\begin{array}{c}3\\-1\end{array}\right]=\left[\begin{array}{c}15\\-5\end{array}\right]\]

向量$\left[\begin{array}{r}x\\y\end{array}\right]$的几何含义: 由原点(0,0)指向点(x,y)的有向线段\\
所有元素都是零的向量称为\textbf{零向量}, 用\textbf{0}表示(\textbf{0}中元素的个数可由上下文确定)\\
\begin{law}
对$\mathbb{R}^n$中一切向量$\mathbf{u}$,$\mathbf{v}$,$\mathbf{w}$以及标量$c$和$d$:
\begin{equation*}
\begin{array}{l@{}c@{}l l l@{}c@{}l l}
( & i & ) & \mathbf{u}+\mathbf{v}=\mathbf{v}+\mathbf{u} & ( & v & ) & c(\mathbf{u}+\mathbf{v})=c\mathbf{u}+c\mathbf{v}\\
( & ii & ) & (\mathbf{u}+\mathbf{v})+\mathbf{w}=\mathbf{u}+(\mathbf{v}+\mathbf{w}) & ( & vi & ) & (c+d)\mathbf{u}=c\mathbf{u}+d\mathbf{u}\\
( & iii & ) & \mathbf{u}+\mathbf{0}=\mathbf{0}+\mathbf{u}=\mathbf{u} & ( & vii & ) & c(d\mathbf{u})=(cd)\mathbf{u}\\
( & iv & ) & \mathbf{u}+(-\mathbf{u})=-\mathbf{u}+\mathbf{u}=\mathbf{0} & ( & viii & ) & 1\mathbf{u}=\mathbf{u}
\end{array}
\end{equation*}
\end{law}
给定$\mathbb{R}^n$中向量$\mathbf{v}_1$,$\mathbf{v}_2$,$\cdots$,$\mathbf{v}_p$和标量$c_1$,$c_2$,$\cdots$,$c_p$, 向量
	\[\mathbf{y}=c_1\mathbf{v}_1+\cdots+c_p\mathbf{v}_p\]
称为向量$\mathbf{v}_1$,$\mathbf{v}_2$,$\cdots$,$\mathbf{v}_p$以$c_1$,$c_2$,$\cdots$,$c_p$为\textbf{权}的\textbf{线性组合}.\\[2ex]
\begin{definition}
若$\mathbf{v}_1$,$\mathbf{v}_2$,$\cdots$,$\mathbf{v}_p$是$\mathbb{R}^n$中的向量, 则$v_1$,$v_2$,$\cdots$,$v_p$的所有线性组合所成的组合用记号Span\{$\mathbf{v}_1$,$\mathbf{v}_2$,$\cdots$,$\mathbf{v}_p$\}表示, 称为\textbf{$\text{由}\mathbf{v}_1,\mathbf{v}_2,\cdots,\mathbf{v}_p\text{所生成的}\mathbb{R}^n\text{的子集}$}. 也就是说, Span\{$\mathbf{v}_1$,$\mathbf{v}_2$,$\cdots$,$\mathbf{v}_p$\}是所有形如
	\[c_1\mathbf{v}_1+c_2\mathbf{v}_2+\cdots+c_p\mathbf{v}_p\]
的向量的集合, 其中$v_1$,$v_2$,$\cdots$,$v_p$为标量.
\end{definition}
\end{document}
