\documentclass[UTF8,fontset=ubuntu]{ctexart}
\usepackage{amsmath}
\usepackage{bm}
\usepackage{parskip}
\usepackage{amssymb}
\usepackage{xcolor}
\usepackage{colortbl}
\usepackage{framed}
\usepackage[framed]{ntheorem}
\theorembodyfont{\normalfont}
\theorempreskip{2ex}
\theorempostskip{2ex}
\theoremstyle{plain}
\newtheorem{theorem}{定理}
\theoremstyle{nonumberplain}
\newtheorem{definition}{定义}
\theoremstyle{empty}
\newframedtheorem{law}{定律}
\definecolor{lightgray}{gray}{0.7}
\definecolor{gray}{gray}{0.5}
\definecolor{white}{gray}{1}
\begin{document}
$m\times n$矩阵$A=[a_{ij}]$的\textbf{对角线元素}是$a_{11}$,$a_{22}$,$a_{33}$,$\cdots$, 它们组成$A$的\textbf{主对角线}.\\
\textbf{对角矩阵}是一个方阵, 它的非对角线元素全是0. 例如$n\times n$单位矩阵$\bm{I}_n$.\\
元素全是0的$m\times n$矩阵称为\textbf{零矩阵}, 用$\bm{0}$表示.\\
若两个矩阵有相同的维数(即有相同的行数和列数), 而且对应元素相同, 则称该两个矩阵\textbf{相等}\\
若$r$是标量而$A$是矩阵, 则\textbf{标量乘法}$rA$是一个矩阵, 它的每一列是$A$的对应列的$r$倍.\\
\begin{theorem}
设$A$,$B$,$C$是相同维数的矩阵, $r$与$s$为数, 则有\\
\begin{tabular}{l@{\ }l@{\hspace{5em}}l@{\ }l}
a. & $A+B=B+A$ & b. & $(A+B)+C=A+(B+C)$\\
c. & $A+0=A$ & d. & $r(A+B)=rA+rB$\\
e. & $(r+s)A=rA+sA$ & f. & $r(sA)=(rs)A$
\end{tabular}
\end{theorem}\ \\
\begin{definition}
若$A$是$m\times n$矩阵, $B$是$n\times p$矩阵, $B$的列是$\bm{b}_1$,$\cdots$,$\bm{b}_p$, 则乘积$AB$是$m\times p$矩阵, 它的各列是$A\bm{b}_1$,$\cdots$,$A\bm{b}_p$, 即
\[AB=A[\bm{b}_1\ \bm{b}_2\ \cdots\ \bm{b}_p]=[A\bm{b}_1\ A\bm{b}_2\ \cdots\ A\bm{b}_p]\]
\end{definition}
\begin{law}
$AB$的每一列都是$A$的各列的线性组合, 以$B$的对应列的元素为权.
\end{law}
\begin{law}[计算$AB$的行列法则]\ \\
若乘积$AB$有定义, 则$AB$的第$i$行第$j$列的元素是$A$的第$i$行与$B$的第$j$列对应元素乘积之和. 若($AB$)${}_{ij}$表示$AB$的($i$,$j$)元素, $A$为$m\times n$矩阵, 则
\[(AB)_{ij}=a_{i1}b_{1j}+a_{i2}b_{2j}+\cdots+a_{in}b_{nj}\]
\end{law}\ \\
\begin{theorem}
设$A$为$m\times n$矩阵, $B$和$C$的维数使下列各式的乘积有意义.\\
\begin{tabular}{l@{\ }l l}
a. & $A(BC)=(AB)C$ & (乘法结合律)\\
b. & $A(B+C)=AB+AC$ & (乘法左分配律)\\
c. & $(B+C)A=BA+CA$ & (乘法右分配律)\\
d. & $r(AB)=(rA)B=A(rB)$, r为任意数 &\\
e. & $\bm{I}_mA=A=A\bm{I}_m$ & (矩阵乘法的恒等式)
\end{tabular}
\end{theorem}
乘积$AB$的因子关系为: $A$被$B$\textbf{右乘}, 或$B$被$A$\textbf{左乘}\\
若$AB$=$BA$, 我们称$A$和$B$彼此\textbf{可交换}\\[2ex]
\textbf{警告}\\
1.一般情况下, $AB\neq BA$.\\
2.消去律对矩阵乘法不成立, 即若$AB=AC$, 一般情况下, $B=C$并不成立.\\
3.若乘积$AB$是零矩阵, 一般情况下, 不能断定A=0或B=0.\\[2ex]
给定$m\times n$矩阵, 则$A$的\textbf{转置}是一个$n\times m$矩阵, 用$A^T$表示, 它的列是由$A$的对应行构成的.\\[2ex]
\begin{theorem}
设$A$与$B$表示矩阵, 其维数使下列和与积有定义, 则\\
\begin{tabular}{l@{\ }l}
a. & $(A^T)^T=A$.\\
b. & $(A+B)^T=A^T+B^T$.\\
c. & 对任意数$r$, $(rA)^T=rA^T$.\\
d. & $(AB)^T=B^TA^T$.
\end{tabular}
\end{theorem}\ \\
\begin{law}
若干个矩阵的乘积的转置等于它们的转置的乘积, 但相乘的顺序相反.
\end{law}
$A$为$n\times n$矩阵, 若存在一个$n\times n$矩阵$C$, 使得
\[CA=\bm{I}_n\qquad\text{且}AC=\bm{I}_n\]
则称$A$\textbf{可逆}, 并且$C$是$A$的\textbf{逆}.\\[2ex]
若$A$可逆, 它的逆是唯一的, 我们将它记为$A^{-1}$, 则
\[A_{-1}A=I\qquad\text{且}AA^{-1}=I\]
不可逆矩阵也称为\textbf{奇异矩阵}.\\
可逆矩阵也称为\textbf{非奇异矩阵}.\\
\begin{theorem}
设$A=\left[\begin{array}{r r}3 & 4\\5 & 6\end{array}\right]$. 若$ad-bc\neq 0$, 则$A$可逆且
\[A^{-1}=\frac{1}{ad-bc}\left[\begin{array}{l l}d & -b\\-c & a\end{array}\right]\]
若$ad-bc=0$, 则$A$不可逆.
\end{theorem}
数$ad-bc$称为$A$的\textbf{行列式}, 记为
\[\det A=ad-bc\]
\begin{theorem}
若$A$是可逆$n\times n$矩阵, 则对每一$\mathbb{R}^n$中的$\bm{b}$, 方程$A\bm{x}=\bm{b}$有唯一解$\bm{x}=A^{-1}\bm{b}$.
\end{theorem}\ \\
\begin{law}[胡克定律]\ \\
公式如下
\[\bm{y}=D\bm{f}\]
其中$D$为\textbf{弹性矩阵}, 它的逆称为\textbf{刚性矩阵}, $\bm{f}$表示它在各个点受的力, $\bm{y}$表示各个点的形变量.
\end{law}
\begin{theorem}\ \\
a.若$A$是可逆矩阵, 则$A^{-1}$也可逆而且$(A^{-1})^{-1}=A$.\\
b.若$A$和$B$都是$n\times n$可逆矩阵, 则$AB$也可逆, 且其逆是$A$和$B$的逆矩阵按相反顺序的乘积, 即
\[(AB)^{-1}=B^{-1}A^{-1}\]
c.若$A$可逆, 则$A^T$也可逆, 且其逆是$A^{-1}$的转置, 即$(A^T)^{-1}=(A^{-1})^T$.
\end{theorem}
\begin{law}
若干个$n\times n$可逆矩阵的积也是可逆的, 其逆等于这些矩阵的逆按相反顺序的乘积.
\end{law}
把单位矩阵进行一次初等行变换, 就得到\textbf{初等矩阵}.\\
\begin{law}
若对$m\times n$矩阵$A$进行某种初等行变换, 所得矩阵可写成$EA$, 其中$E$是$m\times m$矩阵, 是由$I_m$进行同一行变换所得.\\
\end{law}
\begin{law}
每个初等矩阵$E$是可逆的, $E$的逆是一个同类型的初等矩阵, 它把$E$变回$I$.
\end{law}\ \\
\begin{theorem}
$n\times n$矩阵$A$是可逆的, 当且仅当$A$行等价于$I_n$, 这时, 把$A$化简为$I_m$的一系列初等行变化同时把$I_n$变成$A^{-1}$.
\end{theorem}
\begin{law}[求$A^{-1}$的算法]\ \\
把增广矩阵$[A\ I]$进行行化简. 若$A$行等价于$I$, 则$[A\ I]$行等价于$[I\ A^{-1}]$, 否则$A$没有逆.
\end{law}
\begin{theorem}[可逆矩阵定理]
设$A$为$n\times n$矩阵, 则下列命题是等价的, 即对某一特定的$A$, 它们同时为真或同时为假.\\
\begin{tabular}{l@{\ }l}
a. & $A$是可逆矩阵.\\
b. & $A$行等价于$n\times n$单位矩阵.\\
c. & $A$有$n$个主元位置.\\
d. & 方程$A\bm{x}=\bm{0}$仅有平凡解.\\
e. & $A$的各列线性无关.\\
f. & 线性变换$\bm{x}\mapsto A\bm{x}$是一对一的.\\
g. & 对$\mathbb{R}^n$中任意$\bm{b}$, 方程$A\bm{x}=\bm{b}$至少有一个解.\\
h. & $A$的各列生成$\mathbb{R}^n$,\\
i. & 线性变换$\bm{x}\mapsto A\bm{x}$把$\mathbb{R}^n$映上到$\mathbb{R}^n$.\\
j. & 存在$n\times n$矩阵$C$使$CA=I$.\\
k. & 存在$n\times n$矩阵$D$使$AD=I$.\\
l. & $A^T$是可逆矩阵.
\end{tabular}
\end{theorem}
\begin{law}
设$A$和$B$为方阵, 若$AB=I$, 则$A$和$B$都是可逆的, 且$B=A^{-1}$, $A=B^{-1}$.
\end{law}
\begin{theorem}
设$T$:$\mathbb{R}^n\rightarrow\mathbb{R}^n$为线性变换, $A$为$T$的标准矩阵. 则$T$可逆当且仅当$A$是可逆矩阵.
\end{theorem}
形如
\[A=\left[
\begin{array}{r r r|r r|r}
3 & 0 & -1 & 5 & 9 & -2\\
-5 & 2 & 4 & 0 & -3 & 1\\\hline
-8 & -6 & 3 & 1 & 7 & -4
\end{array}
\right]\]
为矩阵$A$的$2\times 3$\textbf{分块矩阵}, 也可表示为
\[A=\left[
\begin{array}{l l l}
A_{11} & A_{12} & A_{13}\\
A_{21} & A_{22} & A_{23}
\end{array}
\right]\]
设$A$为$m\times n$矩阵, $B$为$n\times p$矩阵, 当$A$的列的分法与$B$的行的分法一致时, 可计算$AB$. 如下
\[A=\left[
\begin{array}{r r r|r r}
2 & -3 & 1 & 0 & -4\\
1 & 5 & -2 & 3 & -1\\\hline
0 & -4 & -2 & 7 & -1
\end{array}
\right]=\left[
\begin{array}{l l}
A_{11} & A_{12}\\
A_{21} & A_{22}
\end{array}
\right]\text{,}
B=\left[
\begin{array}{r r}
6 & 4\\
-2 & 1\\
-3 & 7\\\hline
-1 & 3\\
5 & 2
\end{array}
\right]=\left[
\begin{array}{l}
B_1\\
B_2
\end{array}
\right]\]
\[AB=\left[
\begin{array}{l l}
A_{11} & A_{12}\\
A_{21} & A_{22}
\end{array}
\right]\left[
\begin{array}{l}
B_1\\
B_2
\end{array}
\right]=\left[
\begin{array}{l}
A_{11}B_1+A_{12}B_2\\
A_{21}B_1+A_{22}B_2
\end{array}
\right]=\left[
\begin{array}{r r}
-5 & 4\\
-6 & 2\\\hline
2 & 1
\end{array}
\right]\]
\begin{theorem}[$AB$的列行展开]\ \\
若$A$是$m\times n$矩阵, $B$是$n\times p$矩阵, 则
\[\begin{array}{l@{}l}
AB & =[col_1(A)\ col_2(A)\ \cdots\ col_n(A)]\left[
\begin{array}{c}
row_1(B)\\
row_2(B)\\
\vdots\\
row_n(B)
\end{array}
\right]\\
& =col_1(A)row_1(B)+\cdots+col_n(A)row_n(B)
\end{array}\]
\end{theorem}
设$A$是$m\times n$矩阵, 它可以行化简为阶梯形(化简步骤不包含对换变换), 则$A$可写成$A=LU$. 其中, $L$是$m\times m$下三角矩阵, 主对角线元素全是1; $U$是$A$的一个$m\times n$阶梯形矩阵.
\begin{law}[$LU$分解的算法]\ \\
1.如果可能的话, 用一系列的行倍加变换把$A$化为阶梯形$U$.\\
2.填充$L$的元素使相同的行变换把$L$变为$I$.
\end{law}
$LU$分解图解:\\
\[\begin{array}{c@{}c@{}c@{}c@{}c@{}c@{}c@{}c@{}c}
A= & \left[\hspace{-1ex}\begin{array}{>{\columncolor{lightgray}[0pt]}r r r r}
\cellcolor{gray}2 & 4 & 5 & -2\\
-4 & -5 & -8 & 1\\
2 & -5 & 1 & 8\\
-6 & 0 & -3 & 1
\end{array}\hspace{-1ex}\right] & \Rightarrow & \left[\!\begin{array}{r >{\columncolor{white}[0pt]}r r r}
2 & 4 & 5 & -2\\
0 & \cellcolor{gray}3 & 2 & -3\\
0 & \cellcolor{lightgray}-9 & -4 & 10\\
0 & \cellcolor{lightgray}12 & 12 & -5
\end{array}\hspace{-1ex}\right] & \Rightarrow & \left[\!\begin{array}{r r >{\columncolor{white}[6pt][0pt]}r r}
2 & 4 & 5 & -2\\
0 & 3 & 2 & -3\\
0 & 0 & \cellcolor{gray}2 & 1\\
0 & 0 & \cellcolor{lightgray}4 & 7
\end{array}\hspace{-1ex}\right] & \Rightarrow & \left[\!\begin{array}{r r r >{\columncolor{white}[0pt]}r}
2 & 4 & 5 & -2\\
0 & 3 & 2 & -3\\
0 & 0 & 2 & 1\\
0 & 0 & 0 & \cellcolor{gray}5
\end{array}\hspace{-1ex}\right] & =U\\
& \downarrow & & \downarrow & & \downarrow & & \downarrow &\\
L= & \left[\hspace{-1ex}\begin{array}{>{\columncolor{lightgray}[0pt]}r r r r}
\cellcolor{gray}1 & 0 & 0 & 0\\
-2 & 1 & 0 & 0\\
1 & & 1 & 0\\
-3 & & & 1
\end{array}\hspace{-1ex}\right] & & \left[\!\begin{array}{r >{\columncolor{white}[0pt]}r r r}
1 & 0 & 0 & 0\\
-2 & \cellcolor{gray}1 & 0 & 0\\
1 & \cellcolor{lightgray}-3 & 1 & 0\\
-3 & \cellcolor{lightgray}4 & & 1
\end{array}\hspace{-1ex}\right] & & \left[\!\begin{array}{r r >{\columncolor{white}[6pt][0pt]}r r}
1 & 0 & 0 & 0\\
-2 & 1 & 0 & 0\\
1 & -3 & \cellcolor{gray}1 & 0\\
-3 & 4 & \cellcolor{lightgray}2 & 1
\end{array}\hspace{-1ex}\right] & & \left[\!\begin{array}{r r r >{\columncolor{white}[6pt][0pt]}r}
1 & 0 & 0 & 0\\
-2 & 1 & 0 & 0\\
1 & -3 & 1 & 0\\
-3 & 4 & 2 & \cellcolor{gray}1
\end{array}\!\right] &
\end{array}\]
\begin{law}[列昂惕夫投入产出模型或生产方程]\ \\
\[\bm{x}=C\bm{x}+\bm{d}\]
\end{law}
\begin{theorem}
设$C$为某一经济体系的消耗矩阵, $\bm{d}$为最终需求. 若$C$和$\bm{d}$的元素非负, $C$的每一列的和小于1, 则$(I-C)^{-1}$存在, 产出向量
\[\bm{x}=(I-C)^{-1}\bm{d}\]
有非负元素, 且是下列方程的唯一解:
\[\bm{x}=C\bm{x}+d\]
\end{theorem}
物体的平移并不直接对应于矩阵乘法, 因为平移并非线性变换, 引入\textbf{齐次坐标}\\
$\mathbb{R}^2$中每个点$(x,y)$对应于$\mathbb{R}^3$中的点$(x,y,1)$, $(x,y,1)$为$(x,y)$的\textbf{齐次坐标}\\
$(x,y,1)\mapsto(x+h,y+k,1)$的平移变换实现:\\
\[\left[\begin{array}{r r r}
1 & 0 & h\\
0 & 1 & k\\
0 & 0 & 1
\end{array}\right]
\left[\begin{array}{r}
x\\
y\\
1
\end{array}\right]=
\left[\begin{array}{c}
x+h\\
y+k\\
1
\end{array}\right]\]
$mathbb{R}^2$中任意线性变换可以通过齐次坐标乘以分块矩阵$\left[\begin{array}{r r}A & 0\\0 & 1\end{array}\right]$实现, 其中$A$是$2\times 2$矩阵.\\
$(x,y,z,1)$是$\mathbb{R}^3$中点$(x,y,z)$的齐次坐标.\\
若$H\neq 0$, 则$(X,Y,Z,H)$为$(x,y,z)$的齐次坐标, 且
\[x=\frac{X}{H},y=\frac{Y}{H},z=\frac{Z}{H}\]
点$(x,y,z)$在$xy$平面上的透视投影坐标为$(\frac{x}{1-z/d}, \frac{y}{1-z/d}, 0)$. 其中, $d$为$z$轴观测位置$(0,0,d)$\\
绕$\mathbb{R}^2$中一点$p$的旋转是这样实现的: 首先把图形平移$-p$, 然后绕原点旋转, 最后平移$p$.
\end{document}
