\chapter{向量空间}
\section{一、向量空间和子空间}
\begin{definition}
一个向量空间是由一些被称为向量的对象构成的非空集合V, 在这个集合上定义两个运算, 称为加法和标量乘法(标量取实数), 服从以下公理(或法则), 这些公理必须对V中所有向量$\mathbf{u}$,$\mathbf{v}$,$\mathbf{w}$及所有标量c和d均成立.\\
1.\quad$\mathbf{u}$,$\mathbf{v}$之和表示为$\mathbf{u}+\mathbf{v}$, 仍在V中\\
2.\quad$\mathbf{u+v=v+u}$\\
3.\quad$\mathbf{(u+v)+w=u+(v+w)}$\\
4.\quad V中存在一个零向量$\mathbf{0}$, 使得$\mathbf{u+0=u}$\\
5.\quad 对V中每个向量$\mathbf{u}$, 存在V中向量$-\mathbf{u}$, 使得$\mathbf{u+(-u)=0}$\\
6.\quad$\mathbf{u}$与标量c的标量乘法记为$c\mathbf{u}$, 仍在V中\\
7.\quad$c\mathbf{(u+v)}=c\mathbf{u}+c\mathbf{v}$\\
8.\quad$(c+d)\mathbf{u}=c\mathbf{u}+d\mathbf{u}$\\
9.\quad$c(d\mathbf{u})=(cd)\mathbf{u}$\\
10.\quad$1\mathbf{u=u}$\\
\end{definition}
