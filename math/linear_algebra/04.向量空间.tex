\chapter{向量空间}
\section{向量空间和子空间}
\begin{definition}
一个向量空间是由一些被称为向量的对象构成的非空集合V, 在这个集合上定义两个运算, 称为加法和标量乘法(标量取实数), 服从以下公理(或法则), 这些公理必须对V中所有向量$\mathbf{u}$,$\mathbf{v}$,$\mathbf{w}$及所有标量c和d均成立.\\
1.\quad$\mathbf{u}$,$\mathbf{v}$之和表示为$\mathbf{u}+\mathbf{v}$, 仍在V中\\
2.\quad$\mathbf{u+v=v+u}$\\
3.\quad$\mathbf{(u+v)+w=u+(v+w)}$\\
4.\quad V中存在一个零向量$\mathbf{0}$, 使得$\mathbf{u+0=u}$\\
5.\quad 对V中每个向量$\mathbf{u}$, 存在V中向量$-\mathbf{u}$, 使得$\mathbf{u+(-u)=0}$\\
6.\quad$\mathbf{u}$与标量c的标量乘法记为$c\mathbf{u}$, 仍在V中\\
7.\quad$c\mathbf{(u+v)}=c\mathbf{u}+c\mathbf{v}$\\
8.\quad$(c+d)\mathbf{u}=c\mathbf{u}+d\mathbf{u}$\\
9.\quad$c(d\mathbf{u})=(cd)\mathbf{u}$\\
10.\quad$1\mathbf{u=u}$\\
\end{definition}\vspace{4ex}

{\par\raggedright
\framebox{\begin{minipage}{\textwidth}
对V中每个向量$\mathbf{u}$和任意标量$c$,有
\[0\mathbf{u=0}\]
\[c\mathbf{0=0}\]
\[-\mathbf{u}=(-1)\mathbf{u}\]
\end{minipage}}
\par}\vspace{4ex}

\begin{definition}
向量空间V的一个\textbf{子空间}是V的一个满足以下三个性质的子集H:
\begin{enumerate}
\item V中的零向量在H中
\item H对向量加法封闭,即对H中任意向量$\mathbf{u}$,$\mathbf{v}$,和$\mathbf{u+v}$仍在H中
\item H对标量乘法封闭,即对H中任意向量$\mathbf{u}$和任意标量$c$,向量$c\mathbf{u}$仍在H中
\end{enumerate}
\end{definition}\vspace{4ex}

\begin{TheoremOne}
若$v_1$,$v_2$,$\cdots$,$v_p$在向量空间V中,则$\Span\{v_1,\cdots,v_p\}$是V的一个子空间
\end{TheoremOne}

\section{零空间、列空间和线性变换}
考虑下列齐次方程组:
\begin{equation}
\begin{array}{r@{\hspace{0pt}}l@{\hspace{0pt}}r@{\hspace{0pt}}l@{\hspace{0pt}}r@{\hspace{0pt}}l}
x_1 & - & 3x_2 & - & 2x_3 & =0\\
-5x_1 & + & 9x_2 & + & x_3 & =0
\end{array}\label{eq:01}
\end{equation}
用矩阵的形式,此方程组可写成$A\mathbf{x=0}$,其中
\[A=\left[\begin{array}{r r r}
1 & -3 & -2\\
-5 & 9 & 1
\end{array}\right]\]
所有满足\eqref{eq:01}的$\mathbf{x}$的集合称为方程组\eqref{eq:01}的\textbf{解集}\\
我们成满足$A\mathbf{x=0}$的所有$\mathbf{x}$的集合为矩阵A的\textbf{零空间}\\[2ex]

\begin{definition}
矩阵A的零空间写成$\Nul A$,是齐次方程$A\mathbf{x=0}$的全体解的集合. 用集合符号表示,即
\[\Nul A=\{\mathbf{x}:\mathbf{x}\in\mathbb{R}^n, A\mathbf{x=0}\}\]
\end{definition}\vspace{4ex}

\begin{TheoremOne}
$m\times n$矩阵A的零空间是$\mathbb{R}^n$的一个子空间. 等价地,$m$个方程,$n$个未知数的齐次线性方程组$A\mathbf{x=0}$的全体解的集合是$\mathbb{R}^n$的一个子空间
\end{TheoremOne}\vspace{4ex}

\begin{definition}
$m\times n$矩阵A的\textbf{列空间}(记为$\Col A$)是由A的列的所有线性组合组成的集合. 若$A=[\mathbf{a}_1,\cdots,\mathbf{a}_n]$,则$\Col A=\Span\{\mathbf{a}_1,\cdots,\mathbf{a}_n\}$
\end{definition}\vspace{4ex}

\begin{TheoremOne}
$m\times n$矩阵A的列空间是$\mathbb{R}^m$的一个子空间
\end{TheoremOne}\vspace{4ex}

{\par\raggedright
\framebox{\begin{minipage}{\textwidth}
$m\times n$矩阵A的列空间等于$\mathbb{R}^m$当且仅当方程$A\mathbf{x=b}$对$\mathbb{R}^m$中每个$\mathbf{b}$有一个解
\end{minipage}}
\par}\vspace{4ex}

\begin{table}[H]
\caption{对$m\times n$矩阵A,$\Nul A$与$\Col A$之间的对比}
\begin{tabular}{>{\scriptsize}p{6cm}|>{\scriptsize}p{6cm}}
\hline
\multicolumn{1}{c|}{$\Nul A$} & \multicolumn{1}{c}{$\Col A$}\\
\hline
\begin{enumerate}
\item $\Nul A$是$\mathbb{R}^n$的一个子空间
\item $\Nul A$是隐式定义的,即仅给出了一个$\Nul A$中向量必须满足的条件($A\mathbf{x=0}$)
\item 求$\Nul A$中的向量需要时间,需要对$[A\quad\mathbf{0}]$作行变换
\item $\Nul A$与A的元素之间没有明显的关系
\item $\Nul A$中的一个典型向量$\mathbf{v}$具有$A\mathbf{v=0}$的性质
\item 给定一个特定的向量$\mathbf{v}$,容易判断$\mathbf{v}$是否在$\Nul A$中. 仅需计算$A\mathbf{v}$
\item $\Nul A=\{\mathbf{0}\}$当且仅当方程$A\mathbf{x=0}$仅有一个平凡解
\item $\Nul A=\{\mathbf{0}\}$当且仅当线性变换$\mathbf{x}\mapsto A\mathbf{x}$是一对一的
\end{enumerate} & \begin{enumerate}
\item $\Col A$是$\mathbb{R}^m$的一个子空间
\item $\Col A$是显式定义的,即明确指出如何构建$\Col A$中的向量
\item 容易求出$\Col A$中的向量. A的列就是$\Col A$中的向量,其余的可由A的列表示出来
\item $\Col A$与A的元素之间有明显的关系,因为A的列就在$\Col A$中
\item $\Col A$中一个典型向量$\mathbf{v}$具有方程$A\mathbf{x=v}$是相容的性质
\item 给定一个特定的向量$\mathbf{v}$,弄清$\mathbf{v}$是否在$\Col A$中需要时间,需要对$[A\quad\mathbf{v}]$作行变换
\item $\Col A=\mathbb{R}^m$当且仅当方程$A\mathbf{x=b}$对每一个$\mathbf{b}\in\mathbb{R}^m$有一个解
\item $\Col A=\mathbb{R}^m$当且仅当线性变换$\mathbf{x}\mapsto A\mathbf{x}$将$\mathbb{R}^n$映上到$\mathbb{R}^m$
\end{enumerate}\\
\hline
\end{tabular}
\end{table}\vspace{4ex}

\begin{definition}
由向量空间V映射到向量空间W内的\textbf{线性变换}T是一个规划,它将V中每个向量$\mathbf{x}$映射成W中唯一向量$T(\mathbf{x})$,且满足:
\begin{enumerate}
\item $T(\mathbf{u+v})=T(\mathbf{u})+T(\mathbf{v})$,对V中所有$\mathbf{u}$,$\mathbf{v}$均成立
\item $T(c\mathbf{u})=cT(\mathbf{u})$,对V中所有$\mathbf{u}$及所有数$c$均成立
\end{enumerate}
\end{definition}\vspace{4ex}

\section{线性无关集和基}
V中向量的一个指标集$\{v_1,\cdots,v_p\}$称为是\textbf{线性无关}的,如果向量方程
\begin{equation}
c_1\mathbf{v}_1+c_2\mathbf{v}_2+\cdots+c_p\mathbf{v}_p=\mathbf{0}\label{eq:02}
\end{equation}
只有平凡解,即$c_1=0$,$\cdots$,$c_p=0$.\\
集合$\{\mathbf{v}_1,\cdots,\mathbf{v}_p\}$称为\textbf{线性相关},如果\eqref{eq:02}有一个非平凡的解,即存在某些权$c_1$,$\cdots$,$c_p$不全为零,使得\eqref{eq:02}式成立. 此时\eqref{eq:02}式称为$\mathbf{v}_1$,$\cdots$,$\mathbf{v}_p$之间的一个\textbf{线性相关关系}\vspace{4ex}

\begin{TheoremOne}
两个或多个向量组成的有编号的向量集合$\{\mathbf{v}_1,\cdots,\mathbf{v}_p\}$(如果$\mathbf{v}_1\neq\mathbf{0}$)是线性相关的,当且仅当某$\mathbf{v}_j(j>1)$是其前面向量$\mathbf{v}_1$,$\cdots$,$\mathbf{v}_{j-1}$的线性组合
\end{TheoremOne}\vspace{4ex}

\begin{definition}
令H是向量空间V的一个子空间. V中向量的指标集$\mathcal{B}=\{\mathbf{b}_1,\cdots,\mathbf{b}_p\}$称为H的一个\textbf{基},如果
\begin{enumerate}
\item $\mathcal{B}$是一线性无关集
\item 由$\mathcal{B}$生成的子空间与H相同,即$H=\Span\{\mathbf{b}_1,\cdots,\mathbf{b}_p\}$
\end{enumerate}
\end{definition}\vspace{4ex}

令A是一个可逆的$n\times n$矩阵,比如$A=\{\mathbf{a}_1,\cdots,\mathbf{a}_n\}$。则由可逆矩阵定理,A的列组成$\mathbb{R}^n$的一个基,这是因为它们是线性无关的且它们可以生成$\mathbb{R}^n$\\[2ex]

令$\mathbf{e}_1$,$\cdots$,$\mathbf{e}_n$是$n\times n$单位矩阵$I_n$的列,即
\[\mathbf{e}_1=\left[
\begin{array}{c}
1\\
0\\
\vdots\\
0
\end{array}
\right],\mathbf{e}_2=\left[
\begin{array}{c}
0\\
1\\
\vdots\\
0
\end{array}
\right],\cdots,\mathbf{e}_n=\left[
\begin{array}{c}
0\\
\vdots\\
0\\
1
\end{array}
\right]\]
集合$\{\mathbf{e}_1,\cdots,\mathbf{e}_n\}$称为$\mathbb{R}^n$的\textbf{标准基}\\[2ex]

\begin{TheoremTwo}[生成集定理]
令$S=\{\mathbf{v}_1,\cdots,\mathbf{v}_p\}$是V中的向量集,$H=\Span\{\mathbf{v}_1,\cdots,\mathbf{v}_p\}$
\begin{enumerate}
\item 若S中某一个向量(比如说$\mathbf{v}_k$)是S中其余向量的线性组合,则S中去掉$\mathbf{v}_k$后形成的集合仍然可以生成H
\item 若$H\neq\{\mathbf{0}\}$,则S的某一子集是H的一个基
\end{enumerate}
\end{TheoremTwo}\vspace{4ex}

当$\Nul A$包含非零向量时,我们的方法总可以产生一个线性无关集,从而由该方法可以得到$\Nul A$的一个基\\[2ex]

\begin{TheoremOne}
矩阵A的主元列构成$\Col A$的一个基
\end{TheoremOne}\vspace{4ex}

基是尽可能大的线性无关集

\section{坐标系}
\begin{TheoremTwo}[唯一表示定理]
令$\mathcal{B}=\{\mathbf{b}_1,\cdots,\mathbf{b}_n\}$是向量空间V的一个基,则对V中每个向量$\bm{x}$,存在唯一的一组数$c_1,\cdots,c_n$使得
\[\bm{x}=c_1\bm{b}_1+\cdots+c_n\bm{b}_n\]
\end{TheoremTwo}\vspace{4ex}

\begin{definition}
假设$\mathcal{B}=\{\bm{b}_1,\cdots,\bm{b}_n\}$是V的一个基,$\bm{x}$在V中,$\bm{x}$\textbf{相对于基}$\mathcal{B}$\textbf{的坐标}(或$\bm{x}$\textbf{的}$\mathcal{B}$-\textbf{坐标})是使得$\bm{x}=c_1\bm{b}_1+\cdots+c_n\bm{b}_n$的权$c_1$,$\cdots$,$c_n$\\
若$c_1$,$\cdots$,$c_n$是$\bm{x}$的$\mathcal{B}$-坐标,则$\mathbb{R}^n$中的向量
\[[\bm{x}]_{\mathcal{B}}=\left[
\begin{array}{c}
c_1\\
\vdots\\
c_n
\end{array}
\right]\]
是$\bm{x}$(\textbf{相对于}$\mathcal{B}$)\textbf{的坐标向量}或$\bm{x}$\textbf{的}$\mathcal{B}$-\textbf{坐标向量},映射$x\mapsto[\bm{x}]_{\mathcal{B}}$称为(\textbf{由}$\mathcal{B}$\textbf{确定的})\textbf{坐标映射}
\end{definition}\vspace{4ex}

令
\[P_{\mathcal{B}}=[\bm{b}_1\quad\bm{b}_2\quad\cdots\quad\bm{b}_n]\]
则向量方程
\[\bm{x}=c_1\bm{b}_1+c_2\bm{b}_2+\cdots+c_n\bm{b}_n\]
等价于
\[\bm{x}=P_{\mathcal{B}}[\bm{x}]_{\mathcal{B}}\]
我们称$P_{\mathcal{B}}$为从$\mathcal{B}$到$\mathbb{R}^n$中标准基的\textbf{坐标变换矩阵}\\[2ex]

\begin{TheoremOne}
令$\mathcal{B}=\{\bm{b}_1,\cdots,\bm{b}_n\}$是向量空间V的一个基,则坐标映射$\bm{x}\mapsto[\bm{x}]_{\mathcal{B}}$是一个由V映上到$\mathbb{R}^n$的一对一的线性变换
\end{TheoremOne}\vspace{4ex}

\section{向量空间的维数}
\begin{TheoremOne}
若向量空间V具有一组基$\mathcal{B}=\{\bm{b}_1,\cdots,\bm{b}_n\}$,则V中任意包含多余$n$个向量的集合一定线性相关
\end{TheoremOne}\vspace{4ex}

\begin{TheoremOne}
若向量空间V有一组基含有$n$个向量,则V的每一组基一定恰好含有$n$个向量
\end{TheoremOne}\vspace{4ex}

\begin{definition}
若V由一个有限集生成,则V称为\textbf{有限维的},V的维数写成$\dim V$,是V的基中向量的个数. 零向量空间$\{\mathbf{0}\}$的维数定义为零. 如果V不是由一有限集生成,则V称为\textbf{无穷维的}
\end{definition}\vspace{4ex}

\begin{TheoremOne}
令H是有限维向量空间V的子空间,若有必要的话,H中任一个线性无关集均可以扩充为H的一个基. H也是有限维的并且
\[\dim H\leqslant\dim V\]
\end{TheoremOne}\vspace{4ex}

\begin{TheoremTwo}[基定理]
令V是一个$p$维向量空间,$p\geqslant 1$,V中任意含有$p$个元素的线性无关集必然是V的一个基. 任意含有$p$个元素且生成V的集合自然是V的一个基
\end{TheoremTwo}\vspace{4ex}

{\par\centering
\framebox{\begin{minipage}{\textwidth}
$\Nul A$的维数是方程$A\bm{x=0}$中自由变量的个数,$\Col A$的维数是A中主元列的个数
\end{minipage}}
\par}\vspace{4ex}

\section{秩}
矩阵A中线性无关列的最大个数和$A^T$中线性无关列的最大个数(即A中线性无关行的最大个数)是相同的,这个公共值是矩阵A的\textbf{秩}\\[2ex]

若A是一个$m\times n$矩阵,A的每一行具有$n$个元素,即可以视为$\mathbb{R}^n$中一个向量. 其行向量的所有线性组合的集合称为A的\textbf{行空间},记为$\Row A$\\[2ex]

\begin{TheoremOne}
若两个矩阵A和B行等价,则它们的行空间相同. 若B是阶梯形矩阵,则B的非零行构成A的行空间的一个基同时也是B的行空间的一个基
\end{TheoremOne}\vspace{4ex}

\begin{definition}
A的秩即A的列空间的维数
\end{definition}\vspace{4ex}

\begin{TheoremTwo}[秩定理]
$m\times n$矩阵A的列空间和行空间的维数相等,这个公共的维数(即A的秩)还等于A的主元位置的个数且满足方程
\[\rank A+\dim\Nul A=n\]
\end{TheoremTwo}\vspace{4ex}

\begin{TheoremTwo}[可逆矩阵定理(续)]
令A是一个$n\times n$矩阵,则下列命题中的每一个均等价于A是可逆矩阵:
\begin{list}{}{\setlength{\parsep}{0pt}\setlength{\parskip}{0pt}}
\item[m.] A的列构成$\mathbb{R}^n$的一个基
\item[n.] $\Col A=\mathbb{R}^n$
\item[o.] $\dim\Col A=n$
\item[p.] $\rank A=n$
\item[q.] $\Nul A=\{\mathbf{0}\}$
\item[r.] $\dim\Nul A=0$
\end{list}
\end{TheoremTwo}\vspace{4ex}

\section{基的变换}
\begin{TheoremOne}
设$\mathcal{B}=\{\bm{b}_1,\cdots,\bm{b}_n\}$和$\mathcal{C}=\{\bm{c}_1,\cdots,\bm{c}_n\}$是向量空间V的基,则存在一个$n\times n$矩阵$\displaystyle\bm{\convert}_{\mathcal{C}\leftarrow\mathcal{B}}$使得
\[[x]_{\mathcal{C}}=\displaystyle\bm{\convert}_{\mathcal{C}\leftarrow\mathcal{B}}[\bm{x}]_{\mathcal{B}}\]
$\displaystyle\bm{\convert}_{\mathcal{C}\leftarrow\mathcal{B}}$的列是基$\mathcal{B}$中向量的$\mathcal{C}$-坐标向量,即
\[\displaystyle\bm{\convert}_{\mathcal{C}\leftarrow\mathcal{B}}=[[\bm{b}_1]_{\mathcal{C}}\quad[\bm{b}_2]_{\mathcal{C}}\quad\cdots\quad[\bm{b}_n]_{\mathcal{C}}]\]
\end{TheoremOne}\vspace{4ex}

\section{差分方程中的应用}
\begin{equation}
\left[
\begin{array}{l l l}
u_k & v_k & w_k\\
u_{k+1} & v_{k+1} & w_{k+1}\\
u_{k+2} & v_{k+2} & w_{k+2}
\end{array}
\right]\left[
\begin{array}{l}
c_1\\
c_2\\
c_3
\end{array}
\right]=\left[
\begin{array}{l}
0\\
0\\
0
\end{array}
\right]\qquad\text{对所有}k\text{成立}\label{eq:03}
\end{equation}\vspace{1ex}
这个方程组的系矩阵称为信号的\textbf{Casorati矩阵}\\
如果对至少一个$k$值Casorati矩阵可逆,则\eqref{eq:03}将蕴含$c_1=c_2=c_3=0$,这就证明这三个信号是线性无关的\\[2ex]

给定数$a_0$,$\cdots$,$a_n$,$a_0$和$a_n$不为零,给定一个信号$\{z_k\}$,方程
\[a_0y_{k+n}+a_1y_{k+n-1}+\cdots+a_{n-1}y_{k+1}+a_ny_k=z_k\qquad\text{对所有}k\text{成立}\]
称为一个\textbf{$n$阶线性差分方程}(或\textbf{线性递归关系})\\
若$\{z_k\}$是零序列,则方程是\textbf{齐次的};否则,方程为\textbf{非齐次的}\\[2ex]

\begin{TheoremOne}
若$a_n\neq 0$且$\{z_k\}$给定,只要$y_0$,$\cdots$,$y_{n-1}$给定,方程
\[y_{k+n}+a_1y_{k+n-1}+\cdots+a_{n-1}y_{k+1}+a_ny_k=z_k\]\qquad\text{对所有}k\text{成立}
有唯一解
\end{TheoremOne}\vspace{4ex}

\begin{TheoremOne}
$n$阶齐次线性差分方程
\[y_{k+n}+a_1y_{k+n-1}+\cdots+a_{n-1}y_{k+1}+a_ny_k=0\qquad\text{对所有}k\text{成立}\]
的解集H是一个$n$维向量空间
\end{TheoremOne}\vspace{4ex}

一个具有非负元素且各元素的数值相加等于1的向量称为\textbf{概率向量}\\[2ex]

各向量均为概率向量的方阵为\textbf{随机矩阵}\\[2ex]

\textbf{马尔科夫链}是一个概率向量序列$\bm{x}_0$,$\bm{x}_1$,$\bm{x}_2$,$\cdots$和一个随机矩阵$P$,满足
\[\bm{x}_1=P\bm{x}_0,\bm{x}_2=P\bm{x}_1,\bm{x}_3=P\bm{x}_2,\cdots\]
用一阶差分方程描述:
\[bm{x}_{k+1}=P\bm{x}_k,k=0,1,2,\cdots\]\\[2ex]

若P是一个随机矩阵,则相对于P的\textbf{稳态向量}(或\textbf{平衡向量})是一个满足\[P\bm{q=q}\]
的概率向量$\bm{q}$\\
每一个随机矩阵有一个稳态向量\\[2ex]

如果矩阵的某次幂$P^k$仅包含严格正的元素,则随机矩阵是正则的\\[2ex]

一个向量序列$\{\bm{x}_k:k=1,2,\cdots\}$当$k\to\infty$时\textbf{收敛}到一个向量$\bm{q}$,如果当$k$充分大时,$\bm{x}_k$中的元素无线接近$\bm{q}$中对应的元素\\[2ex]

\begin{TheoremOne}
若P是一个$n\times n$的正则随机矩阵, 则P具有唯一的稳态向量$\bm{q}$. 进一步,若$\bm{x}_0$是任一个初始状态, 且$\bm{x}_{k+1}=P\bm{x}_k,k=0,1,2,\cdots$, 则当$k\to\infty$时, 马尔科夫链$\{\bm{x}_k\}$收敛到$\bm{q}$
\end{TheoremOne}
