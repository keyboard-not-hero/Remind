\chapter{向量空间}
\section{向量空间和子空间}
\begin{definition}
一个向量空间是由一些被称为向量的对象构成的非空集合V, 在这个集合上定义两个运算, 称为加法和标量乘法(标量取实数), 服从以下公理(或法则), 这些公理必须对V中所有向量$\mathbf{u}$,$\mathbf{v}$,$\mathbf{w}$及所有标量c和d均成立.\\
1.\quad$\mathbf{u}$,$\mathbf{v}$之和表示为$\mathbf{u}+\mathbf{v}$, 仍在V中\\
2.\quad$\mathbf{u+v=v+u}$\\
3.\quad$\mathbf{(u+v)+w=u+(v+w)}$\\
4.\quad V中存在一个零向量$\mathbf{0}$, 使得$\mathbf{u+0=u}$\\
5.\quad 对V中每个向量$\mathbf{u}$, 存在V中向量$-\mathbf{u}$, 使得$\mathbf{u+(-u)=0}$\\
6.\quad$\mathbf{u}$与标量c的标量乘法记为$c\mathbf{u}$, 仍在V中\\
7.\quad$c\mathbf{(u+v)}=c\mathbf{u}+c\mathbf{v}$\\
8.\quad$(c+d)\mathbf{u}=c\mathbf{u}+d\mathbf{u}$\\
9.\quad$c(d\mathbf{u})=(cd)\mathbf{u}$\\
10.\quad$1\mathbf{u=u}$\\
\end{definition}\vspace{4ex}

{\par\raggedright
\framebox{\begin{minipage}{\textwidth}
对V中每个向量$\mathbf{u}$和任意标量$c$,有
\[0\mathbf{u=0}\]
\[c\mathbf{0=0}\]
\[-\mathbf{u}=(-1)\mathbf{u}\]
\end{minipage}}
\par}\vspace{4ex}

\begin{definition}
向量空间V的一个\textbf{子空间}是V的一个满足以下三个性质的子集H:
\begin{enumerate}
\item V中的零向量在H中
\item H对向量加法封闭,即对H中任意向量$\mathbf{u}$,$\mathbf{v}$,和$\mathbf{u+v}$仍在H中
\item H对标量乘法封闭,即对H中任意向量$\mathbf{u}$和任意标量$c$,向量$c\mathbf{u}$仍在H中
\end{enumerate}
\end{definition}\vspace{4ex}

\begin{theorem}
若$v_1$,$v_2$,$\cdots$,$v_p$在向量空间V中,则$\Span\{v_1,\cdots,v_p\}$是V的一个子空间
\end{theorem}

\section{零空间、列空间和线性变换}
考虑下列齐次方程组:
\begin{equation}
\begin{array}{r@{\hspace{0pt}}l@{\hspace{0pt}}r@{\hspace{0pt}}l@{\hspace{0pt}}r@{\hspace{0pt}}l}
x_1 & - & 3x_2 & - & 2x_3 & =0\\
-5x_1 & + & 9x_2 & + & x_3 & =0
\end{array}\label{eq:01}
\end{equation}
用矩阵的形式,此方程组可写成$A\mathbf{x=0}$,其中
\[A=\left[\begin{array}{r r r}
1 & -3 & -2\\
-5 & 9 & 1
\end{array}\right]\]
所有满足\eqref{eq:01}的$\mathbf{x}$的集合称为方程组\eqref{eq:01}的\textbf{解集}\\
我们成满足$A\mathbf{x=0}$的所有$\mathbf{x}$的集合为矩阵A的\textbf{零空间}\\[2ex]

\begin{definition}
矩阵A的零空间写成$\Nul A$,是齐次方程$A\mathbf{x=0}$的全体解的集合. 用集合符号表示,即
\[\Nul A=\{\mathbf{x}:\mathbf{x}\in\mathbb{R}^n, A\mathbf{x=0}\}\]
\end{definition}\vspace{4ex}

\begin{theorem}
$m\times n$矩阵A的零空间是$\mathbb{R}^n$的一个子空间. 等价地,$m$个方程,$n$个未知数的齐次线性方程组$A\mathbf{x=0}$的全体解的集合是$\mathbb{R}^n$的一个子空间
\end{theorem}\vspace{4ex}

\begin{definition}
$m\times n$矩阵A的\textbf{列空间}(记为$\Col A$)是由A的列的所有线性组合组成的集合. 若$A=[\mathbf{a}_1,\cdots,\mathbf{a}_n]$,则$\Col A=\Span\{\mathbf{a}_1,\cdots,\mathbf{a}_n\}$
\end{definition}\vspace{4ex}

\begin{theorem}
$m\times n$矩阵A的列空间是$\mathbb{R}^m$的一个子空间
\end{theorem}\vspace{4ex}

{\par\raggedright
\framebox{\begin{minipage}{\textwidth}
$m\times n$矩阵A的列空间等于$\mathbb{R}^m$当且仅当方程$A\mathbf{x=b}$对$\mathbb{R}^m$中每个$\mathbf{b}$有一个解
\end{minipage}}
\par}\vspace{4ex}

\begin{table}[H]
\caption{对$m\times n$矩阵A,$\Nul A$与$\Col A$之间的对比}
\begin{tabular}{>{\scriptsize}p{6cm}|>{\scriptsize}p{6cm}}
\hline
$\Nul A$ & $\Col A$\\
\hline
\begin{enumerate}
\item $\Nul A$是$\mathbb{R}^n$的一个子空间
\item $\Nul A$是隐式定义的,即仅给出了一个$\Nul A$中向量必须满足的条件($A\mathbf{x=0}$)
\item 求$\Nul A$中的向量需要时间,需要对$[A\quad\mathbf{0}]$作行变换
\item $\Nul A$与A的元素之间没有明显的关系
\item $\Nul A$中的一个典型向量$\mathbf{v}$具有$A\mathbf{v=0}$的性质
\item 给定一个特定的向量$\mathbf{v}$,容易判断$\mathbf{v}$是否在$\Nul A$中. 仅需计算$A\mathbf{v}$
\item $\Nul A=\{\mathbf{0}\}$当且仅当方程$A\mathbf{x=0}$仅有一个平凡解
\item $\Nul A=\{\mathbf{0}\}$当且仅当线性变换$\mathbf{x}\mapsto A\mathbf{x}$是一对一的
\end{enumerate} & \begin{enumerate}
\item $\Col A$是$\mathbb{R}^m$的一个子空间
\item $\Col A$是显式定义的,即明确指出如何构建$\Col A$中的向量
\item 容易求出$\Col A$中的向量. A的列就是$\Col A$中的向量,其余的可由A的列表示出来
\item $\Col A$与A的元素之间有明显的关系,因为A的列就在$\Col A$中
\item $\Col A$中一个典型向量$\mathbf{v}$具有方程$A\mathbf{x=v}$是相容的性质
\item 给定一个特定的向量$\mathbf{v}$,弄清$\mathbf{v}$是否在$\Col A$中需要时间,需要对$[A\quad\mathbf{v}]$作行变换
\item $\Col A=\mathbb{R}^m$当且仅当方程$A\mathbf{x=b}$对每一个$\mathbf{b}\in\mathbb{R}^m$有一个解
\item $\Col A=\mathbb{R}^m$当且仅当线性变换$\mathbf{x}\mapsto A\mathbf{x}$将$\mathbb{R}^n$映上到$\mathbb{R}^m$
\end{enumerate}\\
\hline
\end{tabular}
\end{table}
