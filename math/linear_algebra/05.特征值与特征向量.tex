\chapter{特征值与特征向量}
\section{特征向量与特征值}
\begin{definition}
A为$n\times n$矩阵,$x$为非零向量,若存在数$\lambda$使$A\bm{x}=\lambda\bm{x}$有非平凡解$\bm{x}$,则称$\lambda$为A的特征值,$\bm{x}$称为对应于$\lambda$的特征向量
\end{definition}\vspace{4ex}

$\lambda$是A的特征值当且仅当方程
\begin{equation}
(A-\lambda I)\bm{x}=\mathbf{0}\label{eq:04}
\end{equation}
有非平凡解\\
方程\eqref{eq:04}的所有解的集合就是矩阵$A-\lambda I$的零空间\\
该集合是$\mathbb{R}^n$的子空间,称为A的对应于$\lambda$的\textbf{特征空间}\\
特征空间由零向量和所有对应于$\lambda$的特征向量组成\\[2ex]

\begin{TheoremOne}
三角矩阵的主对角线的元素是其特征值
\end{TheoremOne}\vspace{4ex}

\begin{TheoremOne}
$\lambda_1$,$\cdots$,$\lambda_r$是$n\times n$矩阵A相异的特征值,$\bm{v}_1$,$\cdots$,$\bm{v}_r$是与$\lambda_1$,$\cdots$,$\lambda_r$对应的特征向量,那么向量集合$\{\bm{v}_1,\cdots,\bm{v}_r\}$线性无关
\end{TheoremOne}\vspace{4ex}

\section{特征方程}
设A是$n\times n$矩阵,$U$是对A作行替换和行交换(不作行倍乘)所得到的任一阶梯形矩阵,$r$是行交换的次数,那么A的行列式$\det A=(-1)^ru_{11}\cdots u_{nn}$. 如果A可逆,那么$u_{11}$,$\cdots$,$u_{nn}$都是主元(因为$A\sim I_n$且$u_{ii}$没有归一化). 否则,至少有$u_{nn}$为零,从而乘积$u_{11}\cdots u_{nn}$为零. 因此
\[\det A=\left\{
\begin{array}{l l}
(-1)^ru_{11}\cdots u_{nn} & 当A可逆\\
0 & 当A不可逆
\end{array}
\right.\]\\[2ex]

\begin{TheoremTwo}[可逆矩阵定理(续)]
设A是$n\times n$矩阵,则A是可逆的当且仅当\\
s.0不是A的特征值\\
t.A的行列式不等于零
\end{TheoremTwo}\vspace{4ex}

\begin{TheoremTwo}[行列式的性质]
设A和B是$n\times n$矩阵\\
a.A可逆的充要条件是$\det A\neq 0$\\
b.$\det AB=(\det A)(\det B)$\\
c.$\det A^T=\det A$\\
d.若A是三角形矩阵,那么$\det A$是A主对角线元素的乘积\\
e.对A作行替换不改变其行列式值. 作一次行交换,行列式值符号改变一次. 数乘一行后,行列式值等于用此数乘原来的行列式值
\end{TheoremTwo}\vspace{4ex}

数值方程$\det(A-\lambda I)=0$称为A的\textbf{特征方程}\\[2ex]

{\par\centering
\framebox{\begin{minipage}{\textwidth}
数$\lambda$是$n\times n$矩阵A的特征值的充要条件是$\lambda$是特征方程$\det(A-\lambda I)=0$的根
\end{minipage}}
\par}\vspace{4ex}

如果A是$n\times n$矩阵,那么$\det(A-\lambda I)$是$n$次多项式,称为A的\textbf{特征多项式}\\[1ex]

把特征值$\lambda$作为特征方程根出现的次数称为$\lambda$的\textbf{(代数)重数}\\[2ex]

假如A和B是$n\times n$矩阵,如果存在可逆矩阵$P$,使得$P_{-1}AP=B$,或等价地$A=PBP_{-1}$,则称\textbf{A相似于B}. 或简单说A和B是\textbf{相似的}. 把A变成$P_{-1}AP$的变换称为\textbf{相似变换}\\[2ex]

\begin{TheoremOne}
若$n\times n$矩阵A和B是相似的,那么它们有相同的特征多项式,从而有相同的特征值(和相同的重数)
\end{TheoremOne}\vspace{4ex}

\section{对角化}
若
\[D=\left[
\begin{array}{r r}
5 & 0\\
0 & 3
\end{array}
\right]\]
则
\[\text{对}k\geqslant 1,D^k=\left[
\begin{array}{r r}
5^k & 0\\
0 & 3^k
\end{array}
\right]\]\\[2ex]

若给定\\
$A=PDP^{-1}$\\
因此\\
$A^2=(PDP^{-1})(PDP^{-1})=PD(P^{-1}P)DP^{-1}=PD^2P^{-1}$\\
同理\\
$A^3=(PDP^{-1})A^2=(PDP^{-1})(PD^2P^{-1})=PD(P^{-1}P)D^2P^{-1}=PD^3P^{-1}$\\
一般对$k\geqslant 1$,有\\
\[A^k=PD^KP^{-1}\]\\[2ex]

如果方阵A相似于对角矩阵,即存在可逆矩阵$P$和对角矩阵$D$,有$A=PDP^{-1}$,则称A\textbf{可对角化}\\[2ex]

\begin{TheoremTwo}[对角化定理]
$n\times n$矩阵A可对角化的充分必要条件是A有$n$个线性无关的特征向量\\
事实上,$A=PDP^{-1}$,$D$为对角矩阵的充分必要条件是$P$的列向量是A的$n$个线性无关的特征向量. 此时,$D$的主对角线上的元素分别是A的对应于$P$中特征向量的特征值
\end{TheoremTwo}\vspace{4ex}

A可对角化的充分必要条件是有足够的特征向量形成$\mathbb{R}^n$的基,我们称这样的基为\textbf{特征向量基}\\[2ex]

对角化步骤:\\
1.求出A的特征值\\
2.求A的三个线性无关的特征向量\\
3.使用特征向量构造矩阵P\\
4.用与特征向量顺序对应的特征值构造矩阵D\\[2ex]

\begin{TheoremOne}
有$n$个相异特征值的$n\times n$矩阵可对角化
\end{TheoremOne}\vspace{4ex}

\begin{TheoremOne}
设A是$n\times n$矩阵,其相异的特征值是$\lambda_1$,$\cdots$,$\lambda_p$\\
a.对于$a\leqslant k\leqslant p$,$\lambda_k$的特征空间的维数小于或等于$\lambda_k$的代数重数\\
b.矩阵A可对角化的充分必要条件是所有不同特征空间的维数之和为$n$. 即\\
\phantom{\quad}(i)特征多项式可完全分解为线性因子\\
\phantom{\quad}(ii)每个$\lambda_k$的特征空间的维数等于$\lambda_k$的代数重数\\
c.若A可对角化,$\mathcal{B}_k$是对应于$\lambda_k$的特征空间的基,则集合$\mathcal{B}_1$,$\cdots$,$\mathcal{B}_p$中所有向量的集合是$\mathbb{R}^n$的特征向量基
\end{TheoremOne}\vspace{4ex}

\section{特征向量与线性变换}
设V是$n$维向量空间,W是$m$维向量空间,T是V到W的线性变换. V的基$\mathcal{B}$是$\{\bm{b}_1,\cdots,\bm{b}_n\}$. 若$\bm{x}=r_1\bm{b}_1+\cdots+r_n\bm{b}_n$,则
\[[\bm{x}]_{\mathcal{B}}=\left[\begin{array}{c}
r_1\\
\vdots\\
r_n
\end{array}\right]\]
因为T是线性的,故
\begin{equation}
T(\bm{x})=T(r_1\bm{b}_1+\cdots+r_n\bm{b}_n)=r_1T(\bm{b}_1)+\cdots+r_nT(\bm{b}_n)\label{eq:05}
\end{equation}
因为从W到$\mathbb{R}^m$的坐标映射是线性的,故等式\eqref{eq:05}可推出
\begin{equation}
[T(\bm{x})]_{\mathcal{C}}=r_1[T(\bm{b}_1)]_{\mathcal{C}}+\cdots+r_n[T(\bm{b}_n)]_{\mathcal{C}}\label{eq:06}
\end{equation}
因为这些$\mathcal{C}$-坐标向量都属于$\mathbb{R}^m$,故向量等式\eqref{eq:06}可以写为矩阵等式
\[[T(\bm{x})]_{\mathcal{C}}=M[\bm{x}]_{\mathcal{B}}\]
其中
\begin{equation}
M=[[T(\bm{b}_1)]_{\mathcal{C}}\quad[T(\bm{b}_2)]_{\mathcal{C}}\quad\cdots\quad[T(\bm{b}_n)]_{\mathcal{C}}]\label{eq:07}
\end{equation}
矩阵M是T的矩阵表示,称为\textbf{T相对于基$\mathcal{B}$和$\mathcal{C}$的矩阵}\\[2ex]

当W=V,$\mathcal{C=B}$时,\eqref{eq:07}中的M称为\textbf{T相对于$\mathcal{B}$的矩阵},或简称为\textbf{T的$\mathcal{B}$-矩阵},记为$[T]_{\mathcal{B}}$\\[2ex]

\begin{TheoremTwo}[(对角矩阵表示)]
设$A=PDP^{-1}$,其中$D$为$n\times n$对角矩阵,若$\mathbb{R}^n$的基$\mathcal{B}$由$P$的列向量组成,那么$D$是变换$\bm{x}\mapsto A\bm{x}$的$\mathcal{B}$-矩阵
\end{TheoremTwo}\vspace{4ex}

\section{复特征值}
一个复数$\lambda$满足$\det(A-\lambda I)=0$当且仅当在$\mathbb{C}^n$中存在一个非零向量$\bm{x}$,使得$A\bm{x}=\lambda\bm{x}$. 我们称这样的$\lambda$是(复)特征值,$\bm{x}$是对应于$\lambda$的(复)特征向量\\[2ex]

$\mathbb{C}^n$中复向量$\bm{x}$的共轭向量$\bar{\bm{x}}$也是$\mathbb{C}^n$中的向量,它的分量是$\bm{x}$中对应分量的共轭复数,向量$\real\bm{x}$和$\imag\bm{x}$称为复向量$\bm{x}$的\textbf{实部}和\textbf{虚部},分别由$\bm{x}$的分量的实部和虚部组成\\[2ex]

\begin{TheoremOne}
设A是$2\times 2$实矩阵,有复特征值$\lambda=a-bi(b\neq 0)$及对应的$\mathbb{C}^2$中的复特征向量$\bm{v}$,那么
\[A=PCP^{-1},\quad\text{其中}P=[\real\bm{v}\quad\imag\bm{v}],C=\left[\begin{array}{r r}a & -b\\b & a\end{array}\right]\]
\end{TheoremOne}
