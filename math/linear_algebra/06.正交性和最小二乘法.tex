\chapter{正交性和最小二乘法}
\section{内积、长度和正交性}
如果$\bm{u}$和$\bm{v}$是$\mathbb{R}^n$中的向量,则可以将$\bm{u}$和$\bm{v}$作为$n\times 1$矩阵. 转置矩阵$\bm{u}^T$是$1\times n$矩阵,且矩阵乘积$\bm{u}^T\bm{v}$是一个$1\times 1$矩阵,我们将其记为一个不加括号的实数(标量). 数$\bm{u}^T\bm{v}$称为$\bm{u}$和$\bm{v}$的\textbf{内积},通常记作$\bm{u}\cdot\bm{v}$,也称为\textbf{点积}\\[2ex]

\begin{TheoremOne}
设$\bm{v}$,$\bm{u}$和$\bm{w}$是$\mathbb{R}^n$中的向量,$c$是一个数,那么\\
a.\ $\bm{u\cdot v=v\cdot u}$\\
b.\ $\bm{(u+v)\cdot w=u\cdot w+v\cdot w}$\\
c.\ $(c\bm{u})\cdot v=c(\bm{u\cdot v})=\bm{u}\cdot(c\bm{v})$\\
d.\ $\bm{u\cdot u}\geqslant 0$,并且$\bm{u\cdot u}=0$成立的充分必要条件是$\bm{u=0}$
\end{TheoremOne}\vspace{4ex}

\begin{definition}
向量$\bm{v}$的\textbf{长度}(或\textbf{范数})是非负数$||\bm{v}||$,定义为
\[||\bm{v}||=\sqrt{\bm{v}\cdot \bm{v}}=\sqrt{v_1^2+v_2^2+\cdots+v_n^2}\text{\ 且\ }||\bm{v}||^2=\bm{v}\cdot \bm{v}\]
\end{definition}\vspace{4ex}

长度为1的向量称为\textbf{单位向量}. 如果把一个非零向量除以其自身的长度,即乘$\dfrac{1}{||\bm{v}||}$,就可以得到一个单位向量,即$\bm{u}=\dfrac{\bm{v}}{||\bm{v}||}$\\[1ex]

把向量$\bm{v}$化成单位向量$\bm{u}$的过程,称为向量$\bm{v}$的\textbf{单位化}\\[2ex]

\begin{definition}
如果$\bm{u\cdot v}=0$,则$\mathbb{R}^n$中的两个向量$\bm{u}$和$\bm{v}$是(相互)\textbf{正交的}
\end{definition}\vspace{4ex}

\begin{TheoremTwo}[毕达哥拉斯(勾股)定理]
两个向量$\bm{u}$和$\bm{v}$正交的充分必要条件是$||\bm{u+v}||^2=||\bm{u}||^2+||\bm{v}||^2$
\end{TheoremTwo}\vspace{4ex}

如果向量$\bm{z}$与$\mathbb{R}^n$的子空间W中的任意向量都正交,则称$\bm{z}$\textbf{正交于W}\\[1ex]

与子空间W正交的向量$\bm{z}$的全体组成的集合称为W的\textbf{正交补},记作$W^\bot$\\[2ex]

{\par\centering
\framebox{\begin{minipage}{\textwidth}
1.向量$\bm{x}$属于$W^\bot$的充分必要条件是向量$\bm{x}$与生成空间W的任一向量都正交\\
2.$W^\bot$是$\mathbb{R}^n$的一个子空间
\end{minipage}}
\par}\vspace{4ex}

\begin{TheoremOne}
假设A是$m\times n$矩阵,那么A的行空间的正交补是A的零空间,且A的列空间的正交补是$A^T$的零空间:
\[(\Row A)^\bot=\Nul A\text{\ 且\ }(\Col A)^\bot=\Nul A^T\]
\end{TheoremOne}\vspace{4ex}

\section{正交集}
$\mathbb{R}^n$中的向量集合$\{\bm{u}_1,\cdots,\bm{u}_p\}$称为\textbf{正交集},如果集合中的任意两个不同向量都正交,即当$i\neq j$时,$\bm{u_i\cdot u_j}=0$\\[2ex]

\begin{TheoremOne}
如果$S=\{\bm{u}_1,\cdots,\bm{u}_p\}$是由$\mathbb{R}^n$中非零向量构成的正交集,那么$S$是线性无关集,因此构成$S$所生成的子空间的一组基
\end{TheoremOne}\vspace{4ex}

\begin{definition}
$\mathbb{R}^n$中子空间W的一个\textbf{正交基}是W的一个基,也是正交集
\end{definition}\vspace{4ex}

\begin{TheoremOne}
假设$\{\bm{u}_1,\cdots,\bm{u}_p\}$是$\mathbb{R}^n$中子空间W的正交基,对W中的每个向量$\bm{y}$,线性组合$\bm{y}=c_1\bm{u}_1+\cdots+c_p\bm{u}_p$中的权可以由$c_j=\dfrac{\bm{y\cdot u_j}}{\bm{u_j\cdot u_j}}$($j=1,\cdots,p$)计算
\end{TheoremOne}\vspace{4ex}

对$\mathbb{R}^n$中给出的非零向量$\bm{u}$,考虑$\mathbb{R}^n$中一个向量$\bm{y}$分解为两个向量之和的问题,一个向量是向量$\bm{u}$的倍数,另一个向量与$\bm{u}$正交. 我们期望写成
\begin{equation}
\bm{y}=\hat{\bm{y}}+\bm{z}\label{eq:08}
\end{equation}
其中$\hat{\bm{y}}=\alpha\bm{u}$,$\alpha$是一个数,$\bm{z}$是一个垂直于$\bm{u}$的向量. 对给定数$\alpha$,记$\bm{z}=\bm{y}-\alpha\bm{u}$,则方程\eqref{eq:08}可以满足. 那么$\bm{y}-\hat{\bm{y}}$和$\bm{u}$正交的充分必要条件是
\[(\bm{y}-\alpha\bm{u})\cdot\bm{u}=\bm{y\cdot u}-(\alpha\bm{u})\cdot\bm{u}=\bm{y\cdot u}-\alpha(\bm{u\cdot u})=0\]
也就是满足方程\eqref{eq:08}且$\bm{z}$与$\bm{u}$正交的充分必要条件是$\alpha=\dfrac{\bm{y\cdot u}}{\bm{u\cdot u}}$和$\hat{\bm{y}}=\proj_L\bm{y}=\dfrac{\bm{y\cdot u}}{\bm{u\cdot u}}\cdot\bm{u}$. 向量$\hat{\bm{y}}$称为\textbf{$\bm{y}$在$\bm{u}$上的正交投影},向量$\bm{z}$称为\textbf{$\bm{y}$与$\bm{u}$正交的分量}\\[2ex]

如果集合$\{\bm{u}_1,\cdots,\bm{u}_p\}$是由单位向量构成的正交集,那么它是一个\textbf{单位正交集}\\[1ex]

如果W是一个由单位正交集合生成的子空间,那么$\{\bm{u}_1,\cdots,\bm{u}_p\}$是W的\textbf{单位正交基}\\[2ex]

\begin{TheoremOne}
一个$m\times n$矩阵$U$具有单位正交列向量的充分必要条件是$U^TU=I$
\end{TheoremOne}\vspace{4ex}

\begin{TheoremOne}
假设$U$是一个具有单位正交列的$m\times n$矩阵,且$\bm{x}$和$\bm{y}$是$\mathbb{R}^n$中的向量,那么\\
a.\ $||U\bm{x}||=||\bm{x}||$\\
b.\ $(U\bm{x})\cdot(U\bm{y})=\bm{x\cdot y}$\\
c.\ $(U\bm{x})\cdot(U\bm{y})=0$的充分必要条件是$\bm{x}\cdot\bm{y}=0$
\end{TheoremOne}\vspace{4ex}

\section{正交投影}
对给定向量$\bm{y}$和$\mathbb{R}^n$中子空间W,存在属于W的向量$\hat{\bm{y}}$满足:\\
(1)\ W中有唯一向量$\hat{\bm{y}}$,使得$\bm{y}-\hat{\bm{y}}$与W正交\\
(2)\ $\hat{\bm{y}}$是W中唯一最接近$\bm{y}$的向量\\[2ex]

\begin{TheoremTwo}[正交分解定理]
若W是$\mathbb{R}^n$的一个子空间,那么$\mathbb{R}^n$中每一个向量$\bm{y}$可以唯一表示为
\[\bm{y}=\hat{\bm{y}}+\bm{z}\]
其中$\hat{\bm{y}}$属于W而$\bm{z}$属于$W^\bot$. 实际上,如果$\{\bm{u}_1,\cdots,\bm{u}_p\}$是W的任意正交基,那么
\[\hat{\bm{y}}=\frac{\bm{y\cdot u_1}}{\bm{u_1\cdot u_1}}\bm{u}_1+\cdots+\frac{\bm{y\cdot u_p}}{\bm{u_p\cdot u_p}}\bm{u}_p\]
且$\bm{z}=\bm{y}-\hat{\bm{y}}$
\end{TheoremTwo}\vspace{4ex}

\begin{TheoremTwo}[最佳逼近定理]
假设W是$\mathbb{R}^n$的一个子空间,$\bm{y}$是$\mathbb{R}^n$中的任意向量,$\hat{\bm{y}}$是$\bm{y}$在W上的正交投影,那么$\hat{\bm{y}}$是W中最接近$\bm{y}$的点,也就是
\[||\bm{y}-\hat{\bm{y}}||<||\bm{y}-\bm{v}||\]
对所有属于W又异于$\hat{\bm{y}}$的$\bm{v}$成立
\end{TheoremTwo}\vspace{4ex}

\begin{TheoremOne}
如果$\{\bm{u}_1,\cdots,\bm{u}_p\}$是$\mathbb{R}^n$中子空间W的单位正交基,那么
\[\proj_W\bm{y}=(\bm{y}\cdot\bm{u}_1)\bm{u}_1+(\bm{y}\cdot\bm{u}_2)\bm{u}_2+\cdots+(\bm{y}\cdot\bm{u}_p)\bm{u}_p\]
如果$U=[\bm{u}_1\quad\bm{u}_2\quad\cdots\quad\bm{u}_p]$,则
\[\proj_W\bm{y}=UU^T\bm{y},\text{\ 对所有}\bm{y}\in\mathbb{R}^n\text{成立}\]
\end{TheoremOne}\vspace{4ex}

\section{格拉姆-施密特方法}
\begin{TheoremTwo}[格拉姆-施密特方法]
对$\mathbb{R}^n$的子空间W的一个基$\{\bm{x}_1,\cdots,\bm{x}_p\}$,定义
\[\begin{array}{>{\displaystyle}r >{\displaystyle}l}
\bm{v}_1 & =\bm{x}_1\\[1ex]
\bm{v}_2 & =\bm{x}_2-\frac{\bm{x}_2\cdot\bm{v}_1}{\bm{v}_1\cdot\bm{v}_1}\bm{v}_1\\[2ex]
\bm{v}_3 & =\bm{x}_3-\frac{\bm{x}_3\cdot\bm{v}_2}{\bm{v}_2\cdot\bm{v}_2}\bm{v}_2-\frac{\bm{x}_3\cdot\bm{v}_1}{\bm{v}_1\cdot\bm{v}_1}\bm{v}_1\\[1ex]
\vdots &\\[1ex]
\bm{v}_p & =\bm{x}_p-\frac{\bm{x}_p\cdot\bm{v}_{p-1}}{\bm{v}_{p-1}\cdot\bm{v}_{p-1}}\bm{v}_{p-1}-\cdots-\frac{\bm{x}_p\cdot\bm{v}_2}{\bm{v}_2\cdot\bm{v}_2}\bm{v}_2-\frac{\bm{x}_p\cdot\bm{v}_1}{\bm{v}_1\cdot\bm{v}_1}\bm{v}_1
\end{array}\]
那么$\{\bm{v}_1,\cdots,\bm{v}_p\}$是W的一个正交基. 此外,
\[\Span\{\bm{v}_1,\cdots,\bm{v}_k\}=\Span\{\bm{x}_1,\cdots,\bm{x}_k\},\text{\ 其中}1\leqslant k\leqslant p\]
\end{TheoremTwo}\vspace{4ex}

\begin{TheoremTwo}[QR分解]
如果$m\times n$矩阵A的列线性无关,那么A可以分解为A=QR,其中Q是一个$m\times n$矩阵,其列形成$\Col A$的一个标准正交基,R是一个$n\times n$上三角可逆矩阵且在对角线上的元素为正数
\end{TheoremTwo}\vspace{4ex}

\section{最小二乘问题}

