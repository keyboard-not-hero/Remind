\chapter{积分的方法 I}
1.换元法(链式求导的逆)
{\par\centering
\framebox{$\displaystyle\int f'(x)g(f(x))\dif x=h(f(x))+C$.}
\par}
例1.\\[1ex]
$\displaystyle\int\frac{x}{x^2+8}\dif x$\\[1ex]
推导过程:\\[1ex]
$\displaystyle\because\frac{\dif}{\dif x}(x^2+8)=2x$\\[1ex]
$\therefore$使用换元法, 设$t=x^2+8$.\\[1ex]
\phantom{$\therefore$}得到$\dif t=2x\dif x$\\[1ex]
$\therefore\displaystyle\int\frac{x}{x^2+8}\dif x=\frac{1}{2}\int\frac{1}{t}\dif t=\frac{1}{2}\ln|t|+C=\frac{1}{2}\ln|x^2+8|+C$\\[2ex]

例2.\\[1ex]
$\displaystyle\int\frac{3x^2+7}{x^3+7x+9}\dif x$\\[1ex]
推导过程:\\[1ex]
$\displaystyle\because\frac{\dif}{\dif x}(x^3+7x-9)=3x^2+7$\\[1ex]
$\therefore$使用换元法, 设$t=x^3+7x-9$.\\[1ex]
\phantom{$\therefore$}得到$\dif t=(3x^2+7)\dif x$\\[1ex]
$\therefore\displaystyle\int\frac{3x^2+7}{x^3+7x-9}\dif x=\frac{1}{t}=\ln|t|+C=\ln|x^3+7x-9|+C$\\[2ex]

2.分部积分法(乘法求导法则的逆)
{\par\centering
\framebox{$\displaystyle\int u\frac{\dif v}{\dif x}\dif x=uv-\int v\frac{\dif u}{\dif x}\dif x$.}
\par}
例1.\\[1ex]
$\displaystyle\int xe^x\dif x$\\[1ex]
推导过程:\\[1ex]
设$u=x$, $\frac{\dif v}{\dif x}=e^x$, 则:\\[1ex]
\phantom{设}$v=e^x$, $\frac{\dif u}{\dif x}=1$\\[1ex]
$\displaystyle\int xe^x\dif x=xe^x-\int e^x\dif x$\\[1ex]
$\displaystyle\int xe^x\dif x=xe^x-e^x+C$\\[2ex]

例2.\\[1ex]
$\displaystyle\int\ln x\dif x$\\[1ex]
推导过程:\\[1ex]
设$u=\ln x$, $\frac{\dif v}{\dif x}=1$, 则:\\[1ex]
\phantom{设}$v=x$, $\frac{\dif u}{\dif x}=\frac{1}{x}$\\[1ex]
$\displaystyle\int\ln x\dif x=x\ln x-\int x\cdot\frac{1}{x}\dif x=x\ln x-x+C$\\[1ex]
总结:\\[1ex]
1.内容存在$\sin$/$\cos$或$e^x$时, 视为已微分部分\\[1ex]
2.内容存在$\ln$时, 视为未微分部分\\[1ex]
3.内容不存在乘积时, 整个内容视为未微分部分\\[70ex]

3.部分分式(有理函数积分)\\
有理函数: 形如$\frac{P(x)}{Q(x)}$的函数
{\par\centering
\framebox{
\begin{minipage}{\linewidth}
部分分式处理步骤:\\
(1)确保分子的次数小于分母的次数, 如有必要(分子次数$\geqslant$分母次数)做除法;\\
(2)对分母进行因式分解;\\
(3)进行"分部"(将分母按隐式分解进行拆分). 分部类别如下:\\
	\phantom{\qquad}1)\,线性式:
		\[\frac{mx+n}{(x+a)(x+b)}=\frac{A}{x+a}+\frac{B}{x+a}\]
	\phantom{\qquad}2)\,线性式的平方:
		\[\frac{mx+b}{(x+a)^2}=\frac{A}{(x+a)^2}+\frac{B}{x+a}\]
	\phantom{\qquad}3)\,二次多项式:
		\[\frac{Ax+B}{x^2+ax+b}\]
	\phantom{\qquad}4)\,线性式的三次方:
		\[\frac{mx^2+nx+k}{(x+a)^3}=\frac{A}{(x+a)^3}+\frac{B}{(x+a)^2}+\frac{C}{x+a}\]
	\phantom{\qquad}5)\,线性式的四次方:
		\[\frac{mx^3+nx^2+kx+h}{(x+a)^4}=\frac{A}{(x+a)^4}+\frac{B}{(x+a)^3}+\frac{C}{(x+a)^2}+\frac{D}{x+a}\]
(4)对各分部进行通分, 计算分部中A/B/C/D等常数项的值;\\
(5)对各分部进行积分\\
\phantom{\qquad}1)分母为单项, 该项为$x$的幂 - 积分结果为对数或$x$的负次幂;\\
\phantom{\qquad}2)分母为二次多项式 - 对分母先进行配方, 然后进行换元, 最后根据分子的项数, 拆分为多个积分
		\[\int\frac{1}{x^2+a^2}\dif x=\frac{1}{a}\tan^{-1}(\frac{x}{a})+C\]
\end{minipage}}
\par}
例.\\[1ex]
$\displaystyle\int\frac{x+2}{x^2-1}\dif x$\\[1ex]
推导过程:\\[1ex]
对分母$x^2-1$进行因式分解:\\[1ex]
$x^2-1=(x+1)(x-1)$\\[1ex]
进行分部:\\[1ex]
$\displaystyle\frac{x+2}{x^2-1}=\frac{A}{x+1}+\frac{B}{x-1}$\\[1ex]
求分子常数的值:\\[1ex]
$\displaystyle A(x-1)+B(x+1)=x+2\quad\Rightarrow\quad\left\{
\begin{array}{l}
A+B=1\\
B-A=2
\end{array}\right.\quad\Rightarrow\quad A=-\frac{1}{2},B=\frac{3}{2}$\\[1ex]
求解分母为线性次幂的积分:\\[1ex]
$\displaystyle\int\frac{x+2}{x^2-1}\dif x=\frac{3}{2}\int\frac{1}{x-1}\dif x-\frac{1}{2}\int\frac{1}{x+1}\dif x=\frac{3}{2}\ln|x-1|-\frac{1}{2}\ln|x+1|+C$
