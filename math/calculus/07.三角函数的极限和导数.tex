\chapter{三角函数的极限和导数}
1.三角函数的极限:\\
(1)$x\to 0$的情况\\[2ex]
\framebox{$\displaystyle\lim_{x\to 0}\frac{\sin x}{x}=1$}\\[2ex]
\framebox{$\displaystyle\lim_{x\to 0}\frac{\tan x}{x}=1$}\\[2ex]
\framebox{$\displaystyle\lim_{x\to 0}\cos x=1$}\\[2ex]
\framebox{$\displaystyle\lim_{x\to 0}\frac{1-\cos x}{x}=0$}\\[2ex]
例1.\\
\phantom{例}$\displaystyle\lim_{x\to 0}\frac{\sin5x}{x}$\\
$\displaystyle\lim_{x\to 0}\frac{\sin5x}{x}=\lim_{x\to 0}\frac{\frac{\sin5x}{5x}}{x}\times5x=1\times5=5$\\

例2.\\
\phantom{例}$\displaystyle\lim_{x\to 0}\frac{\sin^3(2x)\cos(5x^{19})}{x\tan(5x^2)}$
$\displaystyle\lim_{x\to 0}\frac{\sin^3(2x)\cos(5x^{19})}{x\tan(5x^2)}=\lim_{x\to 0}\frac{[\frac{(\sin(2x))^3}{(2x)^3}\times (2x)^3]\cos(5x^{19})}{x[\frac{\tan(5x^2)}{5x^2}\times (5x^2)]}$\\
$\displaystyle\phantom{\lim_{x\to 0}\frac{\sin^3(2x)\cos(5x^{19})}{x\tan(5x^2)}}=\lim_{x\to 0}\frac{\frac{(\sin(2x))^3}{(2x)^3}\cdot\cos(5x^{19})}{\frac{\tan(5x^2)}{5x^2}}\times\frac{(2x)^3}{x(5x^2)}$\\
$\displaystyle\phantom{\lim_{x\to 0}\frac{\sin^3(2x)\cos(5x^{19})}{x\tan(5x^2)}}=\lim_{x\to 0}\frac{8x^3}{5x^3}$\\
$\displaystyle\phantom{\lim_{x\to 0}\frac{\sin^3(2x)\cos(5x^{19})}{x\tan(5x^2)}}=\frac{8}{5}$\\

例3.\\
\phantom{例}$\displaystyle\lim_{x\to 0}x\sin(\frac{5}{x})$\\
$\displaystyle\lim_{x\to 0}x\sin(\frac{5}{x})=\lim_{x\to 0}\frac{\sin(frac{5}{x})}{\frac{5}{x}}\times5=1\times5=5$\\

例4.\\
\phantom{例}$\displaystyle\lim_{x\to 0}\frac{1-\cos x}{x^2}$\\
$\displaystyle\lim_{x\to 0}\frac{1-\cos x}{x^2}=\lim_{x\to 0}\frac{1-\cos x}{x^2}\times\frac{1+\cos x}{1+\cos x}=\lim_{x\to 0}\frac{1-\cos^2x}{x^2(1+\cos x)}$\\
$\displaystyle\phantom{\lim_{x\to 0}\frac{1-\cos x}{x^2}}=\lim_{x\to 0}\frac{\sin^2x}{x^2}\times\frac{1}{1+\cos x}=1\times\frac{1}{2}=\frac{1}{2}$\\[2ex]

(2)$x\to\infty/-\infty$的情况\\
例.\\
\phantom{例}$\displaystyle\lim_{x\to\infty}\frac{x\sin(11x^7)-\frac{1}{2}}{2x^4}$\\
$\displaystyle\because-1\leqslant\sin(11x^7)\leqslant1$\\
$\displaystyle\therefore-x-\frac{1}{2}\leqslant x\sin(11x^7)-\frac{1}{2}\leqslant x-\frac{1}{2}$\\
$\displaystyle\phantom{\therefore}\frac{-x-\frac{1}{2}}{2x^4}\leqslant\frac{x\sin(11x^7)-\frac{1}{2}}{2x^4}\leqslant\frac{x-\frac{1}{2}}{2x^4}$\\
而$\displaystyle\lim_{x\to\infty}\frac{-x-\frac{1}{2}}{2x^4}=\lim_{x\to\infty}-\frac{1}{2x^3}-\frac{1}{4x^4}=0$\\
\phantom{而}$\displaystyle\lim_{x\to\infty}\frac{x-\frac{1}{2}}{2x^4}=\lim_{x\to\infty}\frac{1}{2x^3}-\frac{1}{4x^4}=0$\\
$\displaystyle\therefore$根据夹逼定理\\
$\displaystyle\phantom{\therefore}\lim_{x\to\infty}\frac{x\sin(11x^7)-\frac{1}{2}}{2x^4}=0$\\[2ex]

(3)其他情况\\
面对$x\to a$的极限, 而$a\neq 0$时, 使用$t=x-a$作替换, 将问题转化为$t\to 0$\\
例.\\
\phantom{例}$\displaystyle\lim_{x\to\frac{\pi}{2}}\frac{\cos x}{x-\frac{\pi}{2}}$\\
设$t=x-\frac{\pi}{2}$, 则\\

2.三角函数的导数\\[2ex]
\framebox{$\displaystyle\frac{\dif}{\dif x}\sin(x)=\cos(x)$}\\[2ex]
\framebox{$\displaystyle\frac{\dif}{\dif x}\cos(x)=-\sin(x)$}\\[2ex]
\framebox{$\displaystyle\frac{\dif}{\dif x}\tan(x)=\sec^2(x)$}\\[2ex]
\framebox{$\displaystyle\frac{\dif}{\dif x}\cot(x)=-\csc^2(x)$}\\[2ex]
\framebox{$\displaystyle\frac{\dif}{\dif x}\sec(x)=\sec(x)\tan(x)$}\\[2ex]
\framebox{$\displaystyle\frac{\dif}{\dif x}\csc(x)=-\csc(x)\cot(x)$}\\[2ex]
