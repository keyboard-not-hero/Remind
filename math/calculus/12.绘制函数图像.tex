\documentclass[UTF8, fontset=ubuntu]{ctexart}
\usepackage{parskip}
\begin{document}
1.建立原函数的符号表格\\
(1)建立一个两行的表格, 第一行为$x$取值, 第二行为$f(x)$对应值;\\
(2)在第一行以递增顺序列出$x$所有的关于$f(x)$的零点和不连续点, 并且每个数的左右都要留出表格;\\
(3)填充第二行, 零点直接填0, 不连续点以'*'填充;\\
(4)在第一行第一个数左边填上小于该数的数字, 在中间两个数之间填上介于之间的数字, 在最后一个数字右边填上大于该数的数字;\\
(5)根据第(4)步的值, 在第二行计算对应$f(x)$的值, 大于0则填上'+', 小于0则填上'-'.\\[2ex]

2.建立一阶导数的符号表格\\
(1)建立一个三行行的表格, 第一行为$x$取值, 第二行为$f'(x)$对应值, 第三行为趋势图;\\
(2)在第一行以递增顺序列出$x$所有的关于$f'(x)$的零点和不连续点, 并且每个数的左右都要留出表格;\\
(3)填充第二行, 零点直接填0, 不连续点以'*'填充;\\
(4)在第一行第一个数左边填上小于该数的数字, 在中间两个数之间填上介于之间的数字, 在最后一个数字右边填上大于该数的数字;\\
(5)根据第(4)步的值, 在第二行计算对应$f'(x)$的值, 大于0则填上'+', 小于0则填上'-';\\
(6)在第三行根据第二行的内容, 在该列填上对应内容, '+'对应'/', '0'对应'—', '-'对应'\'.\\[2ex]

3.建立二阶导数的符号表格\\
(1)建立一个两行的表格, 第一行为$x$取值, 第二行为$f''(x)$对应值, 第三行为趋势图;\\
(2)在第一行以递增顺序列出$x$所有的关于$f''(x)$的零点和不连续点, 并且每个数的左右都要留出表格;\\
(3)填充第二行, 零点直接填0, 不连续点以'*'填充;\\
(4)在第一行第一个数左边填上小于该数的数字, 在中间两个数之间填上介于之间的数字, 在最后一个数字右边填上大于该数的数字;\\
(5)根据第(4)步的值, 在第二行计算对应$f'(x)$的值, 大于0则填上'+', 小于0则填上'-';\\
(6)在第三行根据第二行的内容, 在该列填上对应内容, '+'对应'∪'', '0'对应'.', '-'对应'∩'.\\[2ex]

4.绘制函数图像的完整步骤\\
(1)对称性 - 通过$-x$替换$x$, 来验证函数的奇偶性;\\
(2)$y$轴截距 - 通过$x=0$来求$y$轴截距;\\
(3)$x$轴截距 - 通过$y=0$来求$x$轴截距;\\
(4)定义域 - 除已直接给出定义域的情况, 可剔除使得分母为0、偶数根号下的量为负数、对数符号里的量为负数或0的数, 并且反三角函数也需注意;\\
(5)垂直渐近线 - 分母为0且分子不为0的位置, 或对数式;\\
(6)函数的正负 - 建立关于$f(x)$的符号表格;\\
(7)水平渐近线 - 通过计算$\lim_{x\to\infty}f(x)$和$\lim_{x\to -\infty}f(x)$来找出函数的水平渐近线;
(8)导数的正负 - 绘制关于一阶导数的符号表格;\\
(9)最大值和最小值 - 根据(8)的符号表格, 计算所有局部最大/小值, 找出全局最大/小值;\\
(10)二阶导数的正负 - 绘制关于二阶导数的符号表格;\\
(11)拐点 - 拐点的二阶导数为0, 并且在该点两侧导数的正反符号相反.\\[2ex]
\end{document}
