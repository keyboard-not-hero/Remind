\documentclass[UTF8, fontset=ubuntu]{ctexart}
\usepackage{parskip}
\usepackage{tikz}
\usepackage{amssymb}
\begin{document}
1.导数与反函数\\
如果$f$在其定义域$(a,b)$上可导且满足以下条件中的任意一条:\\
(1)对于所有的在$(a,b)$中的$x$, $f'(x)>0$;\\
(2)对于所有的在$(a,b)$中的$x$, $f'(x)<0$;\\
(3)对于所有的在$(a,b)$中的$x$, $f'(x)\geqslant 0$且对于有限个数的$x$, $f'(x)=0$;\\
(4)对于所有的在$(a,b)$中的$x$, $f'(x)\leqslant 0$且对于有限个数的$x$, $f'(x)=0$.\\
则$f$有反函数.\\[2ex]

2.反函数的导数\\[1ex]
\framebox{如果$\displaystyle y=f^{-1}(x)$, 则$\displaystyle\frac{dy}{dx}=\frac{1}{f'(y)}=\frac{1}{f'(f^{-1}(x))}$}\\[1ex]
** $f(y)$是将$f(x)$中的$x$替换为$y$的版本, $f'(y)$类似.\\[2ex]

3.反三角函数\\
(1)$\sin^{-1}$是奇函数; 其定义域为$[-1,1]$, 值域为$\displaystyle[-\frac{\pi}{2},\frac{\pi}{2}]$\\[1ex]
(2)$\displaystyle\frac{d}{dx}\sin^{-1}(x)=\frac{1}{\sqrt{1-x^2}}$, 其中$-1<x<1$.\\[1ex]
(3)$\cos^{-1}$既不是偶函数也不是奇函数; 其定义域为$[-1,1]$, 值域为$[0,\pi]$.\\[1ex]
(4)$\displaystyle\frac{d}{dx}\cos^{-1}(x)=-\frac{1}{\sqrt{1-x^2}}$, 其中$-1<x<1$.\\[1ex]
(5)$\tan^{-1}$是奇函数; 其定义域是$\mathbb{R}$且值域是$\displaystyle(-\frac{\pi}{2},\frac{\pi}{2})$.\\[1ex]
(6)对于所有的实数$x$, $\displaystyle\frac{d}{dx}\tan^{-1}(x)=\frac{1}{1+x^2}$.\\[1ex]
(7)$\cot^{-1}$既不是奇函数也不是偶函数; 其定义域为$\mathbb{R}$且值域是$(0,\pi)$\\[1ex]
(8)对于所有的实数$x$, $\displaystyle\frac{d}{dx}\cot^{-1}(x)=-\frac{1}{1+x^2}$.\\[1ex]
(9)$\sec^{-1}$既不是奇函数也不是偶函数; 其定义域是$(-\infty,-1]\cup[1,\infty)$且值域是$\displaystyle[0,\frac{\pi}{2})\cup(\frac{\pi}{2},\pi]$.\\[1ex]
(10)对于$x>1$或$x<-1$, $\displaystyle\frac{d}{dx}\sec^{-1}(x)=\frac{1}{|x|\sqrt{x^2-1}}$.\\[1ex]
(11)$\csc^{-1}$是奇函数; 其定义域为$(-\infty,-1]\cup[1,\infty)$且值域是$\displaystyle[-\frac{\pi}{2},0)\cup(0,\frac{\pi}{2}]$.\\[1ex]
(12)对于$x>1$或$x<-1$, $\displaystyle\frac{d}{dx}\csc^{-1}(x)=-\frac{1}{|x|\sqrt{x^2-1}}$.\\[2ex]

4.计算反三角函数\\
(1)化简形如$\sin^{-1}(\sin(\alpha))$的三角函数:\\
\phantom{(1)}获取指定角$\alpha$的参照角\\
\phantom{(1)}找到反三角函数定义域中拥有该参照角的角\\
\phantom{(1)}确定该角的正弦值与$\alpha$参照角的正弦值符号一致

(2)
\end{document}
