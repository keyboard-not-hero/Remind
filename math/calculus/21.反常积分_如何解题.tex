\chapter{反常积分:如何解题}
1.如何解题\\
1)拆分积分步骤\\
\ding{172}确定区间$[a,b]$上的所有瑕点;\\
\ding{173}将积分拆分为若干个积分之和,使拆分后的每个积分最多包含一个瑕点,这些瑕点作为相应积分的上限或下限;\\
\ding{174}分别讨论每个积分,如果所有积分收敛,则整个积分收敛; 否则整个积分发散.\\[2ex]

2)如何处理负函数值\\
\ding{172}如果被积函数$f(x)$在区间$[a,b]$上既有正值又有负值,考虑使用绝对收敛判别法\\
\ding{173}如果被积函数$f(x)$在区间$[a,b]$上恒为非正, 即在$[a,b]$上$f(x)\leq0$, 则:
\[\int_a^bf(x)\dif x=-\int_a^b-f(x)\dif x\]
\ding{174}如果被积函数$f(x)$在区间$[a,b]$上正负无损/增益震荡, 积分发散. 如: $\sin x$, $\cos x$\\[4ex]


2.积分判别法汇总\\
1)比较判别法
\begin{center}
\begin{boxedminipage}{\textwidth}
如果在区间$(a,b)$内,函数$f(x)\geqslant g(x)\geqslant 0$,且积分$\displaystyle\int_a^bg(x)\dif x$是发散的,那么积分$\displaystyle\int_a^bf(x)\dif x$也是发散的. 即
\[\int_a^bf(x)\dif x\geqslant\int_a^bg(x)\dif x=\infty\]
\end{boxedminipage}
\end{center}\vspace{2ex}

\begin{center}
\begin{boxedminipage}{\textwidth}
如果在区间$(a,b)$内,函数$0\leqslant f(x)\leqslant g(x)$,且积分$\displaystyle\int_a^bg(x)\dif x$是收敛的,那么积分$\displaystyle\int_a^bf(x)\dif x$也一定是收敛的. 即
\[\int_a^bf(x)\dif x\leqslant\int_a^bg(x)\dif x<\infty\]
\end{boxedminipage}
\end{center}\vspace{2ex}

2)极限比较判别法
\begin{center}
\framebox{当$x\to a$时,$f(x)\sim g(x)$同$\displaystyle\lim_{x\to a}\frac{f(x)}{g(x)}=1$有着同样的意义}\\[2ex]
\begin{boxedminipage}{\textwidth}
如果当$x\to a$时$f(x)\sim g(x)$,且这两个函数在区间$[a,b]$上没有其他瑕点,那么积分$\displaystyle\int_a^bf(x)\dif x$和$\displaystyle\int_a^bg(x)\dif x$是同时收敛或同时发散的,这称为\textbf{极限比较判别法}
\end{boxedminipage}
\end{center}
注解: 如果函数$g(x)$有原函数没有的瑕点, 需要将原函数在该瑕点处进行拆分\\[2ex]

3)p判别法
$\displaystyle\cdot\int^{\infty}$的情况:对于任意有限值$a>0$,积分
\[\int_a^{\infty}\frac{1}{x^p}\dif x\]
\phantom{$\cdot$}在$p>1$时是收敛的,在$p\leqslant 1$时是发散的\\

$\displaystyle\cdot\int_0$的情况:对于任意有限值$a>0$,积分
\[\int_0^a\frac{1}{x^p}\dif x\]
\phantom{$\cdot$}在$p<1$时是收敛的,在$p\geqslant 1$时是发散的\\[2ex]

4)绝对收敛判别法
\begin{center}
\framebox{如果$\displaystyle\int_a^b|f(x)|\dif x$是收敛的,那么$\displaystyle\int_a^bf(x)\dif x$也是收敛的}
\end{center}\vspace{4ex}

3.极限判别法的$g(x)$\\
1)多项式和多项式型函数在$\infty$和$-\infty$附近的表现
\begin{center}
\framebox{若$P(x)$的最高次项是$ax^n$,则当$x\to\infty$或$x\to-\infty$时,有$P(x)\backsim ax^n$}
\end{center}

例1.\\
\phantom{例}$\displaystyle\int_1^{\infty}\frac{1}{2+20\sqrt{x}}\dif x$\\
推导过程:\\
当$x\to\infty$时, $2+20\sqrt{x}\backsim20\sqrt{x}$\\
$\displaystyle\therefore\frac{1}{2+20\sqrt{x}}\backsim\frac{1}{20\sqrt{x}}$\\
由p判别法, 得:\\
$\displaystyle\phantom{\because}\int_1^{\infty}\frac{1}{20\sqrt{x}}\dif x$发散\\
由极限判别法, 得:\\
$\displaystyle\therefore\int_1^{\infty}\frac{1}{2+20\sqrt{x}}\dif x$发散\\[1ex]

例2.\\
\phantom{例}$\displaystyle\int_0^\infty\frac{1}{x^5+4x^4+1}\dif x$\\
推导过程:\\
当$x\to\infty$时, $x^5+4x^4+1\backsim x^5$\\
$\displaystyle\therefore\frac{1}{x^5+4x^4+1}\backsim\frac{1}{x^5}$\\
$\displaystyle\because\int_0^\infty\frac{1}{x^5}\dif x在[0,\infty)有两个瑕点$\\
$\displaystyle\therefore\int_0^\infty\frac{1}{x^5+4x^4+1}\dif x=\int_0^1\frac{1}{x^5+4x^4+1}\dif x+\int_1^\infty\frac{1}{x^5+4x^4+1}\dif x$\\
$\displaystyle\int_0^1\frac{1}{x^5+4x^4+1}\dif x$自然收敛\\
由p判别法, 得:\\
$\displaystyle\int_1^\infty\frac{1}{x^5}\dif x$收敛\\
由极限比较判别法, 得:\\
$\displaystyle\int_1^\infty\frac{1}{x^5+4x^4+1}\dif x$收敛\\
$\displaystyle\therefore\int_0^\infty\frac{1}{x^5+4x^4+1}\dif x$收敛\\[1ex]

例3.\\
\phantom{例}$\displaystyle\int_2^\infty\frac{3x^5+2x^2+9}{x^6+22x^4+\sqrt{4x^{13}+18x}}\dif x$\\
推导过程:\\
当$x\to\infty$时, $\displaystyle\frac{3x^5+2x^2+9}{x^6+22x^4+\sqrt{4x^{13}+18x}}\backsim\frac{3x^5}{2x^{\frac{13}{2}}}=\frac{3}{2x^\frac{3}{2}}$\\
由p判别法, 得:\\
$\displaystyle\int_2^\infty\frac{3}{2x^\frac{3}{2}}\dif x$收敛\\
由极限比较判别法, 得:\\
$\displaystyle\int_2^\infty\frac{3x^5+2x^2+9}{x^6+22x^4+\sqrt{4x^{13}+18x}}\dif x$收敛\\[1ex]

例4.\\
\phantom{例}$\displaystyle\int_9^\infty\frac{1}{\sqrt{x^4+8x^3-9}-x^2}\dif x$\\
推导过程:\\
分子分母同时乘以分母的共轭表达式:\\
$\displaystyle\int_9^\infty\frac{1}{\sqrt{x^4+8x^3-9}-x^2}\dif x=\displaystyle\int_9^\infty\frac{1}{\sqrt{x^4+8x^3-9}-x^2}\times\frac{\sqrt{x^4+8x^3-9}+x^2}{\sqrt{x^4+8x^3-9}+x^2}\dif x$\\
$\displaystyle\phantom{\int_9^\infty\frac{1}{\sqrt{x^4+8x^3-9}-x^2}\dif x}=\int_9^\infty\frac{\sqrt{x^4+8x^3-9}+x^2}{8x^3-9}\dif x$\\
当$x\to\infty$时, $\displaystyle\frac{\sqrt{x^4+8x^3-9}+x^2}{8x^3-9}\backsim\frac{2x^2}{8x^3}=\frac{1}{4x}$\\
由p判别法, 得:\\
$\displaystyle\int_9^\infty\frac{1}{4x}\dif x$发散\\
由极限比较判别法, 得:\\
$\displaystyle\int_9^\infty\frac{1}{\sqrt{x^4+8x^3-9}-x^2}\dif x$发散\\[2ex]

2)三角函数在$\infty$或$-\infty$附近的表现
\begin{center}
\framebox{$|\sin(A)|\leqslant 1$}\qquad
\framebox{$|\cos(A)|\leqslant 1$}
\end{center}

例.\\
\phantom{例}$\displaystyle\int_5^{\infty}\frac{|\sin(x^4)|}{\sqrt{x}+x^2}\dif x$\\
推导过程:\\
当$x\to\infty$时,$\displaystyle\frac{1}{\sqrt{x}+x^2}\backsim\frac{1}{x^2}$\\
由p判别法, 得:\\
$\displaystyle\phantom{\because}\int_5^{\infty}\frac{1}{x^2}\dif x$收敛\\
由极限比较判别法, 得:\\
$\displaystyle\int_5^\infty\frac{1}{\sqrt{x}+x^2}\dif x$收敛\\
$\displaystyle\because\int_5^{\infty}\frac{|\sin(x^4)|}{\sqrt{x}+x^2}\dif x\leqslant\int_5^{\infty}\frac{1}{\sqrt{x}+x^2}\dif x<\infty$\\
$\displaystyle\therefore\int_5^{\infty}\frac{|\sin(x^4)|}{\sqrt{x}+x^2}\dif x$收敛\\[2ex]

3)指数在$\infty$和$-\infty$附近表现
\begin{center}
\framebox{对所有的$x>0$,$e^{-x}\leqslant\frac{C}{x^n}$}
\end{center}

例1.\\
\phantom{例}$\displaystyle\int_{10}^\infty(x^{1000}+x^2+\sin(x))e^{-x^2+6}\dif x$\\
推导过程:\\
当$x\to\infty$时,$(x^{1000}+x^2+\sin x)e^{-x^2+6}\backsim x^{1000}e^{-x^2+6}$\\
$\displaystyle\because e^{-x^2+6}\leqslant\frac{C}{x^{1002}}$\\
$\displaystyle\therefore\int_{10}^\infty x^{1000}e^{-x^2+6}\dif x\leqslant C\int_{10}^{\infty}\frac{1}{x^2}\dif x$\\
由p判别法, 得:\\
$\displaystyle C\int_{10}^\infty\frac{1}{x^2}$收敛\\
即$\displaystyle\int_{10}^\infty x^{1000}e^{-x^2+6}\dif x$收敛\\
由极限比较判别法, 得:\\
$\displaystyle\int_{10}^\infty(x^{1000}+x^2+\sin x)e^{-x^2+6}\dif x$收敛\\[1ex]

例2.\\
\phantom{例}$\displaystyle\int_9^\infty\frac{x^{10}}{e^x-5x^{20}}\dif x$\\
推导过程:\\
当$x\to\infty$时, $e^x-5x^{20}\backsim e^x$\\
$\displaystyle\because e^{-x}\leqslant\frac{C}{x^{12}}$\\
$\displaystyle\therefore\int_9^\infty\frac{x^{10}}{e^x}\dif x\leqslant C\int_9^\infty\frac{1}{x^2}\dif x$\\
由p判别法, 得:\\
$\displaystyle C\int_9^\infty\frac{1}{x^2}\dif x$收敛\\
由极限比较判别法, 得:\\
$\displaystyle\int_9^\infty\frac{x^10}{e^x-5x^{20}}\dif x$收敛\\[1ex]

例3.\\
\phantom{例}$\displaystyle\int_{18}^\infty\frac{x^2}{7^x-4^x}\dif x$\\
推导过程:\\
当$x\to\infty$时, $7^x-4^x\backsim 7^x$\\
$\because x>0$时, $\displaystyle 7^{-x}\leqslant\frac{C}{x^4}$\\
$\displaystyle\therefore\int_{18}^\infty\frac{x^2}{7^x}\dif x\leqslant C\int_{18}^\infty\frac{1}{x^2}\dif x$\\
由p判别法, 得:\\
$\displaystyle C\int_{18}^\infty\frac{1}{x^2}\dif x$收敛\\
由极限比较判别法, 得:\\
$\displaystyle\int_{18}^\infty\frac{x^2}{7^x-4^x}\dif x$收敛\\[2ex]

(4)对数在$\infty$附近的表现
\begin{center}
\framebox{对所有$x>1$,$\ln(x)\leqslant Cx^{\alpha}$}
\end{center}

例1.\\
\phantom{例}$\displaystyle\int_2^{\infty}\frac{\ln(x)}{x^{1.001}}\dif x$\\
推导过程:\\
$\because x>1$时, $\ln x\leqslant Cx^{0.0001}$\\
$\displaystyle\therefore\int_2^\infty\frac{\ln x}{x^{1.001}}\dif x\leqslant C\int_2^\infty\frac{1}{x^{1.0009}}\dif x$\\
由p判别法, 得:\\
$\displaystyle\int_2^{\infty}\frac{1}{x^{1.0009}}\dif x$收敛\\
由比较判别法, 得:\\
$\displaystyle\int_2^{\infty}\frac{\ln(x)}{x^{1.001}}\dif x$收敛\\[1ex]

例2.\\
$\displaystyle\int_2^{\infty}\frac{1}{x^{1.001}\ln(x)}\dif x$\\
推导过程:\\
当$x\geqslant 2$时, $\ln x\geqslant\ln 2$\\
$\displaystyle\int_2^{\infty}\frac{1}{x^{1.001}\ln(x)}\dif x\leqslant\frac{1}{\ln(2)}\int_2^{\infty}\frac{1}{x^{1.001}}\dif x$\\
由p判别法, 得:\\
$\displaystyle\phantom{\because}\frac{1}{\ln(2)}\int_2^{\infty}\frac{1}{x^{1.001}}\dif x<\infty$\\
由比较判别法, 得:\\
$\displaystyle\int_2^{\infty}\frac{1}{x^{1.001}\ln(x)}\dif x$收敛\\[1ex]

例3.\\
\phantom{例}$\displaystyle\int_2^\infty\frac{\ln x}{x}\dif x$\\
推导过程:\\
当$x\geqslant 2$时, $\ln x\geqslant\ln 2$\\
$\displaystyle\therefore\int_2^\infty\frac{\ln x}{x}\dif x\geqslant\ln 2\int_2^\infty\frac{1}{x}\dif x$\\
由p判别法, 得:\\
$\displaystyle\int_2^\infty\frac{1}{x}\dif x$发散\\
由比较判别法, 得:\\
$\displaystyle\int_2^\infty\frac{\ln x}{x}\dif x$发散\\[1ex]

例4.\\
\phantom{例}$\displaystyle\int_2^\infty\frac{1}{x\ln x}\dif x$\\
推导过程:\\
令$t=\ln x$, 则$\dif t=\frac{1}{x}\dif x$\\
$\displaystyle\int_2^\infty\frac{1}{x\ln x}\dif x=\int_{\ln 2}^\infty\frac{1}{t}\dif t$\\
由p判别法, 得:\\
$\displaystyle\int_{\ln 2}^\infty\frac{1}{t}\dif t$发散\\
$\displaystyle\therefore\int_2^\infty\frac{1}{x\ln x}\dif x$发散\\[2ex]

5)多项式和多项式型函数在0附近的表现
\begin{center}
\framebox{若$p(x)$的最低次项是$bx^m$, 则当$x\to 0$, $p(x)\backsim bx^m$}
\end{center}

例1.\\
\phantom{例}$\displaystyle\int_0^5\frac{1}{x^2+\sqrt{x}}\dif x$\\
推导过程:\\
当$x\to 0$时, $\displaystyle\frac{1}{x^2+\sqrt{x}}\backsim\frac{1}{\sqrt{x}}$\\
由p判别法, 得:\\
$\displaystyle\int_0^5\frac{1}{\sqrt{x}}\dif x$收敛\\
由极限比较判别法, 得:\\
$\displaystyle\int_0^5\frac{1}{x^2+\sqrt{x}}\dif x$收敛\\[1ex]

例2.\\
\phantom{例}$\displaystyle\int_0^1\frac{x+3}{x+x^5}\dif x$\\
推导过程:\\
当$x\to 0$时, $\displaystyle\frac{x+3}{x+x^5}\backsim\frac{3}{x}$\\
由p判别法, 得:\\
$\displaystyle\int_0^1\frac{3}{x}\dif x$发散\\
由极限比较判别法, 得:\\
$\displaystyle\int_0^1\frac{x+3}{x+x^5}\dif x$发散\\[2ex]

6)三角函数在0附近的表现
\begin{center}
\framebox{当$x\to 0$, $\sin x\backsim x$, $\tan x\backsim x$且$\cos x\backsim 1$}
\end{center}

例1.\\
\phantom{例}$\displaystyle\int_0^1\frac{1}{\tan x}\dif x$\\
推导过程:\\
当$x\to 0$时, $\displaystyle\frac{1}{\tan x}\backsim\frac{1}{x}$\\
由p判别法, 得:\\
$\displaystyle\int_0^1\frac{1}{x}\dif x$发散\\
由极限比较判别法, 得:\\
$\displaystyle\int_0^1\frac{1}{\tan x}\dif x$发散\\[1ex]

例2.\\
\phantom{例}$\displaystyle\int_0^1\frac{1}{\sqrt{\tan x}}\dif x$\\
推导过程:\\
当$x\to 0^+$时, $\displaystyle\frac{1}{\sqrt{\tan x}}\dif x\backsim\frac{1}{\sqrt{x}}$\\
由p判别法, 得:\\
$\displaystyle\int_0^1\frac{1}{\sqrt{x}}\dif x$收敛\\
由极限比较判别法, 得:\\
$\displaystyle\int_0^1\frac{1}{\sqrt{\tan x}}\dif x$收敛\\[1ex]

例3.\\
\phantom{例}$\displaystyle\int_0^1\frac{\sin x}{x^\frac{3}{2}}\dif x$\\
推导过程:\\
当$x\to 0$时, $\displaystyle\frac{\sin x}{x^\frac{3}{2}}\backsim\frac{x}{x^\frac{3}{2}}\backsim\frac{1}{x^\frac{1}{2}}$\\
由p判别法, 得:\\
$\displaystyle\int_0^1\frac{1}{x^\frac{1}{2}}\dif x$收敛\\
由极限比较判别法, 得:\\
$\displaystyle\int_0^1\frac{\sin x}{x^\frac{3}{2}}\dif x$收敛\\[2ex]

7)指数函数在0附近的表现
\begin{center}
\framebox{当$x\to 0$时, $e^x\backsim 1$并且$e^x-1\backsim x$}
\end{center}

例1.\\
\phantom{例}$\displaystyle\int_0^1\frac{e^x}{x\cos x}\dif x$\\
推导过程:\\
当$x\to 0$时, $\displaystyle\frac{e^x}{x\cos x}\backsim\frac{1}{x}$\\
由p判别法, 得:\\
$\displaystyle\int_0^1\frac{1}{x}\dif x$发散\\
由极限比较判别法, 得:\\
$\displaystyle\int_0^1\frac{e^x}{x\cos x}\dif x$发散\\[1ex]

例2.\\
\phantom{例}$\displaystyle\int_0^1\frac{e^{-\frac{1}{x}}}{x^5}\dif x$\\
推导过程:\\
当$x\to 0$时, $\displaystyle e^{-\frac{1}{x}}\leqslant\frac{C}{(\frac{1}{x})^5}=Cx^5$\\
$\displaystyle\therefore\frac{e^{-\frac{1}{x}}}{x^5}\leqslant C$\\
$\displaystyle\because C\int_0^1\dif x$没有瑕点\\
$\displaystyle\therefore C\int_0^1\dif x$自然收敛\\
$\displaystyle\therefore\int_0^1\frac{e^{-\frac{1}{x}}}{x^5}\dif x$收敛\\[1ex]

例3.\\
\phantom{例}$\displaystyle\int_0^2\frac{1}{\sqrt{e^x-1}}\dif x$\\
推导过程:\\
当$x\to 0$时, $e^x-1\backsim x$\\
由p判别法, 得:\\
$\displaystyle\int_0^2\frac{1}{\sqrt{x}}\dif x$收敛\\
由极限比较判别法, 得:\\
$\displaystyle\int_0^2\frac{1}{\sqrt{e^x-1}}\dif x$收敛\\[2ex]

8)对数函数在0附近的表现
\begin{center}
\framebox{对于所有$0<x<1$, $|\ln x|\leqslant\frac{C}{x^\alpha}$}
\end{center}

例1.\\
\phantom{例}$\displaystyle\int_0^1\frac{|\ln x|}{x^{0.9}}\dif x$\\
推导过程:\\
当$0<x<1$时, $|\ln x|\leqslant\frac{C}{x^{0.05}}$\\
$\displaystyle\therefore\frac{|\ln x|}{x^{0.9}}\leqslant\frac{C}{0.95}$\\
由p判别法, 得:\\
$\displaystyle C\int_0^1\frac{1}{x^{0.95}}\dif x$收敛\\
由比较判别法, 得:\\
$\displaystyle\int_0^1\frac{|\ln x|}{x^{0.9}}\dif x$收敛\\[1ex]

例2.\\
\phantom{例}$\displaystyle\int_0^{\frac{1}{2}}\frac{1}{x^2|\ln x|}\dif x$\\
推导过程:\\
当$0<x<1$时, $|\ln x|\leqslant\frac{C}{x}$\\
$\displaystyle\therefore\frac{1}{x^2|\ln x|}\geqslant\frac{1}{Cx}$\\
由p判别法, 得:\\
$\displaystyle\frac{1}{C}\int_0^{\frac{1}{2}}\frac{1}{x}\dif x$发散\\
由比较判别法, 得:\\
$\displaystyle\int_0^{\frac{1}{2}}\frac{1}{x^2|\ln x|}\dif x$发散\\[1ex]

例3.\\
\phantom{例}$\displaystyle\int_0^{\frac{1}{2}}\frac{1}{x^{0.9}|\ln x|}\dif x$\\
推导过程:\\
当$0<x\leqslant\frac{1}{2}$时, $|\ln x|\geqslant|\ln(\frac{1}{2})|=\ln 2$\\
$\displaystyle\therefore\frac{1}{x^{0.9}|\ln x|}\leqslant\frac{1}{\ln2 x^{0.9}}$\\
由p判别法, 得:\\
$\displaystyle\frac{1}{\ln 2}\int_0^{\frac{1}{2}}\frac{1}{x^{0.9}}\dif x$收敛\\
由比较判别法, 得:\\
$\displaystyle\int_0^{\frac{1}{2}}\frac{1}{x^{0.9}|\ln x|}\dif x$收敛\\[2ex]

9)更一般的函数在0附近的表现
\begin{center}
\begin{boxedminipage}{\textwidth}
若一个函数在0附近收敛于该函数的麦克劳林级数, 则函数在$x\to 0$时渐进等价于级数的最低次项.即
		\[f(x)=a_nx^n+a_{n+1}x^{n+1}+...\text{, 当}x\to 0\text{, 则}f(x)\backsim a_nx^n\]
\end{boxedminipage}
\end{center}

例1.\\
\phantom{例}$\displaystyle\int_0^1\frac{1}{1-\cos x}\dif x$\\
推导过程:\\
$\cos x$的麦克劳林级数:\\
\[\cos x=1-\frac{x^2}{2!}+\frac{x^4}{4!}-\cdots\]
$\displaystyle 1-\cos x=\frac{x^2}{2!}-\frac{x^4}{4!}+\cdots$\\
$\therefore$当$x\to 0$时, $\displaystyle\frac{1}{1-\cos x}\backsim\frac{2}{x^2}$\\
由p判别法, 得:\\
$\displaystyle\int_0^1\frac{2}{x^2}\dif x$发散\\
由极限比较判别法, 得:\\
$\displaystyle\int_0^1\frac{1}{1-\cos x}\dif x$发散\\[1ex]

例2.\\
\phantom{例}$\displaystyle\int_0^1\frac{1}{(1-\cos x)^{\frac{1}{3}}}\dif x$\\
推导过程:\\
当$x\to 0$时, $\displaystyle 1-\cos x\backsim \frac{x^2}{2}$\\
$\displaystyle\therefore\frac{1}{(1-\cos x)^{\frac{1}{3}}}\backsim\frac{\sqrt[3]{2}}{x^{\frac{2}{3}}}$\\
由p判别法, 得:\\
$\displaystyle\int_0^1\frac{\sqrt[3]{2}}{x^{\frac{2}{3}}}\dif x$收敛\\
由极限比较判别法, 得:\\
$\displaystyle\int_0^1\frac{1}{(1-\cos x)^{\frac{1}{3}}}\dif x$收敛\\[2ex]

10)不在0或$\infty$或$-\infty$处的瑕点\\
\begin{itemize}
	\item 若积分$\int_a^bf(x)\dif x$的唯一瑕点出现在$x=a$处, 进行$t=x-a$换元
	\item 若积分$\int_a^bf(x)\dif x$的唯一瑕点出现在$x=b$处, 进行$t=b-x$换元
\end{itemize}

例.\\
\phantom{例}$\displaystyle\int_0^3\frac{1}{x(x+1)(x-1)(x-2)}\dif x$\\
$\displaystyle\int_0^3\frac{1}{x(x+1)(x-1)(x-2)}\dif x$在积分范围内包含三个瑕点, 分别为$x=0,1,2$\\
将原积分拆分为以下部分:\\
\ding{172}$\displaystyle\int_0^{\frac{1}{2}}\frac{1}{x(x+1)(x-1)(x-2)}\dif x$\\
当$x\to 0$时, $\displaystyle\frac{1}{x(x+1)(x-1)(x-2)}\backsim\frac{1}{2x}$\\
由p判别法, 得:\\
$\displaystyle\int_0^{\frac{1}{2}}\frac{1}{2x}\dif x$发散\\
由极限比较判别法, 得:\\
$\displaystyle\int_0^{\frac{1}{2}}\frac{1}{x(x+1)(x-1)(x-2)}\dif x$发散\\[1ex]


\ding{173}$\displaystyle\int_{\frac{1}{2}}^1\frac{1}{x(x+1)(x-1)(x-2)}\dif x$\\
设$t=1-x, 则\dif t=-\dif x$\\
$\displaystyle\int_{\frac{1}{2}}^1\frac{1}{x(x+1)(x-1)(x-2)}\dif x=-\int_{\frac{1}{2}}^0\frac{1}{t(1+t)(1-t)(2-t)}\dif t$\\
$\displaystyle\phantom{\int_{\frac{1}{2}}^1\frac{1}{x(x+1)(x-1)(x-2)}\dif x}=\int_0^{\frac{1}{2}}\frac{1}{t(1+t)(1-t)(2-t)}\dif t$\\
当$t\to 0$时, $\displaystyle\frac{1}{t(1+t)(1-t)(2-t)}\backsim\frac{1}{2t}$\\
由p判别法, 得:\\
$\displaystyle\int_0^{\frac{1}{2}}\frac{1}{2t}\dif t$发散\\
由极限比较判别法, 得:\\
$\displaystyle\int_0^{\frac{1}{2}}\frac{1}{t(1+t)(1-t)(2-t)}\dif t$发散\\
即$\displaystyle\int_{\frac{1}{2}}^1\frac{1}{x(x+1)(x-1)(x-2)}\dif x$发散\\[1ex]


\ding{174}$\displaystyle\int_1^{\frac{3}{2}}\frac{1}{x(x+1)(x-1)(x-2)}\dif x$\\
设$t=x-1, 则\dif t=\dif x$\\
$\displaystyle\int_1^{\frac{3}{2}}\frac{1}{x(x+1)(x-1)(x-2)}\dif x=-\int_0^{\frac{1}{2}}\frac{1}{t(t+1)(t+2)(1-t)}\dif t$\\
当$t\to 0$时, $\displaystyle\frac{1}{t(t+1)(t+2)(1-t)}\backsim\frac{1}{2t}$\\
由p判别法, 得:\\
$\displaystyle\int_0^{\frac{1}{2}}\frac{1}{2t}\dif t$发散\\
由极限比较判别法, 得:\\
$\displaystyle\int_0^{\frac{1}{2}}\frac{1}{t(t+1)(t+2)(1-t)}\dif t$发散\\
即$\displaystyle\int_1^{\frac{3}{2}}\frac{1}{x(x+1)(x-1)(x-2)}\dif x$发散\\[1ex]

\ding{175}$\displaystyle\int_{\frac{3}{2}}^2\frac{1}{x(x+1)(x-1)(x-2)}\dif x$\\
设$t=2-x, 则\dif t=-\dif x$\\
$\displaystyle\int_{\frac{3}{2}}^2\frac{1}{x(x+1)(x-1)(x-2)}\dif x=\int_{\frac{1}{2}}^0\frac{1}{t(1-t)(2-t)(3-t)}\dif t$\\
$\displaystyle\phantom{\int_{\frac{3}{2}}^2\frac{1}{x(x+1)(x-1)(x-2)}\dif x}=-\int_0^{\frac{1}{2}}\frac{1}{t(1-t)(2-t)(3-t)}\dif t$\\
当$t\to 0$时, $\displaystyle\frac{1}{t(1-t)(2-t)(3-t)}\backsim\frac{1}{6t}$\\
由p判别法, 得:\\
$\displaystyle\int_0^{\frac{1}{2}}\frac{1}{6t}\dif t$发散\\
由极限比较判别法, 得:\\
$\displaystyle\int_0^{\frac{1}{2}}\frac{1}{t(1-t)(2-t)(3-t)}\dif t$发散\\
即$\displaystyle\int_{\frac{3}{2}}^2\frac{1}{x(x+1)(x-1)(x-2)}\dif x$发散\\[1ex]

\ding{176}$\displaystyle\int_2^3\frac{1}{x(x+1)(x-1)(x-2)}\dif x$\\
设$t=x-2, 则\dif t=\dif x$\\
$\displaystyle\int_2^3\frac{1}{x(x+1)(x-1)(x-2)}\dif x=\int_0^1\frac{1}{t(t+1)(t+2)(t+3)}\dif t$\\
当$t\to 0$时, $\displaystyle\frac{1}{t(t+1)(t+2)(t+3)}\backsim\frac{1}{6t}$\\
由p判别法, 得:\\
$\displaystyle\int_0^{\frac{1}{2}}\frac{1}{6t}\dif t$发散\\
由极限比较判别法, 得:\\
$\displaystyle\int_0^1\frac{1}{t(t+1)(t+2)(t+3)}\dif t$发散\\
即$\displaystyle\int_2^3\frac{1}{x(x+1)(x-1)(x-2)}\dif x$发散\\[1ex]

综上, $\displaystyle\int_0^3\frac{1}{x(x+1)(x-1)(x-2)}\dif x$发散

%最后编辑于: 2022-01-14
