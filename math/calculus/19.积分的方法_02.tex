\documentclass[UTF8, fontset=ubuntu]{ctexart}
\usepackage{parskip}
\usepackage{amsmath}
\usepackage{amssymb}
\usepackage[integrals]{wasysym}
\newcommand*{\dif}{\mathop{}\!\mathrm{d}}
\begin{document}
1.三角恒等式\\[1ex]
(1)形如$\sqrt{1\pm\cos(x)}$\\[1ex]
例.\\[1ex]
$\displaystyle\int_0^{\frac{\pi}{2}}\sqrt{1-\cos(2x)}\dif x$\\[1ex]
推导过程:\\[1ex]
由$\cos(2x)=1-2\sin^2(x)$, 得:\\[1ex]
$\displaystyle\int_0^{\frac{\pi}{2}}\sqrt{1-\cos(2x)}\dif x=\sqrt{2}\int_0^{\frac{\pi}{2}}|\sin(x)|\dif x$\\[1ex]
由于$\sin(x)$在定义域区间$[0,\frac{\pi}{2}]$内为正, 所以:\\[1ex]
$\displaystyle\int_0^{\frac{\pi}{2}}\sqrt{1-\cos(2x)}\dif x=-\sqrt{2}\cos(x)\Big{|}_0^{\frac{\pi}{2}}=0-(-\sqrt{2})=\sqrt{2}$\\[1ex]

(2)形如$\sqrt{1-\sin^2(x)}/\sqrt{1+\tan^2(x)}$\\[1ex]
例.\\[1ex]
$\displaystyle\int_0^{\pi}\sqrt{1-\cos^2(x)}\dif x$\\[1ex]
推导过程:\\[1ex]
由$\sin^2(x)=1-\cos^2(x)$, 得:\\[1ex]
$\displaystyle\int_0^{\pi}\sqrt{1-\cos^2(x)}\dif x=\int_0^{\pi}\sqrt{sin^2(x)}\dif x=\int_0^{\pi}|\sin(x)|\dif x$\\[1ex]
由于$\sin(x)$在区间$[0,\pi]$内为正, 所以:\\[1ex]
$\displaystyle\int_0^{\pi}\sqrt{1-\cos^2(x)}\dif x=\int_0^{\pi}\sin(x)\dif x=-\cos(x)\Big{|}_0^{\pi}=1-(-1)=2$\\[1ex]

(3)形如$\displaystyle\frac{1}{\sec(x)-1}$\\[1ex]
例.\\[1ex]
$\displaystyle\int\frac{1}{\sec(x)-1}\dif x$\\[1ex]
推导过程:\\[1ex]
$\displaystyle\int\frac{1}{\sec(x)-1}\dif x=\int\frac{1}{\sec(x)-1}\times\frac{\sec(x)+1}{\sec(x)+1}\dif x=\int\frac{\sec(x)+1}{\sec^2(x)-1}\dif x$\\[1ex]
\phantom{$\displaystyle\int\frac{1}{\sec(x)-1}\dif x$}$\displaystyle=\int\frac{\sec(x)}{\tan^2(x)}\dif x+\int\frac{1}{\tan^2(x)}\dif x=\int\frac{\cos(x)}{\sin^2(x)}\dif x+\int\frac{1}{\tan^2(x)}\dif x$\\[1ex]
设$t=\sin(x)$, 则$\dif t=\cos(x)\dif x$, 得:\\[1ex]
$\displaystyle\int\frac{\cos(x)}{\sin^2(x)}\dif x=\int\frac{1}{t^2}\dif t=-\frac{1}{t}+C=-\csc(x)+C$\\[1ex]
由$\frac{\dif}{\dif x}(\cot(x))=-\csc^2(x)$, 得:\\[1ex]
$\displaystyle\int\frac{1}{\tan^2(x)}\dif x=\int\cot^2(x)\dif x=\int\csc^2(x)\dif x-\int\dif x=-\cot(x)-x+C$\\[1ex]
$\therefore\displaystyle\int\frac{1}{\sec(x)-1}\dif x=-\csc(x)-\cot(x)-x+C$\\[1ex]

(4)形如$\sin(\alpha)\cos(\beta)$\\[1ex]
例.\\[1ex]
$\displaystyle\int\sin(19x)\cos(3x)\dif x$\\[1ex]
推导过程:\\[1ex]
和/差角公式:\\[-4ex]
\begin{center}
    \framebox{$\sin(A+B)=\sin(A)\cos(B)+\cos(A)\sin(B)$}\\[2ex]
    \framebox{$\cos(A+B)=\cos(A)\cos(B)-\sin(A)\sin(B)$}\\[2ex]
    \framebox{$\sin(A-B)=\sin(A)\cos(B)-\cos(A)\sin(B)$}\\[2ex]
    \framebox{$\cos(A-B)=\cos(A)\cos(B)+\sin(A)\sin(B)$}
\end{center}\vspace{1ex}
可推断出:\\[-4ex]
\begin{center}
	\framebox{$\sin(A)\cos(B)=\dfrac{1}{2}(\sin(A+B)+\sin(A-B))$}\\[2ex]
	\framebox{$\cos(A)\cos(B)=\dfrac{1}{2}(\cos(A+B)+\cos(A-B))$}\\[2ex]
	\framebox{$\sin(A)\sin(B)=\dfrac{1}{2}(\cos(A-B)-\cos(A+B))$}\\[2ex]
\end{center}\newpage
$\displaystyle\therefore\int\sin(19x)\cos(3x)\dif x=\frac{1}{2}\int(\sin(19x+3x)+\sin(19x-3x))\dif x$\\[1ex]
$\displaystyle\phantom{\therefore\int\sin(19x)\cos(3x)\dif x}=\frac{1}{2}\int(\sin(22x)+\sin(16x))\dif x$\\[1ex]
$\displaystyle\phantom{\therefore\int\sin(19x)\cos(3x)\dif x}=\frac{1}{2}\left(-\frac{\cos(22x)}{22}-\frac{\cos(16x)}{16}\right)+C$\\[1ex]
$\displaystyle\phantom{\therefore\int\sin(19x)\cos(3x)\dif x}=-\frac{\cos(22x)}{44}-\frac{\cos(16x)}{32}+C$\\[1ex]
\end{document}
