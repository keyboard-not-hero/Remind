\documentclass[UTF8, fontset=ubuntu, fleqn, fleqn]{ctexart}
\usepackage{parskip}
\usepackage{amsmath}
\usepackage{amssymb}
\usepackage[integrals]{wasysym}
\newcommand*{\dif}{\mathop{}\!\mathrm{d}}
\renewcommand{\baselinestretch}{1.8}
\begin{document}
1.三角恒等式\\
(1)形如$\sqrt{1\pm\cos(x)}$\\
例.\\
$\displaystyle\int_0^{\frac{\pi}{2}}\sqrt{1-\cos(2x)}\dif x$\\
推导原理:\\
将一个数与三角函数的运算, 转化为该三角函数半角的三角函数平方, 便于开根号\\
推导过程:\\
由$\cos(2x)=1-2\sin^2(x)$, 得:\\
$\displaystyle\int_0^{\frac{\pi}{2}}\sqrt{1-\cos(2x)}\dif x=\sqrt{2}\int_0^{\frac{\pi}{2}}|\sin(x)|\dif x$\\
由于$\sin(x)$在定义域区间$[0,\frac{\pi}{2}]$内为正, 所以:\\
$\displaystyle\int_0^{\frac{\pi}{2}}\sqrt{1-\cos(2x)}\dif x=-\sqrt{2}\cos(x)\Big{|}_0^{\frac{\pi}{2}}=0-(-\sqrt{2})=\sqrt{2}$\\

(2)形如$\sqrt{1-\sin^2(x)}/\sqrt{1+\tan^2(x)}$\\
例.\\
$\displaystyle\int_0^{\pi}\sqrt{1-\cos^2(x)}\dif x$\\
推导原理:\\
将根号下一个数字与三角函数平方的运算, 转化为该三角函数角的另一个三角函数的平方, 便于开根号\\
推导过程:\\
由$\sin^2(x)=1-\cos^2(x)$, 得:\\
$\displaystyle\int_0^{\pi}\sqrt{1-\cos^2(x)}\dif x=\int_0^{\pi}\sqrt{sin^2(x)}\dif x=\int_0^{\pi}|\sin(x)|\dif x$\\
由于$\sin(x)$在区间$[0,\pi]$内为正, 所以:\\
$\displaystyle\int_0^{\pi}\sqrt{1-\cos^2(x)}\dif x=\int_0^{\pi}\sin(x)\dif x=-\cos(x)\Big{|}_0^{\pi}=1-(-1)=2$\\

(3)形如$\displaystyle\frac{1}{\sec(x)-1}$\\
例.\\
$\displaystyle\int\frac{1}{\sec(x)-1}\dif x$\\
推导原理:\\
将分子分母同时乘以分母的共轭式\\
推导过程:\\
$\displaystyle\int\frac{1}{\sec(x)-1}\dif x=\int\frac{1}{\sec(x)-1}\times\frac{\sec(x)+1}{\sec(x)+1}\dif x=\int\frac{\sec(x)+1}{\sec^2(x)-1}\dif x$\\
\phantom{$\displaystyle\int\frac{1}{\sec(x)-1}\dif x$}$\displaystyle=\int\frac{\sec(x)}{\tan^2(x)}\dif x+\int\frac{1}{\tan^2(x)}\dif x=\int\frac{\cos(x)}{\sin^2(x)}\dif x+\int\frac{1}{\tan^2(x)}\dif x$\\
设$t=\sin(x)$, 则$\dif t=\cos(x)\dif x$, 得:\\
$\displaystyle\int\frac{\cos(x)}{\sin^2(x)}\dif x=\int\frac{1}{t^2}\dif t=-\frac{1}{t}+C=-\csc(x)+C$\\
由$\displaystyle\frac{\dif}{\dif x}\cot(x)=-\csc^2(x)$, 得:\\
$\displaystyle\int\frac{1}{\tan^2(x)}\dif x=\int\cot^2(x)\dif x=\int\csc^2(x)\dif x-\int\dif x=-\cot(x)-x+C$\\
$\therefore\displaystyle\int\frac{1}{\sec(x)-1}\dif x=-\csc(x)-\cot(x)-x+C$\\

(4)形如$\sin(\alpha)\cos(\beta)$\\
例.\\
$\displaystyle\int\sin(19x)\cos(3x)\dif x$\\
推导原理:\\
利用和/差角公式:\\[-4ex]
\begin{center}
    \framebox{$\sin(A+B)=\sin(A)\cos(B)+\cos(A)\sin(B)$}\\[2ex]
    \framebox{$\cos(A+B)=\cos(A)\cos(B)-\sin(A)\sin(B)$}\\[2ex]
    \framebox{$\sin(A-B)=\sin(A)\cos(B)-\cos(A)\sin(B)$}\\[2ex]
    \framebox{$\cos(A-B)=\cos(A)\cos(B)+\sin(A)\sin(B)$}
\end{center}\vspace{1ex}
可推断出:\\[-4ex]
\begin{center}
	\framebox{$\sin(A)\cos(B)=\dfrac{1}{2}(\sin(A+B)+\sin(A-B))$}\\[2ex]
	\framebox{$\cos(A)\cos(B)=\dfrac{1}{2}(\cos(A+B)+\cos(A-B))$}\\[2ex]
	\framebox{$\sin(A)\sin(B)=\dfrac{1}{2}(\cos(A-B)-\cos(A+B))$}\\[2ex]
\end{center}\newpage
推导过程:\\
$\displaystyle\therefore\int\sin(19x)\cos(3x)\dif x=\frac{1}{2}\int(\sin(19x+3x)+\sin(19x-3x))\dif x$\\
$\displaystyle\phantom{\therefore\int\sin(19x)\cos(3x)\dif x}=\frac{1}{2}\int(\sin(22x)+\sin(16x))\dif x$\\
$\displaystyle\phantom{\therefore\int\sin(19x)\cos(3x)\dif x}=\frac{1}{2}\left(-\frac{\cos(22x)}{22}-\frac{\cos(16x)}{16}\right)+C$\\
$\displaystyle\phantom{\therefore\int\sin(19x)\cos(3x)\dif x}=-\frac{\cos(22x)}{44}-\frac{\cos(16x)}{32}+C$\\[4ex]

二、关于三角函数的幂的积分\\
(1)$\sin$或$\cos$的幂\\
I、至少一个乘积项为奇次幂\\
例.\\
$\int\sin^7(x)\cos^4(x)\dif x$\\
解题思路:\\
将奇次幂分解为1次方和n-1次方, 并且将剩下的n-1偶次幂利用$\cos^2(x)+\sin^2(x)=1$进行转化\\
推导过程:\\
设$t=\cos(x)$, 则$\dif t=-\sin(x)\dif x$, 得:\\
\(\begin{array}{r l}
\int\sin^7(x)\cos^4(x)\dif x & =-\int\sin^6(x)\cos^4(x)(-\sin(x)\dif x)\\
& =-\int(1-\cos^2(x))^3\cos^4(x)(-\sin(x)\dif x)\\
& =-\int(1-t^2)t^4\dif t\\
& =-\int(1-3t^2+3t^4-t^6)t^4\dif t\\
& =-\int(t^4-3t^6+3t^8-t^{10})\dif t\\
& =-\dfrac{1}{5}t^5+\dfrac{3}{7}t^7-\dfrac{1}{3}t^9+\dfrac{1}{11}t^{11}+C\\[1ex]
& =-\dfrac{1}{5}\cos^5(x)+\dfrac{3}{7}\cos^7(x)-\dfrac{1}{3}\cos^9(x)+\dfrac{1}{11}\cos^{11}(x)+C
\end{array}\)\\[2ex]

II、两个乘积项都为偶次幂\\
例.\\
$\int\cos^2(x)\sin^4(x)\dif x$\\
解题思路:\\
利用倍角公式降低幂次\\
推导过程:\\
\(\begin{array}{r l}
\int\cos^2(x)\sin^4(x)\dif x & =\int\dfrac{1}{2}(1+\cos(2x))(\dfrac{1}{2}(1-\cos(2x)))^2\dif x\\[1ex]
& =\dfrac{1}{8}\int(1+\cos(2x))(1-\cos(2x))^2\dif x\\[1ex]
& =\dfrac{1}{8}\int(1-\cos(2x)-\cos^2(2x)+\cos^3(2x))\dif x\\[1ex]
& =\dfrac{1}{8}\int 1\dif x-\dfrac{1}{8}\int\cos(2x)\dif x-\dfrac{1}{8}\int\cos^2(2x)\dif x+\dfrac{1}{8}\int\cos^3(2x)\dif x\\[1ex]
& =\dfrac{1}{8}x-\dfrac{1}{16}\sin(2x)-\dfrac{1}{16}\int(1+\cos(4x))\dif x+\dfrac{1}{8}\int(1-\sin^2(2x))\cos(2x)\dif x\\[1ex]
& =\dfrac{1}{8}x-\dfrac{1}{16}\sin(2x)-\dfrac{1}{16}(x+\dfrac{\sin(4x)}{4})+\dfrac{1}{8}(\dfrac{\sin(2x)}{2}-\dfrac{\sin^3(2x)}{6})+C\\[1ex]
& =\dfrac{1}{16}x-\dfrac{1}{48}\sin^3(2x)-\dfrac{1}{64}\sin(4x)+C
\end{array}\)\\[2ex]

(2)$\tan$的幂($\cot$类似)\\
I、当幂为1\\
例.\\
$\int\tan(x)\dif x$\\
解题思路:\\
转化为$\dfrac{\sin(x)}{\cos(x)}$格式\\
推导过程:\\
假设$t=\cos(x)$,则$\dif t=-\sin(x)\dif x$\\
$\begin{array}{r l}
\int\tan(x)\dif x & =\int\dfrac{\sin(x)}{\cos(x)}\dif x\\
& =-\int\dfrac{1}{t}\dif t\\
& =-\ln|t|+C\\
& =-\ln|\cos(x)|+C
\end{array}$\\[2ex]

II、当幂为2\\
例.\\
$\int\tan^2(x)\dif x$\\
解题思路:\\
将$\tan^2(x)$转化为$\sec^2(x)-1$\\
推导过程:\\
$\int\tan^2(x)\dif x=\int(\sec^2(x)-1)\dif x$\\
$\phantom{\int\tan^2(x)\dif x}=\int\sec^2(x)\dif x-\int 1\dif x$\\
$\phantom{\int\tan^2(x)\dif x}=\tan(x)-x+C$\\[2ex]

III、当幂大于等于3\\
例.\\
$\int\tan^6(x)\dif x$\\
解题思路:\\
首先, 从中提取一个$\tan^2(x)$变化为$\sec^2(x)-1$, 然后被积分部分分成两部分. 第一部分为关于$t=\tan^2(x)$的积分; 第二部分为$\tan(x)$的更低次幂, 继续循环当前操作
推导过程:\\
$\int\tan^6(x)\dif x=\int\tan^4(x)(\sec^2(x)-1)\dif x=\int\tan^4(x)\sec^2(x)\dif x-\int\tan^4(x)\dif x$\\
设$t=\tan(x)$, 则$\dif t=\sec^2(x)\dif x$, 得:\\
$\begin{array}{r l}
\int\tan^4(x)\sec^2(x)\dif x & =\int t^4\dif t\\
& =\dfrac{1}{5}t^5+C\\[1ex]
& =\dfrac{1}{5}\tan^5(x)+C
\end{array}$\\[1ex]
设$t=\tan(x)$, 则$\dif t=\sec^2(x)\dif x$, 得:\\
$\begin{array}{r l}
\int\tan^4(x)\dif x & =\int\tan^2(x)(\sec^2(x)-1)\dif x\\
& =\int\tan^2(x)\sec^2(x)\dif x-\int\tan^2(x)\dif x\\
& =\int\tan^2(x)\sec^2(x)\dif x-\int(\sec^2(x)-1)\dif x\\
& =\int\tan^2(x)\sec^2(x)\dif x-\int\sec^2(x)\dif x+\int 1\dif x\\
& =\int t^2\dif t-\int 1\dif t+\dif 1\dif x\\
& =\dfrac{1}{3}t^3-t+x+C\\[1ex]
& =\dfrac{1}{3}\tan^3(x)-\tan(x)+x+C
\end{array}$\\[1ex]
合并结果, 得:\\
$\int\tan^6(x)=\dfrac{1}{5}\tan^5(x)-\dfrac{1}{3}\tan^3(x)+\tan(x)-x+C$\\[4ex]

(3)$\sec$的特征($\csc$类似)\\
I、当幂等于1\\
例.\\
$\int\sec(x)\dif x$\\
解题思路:\\
分子与分母同时乘以$\sec(x)+\tan(x)$, 得到形如$\int\dfrac{f'(x)}{f(x)}\dif x$的结果\\
推导过程:\\
设$t=\sec(x)+\tan(x)$, 则$\dif t=\sec(x)\tan(x)+\sec^2(x)\dif x$, 得:\\
$\begin{array}{r l}
\int\sec(x)\dif x & =\int\sec(x)\times\dfrac{\sec(x)+\tan(x)}{\sec(x)+\tan(x)}\dif x\\[1ex]
& =\int\dfrac{\sec^2(x)+\sec(x)\tan(x)}{\sec(x)+\tan(x)}\dif x\\[1ex]
& =\int\dfrac{1}{t}\dif t\\[1ex]
& =\ln|t|+C\\
& =\ln|\sec(x)+\tan(x)|+C
\end{array}$\\[2ex]

II、当幂等于2\\
例.\\
$\int\sec^2(x)\dif x$\\
解题思路:\\
$\int\sec^2(x)\dif x=\tan(x)+C$\\[2ex]

III、当幂大于等于3\\
例.\\
$\int\sec^6(x)\dif x$\\
解题思路:\\
提取出$\sec^2(x)$, 利用分部积分公式: $\int u\dif v=uv-\int v\dif u$\\
推导过程:\\
$\int\sec^6(x)\dif x=\int\sec^4(x)\sec^2(x)\dif x$\\
可得到以下结论:\\
$\begin{array}{c c}
u=\sec^4(x) & v=\tan(x)\\
\dfrac{\dif u}{\dif x}=4\sec^4(x)\tan(x) & \dfrac{\dif v}{\dif x}=\sec^2(x)
\end{array}$\\
利用分部积分公式:\\
\begin{equation}
\int\sec^4(x)\sec^2(x)\dif x=\sec^4(x)\tan(x)-4\int\sec^4(x)\tan^2(x)\dif x\label{integral:sec_01}
\end{equation}
\begin{eqnarray}
\int\tan^2(x)\sec^4(x) & = & \int(\sec^2(x)-1)\sec^4(x)\dif x\nonumber\\
& = & \int\sec^6(x)\dif x-\int\sec^4(x)\dif x\label{integral:sec_02}
\end{eqnarray}
将\eqref{integral:sec_02}代入\eqref{integral:sec_01}, 得:\\
$\int\sec^6(x)\dif x=\dfrac{1}{5}\sec^4(x)\tan(x)+\dfrac{4}{5}\int\sec^4(x)\dif x$\\
\end{document}
