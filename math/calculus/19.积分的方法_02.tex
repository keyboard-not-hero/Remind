\chapter{积分的方法 II}
1.三角恒等式相关\\
1)倍角公式\\
\[\cos^2x=\frac{1}{2}(1+\cos2x)\]
\[\sin^2x=\frac{1}{2}(1-\cos2x)\]
例.\\
\phantom{例}$\displaystyle\int_0^{\frac{\pi}{2}}\sqrt{1-\cos2x}\dif x$\\
推导过程:\\
由$\cos2x=1-2\sin^2x$, 得:\\
$\displaystyle\int_0^{\frac{\pi}{2}}\sqrt{1-\cos2x}\dif x=\sqrt{2}\int_0^{\frac{\pi}{2}}|\sin x|\dif x$\\
$\because\sin x$在$x\in[0,\frac{\pi}{2}]$值恒为非负\\
$\therefore\displaystyle\int_0^{\frac{\pi}{2}}\sqrt{1-\cos2x}\dif x=-\sqrt{2}\cos x\Big{|}_0^{\frac{\pi}{2}}=0-(-\sqrt{2})=\sqrt{2}$\\

2)毕达哥拉斯定理\\
\[\sin^2x+\cos^2x=1\]
\[\tan^2x+1=\sec^2x\]
\[\cot^2x+1=\csc^2x\]
例.\\
\phantom{例}$\displaystyle\int\frac{1}{\sec x+1}\dif x$\\
推导过程:\\
分子分母同时乘以共轭表达式\\
$\displaystyle\int\frac{1}{\sec x+1}\dif x=\int\frac{1}{\sec x+1}\times\frac{\sec x-1}{\sec x-1}\dif x=\int\frac{\sec x-1}{\sec^2x-1}\dif x$\\
$\displaystyle\phantom{\int\frac{1}{\sec x+1}\dif x}=\int\frac{\sec x-1}{\tan^2x}\dif x=\int\frac{\sec x}{\tan^2x}\dif x-\int\frac{1}{\tan^2x}\dif x$\\
$\displaystyle\because\int\frac{\sec x}{\tan^2x}\dif x=\int\frac{\cos x}{\sin^2x}\dif x=\int\cot x\csc x\dif x=-\csc x+C$\\
\phantom{$\because$}$\displaystyle\int\frac{1}{\tan^2x}\dif x=\int\cot^2x\dif x=\int(\csc^2x-1)\dif x=\int\csc^2x\dif x-\int\dif x$\\
\phantom{$\displaystyle\because\int\frac{1}{\tan^2x}\dif x$}$=-\cot x-x+C$\\
$\displaystyle\therefore\int\frac{1}{\sec x+1}\dif x=-\csc x-\cot x-x+C$\\


3)角和/差公式\\
\[\cos A\cos B=\frac{1}{2}(\cos(A-B)+\cos(A+B))\]
\[\sin A\sin B=\frac{1}{2}(\cos(A-B)-\cos(A+B))\]
\[\sin A\cos B=\frac{1}{2}(\sin(A-B)+\sin(A+B))\]
例.\\
\phantom{例}$\displaystyle\int\cos3x\sin19x\dif x$\\
推导过程:\\
$\displaystyle\because\sin19x\cos3x=\frac{1}{2}(\sin(19x-3x)+\sin(19x+3x))$\\
$\displaystyle\phantom{\because\sin19x\cos3x}=\frac{1}{2}(\sin16x+\sin22x)$\\
$\displaystyle\therefore\int\cos3x\sin19x\dif x=\frac{1}{2}\int\sin16x\dif x+\frac{1}{2}\int\sin22x\dif x$\\
$\displaystyle\phantom{\therefore\int\cos3x\sin19x\dif x}=-\frac{1}{32}\cos16x-\frac{1}{44}\cos22x+C$\\[2ex]

2.三角函数幂的积分\\
1)$\sin$的幂($\cos$类似)\\
I、一次幂\\
$\displaystyle\int\sin x\dif x=-\cos x+C$\\

II、二次幂\\
$\displaystyle\int\sin^2x\dif x=\frac{1}{2}\int(1-\cos2x)\dif x=\frac{1}{2}\int\dif x-\frac{1}{2}\int\cos2x\dif x$\\
$\displaystyle\phantom{\int\sin^2x\dif x}=\frac{1}{2}x-\frac{1}{4}\sin2x+C$\\

III、三次幂及以上的奇次幂
解题思路:\\
\phantom{解题}提取单次项与dx合并(用于换元), 剩余部分利用$\sin^2x+\cos^2x=1$进行转化\\
例.\\
\phantom{例}$\displaystyle\int\sin^7x\dif x$\\
推导过程:\\
设$t=\cos x$,则$\dif t=-\sin x\dif x$\\
$\displaystyle\int\sin^7x\dif x=-\int(1-\cos^2x)^3\sin x\dif x=-\int(1-t^2)^3\dif t$\\
$\displaystyle\phantom{\int\sin^7x\dif x}=-\int(1-3t^2+3t^4-t^6)\dif t$\\
$\displaystyle\phantom{\int\sin^7x\dif x}=-(t-t^3+\frac{3}{5}t^5-\frac{1}{7}t^7)+C$\\
$\displaystyle\phantom{\int\sin^7x\dif x}=-\cos x+\cos^3x-\frac{3}{5}\cos^5x+\frac{1}{7}\cos^7x+C$\\

IV、四次幂及以上的偶次幂\\
例.\\
\phantom{例}$\displaystyle\int\sin^6x\dif x$\\
\ding{172}使用倍角公式\\
$\displaystyle\int\sin^6x\dif x=\int\frac{1}{8}(1-\cos2x)^3\dif x=\frac{1}{8}\int(1-3\cos2x+3\cos^22x-\cos^32x)\dif x$\\
$\displaystyle\phantom{\int\sin^6x\dif x}=\int\frac{1}{8}\dif x-\frac{3}{8}\int\cos2x\dif x+\frac{3}{8}\int\cos^22x\dif x-\frac{1}{8}\int\cos^32x\dif x$\\
$\displaystyle\because\int\cos^22x\dif x=\frac{1}{2}\int(1+\cos4x)\dif x=\frac{1}{2}(\int\dif x+\int\cos4x\dif x)$\\
$\displaystyle\phantom{\because\int\cos^22x\dif x}=\frac{1}{2}+\frac{1}{8}\sin4x+C$\\
$\displaystyle\phantom{\because}\int\cos^32x\dif x=\int(1-\sin^22x)\cos2x\dif x$\\
\phantom{$\displaystyle\because$}设$t=\sin2x$, 则$\dif t=2\cos2x\dif x$\\
$\displaystyle\phantom{\because}\int\cos^32x\dif x=\frac{1}{2}\int(1-t^2)\dif t=\frac{1}{2}t-\frac{1}{6}t^3+C=\frac{1}{2}\sin2x-\frac{1}{6}\sin^32x+C$\\
$\displaystyle\therefore\int\sin^6x\dif x=\frac{1}{8}\int\dif x-\frac{3}{8}\int\cos2x\dif x+\frac{3}{8}\int\cos^22x\dif x-\frac{1}{8}\int\cos^32x\dif x$\\
$\displaystyle\phantom{\therefore\int\sin^6x\dif x}=\frac{1}{8}x-\frac{3}{16}\sin2x+\frac{3}{8}(\frac{1}{2}x+\frac{1}{8}\sin4x)-\frac{1}{8}(\frac{1}{2}\sin2x-\frac{1}{6}\sin^32x)+C$\\
$\displaystyle\phantom{\therefore\int\sin^6x\dif x}=\frac{5}{16}x-\frac{1}{4}\sin2x+\frac{1}{48}\sin^32x+\frac{3}{64}\sin4x+C$\\

\ding{173}使用分部积分\\
设$u=\sin^5x$, $\frac{\dif v}{\dif x}=\sin x$, 则:\\
\phantom{设}$\frac{\dif u}{\dif x}=5\sin^4x\cos x$, $v=-\cos x$\\
$\displaystyle\int\sin^6x\dif x=-\sin^5x\cos x+\int5\sin^4x\cos^2x\dif x$\\
$\displaystyle\phantom{\int\sin^6x\dif x}=-\sin^5x\cos x+5\int\sin^4x(1-\sin^2x)\dif x$\\
$\displaystyle\phantom{\int\sin^6x\dif x}=-\sin^5x\cos x+5\int\sin^4x\dif x-5\int\sin^6x\dif x$\\
将$\displaystyle5\int\sin^6x\dif x$移动到等式左侧, 得:\\
$\displaystyle\int\sin^6x\dif x=-\frac{1}{6}\sin^5x\cos x+\frac{5}{6}\int\sin^4x\dif x$\\
设$h=\sin^3x\dif x$, $\frac{\dif k}{\dif x}=\sin x$, 则:\\
\phantom{设}$\frac{\dif h}{\dif x}=3\sin^2x\cos x$, $k=-\cos x$\\
$\displaystyle\int\sin^4x\dif x=-\sin^3x\cos x+3\int\sin^2x\cos^2x\dif x$\\
$\displaystyle\phantom{\int\sin^4x\dif x}=-\sin^3x\cos x+3\int\sin^2x(1-\sin^2x)\dif x$\\
$\displaystyle\phantom{\int\sin^4x\dif x}=-\sin^3x\cos x+3\int\sin^2x\dif x-3\int\sin^4x\dif x$\\
将$\displaystyle3\int\sin^4x\dif x$移动到等式左侧, 得:\\
$\displaystyle\int\sin^4x\dif x=-\frac{1}{4}\sin^3x\cos x+\frac{3}{4}\int\sin^2x\dif x$\\
$\displaystyle\phantom{\int\sin^4x\dif x}=-\frac{1}{4}\sin^3x\cos x+\frac{3}{4}(\frac{1}{2}x-\frac{1}{4}\sin2x)+C$\\
$\displaystyle\phantom{\int\sin^4x\dif x}=-\frac{1}{4}\sin^3x\cos x+\frac{3}{8}x-\frac{3}{16}\sin2x+C$\\
$\displaystyle\therefore\int\sin^6x\dif x=-\frac{1}{6}\sin^5x\cos x+\frac{5}{6}\int\sin^4x\dif x$\\
$\displaystyle\phantom{\therefore\int\sin^6x\dif x}=-\frac{1}{6}\sin^5x\cos x+\frac{5}{6}(-\frac{1}{4}\sin^3x\cos x+\frac{3}{8}x-\frac{3}{16}\sin2x)+C$\\
$\displaystyle\phantom{\therefore\int\sin^6x\dif x}=-\frac{1}{6}\sin^5x\cos x-\frac{5}{24}\sin^3x\cos x+\frac{5}{16}x-\frac{5}{32}\sin2x+C$\\
$\displaystyle\phantom{\therefore\int\sin^6x\dif x}=-\frac{1}{48}(1-\cos2x)^2\sin2x-\frac{5}{96}(1-\cos2x)\sin2x+\frac{5}{16}x-\frac{5}{32}\sin2x+C$\\
$\displaystyle\phantom{\therefore\int\sin^6x\dif x}=-\frac{1}{48}(2\sin2x-\sin4x-\sin^32x)-\frac{5}{96}\sin2x+\frac{5}{192}\sin4x+\frac{5}{16}x-\frac{5}{32}\sin2x+C$\\
$\displaystyle\phantom{\therefore\int\sin^6x\dif x}=\frac{5}{16}x-(\frac{1}{24}+\frac{5}{96}+\frac{5}{32})\sin2x+\frac{1}{48}\sin^32x+(\frac{1}{48}+\frac{5}{192})\sin4x$\\
$\displaystyle\phantom{\therefore\int\sin^6x\dif x}=\frac{5}{16}x-\frac{1}{4}\sin2x+\frac{1}{48}\sin^32x+\frac{3}{64}\sin4x$\\

V、$\sin$与$\cos$混合, 并且至少其中一个为奇次幂\\
解题思路:\\
提取奇数幂项的单次幂与$\dif x$放置一起(当两个都为奇次幂时, 提取次方幂小的项)\\
例.\\
\phantom{例}$\displaystyle\int\cos^7x\sin^{10}x\dif x$\\
推导过程:\\
设$t=\sin x$,则$\dif t=\cos x\dif x$\\
$\displaystyle\int\cos^7x\sin^{10}x\dif x=\int(1-\sin^2x)^3\sin^{10}x\cos x\dif x=\int(1-t^2)\dif t$\\
$\displaystyle\phantom{\int\cos^7x\sin^{10}x\dif x}=\int(t^{10}-3t^{12}+3t^{14}-t^{16})\dif t$\\
$\displaystyle\phantom{\int\cos^7x\sin^{10}x\dif x}=\frac{1}{11}t^{11}-\frac{3}{13}t^{13}+\frac{1}{5}t^{15}-\frac{1}{17}t^{17}+C$\\
将$t=\sin x$代入上式:\\
$\displaystyle\int\cos^7x\sin^{10}x\dif x=\frac{1}{11}\sin^{11}x-\frac{3}{13}\sin^{13}x+\frac{1}{5}\sin^{15}x-\frac{1}{17}\sin^{17}x+C$\\

VI、$\sin$与$\cos$混合, 并且都为偶次幂\\
解题思路:\\
使用倍角公式, 降低次幂\\
例.\\
\phantom{例}$\displaystyle\int\cos^2x\sin^4x\dif x$\\
推导过程:\\
$\displaystyle\int\cos^2x\sin^4x\dif x=\frac{1}{8}\int(1+\cos2x)(1-2\cos2x+\cos^22x)\dif x$\\
$\displaystyle\phantom{\int\cos^2x\sin^4x\dif x}=\frac{1}{8}\int(1-\cos2x-\cos^22x+\cos^32x)\dif x$\\
$\displaystyle\phantom{\int\cos^2x\sin^4x\dif x}=\frac{1}{8}\int\dif x-\int\cos2x\dif x-\int\cos^22x\dif x+\int\cos^32x\dif x$\\
$\displaystyle\because\int\cos^2x\dif x=\frac{1}{2}\int(1+\cos4x)\dif x=\frac{1}{2}x+\frac{1}{8}\sin4x+C$\\
$\displaystyle\phantom{\because}\int\cos^32x\dif x=\int(1-\sin^22x)\cos2x\dif x=\int\cos2x\dif x-\int\sin^22x\cos2x\dif x$\\
$\displaystyle\phantom{\because\int\cos^32x\dif x}=\frac{1}{2}\sin2x-\int\sin^22x\cos2x\dif x$\\
\phantom{$\displaystyle\because$}设$t=\sin2x$, 则$\dif t=2\cos2x\dif x$\\
$\displaystyle\therefore\int\cos^32x\dif x=\frac{1}{2}t-\frac{1}{2}\int t^2\dif t=\frac{1}{2}t-\frac{1}{6}t^3+C$\\
$\displaystyle\phantom{\therefore\int\cos^32x\dif x}=\frac{1}{2}\sin2x-\frac{1}{6}\sin^32x+C$\\
$\displaystyle\therefore\int\cos^2x\sin^4x\dif x=\frac{1}{8}(x-\frac{1}{2}\sin2x-\frac{1}{2}x-\frac{1}{8}\sin4x+\frac{1}{2}\sin2x-\frac{1}{6}\sin^32x)+C$\\
$\displaystyle\phantom{\therefore\int\cos^2x\sin^4x\dif x}=\frac{1}{16}x-\frac{1}{48}\sin^32x-\frac{1}{64}\sin4x+C$\\

2)$\tan$的幂($\cot$类似)\\
I、一次幂\\
$\displaystyle\tan x\dif x=\int\frac{\sin x}{\cos x}\dif x=-\int\frac{1}{\cos x}-\sin x\dif x$\\
设$t=\cos x$, 则$\dif t=-\sin x\dif x$\\
$\displaystyle\int\tan x\dif x=-\int\frac{1}{t}\dif t=-\ln|t|+C=-\ln|\cos x|+C$\\

II、二次幂及以上次幂\\
解题思路:\\
依次提取$\tan^2x$, 转化为$\sec^2x-1$\\
例.\\
\phantom{例}$\displaystyle\int\tan^7x\dif x$\\
推导过程:\\
$\displaystyle\int\tan^7x\dif x=\int\tan^5x(\sec^2x-1)\dif x=\int\tan^5x\sec^2x\dif x-\int\tan^5x\dif x$\\
$\displaystyle\phantom{\int\tan^7x\dif x}=\int\tan^5x\sec^2x\dif x-\int\tan^3x(\sec^2x-1)\dif x$\\
$\displaystyle\phantom{\int\tan^7x\dif x}=\int\tan^5x\sec^2x\dif x-\int\tan^3x\sec^2x\dif x+\int\tan^3x\dif x$\\
$\displaystyle\phantom{\int\tan^7x\dif x}=\int\tan^5x\sec^2x\dif x-\int\tan^3x\sec^2x\dif x+\int\tan x\sec^2x\dif x-\int\tan x\dif x$\\
设$t=\tan x$, 则$\dif t=\sec^2x\dif x$\\
$\displaystyle\int\tan^5x\sec^2x\dif x=\int t^5\dif t=\frac{1}{6}t^6+C$\\
$\displaystyle\int\tan^3x\sec^2x\dif x=\int t^3\dif t=\frac{1}{4}t^4+C$\\
$\displaystyle\int\tan x\sec^2x\dif x=\int t\dif t=\frac{1}{2}t^2+C$\\
$\displaystyle\therefore\int\tan^7x\dif x=\frac{1}{6}t^6-\frac{1}{4}t^4+\frac{1}{2}t^2+\ln|\cos x|+C$\\
$\displaystyle\phantom{\therefore\int\tan^7x\dif x}=\frac{1}{6}\tan^6x-\frac{1}{4}\tan^4x+\frac{1}{2}\tan^2x+\ln|\cos x|+C$\\
总结:\\
$\displaystyle I_n=\frac{1}{n-1}\tan^{n-1}x-I_{n-2}$\\

3)$\sec$的幂($\csc$类似)\\
I、一次幂\\
$\displaystyle\int\sec x\dif x=\int\sec x\times\frac{\sec x+\tan x}{\sec x+\tan x}\dif x=\int\frac{\sec^2x+\tan x\sec x}{\sec x+\tan x}\dif x$\\
设$t=\sec x+\tan x$, 则$\dif t=(\tan x\sec x+\sec^2x)\dif x$\\
$\displaystyle\int\sec x\dif x=\int\frac{1}{t}\dif t=\ln|t|+C=\ln|\sec x+\tan x|+C$\\

II、二次幂\\
$\displaystyle\int\sec^2x\dif x=\tan x+C$\\

III、三次幂及以上次幂\\
解题思路:\\
提取$\sec^2x$, 应用分部积分法, 将其视为$\frac{\dif v}{\dif x}$部分\\
例.\\
\phantom{例}$\displaystyle\sec^6x\dif x$\\
推导过程:\\
设$u=\sec^4x$, $\frac{\dif v}{\dif x}=\sec^2x$, 则:\\
\phantom{设}$\frac{\dif u}{\dif x}=4\sec^4x\tan x$, $v=\tan x$\\
$\displaystyle\int\sec^6x\dif x=\sec^4x\tan x-4\int\sec^4x\tan^2x\dif x$\\
$\displaystyle\phantom{\int\sec^6x\dif x}=\sec^4x\tan x-4\int\sec^4x(\sec^2x-1)\dif x$\\
$\displaystyle\phantom{\int\sec^6x\dif x}=\sec^4x\tan x-4\int\sec^6x\dif x+4\int\sec^4x\dif x$\\
将$4\int\sec^6x\dif x$移动到等式左侧, 得:\\
$\displaystyle\int\sec^6x\dif x=\frac{1}{5}\sec^4x\tan x+\frac{4}{5}\int\sec^4x\dif x$\\
设$h=\sec^2x$, $\frac{\dif k}{\dif x}=\sec^2x$, 则:\\
\phantom{设}$\frac{\dif h}{\dif x}=2\sec^2x\tan x$, $k=\tan x$\\
$\displaystyle\int\sec^4x\dif x=\sec^2x\tan x-2\int\sec^2x\tan^2x\dif x$\\
$\displaystyle\phantom{\int\sec^4x\dif x}=\sec^2x\tan x-2\int\sec^2x(\sec^2x-1)\dif x$\\
$\displaystyle\phantom{\int\sec^4x\dif x}=\sec^2x\tan x-2\int\sec^4x\dif x+2\int\sec^2x\dif x$\\
将$2\int\sec^4x\dif x$移动到等式左侧, 得:\\
$\displaystyle\int\sec^4x\dif x=\frac{1}{3}\sec^2x\tan x+\frac{2}{3}\int\sec^2x\dif x$\\
$\displaystyle\therefore\int\sec^6x\dif x=\frac{1}{5}\sec^4x\tan x+\frac{4}{5}(\frac{1}{3}\sec^2x\tan x+\frac{2}{3}\int\sec^2x\dif x)$\\
$\displaystyle\phantom{\therefore\int\sec^6x\dif x}=\frac{1}{5}\sec^4x\tan x+\frac{4}{15}\sec^2x\tan x+\frac{8}{15}\int\sec^2x\dif x$\\
$\displaystyle\phantom{\therefore\int\sec^6x\dif x}=\frac{1}{5}\sec^4x\tan x+\frac{4}{15}\sec^2x\tan x+\frac{8}{15}\tan x+C$\\[2ex]

3.三角换元法的积分\\
1)$\sqrt{a^2-x^2}$的奇次幂\\
解题思路:\\
设$x=a\sin\theta$, 将$x$的积分转化为$\theta$的积分\\
例.\\
\phantom{例}$\displaystyle\int\frac{x^2}{(9-x^2)^{\frac{3}{2}}}\dif x$\\
推导过程:\\
设$x=3\sin\theta$, 则$\dif t=3\cos\theta\dif\theta$\\
$\displaystyle\int\frac{x^2}{(9-x^2)^{\frac{3}{2}}}\dif x=\int\frac{9\sin^2\theta}{27\cos^3\theta}\times3\cos\theta\dif\theta=\int\tan^2\theta\dif\theta$\\
$\displaystyle\phantom{\int\frac{x^2}{(9-x^2)^{\frac{3}{2}}}\dif x}\int(\sec^2\theta-1)\dif\theta=\tan\theta-\theta+C$\\
$\displaystyle\because\sin\theta=\frac{x}{3}$\\
$\displaystyle\therefore\cos\theta=\frac{\sqrt{9-x^2}}{3}$, $\theta=\sin^{-1}(\frac{x}{3})$\\
$\displaystyle\therefore\int\frac{x^2}{(9-x^2)^{\frac{3}{2}}}\dif x=\frac{x}{\sqrt{9-x^2}-\sin^{-1}(\frac{x}{3})}+C$\\

2)$\sqrt{x^2+a^2}$的奇次幂\\
解题思路:\\
设$x=a\tan\theta$, 将$x$的积分转化为$\theta$的积分\\
例.\\
\phantom{例}$\displaystyle\int(x^2+15)^{-\frac{5}{2}}\dif x$\\
推导过程:\\
设$x=\sqrt{15}\tan\theta$, 则$\dif x=\sqrt{15}\sec^2\theta\dif\theta$\\
$\displaystyle\int(x^2+15)^{-\frac{5}{2}}\dif x=\int\frac{1}{(\sqrt{15}\sec\theta)^5}\times\sqrt{15}\sec^2\theta\dif\theta=\frac{1}{225}\int\frac{1}{\sec^3\theta}\dif\theta$\\
$\displaystyle\phantom{\int(x^+15)^{-\frac{5}{2}}\dif x}=\frac{1}{225}\int\cos^3\theta\dif\theta$\\
$\displaystyle\because\int\cos^3\theta\dif\theta=\int(1-\sin^2\theta)\cos\theta\dif\theta$\\
\phantom{$\displaystyle\because$}设$t=\sin\theta$, 则$\dif t=\cos\theta\dif\theta$\\
$\displaystyle\phantom{\because}\int\cos^3\theta\dif\theta=\int(1-t^2)\dif t=t-\frac{1}{3}t^3+C=\sin\theta-\frac{1}{3}\sin^3\theta+C$\\
$\displaystyle\therefore\int(x^2+15)^{-\frac{5}{2}}\dif x=\frac{1}{225}(\sin\theta-\frac{1}{3}\sin^3\theta)+C$\\
$\displaystyle\because\tan\theta=\frac{x}{\sqrt{15}}$\\
$\displaystyle\therefore\sin\theta=\frac{x}{\sqrt{x^2+15}}$\\
$\displaystyle\therefore\int(x^2+15)^{-\frac{5}{2}}\dif x=\frac{1}{225}(\frac{x}{\sqrt{x^2+15}}-\frac{x^3}{3(x^2+15)^{\frac{3}{2}}})+C$\\

3)$\sqrt{x^2-a^2}$的奇次幂\\
解题思路:\\
设$x=a\sec\theta$, 将$x$的积分转化为$\theta$的积分\\
例.\\
\phantom{例}$\displaystyle\int\frac{1}{x^3\sqrt{x^2-4}}\dif x$\\
推导过程:\\
设$x=2\sec\theta$, 则$\dif x=2\tan\theta\sec\theta$\\
$\displaystyle\int\frac{1}{x^3\sqrt{x^2-4}}\dif x=\int\frac{2\tan\theta\sec\theta}{8\sec^3\theta\sqrt{4\tan^2\theta}}\dif\theta=\frac{1}{8}\int\frac{1}{\sec^2\theta}\dif\theta$\\
$\displaystyle\phantom{\int\frac{1}{x^3\sqrt{x^2-4}}\dif x}=\frac{1}{8}\int\cos^2\theta\dif\theta$\\
$\displaystyle\because\int\cos^2\theta\dif\theta=\frac{1}{2}\int(1+\cos2\theta)\dif\theta=\frac{\theta}{2}+\frac{\sin2\theta}{4}+C$\\
$\displaystyle\therefore\int\frac{1}{x^3\sqrt{x^2-4}}\dif x=\frac{1}{16}(\theta+\frac{\sin2\theta}{2})+C$\\
$\displaystyle\because\cos\theta=\frac{1}{\sec\theta}=\frac{2}{x}$\\
$\displaystyle\therefore\sin\theta=\frac{\sqrt{x^2-4}}{x},\sin2\theta=2\sin\theta\cos\theta=\frac{4\sqrt{x^2-4}}{x^2},\theta=\sec^{-1}(\frac{x}{2})$\\
$\displaystyle\therefore\int\frac{1}{x^3\sqrt{x^2-4}}\dif x=\frac{1}{16}(\sec^{-1}(\frac{x}{2})+\frac{2\sqrt{x^2-4}}{x^2})+C$\\

%最后编辑于: 2021-12-20
