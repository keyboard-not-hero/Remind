\chapter{指数函数和对数函数}
1.指数法则:\\
(1)\quad$b^0=1$\\
(2)\quad$b^1=b$\\
(3)\quad$b^xb^y=b^{x+y}$\\[1ex]
(4)\quad$\displaystyle\frac{b^x}{b^y}=b^{x-y}$\\[1ex]
(5)\quad$(b^x)^y=b^{xy}$\\

2.对数法则:\\
(1)\quad$\log_b(1)=0$\\
(2)\quad$\log_b(b)=1$\\
(3)\quad$\log_b(xy)=\log_b(x)+\log_b(y)$\\[1ex]
(4)\quad$\displaystyle\log_b(\frac{x}{y})=\log_b(x)-\log_b(y)$\\[1ex]
(5)\quad$\log_b(x^y)=y\log_b(x)$\\
(6)\quad 换底法则:
\[\log_b(x)=\frac{\log_c(x)}{\log_c(b)}\]

3.自然数$e$相关\\[1ex]
(1)\quad$\displaystyle\lim_{n\to\infty}(1+\frac{x}{n})^n=e^x$\\[1ex]
(2)\quad$\displaystyle\lim_{n\to\infty}(1+\frac{1}{n})^n=e$\\[1ex]
** $\log_e(x)/\ln(x)/\log(x)$具有相同意义

4.对数函数求导\\[1ex]
(1)\quad$\displaystyle\frac{\dif }{\dif x}\ln(x)=\frac{1}{x}$\\[1ex]
(2)\quad$\displaystyle\frac{\dif }{\dif x}\log_b(x)=\frac{1}{x\ln(b)}$\\[1ex]

5.指数函数求导\\[1ex]
(1)\quad$\displaystyle\frac{\dif }{\dif x}(b^x)=b^x\ln(b)$\\[1ex]
(2)\quad$\displaystyle\frac{\dif }{\dif x}(e^x)=e^x$\\[1ex]

6.指数函数在0附近的行为\\[1ex]
$\displaystyle\lim_{h\to 0}\frac{e^h-1}{h}=\lim_{h\to 0}\frac{e^{0+h}-e^0}{h}=\{(e^x)'|x=0\}=1$\\[1ex]

7.对数函数在1附近的行为\\[1ex]
$\displaystyle\lim_{h\to 0}\frac{\ln(1+h)}{h}=\lim_{h\to 0}\frac{\ln(1+h)-\ln(1)}{h}=\{\ln'(x)|x=1\}=1$\\[1ex]

8.指数函数在$\infty$或$-\infty$附近的行为\\[1ex]
(1)\quad$\displaystyle\lim_{x\to\infty}e^x=\infty$\\[1ex]
(2)\quad$\displaystyle\lim_{x\to -\infty}e^x=0$\\[1ex]
(3)\quad$\displaystyle\lim_{x\to\infty}\frac{x^n}{e^x}=0$\\[1ex]
(4)\quad$\displaystyle\lim_{x\to\infty}\frac{e^x}{x!}=0$\\[1ex]

9.对数函数在$\infty$附近的行为\\[1ex]
(1)\quad$\displaystyle\lim_{x\to\infty}\ln(x)=\infty$\\[1ex]
(2)\quad$\displaystyle\lim_{x\to\infty}\frac{\ln(x)}{x^a}=0$, 其中$a>0$\\[1ex]

10.对数函数在0附近的行为\\[1ex]
(1)\quad$\displaystyle\lim_{x\to 0^+}\ln(x)=-\infty$\\[1ex]
(2)\quad$\displaystyle\lim_{x\to 0^+}x^a\ln(x)=0$\\[1ex]

11.取对数求导法\\
当函数的底数和指数均为关于$x$的函数时, 通过对函数进行取对数, 让指数转化为乘数, 从而使用复合求导中的乘数求导法则解决问题\\
例.\\[1ex]
$\displaystyle y=x^{\sin(x)}$\\[1ex]
由原方程式等号两边取对数, 得:\\
\begin{equation}
\ln(y)=\sin(x)\ln(x)\label{eq:logarithm}
\end{equation}
公式\eqref{eq:logarithm}两边关于$x$隐式求导, 得:\\[1ex]
$\displaystyle\frac{1}{y}\frac{\dif y}{\dif x}=\ln(x)\cos(x)+\frac{\sin(x)}{x}$\\[1ex]
$\displaystyle\frac{\dif y}{\dif x}=\ln(x)\cos(x)x^{\sin(x)}+\sin(x)x^{\sin(x)-1}$\\[1ex]

12.指数增长与指数衰变\\
(1)指数增长方程: $P(t)=P_0e^{kt}$\\
(2)指数衰变方程: $P(t)=P_0e^{-kt}$\\

13.双曲函数\\[1ex]
(1)双曲余弦: $\displaystyle\cosh(x)=\frac{e^x+e^{-x}}{2}$\\[1ex]
(2)双曲正弦: $\displaystyle\sinh(x)=\frac{e^x-e^{-x}}{2}$\\[1ex]
(3)双曲线方程: $\cosh^2(x)-\sinh^2(x)=1$\\[1ex]
(4)导数:\\[1ex]
$\displaystyle\frac{\dif }{\dif x}\sinh(x)=\cosh(x)$\qquad$\displaystyle\frac{\dif }{\dif x}\cosh(x)=\sinh(x)$\\[1ex]
