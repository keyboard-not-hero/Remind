\chapter{方法、图像和直线}
1.函数\\
\textbf{函数}是将一个对象转化为另一个对象的规则. 起始对象称为\textbf{输入}, 来自称为\textbf{定义域}的集合. 返回对象称为\textbf{输出}, 来自称为\textbf{上域}的集合.

一个函数必须给每一个有效的输入指定唯一的输出.

\textbf{值域}是所有可能的输出所组成的集合.

\textbf{例1}.\\
$f(x)=x^2(x\in\mathbb{R}, f(x)\in\mathbb{R})$\\
在该示例中, 定义域为$\mathbb{R}$, 值域为$\mathbb{R}^+$, 上域为$\mathbb{R}$

区间定义:\\
$[a,b]$表示所有介于$a$和$b$ 之间(包括$a$和$b$)的实数的集合, 称为\textbf{闭区间}.\\
$(a,b)$表示所有介于$a$和$b$之间(不包括$a$和$b$)的实数的集合, 称为\textbf{开区间}.\\
$[a,b)$表示所有介于$a$和$b$之间(包括$a$, 不包括$b$)的实数的集合, 称为\textbf{半开区间}.

注意事项:\\
(1)分数的分母不能是零.\\
(2)不能取负数的偶次方根.\\
(3)不能取负数或零的对数.

垂线检验: 当任何一条垂直线与图像相交多于一次时, 该图像不是函数; 反之则图像为函数\\[2ex]

2.反函数\\
反函数的条件:\\
1)从一个函数$f$出发, 使得对于在$f$值域中的任意$y$, 都只有唯一的$x$值满足$f(x)=y$;\\
2)$f^{-1}$的定义域和$f$的值域相同;\\
3)$f^{-1}$的值域和$f$的定义域相同;\\
4)$f^{-1}(y)$的值就是满足$f(x)=y$的$x$. 即
\[\text{如果}\ f(x)=y, \text{则}\ f^{-1}(y)=x\]

水平线检验: 如果每一条水平线和一个函数的图像相交至多一次, 那么这个函数有反函数; 如果即使只有一条水平线和函数的图像相交多余一次, 那么这个函数没有反函数.

原始函数与反函数关于$y=x$对称

对于反函数, 有如下规则:\\
1)对于$f$值域中的所有$y$, 都有$f(f^{-1}(y))=y$;\\
2)对于$f$定义域中的$x$, 只有当$f$满足水平线检验时, 才有$f^{-1}(f(x))=x$.

例.\\
\phantom{例}$f(x)=\sin(x), \text{求}f^{-1}(f(x))$\\
推导过程:\\
当$f(x)$满足水平线检验时, $x\in[-\frac{\pi}{2},\frac{\pi}{2}]$\\
此时$f^{-1}(f(x))=x$\\[2ex]

3.复合函数\\
令$g(x)=x^2, h(x)=\cos(x)$, 而$f(x)=\cos(x^2)$, 则$f(x)=h(g(x))$, 也可表示为$f=h\circ g$, $f$为$g$与$h$的\textbf{复合函数}.\\[2ex]

4.奇函数与偶函数\\
如果$f$对定义域内的所有$x$有$f(-x)=f(x)$, 则$f$为\textbf{偶函数}.\\
如果$f$对定义域内的所有$x$有$f(-x)=-f(x)$, 则$f$为\textbf{奇函数}.

偶函数的图像关于y轴具有镜面对称性.\\
奇函数的图像关于原点有$180^o$的点对称性.\\[2ex]

5.线性函数\\
形如$f(x)=mx+b$的函数叫做\textbf{线性函数}.

点斜式方程:
\begin{center}
\begin{boxedminipage}{\textwidth}
	如果已知直线通过点$(x_0,y_0)$, 斜率为$m$, 则它的方程为$y-y_0=m(x-x_0)$
\end{boxedminipage}
\end{center}\vspace{4ex}

\begin{center}
\begin{boxedminipage}{11cm}
	如果一条直线通过点$(x_1,y_1)$和$(x_2,y_2)$, 则它的斜率等于$\displaystyle\frac{y_2-y_1}{x_2-x_1}$
\end{boxedminipage}
\end{center}\vspace{8ex}

6.常见函数\\
1)多项式\\
形如$f(x)=5x^4-4x^3+10$的函数\\[1ex]
基本项$x^n$的倍数叫做$x^n$的\textbf{系数}\\[1ex]
最大的幂指数$n$(该项系数不能为零)叫做多项式的\textbf{次数}\\[1ex]
最高次数项的系数称为\textbf{首项系数}\\

2)有理函数\\
形如$\displaystyle\frac{p(x)}{q(x)}$, 其中$p$和$q$为多项式\\

3)指数函数和对数函数\\
形如$y=a^x$的函数, 称为指数函数\\[1ex]
形如$\log_a(x)$的函数, 称为对数函数\\

4)带有绝对值的函数\\
\begin{center}
\begin{boxedminipage}{5cm}
\[|x|=\left\{
	\begin{array}{r l}
			x & \text{if}\ x\geqslant0\\
			-x & \text{if}\ x<0
	\end{array}
\right.\]
\end{boxedminipage}
\end{center}

%最后编辑于: 2022-01-19
