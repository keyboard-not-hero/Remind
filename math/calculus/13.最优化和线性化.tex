\chapter{最优化和线性化}
1.最优化方案\\
(1)识别可能用到的所有变量;\\
(2)在极端情况下, 变量的取值范围;\\
(3)列出关联不同变量的方程组;\\
(4)通过方程组消去变量, 使得因变量(目标)可以表示为只关于一个自变量的函数;\\
(5)对因变量关于自变量求导, 找出临界点;\\
(6)通过一阶或二阶导数的符号表格求出最大值或最小值;\\
(7)得出最终结论.\\[2ex]

2.线性化方案\\
(1)将估算量写成适当的函数$f(x)$, 则当前值为$f(a)$;\\
(2)选取某个与值$a$接近的自变量值$b$, 并且$f(b)$便于计算;\\
(3)找出通过曲线$f(x)$上点$(b,f(b))$的切线, 方程为: $g(x)-f(b)=f'(b)(x-b)$;\\
(4)最后结果$f(x)\approx g(x)=f'(b)(x-b)+f(b)$, 函数$g(x)$称为$f(x)$在$x=b$处的\textbf{线性化}.\\[2ex]

3.近似估算 - 牛顿法\\[-4ex]
\begin{theorem}[牛顿法]
假设$a$是对方程$f(x)=0$的解的一个近似. 如果令
\[b=a-\frac{f(a)}{f'(a)}\]
则在很多情况下, $b$是个比$a$更好的近似.
\end{theorem}
牛顿法不起作用的四个情况:\\
(1)$f'(a)$的值接近于$0$;\\
(2)如果$f(x)=0$有不止一个解, 可能得到的不是你想要的那个解;\\
(3)近似可能变得越来越糟. 如: $f(x)=x^{\frac{1}{3}}$;\\
(4)陷入循环.\\
