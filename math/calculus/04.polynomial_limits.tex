\documentclass[UTF8, fontset=ubuntu]{ctexart}
\usepackage{parskip}
\usepackage{amsmath}
\usepackage{amssymb}
\begin{document}
1.$x\to a$时的有理函数的极限\\
有理函数: 形如$f(x)=\frac{p(x)}{q(x)}$的函数, 其中$p(x),q(x)$都是多项式.\\
i.当$f(a)=\frac{p(a)}{q(a)}=\frac{m}{n}$时:\\
$\displaystyle\lim_{x\to a}f(x)=\frac{m}{n}$\\
例.
\[\lim_{x\to -1}\frac{x^2-9}{x-2}=\lim_{x\to -1}\frac{x-9}{x-2}=\frac{1-9}{1-2}=8\]

ii.当$f(a)=\frac{p(a)}{q(a)}=\frac{0}{0}$时:\\
$\displaystyle\lim_{x\to a}f(x)$进行分子分母约分\\
例.
\[\lim_{x\to 2}\frac{x^2-3x+2}{x-2}=\lim_{x\to 2}\frac{(x-1)(x-2)}{x-2}=\lim{x\to 2}x-1=2-1=1\]

iii.当$f(a)=\frac{p(a)}{q(a)}=\frac{m}{0}$时:\\
$\displaystyle\lim_{x\to a}f(x)$判断极限点两边的极限是否同为$\infty$或$-\infty$\\
例.
\begin{displaymath}
\begin{array}{l}
    \displaystyle\lim_{x\to 1}\frac{2x^2-x-6}{x(x-1)^3}=\lim_{x\to 1}\frac{(2x+3)(x-2)}{x(x-1)^3}\\
	\displaystyle\mathbf{\because}\lim_{x\to 1^-}\frac{(2x+3)(x-2)}{x(x-1)^3}=\frac{(+)(-)}{(+)(-)}=+\\
    \displaystyle\phantom{\because}\lim_{x\to 1^+}\frac{(2x+3)(x-2)}{x(x-1)^3}=\frac{(+)(-)}{(+)(+)}=-\\
	\displaystyle\mathbf{\therefore}\,f(x)\text{无极限值}
\end{array}
\end{displaymath}\vspace{2ex}

2.$x\to a$时的平方根的极限\\
共轭因式: 若S是含有根式的已知表达式, 若存在一个不恒等于零的表达式M, 使乘积SM不含根式, 则M为S的共轭因式. 反之, S也为M的共轭因式.\\
设$f(x)=\frac{g(x)\pm h(x)}{p(x)\pm q(x)}$, 其中, g(x)/h(x)/p(x)/q(x)其中一个为根式\\
当$f(a)=\frac{g(a)-h(a)}{p(a)-q(a)}=\frac{0}{0}$时, 将分子分母同时乘以含根号部分的共轭因式.\\
例.
\begin{displaymath}
\begin{array}{l}
    \displaystyle\lim_{x\to 5}\frac{\sqrt{x^2-9}-4}{x-5}=\lim_{x\to 5}\frac{\sqrt{x^2-9}-4}{x-5}\times\frac{\sqrt{x^2-9}+4}{\sqrt{x^2-9}+4}=\lim_{x\to 5}\frac{x^2-25}{(x-5)(\sqrt{x^2-9}+4)}\\
    \displaystyle\phantom{\lim_{x\to 5}\frac{\sqrt{x^2-9}-4}{x-5}}=\lim_{x\to 5}\frac{x+5}{\sqrt{x^-9}+4}=\frac{5+5}{\sqrt{25-9}+4}=\frac{5}{4}
\end{array}
\end{displaymath}\vspace{2ex}
3.
\end{document}
