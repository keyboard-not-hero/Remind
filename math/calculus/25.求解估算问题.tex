\chapter{求解估算问题}
1.求泰勒级数步骤\\
(1)构造一个导数表($n/f^{(n)}(x)/f^{(n)}(a)$)\\
(2)将求导结果代入泰勒级数\\
例.\\
$f(x)=e^x$关于$x=-2$的泰勒级数\\
推导过程:\\
导数表如下\\[1ex]
$\begin{array}{r|r|r}
\hline
n & f^{(n)}(x) & f^{(n)}(a)\\
\hline
0 & e^x & e^{-2}\\
1 & e^x & e^{-2}\\
2 & e^x & e^{-2}\\
3 & e^x & e^{-2}\\
\hline
\end{array}$\\[1ex]
函数关于$x=-2$的泰勒级数\\
$\displaystyle f(a)+f'(a)(x-a)+f''(a)(x-a)^2+\cdots=e^{-2}+e^{-2}(x+2)+e^{-2}(x+2)^2+\cdots$\\[4ex]

2.用误差项估算问题\\
(1)构造相关函数$f(x)$\\
(2)选一个接近$x$的值$a$,使$f(a)/f'(a)$易于计算\\
(3)构造$f$的导数表\\
(4)$R_N$的公式:$\displaystyle R_N(x)=\frac{f^{(N+1)}(x)}{(N+1)!}(x-a)^{N+1}$\\
(5)确定泰勒多项式的阶$N$(未知时可反复代入进行测试)\\
(6)确定$x$的值,并且$c$介于$a$与$x$之间\\
(7)求$|R_N(x)|$的最大值\\
(8)根据估算,得出泰勒多项式$P_N(x)=f(a)+f'(a)(x-a)+\cdots+f^{(N)}(a)(x-a)^N$\\
例1.\\
用二阶泰勒多项式估算$e^{\frac{1}{3}}$,并估算误差\\
推导过程:\\
构造函数$f(x)=e^x$\\
由于$e^0=1$,取值$a=0$\\
$f(x)$关于$x=0$的导数表\\[1ex]
$\begin{array}{c|c|c}
\hline
n & f^{(n)}(x) & f^{(n)}(a)\\
\hline
0 & e^x & 1\\
1 & e^x & 1\\
2 & e^x & 1\\
3 & e^x & 1\\
\hline
\end{array}$\\[1ex]
$\displaystyle R_2(x)=\frac{f^{(3)}(c)}{3!}x^3=\frac{e^c}{3!}x^3$\\
由于$c$介于$a$与$x$之间\\
$\displaystyle 0<c<\frac{1}{3}$\\
$\displaystyle|R_2(\frac{1}{3})|\leqslant\frac{e^{\frac{1}{3}}}{3!}\times(\frac{1}{3})^3=\frac{e^{\frac{1}{3}}}{162}<\frac{8^{\frac{1}{3}}}{162}=\frac{1}{81}$\\
$\displaystyle P_2(x)=f(0)+f'(0)x+\frac{f''(0)}{2!}x^2=1+x+\frac{x^2}{2}$\\
$\displaystyle P_2(\frac{1}{3})=1+\frac{1}{3}+frac{1}{2}\times(\frac{1}{3})^2=\frac{25}{18}$\\
$\displaystyle e^{\frac{1}{3}}=f(\frac{1}{3})\approx P_2(\frac{1}{3})=\frac{25}{18}$\\[1ex]

例2.\\
估算$e^{\frac{1}{3}}$,且误差不得大于$\displaystyle\frac{1}{10000}$\\
推导过程:\\
构造函数$f(x)=e^x$\\
由于$e^0=1$,取值$a=0$\\
$f(x)$关于$x=0$的导数表\\[1ex]
$\begin{array}{c|c|c}
\hline
n & f^{(n)}(x) & f^{(n)}(a)\\
\hline
0 & e^x & 1\\
1 & e^x & 1\\
2 & e^x & 1\\
\vdots & \vdots & \vdots\\
\hline
\end{array}$\\[1ex]
$\displaystyle R_N(x)=\frac{f^{N+1}(c)}{(N+1)!}x^{N+1}$\\
由例1可知,$N=2$时,$\displaystyle R_2(\frac{1}{3})=\frac{1}{81}>\frac{1}{10000}$\\
将$N=3$代入误差项,得:\\
$\displaystyle R_3(\frac{1}{3})=\frac{f^{(4)}(c)}{4!}\times(\frac{1}{3})^4=\frac{e^c}{1944}<\frac{8^{\frac{1}{3}}}{1944}=\frac{1}{972}$\\
而$\displaystyle\frac{1}{972}>\frac{1}{10000}$,$N=3$不符合\\
将$N=4$代入误差项,得:\\
$\displaystyle R_4(\frac{1}{3})=\frac{f^{(5)}(c)}{5!}\times(\frac{1}{3})^5=\frac{e^c}{29160}<\frac{8^{\frac{1}{3}}}{29160}=\frac{1}{14580}$\\
$\displaystyle\frac{1}{14580}<\frac{1}{10000}$,$N=4$符合误差要求\\
$\begin{array}{>{\displaystyle}r >{\displaystyle}l}
P_4(\frac{1}{3}) & =f(0)+f'(0)x+\frac{f''(0)}{2!}x^2+\frac{f^{(3)}(0)}{3!}x^3+\frac{f^{(4)}(0)}{4!}x^4\\
& =1+\frac{1}{3}+\frac{1}{2}\times\frac{1}{9}+\frac{1}{6}\times\frac{1}{27}+\frac{1}{24}\times\frac{1}{81}\\
& =\frac{2713}{1944}
\end{array}$\\
$\displaystyle e^{\frac{1}{3}}=f(\frac{1}{3})\approx P_4(\frac{1}{3})=\frac{2713}{1944}$\\[2ex]

3.满足交错级数判别法的估算
{\par\centering
\framebox{\begin{minipage}{\textwidth}
泰勒级数若是各项绝对值递减趋于0的交错级数,则误差小于下一项. 即
\[\displaystyle|R_N(x)|\leqslant\bigg|\frac{f^{(N+1)}(a)}{(N+1)!}(x-a)^{N+1}\bigg|\]
\end{minipage}}
\par}
例.\\
使用麦克劳林级数求定积分$\displaystyle\int_0^{\frac{1}{2}}\frac{\sin(t)}{t}\dif t$,并且误差为$\displaystyle\frac{1}{1000}$\\
推导过程:\\
构造函数$f(x)$,使得\\
$\displaystyle f(x)=\int_0^x\frac{\sin(t)}{t}\dif t$\\
$\sin(t)$的关于$t=0$的导数表\\[1ex]
$\begin{array}{c|c|c}
\hline
n & f^{(n)}(x) & f^{(n)}(a)\\
\hline
0 & \sin(x) & 0\\
1 & \cos(x) & 1\\
2 & -\sin(x) & 0\\
3 & -\cos(x) & -1\\
\hline
\end{array}$\\[1ex]
$\sin(t)$的麦克劳林级数如下:\\
$\begin{array}{>{\displaystyle}r >{\displaystyle}l}
\sin(t) & =f(0)+f'(0)t+\frac{f''(0)}{2!}t^2+\frac{f^{(3)}(0)}{3!}t^3+\cdots\\[1ex]
& =t-\frac{t^3}{3!}+\frac{t^5}{5!}-\frac{t^7}{7!}+\cdots
\end{array}$\\[1ex]
将$\sin(t)$的麦克劳林级数代入函数$f(x)$,得:\\
$\begin{array}{>{\displaystyle}r >{\displaystyle}l}
f(x) & =\int_0^x(1-\frac{t^2}{3!}+\frac{t^4}{5!}-\frac{t^6}{7!}+\cdots)\dif t\\[1ex]
& =\left(t-\frac{t^3}{3\times 3!}+\frac{t^5}{5\times 5!}-\frac{t^7}{7\times 7!}+\cdots\right)\Big|_0^x\\[1ex]
& =x-\frac{x^3}{3\times 3!}+\frac{x^5}{5\times 5!}-\frac{x^7}{7\times 7!}+\cdots
\end{array}$\\[1ex]
$\displaystyle f(\frac{1}{2})=\frac{1}{2}-\frac{1}{3\times 3!}\times(\frac{1}{2})^3+\frac{1}{5\times 5!}\times(\frac{1}{2})^5-\frac{1}{7\times 7!}\times(\frac{1}{2})^7+\cdots$\\
由于$f(x)$符合交错级数判别法\\
首先使用第一项近似积分,得:\\
$\displaystyle|R_1(\frac{1}{2})|\leqslant|-\frac{1}{3\times 3!}\times(\frac{1}{2})^3|=\frac{1}{18}\times\frac{1}{8}=\frac{1}{144}$\\
但是$\displaystyle\frac{1}{144}>\frac{1}{1000}$\\
继续往后取项,使用前两项近似积分,得:\\
$\displaystyle|R_2(\frac{1}{2})|\leqslant|\frac{1}{5\times 5!}\times(\frac{1}{2})^5|=\frac{1}{600}\times\frac{1}{64}=\frac{1}{38400}$\\
由于$\displaystyle\frac{1}{38400}<\frac{1}{1000}$\\
$\displaystyle f(\frac{1}{2})\approx\frac{1}{2}-\frac{1}{3\times 3!}\times(\frac{1}{2})^3=\frac{1}{2}-\frac{1}{144}=\frac{71}{144}$


