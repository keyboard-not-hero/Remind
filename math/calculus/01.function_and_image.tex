\documentclass[UTF8, fontset=ubuntu]{ctexart}
\usepackage{parskip}
\usepackage{amssymb}
\begin{document}
\textbf{函数}是将一个对象转化为另一个对象的规则. 其实对象称为\textbf{输入}, 来自称为\textbf{定义域}的集合. 返回对象成为{输出}, 来自称为{上域}的集合.

\textbf{值域}是所有可能的输出所组成的集合.\\
\textbf{例1}.\\
$f(x)=x^2(x\in\mathbb{R}, f(x)\in\mathbb{R})$\\
在该示例中, 定义域为$\mathbb{R}$, 值域为$\mathbb{R}^+_0$, 上域为$\mathbb{R}$

$[a,b]$的含义为$a\leqslant x\leqslant b$, 称为\textbf{闭区间}.\\
$(a,b)$的含义为$a<x<b$, 称为\textbf{开区间}.\\
$[a,b)$的含义为$a\leqslant x<b$, 称为\textbf{半开半闭区间}.

注意事项:\\
(1)分数的分母不能是零.\\
(2)不能取负数的偶次方根.\\
(3)不能取负数或零的对数.

垂线检验: 当任何一条垂直线与图像相交多于一次时, 该图像不是函数; 反之则图像为函数

从输出$y$出发, 这个新的函数发现一个且仅有一个输入$x$满足$f(x)=y$, 这个新的函数称为\textbf{反函数}. 写作$f^{-1}$.

水平线检验: 如果每一条水平线和一个函数的图像相交至多一次, 那么这个函数有反函数; 如果即使只有一条水平线和函数的图像相交多余一次, 那么这个函数没有反函数.

令$g(x)=x^2, h(x)=cos(x)$, 而$f(x)=cos(x^2)$, 则$f(x)=h(g(x))$, 也可表示为$f=h\circ g$, $f$为$g$与$h$的\textbf{复合}, $f(x)$为\textbf{复合函数}.

如果$f$对定义域内的所有$x$有$f(-x)=f(x)$, 则$f$为\textbf{偶函数}.\\
如果$f$对定义域内的所有$x$有$f(-x)=-f(x)$, 则$f$为\textbf{奇函数}.

偶函数的图像关于y轴具有镜面对称性.\\
奇函数的图像关于原点有$180^o$的点对称性.

形如$f(x)=mx+b$的函数叫做\textbf{线性函数}.
\end{document}
