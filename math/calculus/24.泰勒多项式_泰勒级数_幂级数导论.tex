\chapter{泰勒多项式、泰勒级数和幂级数导论}
1.近似值和泰勒多项式\\
\begin{center}
\framebox{\begin{minipage}{\textwidth}
$e^x$曲线在$x=0$处的三阶近似曲线
\[e^x\approx 1+x+\frac{x^2}{2}+\frac{x^3}{6}\]
并且,$x$越趋近于$0$,曲线趋近程度越高
\end{minipage}}
\end{center}

\begin{center}
\framebox{\begin{minipage}{\textwidth}
\textbf{泰勒近似定理:}若$f$在$x=a$光滑,则在所有次数为$N$或更低的多项式中,当$x$在$a$附近时,最近似于$f(x)$的是
\[P_N(x)=f(a)+f'(a)(x-a)+\frac{f''(a)}{2}(x-a)^2+\cdots+\frac{f^{(N)}(a)}{N!}(x-a)^N\]
使用求和符号表示:
\[P_N(x)=\sum_{n=0}^{\infty}\frac{f^{(n)}(a)}{n!}(x-a)^n\]
多项式$P_N(x)$称为$f(x)$在$x=a$处的\textbf{N阶泰勒多项式}
\end{minipage}}
\end{center}

\begin{center}
\framebox{\begin{minipage}{\textwidth}
曲线$f(x)$与近似曲线$P_N(x)$的误差称为\textbf{N阶误差项},也称为\textbf{N阶余项},公式表示为:
\[R_N(x)=f(x)-P_N(x)\]
\end{minipage}}
\end{center}

\begin{center}
\framebox{\begin{minipage}{\textwidth}
\textbf{泰勒定理:}关于$x=a$的N阶余项
\[R_N(x)=\frac{f^{(N+1)}(c)}{(N+1)!}(x-a)^{N+1}\]
其中,$c$是介于$x$与$a$之间的一个数
\end{minipage}}
\end{center}\vspace{4ex}

2.幂级数和泰勒级数\\
\begin{center}
\framebox{\begin{minipage}{\textwidth}
\textbf{幂级数:}关于$x=a$($a$为中心)的幂级数的表达式
\[\sum_{n=0}^{\infty}a_n(x-a)^n=a_0+a_1(x-a)+a_2(x-a)^2+\cdots\]
其中$a_n$是确定的常数
\end{minipage}}
\end{center}

\begin{center}
\framebox{\begin{minipage}{\textwidth}
\textbf{幂级数:}关于$x=0$($0$为中心)的幂级数的表达式
\[\sum_{n=0}^{\infty}a_nx^n=a_0+a_1x+a_2x^2+\cdots\]
其中$a_n$是确定的常数
\end{minipage}}
\end{center}

\begin{center}
\framebox{\begin{minipage}{\textwidth}
$e^x$曲线关于$x=0$的幂级数
\[e^x=\sum_{n=0}^{\infty}\frac{1}{n!}x^n=1+x+\frac{x^2}{2!}+\frac{x^3}{3!}+\cdots\]
幂级数收敛于$e^x$
\end{minipage}}
\end{center}

\begin{center}
\framebox{\begin{minipage}{\textwidth}
$\displaystyle\frac{1}{1-x}$曲线关于$x=0$的幂级数
\[\frac{1}{1-x}=1+x+x^2+x^3+\cdots\]
当$-1<x<1$时,幂级数收敛于$\displaystyle\frac{1}{1-x}$
\end{minipage}}
\end{center}

\begin{center}
\framebox{\begin{minipage}{\textwidth}
使用光滑函数$f$的所有导数定义关于$x=a$的幂级数
\[\sum_{n=0}^{\infty}\frac{f^{(n)}(a)}{n!}(x-a)^n=f(a)+f'(a)(x-a)+f''(a)(x-a)^2+\cdots\]
该幂级数的系数为$\displaystyle a_n=\frac{f{(n)}(a)}{n!}$,该级数称为$f$关于$x=a$的\textbf{泰勒级数}
\end{minipage}}
\end{center}

\begin{center}
\framebox{\begin{minipage}{\textwidth}
使用光滑函数$f$的所有导数定义关于$x=0$的幂级数
\[\sum_{n=0}^{\infty}\frac{f^{(n)}(0)}{n!}x^n=f(0)+f'(0)x+f''(0)x^2+\cdots\]
该幂级数的系数为$\displaystyle a_n=\frac{f{(n)}(0)}{n!}$,该级数称为$f$关于$x=0$的\textbf{麦克劳林级数}

\end{minipage}}
\end{center}
