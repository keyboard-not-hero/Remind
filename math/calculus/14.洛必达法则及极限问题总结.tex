\chapter{洛必达法则及极限问题总结}
1.类型$A$: $\displaystyle\frac{0}{0}$
{\par\raggedright
\framebox{如果$f(a)=g(a)=0$, 那么$\displaystyle\lim_{x\to a}\frac{f(a)}{g(a)}=\lim_{x\to a}\frac{f'(x)}{g'(x)}$}\
\par}\vspace{2ex}
例.\\
\phantom{例}$\displaystyle\lim_{x\to 0}\frac{x-\sin x}{x^3}$\\
$\displaystyle\lim_{x\to 0}\frac{x-\sin x}{x^3}=\lim_{x\to 0}\frac{1-\cos x}{3x^2}=\lim_{x\to 0}\frac{\sin x}{6x}=\frac{1}{6}$\\[2ex]

2.类型$A$: $\displaystyle\frac{\pm\infty}{\pm\infty}$
{\par\raggedright
\framebox{如果$f(a)=g(a)=\pm\infty$, 那么$\displaystyle\lim_{x\to a}\frac{f(a)}{g(a)}=\lim_{x\to a}\frac{f'(x)}{g'(x)}$}
\par}\vspace{2ex}
例1.\\
\phantom{例}$\displaystyle\lim_{x\to\infty}\frac{3x^2+7x}{2x^2-5}$\\
$\displaystyle\lim_{x\to\infty}\frac{3x^2+7x}{2x^2-5}=\lim_{x\to\infty}\frac{6x+7}{4x}=\lim_{x\to\infty}(\frac{6}{4}+\frac{7}{4x})=\frac{3}{2}$\\[2ex]

例2.\\
\phantom{例}$\displaystyle\lim_{x\to 0^+}\frac{\csc x}{1-\ln x}$\\
$\displaystyle\lim_{x\to 0^+}\frac{\csc x}{1-\ln x}=\lim_{x\to 0^+}\frac{-\csc x\cot x}{-\frac{1}{x}}=\lim_{x\to 0^+}\frac{x}{\sin x}\frac{1}{\tan x}=1\times\infty=\infty$\\[2ex]

例3.\\
\phantom{例}$\displaystyle\lim_{x\to\infty}\frac{x}{e^x}$\\
$\displaystyle\lim_{x\to\infty}\frac{x}{e^x}=\lim_{x\to\infty}\frac{1}{e^x}=0$\\[2ex]

3.类型$B1$: $\infty -\infty$\\
方法: 通过通分或者分子/分母同时乘以共轭表达式来转化为类型$A$\\
例.\\
\phantom{例}$\displaystyle\lim_{x\to 0}(\frac{1}{\sin x}-\frac{1}{x})$\\
$\displaystyle\lim_{x\to 0}(\frac{1}{\sin x}-\frac{1}{x})=\lim_{x\to 0}\frac{x-\sin x}{x\sin x}=\lim_{x\to 0}\frac{1-\cos x}{\sin x+x\cos x}$\\
$\displaystyle\phantom{\lim_{x\to 0}(\frac{1}{\sin x}-\frac{1}{x})}=\lim_{x\to 0}\frac{\sin x}{2\cos x-x\sin x}=0$\\[2ex]

4.类型$B2$: $0\times\pm\infty$\\
方法: 选择两个因式中较简单的那个取倒数把它移到分母(尽量不要选用对数做分母, 把它留在分子)\\
例.\\
\phantom{例}$\displaystyle\lim_{x\to 0^+}x\ln x$\\
$\displaystyle\lim_{x\to 0^+}x\ln x=\lim_{x\to 0^+}\frac{\ln x}{\frac{1}{x}}=\lim_{x\to 0^+}\frac{\frac{1}{x}}{-\frac{1}{x^2}}=\lim_{x\to 0^+}(-x)=0$\\[2ex]

5.类型$C$: $1^{\pm\infty}$,$0^0$,$\infty^0$\\
通过取对数, 转化为类型$B2$或$A$, 计算获得极限$L$, 再以$e$为底/$L$为幂获取最终结果\\
例1.\\
\phantom{例}$\displaystyle\lim_{x\to 0^+}x^{\sin x}$\\
$\displaystyle\lim_{x\to 0^+}\ln(x^{\sin x})=\lim_{x\to 0^+}\sin x\ln x=\lim_{x\to 0^+}\frac{\ln x}{\csc x}=\lim_{x\to 0^+}\frac{\frac{1}{x}}{-\csc x\cot x}$\\
$\displaystyle\phantom{\lim_{x\to 0^+}\ln(x^{\sin x})}=\lim_{x\to 0^+}-\frac{\sin x}{x}\times\tan x=0$\\
对两边求指数, 得:\\
$\displaystyle\lim_{x\to 0^+}x^{\sin x}=e^0=1$\\[2ex]

例2.\\
\phantom{例}$\displaystyle\lim_{x\to 0}(1+3\tan x)^{\frac{1}{x}}$\\
$\displaystyle\lim_{x\to 0}\ln((1+3\tan x)^{\frac{1}{x}})=\lim_{x\to 0}\frac{\ln(1+3\tan x)}{x}=\lim_{x\to 0}\frac{\frac{3\sec^2x}{1+3\tan x}}{1}$\\
$\displaystyle\phantom{\lim_{x\to 0}\ln((1+3\tan x)^{\frac{1}{x}})}=\frac{3\times1^2}{1+3\times0}=3$\\
对两边求指数, 得:\\
$\displaystyle\lim_{x\to 0}(1+3\tan x)^{\frac{1}{x}}=e^3$

%最后编辑于: 2021-12-28
