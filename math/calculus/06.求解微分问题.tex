\documentclass[UTF8, fontset=ubuntu]{ctexart}
\usepackage{amsmath}
\usepackage{amssymb}
\usepackage{parskip}
\begin{document}
1.使用定义求导\\
\framebox{$\displaystyle\frac{d}{dx}(x)=1$}\\
证明:\\
$\displaystyle\lim_{h\to 0}\frac{f(x+h)-f(x)}{h}=\lim_{h\to 0}\frac{(x+h)-x}{h}=\lim_{h\to 0}\frac{h}{h}=1$\\[4ex]

\framebox{$\displaystyle\frac{d}{dx}(\sqrt{x})=\frac{1}{2\sqrt{x}}$}\\
证明:\\
\begin{math}
\begin{array}{l}
\displaystyle\lim_{h\to 0}\frac{f(x+h)-f(x)}{h}=\lim_{h\to 0}\frac{\sqrt{x+h}-\sqrt{x}}{h}=\lim_{h\to 0}\frac{\sqrt{x+h}-\sqrt{x}}{h}\times\frac{\sqrt{x+h}+\sqrt{x}}{\sqrt{x+h}+\sqrt{x}}\\
\displaystyle\phantom{\lim_{h\to 0}\frac{f(x+h)-f(x)}{h}}=\lim_{h\to 0}\frac{h}{h(\sqrt{x+h}+\sqrt{x})}=\lim_{h\to 0}\frac{1}{\sqrt{x+h}+\sqrt{x}}=\frac{1}{2\sqrt{x}}
\end{array}
\end{math}\\[4ex]

\framebox{$\displaystyle\frac{d}{dx}(\frac{1}{x})=-\frac{1}{x^2}$}\\
证明:\\
\begin{math}
\begin{array}{l}
\displaystyle\lim_{h\to 0}\frac{f(x+h)-f(x)}{h}=\lim_{h\to 0}\frac{\frac{1}{x+h}-\frac{1}{x}}{h}=\lim_{h\to 0}\frac{-\frac{h}{x(x+h)}}{h}\\
\displaystyle\phantom{\lim_{h\to 0}\frac{f(x+h)-f(x)}{h}}=\lim_{h\to 0}-\frac{1}{x(x+h)}=-\frac{1}{x^2}
\end{array}
\end{math}\\[4ex]

\framebox{$\displaystyle\frac{d}{dx}(x^a)=ax^{a-1}$}\\[4ex]

2.运算法则\\
(1)常数倍\\
$\displaystyle\frac{d}{dx}(Cx^a)=(Ca)x^{a-1}$\\[4ex]
(2)加/减法法则\\
$\displaystyle\frac{d}{dx}(x^a+\sqrt{x})=\frac{d}{dx}(x^a)+\frac{d}{dx}(\sqrt{x})$\\[4ex]
(3)乘积法则\\[-6ex]
\begin{center}
\framebox{\textbf{乘积法则(版本1)}\qquad 如果$h(x)=f(x)g(x)$, 那么$h'(x)=f'(x)g(x)+f(x)g'(x)$.}\\[2ex]
\framebox{
\begin{minipage}[c]{6cm}
\textbf{乘积法则(版本2)}\quad 如果$y=uv$, 则
\[\frac{dy}{dx}=v\frac{du}{dx}+u\frac{dv}{dx}\]
\end{minipage}}\\[4ex]
\framebox{
\begin{minipage}[c]{7cm}
\textbf{乘积法则(三个变量)}\quad 如果$y=uvw$, 则
\[\frac{dy}{dx}=\frac{du}{dx}vw+u\frac{dv}{dx}w+uv\frac{dw}{dx}\]
\end{minipage}}
\end{center}\vspace{4ex}
(4)商法则\\[-4ex]
\begin{center}
\framebox{
\begin{minipage}[c]{8cm}
\textbf{商法则(版本1)}\quad 如果$h(x)=\frac{f(x)}{g(x)}$, 那么
\[h'(x)=\frac{f'(x)g(x)-f(x)g'(x)}{(g(x))^2}\]
\end{minipage}}\\[2ex]
\framebox{
\begin{minipage}[c]{6cm}
\textbf{商法则(版本2)}\quad 如果$y=\frac{u}{v}$, 那么
\[\frac{dy}{dx}=\frac{v\frac{du}{dx}-u\frac{dv}{dx}}{v^2}\]
\end{minipage}}
\end{center}\vspace{4ex}
(5)链式求导法则\\[-6ex]
\begin{center}
\framebox{\textbf{链式求导法则(版本1)}\quad 如果$h(x)=f(g(x))$, 那么$h'(x)=f'(g(x))g'(x)$.}\\[2ex]
\framebox{
\begin{minipage}[c]{12cm}
\textbf{链式求导法则(版本2)}\quad 如果$y$是$u$的函数, 并且$u$是$x$的函数, 那么
\[\frac{dy}{dx}=\frac{dy}{du}\frac{du}{dx}\]
\end{minipage}}
\end{center}\vspace{4ex}

3.导数伪装的极限\\
例.\\
$\displaystyle\lim_{h\to 0}\frac{\sqrt[5]{32+h}-2}{h}$\\
证明:\\
设$f(x)=\sqrt[5]{x}$\\
$\displaystyle\because f'(x)=\lim_{h\to 0}\frac{\sqrt[5]{x+h}-x}{h}=\frac{1}{5}x^{-\frac{4}{5}}$\\
$\displaystyle\therefore f'(32)=\lim_{h\to 0}\frac{\sqrt[5]{32+h}-\sqrt[5]{32}}{h}=\lim_{h\to 0}\frac{\sqrt[5]{32+h}-2}{h}=\frac{1}{5}\times 32^{-\frac{4}{5}}=\frac{1}{5}\times\frac{1}{16}=\frac{1}{80}$\\

4.分段函数的导数\\
检验方式: 分段函数再连接点上极限相等, 并且导数再连接点上的极限也相等\\
例.\\
$f(x)=\left\{
\begin{array}{l l}
1 & \text{如果}x\leqslant0,\\
x^2+1 & \text{如果}x>0.    
\end{array}
\right.$\\
$\because f(x)$在连接点$x=0$上的左极限:\\
$\phantom{\because}\displaystyle\lim_{x\to0^-}f(x)=\lim_{x\to0^-}1=1$\\
$\phantom{\because}f(x)$在连接点$x=0$上的右极限:\\
$\phantom{\because}\displaystyle\lim_{x\to0^+}f(x)=\lim_{x\to0^+}(x^2+1)=1$\\
$\therefore\displaystyle\lim_{x\to0}f(x)=1$\\[2ex]
由于, $f'(x)=\left\{
\begin{array}{l l}
	0 & \text{如果}x\leqslant 0,\\
	2x & \text{如果}x>0.
\end{array}
\right.$\\
$\because f'(x)$在连接点$x=0$上的左极限:\\
$\phantom{\because}\displaystyle\lim_{x\to0^-}f'(x)=\lim_{x\to0^-}0=0$\\
$\phantom{\because}f'(x)$在连接点$x=0$上的右极限:\\
$\phantom{\because}\displaystyle\lim_{x\to0^+}f'(x)=\lim_{x\to0^+}2x=0$\\
$\therefore\displaystyle\lim_{x\to0}f'(x)=0$\\

5.直接画出导函数的图像
\end{document}
