\chapter{导数和图像}
1.函数的极值
\begin{theorem}[极值定理]
假设函数$f$定义在开区间$(a,b)$内, 并且点$c$在$(a,b)$区间内. 如果点$c$为函数的局部最大值或最小值, 那么点$c$一定为该函数的临界点. 也就是说, $f'(c)=0$或$f'(c)$不存在.
\end{theorem}

求解闭区间$[a,b]$内的全局最大值和最小值步骤:\\
(1)求出$f'(x)$, 并列出在$(a,b)$中$f'(x)$不存在或$f'(x)=0$的点. 也就是说, 列出在开区间$(a,b)$内所有的临界点.\\
(2)把端点$x=a$和$x=b$放入列表.\\
(3)对于上述列表中的每个点, 将它们带入$y=f(x)$求出对应函数值.\\
(4)找出最大的函数值以及它所对应的$x$值, 得到全局最大值.\\
(5)类似于(4), 得到全局最小值.\\[2ex]

2.罗尔定理
\begin{theorem}[罗尔定理]
假设函数$f$在闭区间$[a,b]$内连续, 在开区间$(a,b)$内可导. 如果$f(a)=f(b)$, 那么在开区间$(a,b)$内至少存在一点$c$, 使得$f'(c)=0$.
\end{theorem}\vspace{4ex}

3.中值定理
\begin{theorem}[中值定理]
假设函数$f$在闭区间$[a,b]$内连续, 在开区间$(a,b)$内可导, 那么在开区间$(a,b)$内至少有一点$c$使得
\[f'(x)=\frac{f(b)-f(a)}{b-a}\]
\end{theorem}
\framebox{
\begin{minipage}{\linewidth}
如果对于在定义域$(a,b)$内的所有$x$, 都有$f'(x)=0$, 那么函数$f$在开区间$(a,b)$内为常数函数.
\end{minipage}}\\[2ex]
\framebox{
\begin{minipage}{\linewidth}
如果对于任意实数$x$都有$f'(x)=g'(x)$, 那么有$f(x)=g(x)+C$($C$为常数).
\end{minipage}}\\[4ex]

4.二阶导数与图像\\[2ex]
\framebox{如果$x=c$点是函数$f$的拐点, 则有$f''(c)=0$.}\\[2ex]
\framebox{如果$f''(c)=0$, 则$c$点不一定都是函数$f$的拐点.}\\[2ex]

5.导数为零的汇总\\
纯1阶导数分析 - 假设$f'(c)=0$, 此时情况如下:\\
(1)如果从左往右通过$c$点, $f'(x)$的符号由正变负, 那么$c$点为局部最大值;\\
(2)如果从左往右通过$c$点, $f'(x)$的符号由负变正, 那么$c$点为局部最小值;\\
(3)如果从左往右通过$c$点, $f'(x)$的符号不发生变化, 那么$c$点为水平拐点.

1/2阶导数综合分析 - 假设$f'(x)=0$, 则有:\\
(1)如果$f''(c)<0$, 那么$x=c$为局部最大值;\\
(1)如果$f''(c)>0$, 那么$x=c$为局部最小值;\\
(1)如果$f''(c)=0$, 那么无法判断, 需借助纯1阶分析\\
