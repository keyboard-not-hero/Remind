\chapter{求解多项式的极限问题}
1.$x\to a$时的有理函数的极限\\
有理函数: 两个多项式之比$\frac{p(x)}{q(x)}$\\
解题方法:\\
(1)当$f(a)=\frac{m}{n}$($m/n$皆不为0)\\
例.\\
\phantom{例}$\displaystyle\lim_{x\to -1}\frac{x^-3x+2}{x-2}$\\
$\displaystyle\lim_{x\to -1}\frac{x^2-3x+2}{x-2}=\frac{(-1)^2-3\times(-1)+2}{-1-2}=\frac{6}{-3}=2$\\\vspace{1ex}

2)当$f(a)=\frac{0}{0}$, 分子分母进行因式分解, 并删除公因子\\
例.\\
\phantom{例}$\displaystyle\lim_{x\to 2}\frac{x^2-3x+2}{x-2}$\\
$\displaystyle\lim_{x\to 2}\frac{x^2-3x+2}{x-2}=\lim_{x\to 2}\frac{(x-1)(x-2)}{x-2}=\lim_{x\to 2}(x-1)=1$\\\vspace{1ex}

3)当$f(a)=\frac{m}{0}$, 验证极限点左右两边的符号是否相同\\
例.\\
\phantom{例}$\displaystyle\lim_{x\to 1}\frac{2x^2-x-6}{x(x-1)^3}$\\
$\displaystyle\lim_{x\to 1^+}\frac{(2x+3)(x-2)}{x(x-1)^3}=\frac{(+)(-)}{(+)(+)}=-$\\
$\displaystyle\lim_{x\to 1^-}\frac{(2x+3)(x-2)}{x(x-1)^3}=\frac{(+)(-)}{(+)(-)}=+$\\
$\displaystyle\because\lim_{x\to 1}\frac{2x^2-x-6}{x(x-1)^3}$左右两侧极限符不相等, 所有极限不存在\\\vspace{4ex}

2.$x\to a$时的平方根的极限\\
解题方法: 将分子分母同时乘以共轭表达式\\
例.\\
\phantom{例}$\displaystyle\lim_{x\to 5}\frac{\sqrt{x^2-9}-4}{x-5}$\\
$\displaystyle\lim_{x\to 5}\frac{\sqrt{x^2-9}-4}{x-5}=\lim_{x\to 5}\frac{\sqrt{x^2-9}-4}{x-5}\times\frac{\sqrt{x^2-9}+4}{\sqrt{x^2-9}+4}=\lim_{x\to 5}\frac{(x+5)(x-5)}{(x-5)(\sqrt{x^2-9}+4)}$\\
$\displaystyle\phantom{\lim_{x\to 5}\frac{\sqrt{x^2-9}-4}{x-5}}=\lim_{x\to 5}\frac{x+5}{\sqrt{x^2-9}+4}=\frac{5+5}{\sqrt{25-9}+4}=\frac{5}{4}$\\\vspace{4ex}

3.$x\to\infty/-\infty$时的有理函数的极限\\
对于任意的$n>0$, 只要$C$是常数, 满足
\[\lim_{x\to\infty}\frac{C}{x^n}=0\]
解题方法: 分组分母分别除以其最高项内容, 再乘以该内容\\
例1.\\
\phantom{例}$\displaystyle\lim_{x\to\infty}\frac{x-8x^4}{7x^4+5x^3+2000x^2-6}$\\
$\displaystyle\lim_{x\to\infty}\frac{x-8x^4}{7x^4+5x^3+2000x^2-6}=\lim_{x\to\infty}\frac{\frac{x-8x^4}{-8x^4}\times(-8x^4)}{\frac{7x^4++5x^3+2000x^2-6}{7x^4}\times7x^4}$\\
$\displaystyle\phantom{\lim_{x\to\infty}\frac{x-8x^4}{7x^4+5x^3+2000x^2-6}}=\lim_{x\to\infty}\frac{1\times(-8x^4)}{1\times7x^4}=-\frac{8}{7}$\\\vspace{2ex}

例2.\\
\phantom{例}$\displaystyle\lim_{x\to\infty}\frac{(x^4+3x-99)(2-x^5)}{(18x^7+9x^6-3x^2-1)(x+1)}$\\
$\displaystyle\lim_{x\to\infty}\frac{(x^4+3x-99)(2-x^5)}{(18x^7+9x^6-3x^2-1)(x+1)}=\lim_{x\to\infty}\frac{(\frac{x^4+3x-99}{x^4}\times x^4)(\frac{2-x^5}{-x^5}\times(-x^5))}{(\frac{18x^7+9x^6-3x^2-1}{18x^7}\times18x^7)(\frac{x+1}{x}\times x)}$\\
$\displaystyle\phantom{\lim_{x\to\infty}\frac{(x^4+3x-99)(2-x^5)}{(18x^7+9x^6-3x^2-1)(x+1)}}=\lim_{x\to\infty}\frac{x^4\times(-x^5)}{18x^7\times x}$\\
$\displaystyle\phantom{\lim_{x\to\infty}\frac{(x^4+3x-99)(2-x^5)}{(18x^7+9x^6-3x^2-1)(x+1)}}=\lim_{x\to\infty}-\frac{x}{18}$\\
$\displaystyle\phantom{\lim_{x\to\infty}\frac{(x^4+3x-99)(2-x^5)}{(18x^7+9x^6-3x^2-1)(x+1)}}=-\infty$\\\vspace{2ex}

例3.\\
\phantom{例}$\displaystyle\lim_{x\to-\infty}\frac{(x^4+3x-99)(2-x^5)}{(18x^7+9x^6-3x^2-1)(x+1)}$\\
$\displaystyle\lim_{x\to-\infty}\frac{(x^4+3x-99)(2-x^5)}{(18x^7+9x^6-3x^2-1)(x+1)}=\lim_{x\to-\infty}\frac{(\frac{x^4+3x-99}{x^4}\times x^4)(\frac{2-x^5}{-x^5}\times(-x^5))}{(\frac{18x^7+9x^6-3x^2-1}{18x^7}\times18x^7)(\frac{x+1}{x}\times x)}$\\
$\displaystyle\phantom{\lim_{x\to-\infty}\frac{(x^4+3x-99)(2-x^5)}{(18x^7+9x^6-3x^2-1)(x+1)}}=\lim_{x\to-\infty}\frac{x^4\times(-x^5)}{18x^7\times x}$\\
$\displaystyle\phantom{\lim_{x\to-\infty}\frac{(x^4+3x-99)(2-x^5)}{(18x^7+9x^6-3x^2-1)(x+1)}}=\lim_{x\to-\infty}-\frac{x}{18}$\\
$\displaystyle\phantom{\lim_{x\to-\infty}\frac{(x^4+3x-99)(2-x^5)}{(18x^7+9x^6-3x^2-1)(x+1)}}=\infty$\\\vspace{2ex}


情况分布:\\
(1)如果$p$的次数等于$q$的次数, 极限是有限的且非零\\
(2)如果$p$的次数大于$q$的次数, 极限是$\infty$或$-\infty$\\
(3)如果$p$的次数小于$q$的次数, 极限是0.\\[2ex]

4.$x\to\infty/-\infty$时的多项式型函数的极限\\
多项式型函数: 类似于多项式, 但包含分数次数或$n$次根\\
解题方法: 分子分母分别除以其最高项内容, 再乘以该内容\\
例1.\\
\phantom{例}$\displaystyle\lim_{x\to\infty}\frac{\sqrt{16x^4+8}+3x}{2x^2+6x+1}$\\
$\displaystyle\lim_{x\to\infty}\frac{\sqrt{16x^4+8}+3x}{2x^2+6x+1}=\lim_{x\to\infty}\frac{\frac{\sqrt{16x^4+8}+3x}{4x^2}\times4x^2}{\frac{2x^2+6x+1}{2x^2}\times2x^2}$\\
$\displaystyle\phantom{\lim_{x\to\infty}\frac{\sqrt{16x^4+8}+3x}{2x^2+6x+1}}=\lim_{x\to\infty}\frac{1\times4x^2}{1\times2x^2}=2$\\

例2.\\
\phantom{例}$\displaystyle\lim_{x\to\infty}\frac{\sqrt{4x^6-5x^5}-2x^3}{\sqrt[3]{27x^6+8x}}$\\
$\displaystyle\lim_{x\to\infty}\frac{\sqrt{4x^6-5x^5}-2x^3}{\sqrt[3]{27x^6+8x}}=\lim_{x\to\infty}\frac{\sqrt{4x^6-5x^5}-2x^3}{\sqrt[3]{27x^6+8x}}\times\frac{\sqrt{4x^6-5x^5}+2x^3}{\sqrt{4x^6-5x^5}+2x^3}$\\
$\displaystyle\phantom{\lim_{x\to\infty}\frac{\sqrt{4x^6-5x^5}-2x^3}{\sqrt[3]{27x^6+8x}}}=\lim_{x\to\infty}\frac{(4x^6-5x^5)-4x^6}{\sqrt[3]{27x^6+8x}(\sqrt{4x^6-5x^5}+2x^3)}$\\
$\displaystyle\phantom{\lim_{x\to\infty}\frac{\sqrt{4x^6-5x^5}-2x^3}{\sqrt[3]{27x^6+8x}}}=\lim_{x\to\infty}\frac{-5x^5}{(\frac{\sqrt[3]{27x^6+8x}}{3x^2}\times3x^2)(\frac{\sqrt{4x^6-5x^5}+2x^3}{4x^3}\times4x^3)}$\\
$\displaystyle\phantom{\lim_{x\to\infty}\frac{\sqrt{4x^6-5x^5}-2x^3}{\sqrt[3]{27x^6+8x}}}=\lim_{x\to\infty}-\frac{5x^5}{3x^2\times4x^3}$\\
$\displaystyle\phantom{\lim_{x\to\infty}\frac{\sqrt{4x^6-5x^5}-2x^3}{\sqrt[3]{27x^6+8x}}}=-\frac{5}{12}$\\

例3.\\
\phantom{例}$\displaystyle\lim_{x\to-\infty}\frac{\sqrt{4x^6+8}}{2x^3+6x+1}$\\
由于$\sqrt{4x^6}$为正数, 而$2x^3$为负数\\
所以$\sqrt{4x^6}=-2x^3$\\
$\displaystyle\lim_{x\to\-infty}\frac{\sqrt{4x^6+8}}{2x^3+6x+1}=\lim_{x\to-\infty}\frac{\frac{\sqrt{4x^6+8}}{-2x^3}\times(-2x^3)}{\frac{2x^3+6x+1}{2x^3}\times2x^3}=-1$\\

5.包含绝对值的函数的极限\\
例.\\
\phantom{例}$\displaystyle\lim_{x\to-2}\frac{|x+2|}{x+2}$\\
$\displaystyle\because p(x)>0, |p(x)|=p(x)$\\ 
$\displaystyle\phantom{\because}p(x)<0, |p(x)|=-p(x)$\\
$\displaystyle\therefore\lim_{x\to -2^-}\frac{|x+2|}{x+2}=\frac{-(x+2)}{x+2}=-1$\\
$\displaystyle\phantom{\therefore}\lim_{x\to -2^+}\frac{|x+2|}{x+2}=\frac{x+2}{x+2}=1$\\
$\displaystyle\therefore\lim_{x\to -2}\frac{|x+2|}{x+2}\text{无双侧极限}$\\


%最后编辑于: 2021-12-27
