\documentclass[UTF8, fontset=ubuntu]{ctexart}
\usepackage{parskip}
\usepackage{amsmath}
\usepackage{amssymb}
\begin{document}
1.$x\to a$时的有理函数的极限\\
有理函数: 形如$f(x)=\frac{p(x)}{q(x)}$的函数, 其中$p(x),q(x)$都是多项式.\\
1)当$f(a)=\frac{p(a)}{q(a)}=\frac{m}{n}$时:\\
$\displaystyle\lim_{x\to a}f(x)=\frac{m}{n}$\\
例.
\[\lim_{x\to -1}\frac{x^2-9}{x-2}=\lim_{x\to -1}\frac{x-9}{x-2}=\frac{1-9}{1-2}=8\]

2)当$f(a)=\frac{p(a)}{q(a)}=\frac{0}{0}$时:\\
$\displaystyle\lim_{x\to a}f(x)$进行分子分母约分\\
例.
\[\lim_{x\to 2}\frac{x^2-3x+2}{x-2}=\lim_{x\to 2}\frac{(x-1)(x-2)}{x-2}=\lim{x\to 2}x-1=2-1=1\]

3)当$f(a)=\frac{p(a)}{q(a)}=\frac{m}{0}$时:\\
$\displaystyle\lim_{x\to a}f(x)$判断极限点两边的极限是否同为$\infty$或$-\infty$\\
例.
\begin{displaymath}
\begin{array}{l}
    \displaystyle\lim_{x\to 1}\frac{2x^2-x-6}{x(x-1)^3}=\lim_{x\to 1}\frac{(2x+3)(x-2)}{x(x-1)^3}\\
	\displaystyle\mathbf{\because}\lim_{x\to 1^-}\frac{(2x+3)(x-2)}{x(x-1)^3}=\frac{(+)(-)}{(+)(-)}=+\\
    \displaystyle\phantom{\because}\lim_{x\to 1^+}\frac{(2x+3)(x-2)}{x(x-1)^3}=\frac{(+)(-)}{(+)(+)}=-\\
	\displaystyle\mathbf{\therefore}\,f(x)\text{无极限值}
\end{array}
\end{displaymath}\vspace{2ex}

2.$x\to a$时的平方根的极限\\
共轭因式: 若S是含有根式的已知表达式, 若存在一个不恒等于零的表达式M, 使乘积S$\times$M不含根式, 则M为S的共轭因式. 反之, S也为M的共轭因式.\\
设$f(x)=\frac{g(x)\pm h(x)}{p(x)\pm q(x)}$, 其中, g(x)/h(x)/p(x)/q(x)其中一个为根式\\
当$f(a)=\frac{g(a)-h(a)}{p(a)-q(a)}=\frac{0}{0}$时, 将分子分母同时乘以含根号部分的共轭因式.\\
例.
\begin{displaymath}
\begin{array}{l}
    \displaystyle\lim_{x\to 5}\frac{\sqrt{x^2-9}-4}{x-5}=\lim_{x\to 5}\frac{\sqrt{x^2-9}-4}{x-5}\times\frac{\sqrt{x^2-9}+4}{\sqrt{x^2-9}+4}=\lim_{x\to 5}\frac{x^2-25}{(x-5)(\sqrt{x^2-9}+4)}\\
    \displaystyle\phantom{\lim_{x\to 5}\frac{\sqrt{x^2-9}-4}{x-5}}=\lim_{x\to 5}\frac{x+5}{\sqrt{x^-9}+4}=\frac{5+5}{\sqrt{25-9}+4}=\frac{5}{4}
\end{array}
\end{displaymath}\vspace{2ex}

3.$x\to\infty/-\infty$时的有理函数的极限\\
\begin{math}
\begin{array}{l}
	\because\displaystyle\lim_{x\to\infty}\frac{C}{x^n}=0\\
	\therefore\displaystyle\lim_{x\to\infty}f(x)=\lim_{x\to\infty}\frac{p(x)}{q(x)}=\lim_{x\to\infty}\frac{a_{0}x^m+a_{1}x^{m-1}+\cdots+a_{m}}{b_{0}x^n+b_{1}x^{n-1}+\cdots+b_{n}}\\
	\displaystyle\phantom{\therefore\lim_{x\to\infty}f(x)}=\lim_{x\to\infty}\frac{\frac{a_{0}x^m+a_{1}x^{m-1}+\cdots+a_{m}}{a_{0}x^m}\times a_{0}x^m}{\frac{b_{0}x^n+b_{1}x^{n-1}+\cdots+b_{n}}{b_{0}x^n}\times b_{0}x^n}\\
	\displaystyle\phantom{\therefore\lim_{x\to\infty}f(x)}=\lim_{x\to\infty}\frac{a_{0}x^m}{b_{0}x^n}\\
	\displaystyle\phantom{\therefore\lim_{x\to\infty}f(x)}=\frac{a_0}{b_0}\lim_{x\to\infty}\frac{x^m}{x^n}
\end{array}\\[1ex]
\end{math}
情况分布:\\
(1)m=n, 极限为有限的且非零;\\
(2)m>n, 极限为$\infty$或$-\infty$;\\
(3)m<n, 极限为0.\\[2ex]

4.$x\to\infty$时的多项式型函数的极限\\
\begin{math}
\begin{array}{l}
	\displaystyle\lim_{x\to\infty}f(x)=\lim_{x\to\infty}\frac{p(x)}{q(x)}=\lim_{x\to\infty}\frac{\sqrt{a_{0}x^m+a_{1}x^{m-1}+\cdots+a_{m}}}{b_{0}x^n+b_{1}x^{n-1}+\cdots+b_{n}}\\
	\displaystyle\phantom{\lim_{x\to\infty}f(x)}=\lim_{x\to\infty}\frac{\frac{\sqrt{a_{0}x^m+a_{1}x^{m-1}+\cdots+a_{m}}}{\sqrt{a_0}x^{\frac{m}{2}}}\times\sqrt{a_0}x^{\frac{m}{2}}}{\frac{b_{0}x^n+b_{1}x^{n-1}+\cdots+b_{n}}{b_{0}x^n}\times b_{0}x^n}\\
	\displaystyle\phantom{\lim_{x\to\infty}f(x)}=\lim_{x\to\infty}\frac{\sqrt{a_0}x^{\frac{m}{2}}}{b_{0}x^n}\\
	\displaystyle\phantom{\lim_{x\to\infty}f(x)}=\frac{\sqrt{a_0}}{b_0}\lim_{x\to\infty}\frac{x^{\frac{m}{2}}}{x^n}
\end{array}
\end{math}\\[2ex]

5.$x\to-\infty$时的多项式型函数的极限\\
与类型4类似. 但有一种特殊情况:\\[1ex]
\framebox{
\begin{minipage}[c]{10cm}
    如果$x<0$, 并且$\sqrt[n]{x^p}=x^m$, 那么需要在$x^m$之前加一个负号的唯一情形是: n是偶的而m是奇的.
\end{minipage}
}\\[2ex]

6.包含绝对值的函数的极限\\
\begin{math}
\begin{array}{l}
\because p(x)>0, |p(x)|=p(x)\\ 
\phantom{\because}p(x)<0, |p(x)|=-p(x)\\
\displaystyle\therefore\lim_{x\to -2^-}\frac{|x+2|}{x+2}=\frac{-(x+2)}{x+2}=-1\\
\displaystyle\phantom{\therefore}\lim_{x\to -2^+}\frac{|x+2|}{x+2}=\frac{x+2}{x+2}=1\\
\displaystyle\therefore\lim_{x\to -2}\frac{|x+2|}{x+2}\text{无双侧极限}
\end{array}
\end{math}
\end{document}
