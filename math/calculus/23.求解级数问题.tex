\chapter{求解级数问题}
1.级数的讨论\\
1)是否为几何级数\\
\phantom{\quad}当级数为几何级数时, 应用几何级数规则\\
2)级数中的项是否趋于0\\
\phantom{\quad}当级数中的项不趋于0, 包含如下情况:\\
\phantom{\qquad}\ding{172}数列的分子分母都为多项式, 且分子的次数大于等于分母\\
\phantom{\qquad}\ding{173}对数的内容包含多项式, 并且多项式的值不趋于1\\
\phantom{\quad}应用第n项判别法\\
3)级数中是否有负项\\
\phantom{\quad}当级数中包含负的项时, 应用第n项判别法/绝对收敛判别法/交错级数判别法\\
4)级数中是否有阶乘\\
\phantom{\quad}级数中的项包含阶乘, 应用比式判别法\\
5)级数中是否底和指数都包含n\\
\phantom{\quad}当级数的项, 底和指数都包含n时, 应用根式判别法\\
6)级数中是否含1/n和对数\\
\phantom{\quad}当级数中包含1/n和对数时, 应用积分判别法\\
7)上述皆不适用\\
\phantom{\quad}当以上方法都不适用时, 使用反常积分的判别法, 比较判别法/P判别法/极限比较判别法\\[2ex]

2.具体解决方案\\
1)几何级数
{\par\centering
\begin{boxedminipage}{8cm}
若$-1<r<1$,无穷几何级数的和=$\displaystyle\frac{\text{首项}}{1-r}$\\
如果$r\geqslant 1$或$r\leqslant 1$, 级数发散
\end{boxedminipage}
\par}

例1.\\
\phantom{例}$\displaystyle\sum_{n=5}^{\infty}\frac{4}{3^n}$\\
推导过程:\\
$\displaystyle\because\frac{4}{3^n}=4(\frac{1}{3})^n$\\
$\displaystyle\phantom{\because}-1<r=\frac{1}{3}<1$\\
$\displaystyle\sum_{n=5}^{\infty}\frac{4}{3^n}=\frac{4\times(\frac{1}{3})^5}{1-\frac{1}{3}}=\frac{2}{81}$\\[1ex]

例2.\\
\phantom{例}$\displaystyle\sum_{n=2}^\infty\frac{2^{2n}-(-7)^n}{11^n}$\\
推导过程:\\
$\displaystyle\sum_{n=2}^\infty\frac{2^{2n}-(-7)^n}{11^n}=\sum_{n=2}^\infty\frac{4^n}{11^n}-\sum_{n=2}^\infty\frac{(-7)^n}{11^n}$\\
$\displaystyle\because -1<\r_1=\frac{4}{11}<1$\\
$\displaystyle\sum_{n=2}^\infty\frac{4^n}{11^n}=\frac{4^2}{11^2}\times\frac{1}{1-\frac{4}{11}}=\frac{16}{77}$\\
$\displaystyle\because -1<\r_2=\frac{-7}{11}<1$\\
$\displaystyle\sum_{n=2}^\infty\frac{(-7)^n}{11^n}=\frac{(-7)^2}{11^2}\times\frac{1}{1-(-\frac{7}{11})}=\frac{49}{198}$\\
$\displaystyle\therefore\sum_{n=2}^\infty\frac{2^{2n}-(-7)^n}{11^n}=\frac{16}{77}-\frac{49}{198}=-\frac{5}{126}$\\
$\displaystyle\sum_{n=2}^\infty\frac{2^{2n}-(-7)^n}{11^n}$收敛\\[2ex]

2)第$n$项判别法
\begin{center}
\framebox{若$\displaystyle\lim_{n\to\infty}a_n\neq 0$或极限不存在,则级数$\displaystyle\sum_{n=1}^{\infty}a_n$发散}
\end{center}
注解: 该判别法不能用于级数收敛性的判定\\
例1.\\
\phantom{例}$\displaystyle\sum_{n=1}^{\infty}\frac{n^2-3n+7}{4n^2+2n+1}$\\[1ex]
推导过程:\\
$\displaystyle\because\lim_{n\to\infty}\frac{n^2-3n+7}{4n^2+2n+1}=\frac{1}{4}\neq 0$\\
$\therefore$由第n项判别法\\
$\displaystyle\phantom{\therefore}\sum_{n=1}^\infty\frac{n^2-3n+7}{4n^2+2n+1}$发散\\[1ex]

例2.\\
\phantom{例}$\displaystyle\sum_{n=1}^\infty\ln(\frac{n^2+2}{3n^2+5})$\\
推导过程:\\
$\displaystyle\lim_{n\to\infty}\ln(\frac{n^2+2}{3n^2+5})=-\ln3\neq0$\\
由第n项判别式, 得:\\
$\displaystyle\sum_{n=1}^\infty\ln(\frac{n^2+2}{3n^2+5})$发散\\[2ex]

3)含负项的级数\\
\ding{172}若所有项都为负,则在所有项前面添加负号来修改级数\\
例.\\
\phantom{例}$\displaystyle\sum_{n=3}^{\infty}\ln(\frac{1}{n})\frac{1}{\sqrt{n}}$\\[1ex]
推导过程:\\
$\because$在$n\geqslant 3$时,$\ln(\frac{1}{n})<0$,$\ln(\frac{1}{n})\frac{1}{\sqrt{n}}<0$\\
$\therefore\displaystyle\sum_{n=3}^{\infty}-\ln(\frac{1}{n})\frac{1}{\sqrt{n}}=\sum_{n=3}^{\infty}\ln(n)\frac{1}{\sqrt{n}}$\\
$\because$当$n\in[3,\infty)$时,$\ln(n)\geqslant\ln(3)$\\
$\therefore\displaystyle\sum_{n=3}^{\infty}\ln(n)\frac{1}{\sqrt{n}}\geqslant\sum_{n=3}^{\infty}\ln(3)\frac{1}{\sqrt{n}}=\ln(3)\sum_{n=3}^{\infty}\frac{1}{\sqrt{n}}$\\
由p判别法, 得:\\
$\displaystyle\sum_{n=3}^\infty\frac{1}{\sqrt{n}}$发散\\
由比较判别法, 得:\\
$\displaystyle\sum_{n=3}^\infty\ln(n)\frac{1}{\sqrt{n}}$发散\\
$\therefore\displaystyle\sum_{n=3}^{\infty}\ln(\frac{1}{n})\frac{1}{\sqrt{n}}$发散\\[2ex]

\ding{173}若有些项为正,有些项为负,当$n\to\infty$时通项不趋于$0$,用第n项判别法\\
例.\\
\phantom{例}$\displaystyle\sum_{n=1}^{\infty}(-1)^nn^2$\\[1ex]
推导过程:\\
$\displaystyle\because\lim_{n\to\infty}(-1)^nn^2$图像进行增益震荡\\
$\displaystyle\therefore\lim_{n\to\infty}(-1)^nn^2$极限不存在\\
由第n项判别法, 得:\\
$\displaystyle\sum_{n=1}^{\infty}(-1)^nn^2$发散\\[2ex]

\ding{174}若有些项为正,有些项为负,当$n\to\infty$时通项趋于0,用绝对收敛判别法
\begin{center}
\framebox{若$\displaystyle\sum_{n=1}^{\infty}|a_n|$收敛,则$\displaystyle\sum_{n=1}^{\infty}a_n$也收敛}
\end{center}

例.\\
$\displaystyle\sum_{n=1}^{\infty}\frac{\sin(n)}{n^2}$\\[1ex]
推导过程:\\
$\displaystyle\because\sum_{n=1}^{\infty}\frac{|\sin(n)|}{n^2}\leqslant\sum_{n=1}^{\infty}\frac{1}{n^2}$\\
由p判别法, 得:\\
$\displaystyle\sum_{n=1}^\infty\frac{1}{n^2}$收敛\\
由比较判别法, 得:\\
$\displaystyle\sum_{n=1}^{\infty}\frac{|\sin(n)|}{n^2}$收敛\\
由绝对收敛判别法, 得:\\
$\displaystyle\sum_{n=1}^{\infty}\frac{\sin(n)}{n^2}$收敛\\[2ex]

\ding{175}若有些项为正,有些项为负,并且级数不是绝对收敛,用交错级数判别法
{\par\centering
\framebox{\begin{minipage}{\textwidth}
若级数$\displaystyle\sum_{n=1}^{\infty}a_n$是交错的,且各项的绝对值递减趋于$0$,则级数收敛\\
收敛条件列表:\\
1.$a_n$正负交错. 如:$(-1)^n$\\
2.$|a_n|$递减\\
3.$\displaystyle\lim_{n\to\infty}|a_n|=0$
\end{minipage}}
\par}

例1.\\
\phantom{例}$\displaystyle\sum_{n=1}^{\infty}\frac{(-1)^n}{n}$\\[1ex]
推导过程:\\
在区间$[1,\infty$上:\\
1)\,$\displaystyle\sum_{n=1}^{\infty}\frac{(-1)^n}{n}$正负交错\\
2)\,$\displaystyle\because(\frac{1}{n})'=-\frac{1}{n^2}$\\
\phantom{2)\,}$\displaystyle\therefore\Big|\frac{(-1)^n}{n}\Big|=\frac{1}{n}$单调递减\\
3)\,$\displaystyle\lim_{n\to\infty}\Big|\frac{(-1)^n}{n}\Big|=\lim_{n\to\infty}\frac{1}{n}=0$\\
$\therefore$由交错级数判别法, 得:\\
$\phantom{\therefore}\displaystyle\sum_{n=1}^{\infty}\frac{(-1)^n}{n}$收敛\\[1ex]

例2.\\
\phantom{例}$\displaystyle\sum_{n=2}^\infty\frac{(-1)^n}{\ln(n)}$\\
推导过程:\\
在区间$[2,\infty)$上:\\
1)\,$\displaystyle\sum_{n=2}^\infty\frac{(-1)^n}{\ln(n)}$正负交错\\
2)\,$\displaystyle\because(\frac{1}{\ln(n)})'=-\frac{\frac{1}{n}}{\ln^2n}=-\frac{1}{n\ln^2}$\\
\phantom{2)\,}$\displaystyle\therefore\frac{1}{\ln(n)}单调递减$\\
3)$\displaystyle\lim_{n\to\infty}\frac{1}{\ln(n)}=0$\\
$\therefore$由交错级数判别法, 得:\\
$\phantom{\therefore}\displaystyle\sum_{n=2}^\infty\frac{(-1)^n}{\ln(n)}$收敛\\[2ex]

4)比式判别法
{\par\centering
\framebox{\begin{minipage}{\textwidth}
若$\displaystyle L=\lim_{n\to\infty}\Big|\frac{a_{n+1}}{a_n}\Big|$,则$\displaystyle n=\sum_{n=1}^{\infty}a_n$在$L<1$时绝对收敛,在$L>1$时发散;但当$L=1$或极限不存在时,比式判别法无效
\end{minipage}}
\par}

例1.\\
\phantom{例}$\displaystyle\sum_{n=1}^{\infty}\frac{n^{1000}}{2^n}$\\[1ex]
推导过程:\\
$\displaystyle L=\lim_{n\to\infty}\Big|\frac{a_{n+1}}{a_n}\Big|=\lim_{n\to\infty}\left|\frac{\frac{(n+1)^{1000}}{2^{n+1}}}{\frac{n^{1000}}{2^n}}\right|=\frac{1}{2}\lim_{n\to\infty}(\frac{n+1}{n})^{1000}=\frac{1}{2}$\\
由比式判别法, 得:\\
$\displaystyle\sum_{n=1}^{\infty}\frac{n^{1000}}{2^n}$收敛\\[1ex]

例2.\\
\phantom{例}$\displaystyle\sum_{n=2}^\infty\frac{3^n}{n\ln(n)}$\\
推导过程:\\
$\displaystyle L=\lim_{n\to\infty}\left|\frac{\frac{3^{n+1}}{(n+1)\ln(n+1)}}{\frac{3^n}{n\ln(n)}}\right|=3\lim_{n\to\infty}\frac{n}{n+1}\frac{\ln(n)}{\ln(n+1)}$\\
根据洛必达法则, 得:\\
$\displaystyle\lim_{n\to\infty}\frac{\ln(n)}{\ln(n+1)}=\lim_{n\to\infty}\frac{\frac{1}{n}}{\frac{1}{n+1}}=\lim_{n\to\infty}\frac{n+1}{n}=1$\\
$L=3>1$\\
$\displaystyle\sum_{n=2}^\infty\frac{3^n}{n\ln(n)}$发散\\[1ex]

例3.\\
\phantom{例}$\displaystyle\sum_{n=1}^\infty\frac{(2n)!}{(n!)^2}$\\
推导过程:\\
$\displaystyle L=\lim_{n\to\infty}\left|\frac{\frac{(2n+2)!}{((n+1)!)^2}}{\frac{(2n)!}{(n!)^2}}\right|=\lim_{n\to\infty}\frac{(2n+1)(2n+2)}{(n+1)^2}=\lim_{n\to\infty}\frac{4n+2}{n+1}=4>1$\\
由第n项判别法, 得:\\
$\displaystyle\sum_{n=1}^\infty\frac{(2n)!}{(n!)^2}$发散\\[2ex]

5)根式判别法
{\par\centering
\framebox{\begin{minipage}{\textwidth}
若$\displaystyle L=\lim_{n\to\infty}|a_n|^{\frac{1}{n}}$,则$\displaystyle\sum_{n=1}^{\infty}a_n$在$L<1$时绝对收敛,在$L>1$时发散;但当$L=1$或极限不存在时,根式判别法无效
\end{minipage}}
\par}

例.\\
\phantom{例}$\displaystyle\sum_{n=1}^{\infty}(1-\frac{2}{n})^{n^2}$\\[1ex]
推导过程:\\
$\displaystyle L=\lim_{n\to\infty}|a_n|^{\frac{1}{n}}=\lim_{n\to\infty}\Big|(1-\frac{2}{n})^{n^2}\Big|^{\frac{1}{n}}=\lim_{n\to\infty}(1-\frac{2}{n})^n=e^{-2}<1$\\
由根式判别法, 得:\\
$\displaystyle\sum_{n=1}^{\infty}(1-\frac{2}{n})^{n^2}$绝对收敛\\[2ex]

6)积分判别法
{\par\centering
\framebox{若对连续递减函数$f$有$a_n=f(n)$,则$\displaystyle\sum_{n=N}^{\infty}a_n$与$\displaystyle\int_N^{\infty}f(x)\dif x$同时收敛或发散}
\par}

例1.\\
\phantom{例}$\displaystyle\sum_{n=2}^{\infty}\frac{1}{n\ln(n)}$\\[1ex]
推导过程:\\
令$\displaystyle f(x)=\frac{1}{x\ln x}$\\
设$t=\ln x$,则$\displaystyle\dif t=\frac{1}{x}\dif x$\\
$\displaystyle\int_2^{\infty}\frac{1}{x\ln x}\dif x=\int_{\ln 2}^{\infty}\frac{1}{t}\dif t$\\
由p判别法, 得:\\
$\displaystyle\int_{\ln 2}^{\infty}\frac{1}{t}\dif t$发散\\
$\therefore\displaystyle\sum_{n=2}^{\infty}\frac{1}{n\ln(n)}$发散\\[1ex]

例2.\\
\phantom{例}$\displaystyle\sum_{n=2}^\infty\frac{1}{n(\ln(n))^2}$\\
推导过程:\\
令$\displaystyle f(x)=\frac{1}{x(\ln x)^2}$\\
设$t=\ln x$, 则$\displaystyle\dif t=\frac{1}{x}\dif x$\\
$\displaystyle\int_2^{\infty}\frac{1}{x(\ln x)^2}\dif x=\int_{\ln 2}^{\infty}\frac{1}{t^2}\dif t$\\
由p判别法, 得:\\
$\displaystyle\int_{\ln 2}^{\infty}\frac{1}{t^2}\dif t$收敛\\
$\therefore\displaystyle\sum_{n=2}^{\infty}\frac{1}{n(\ln(n))^2}$收敛\\[1ex]


7)反常积分的判别法(比较判别法/极限比较判别法/p判别法)\\
\ding{172}比较判别法
{\par\centering
\framebox{若对所有$n$,有$0\leqslant b_n\leqslant a_n$,且$\displaystyle\sum_{n==1}^{\infty}b_n$发散,则$\displaystyle\sum_{n=1}^{\infty}a_n$也发散}\\[1ex]
\framebox{若对所有$n$,有$b_n\geqslant a_n\geqslant 0$,且$\displaystyle\sum_{n=1}^{\infty}b_n$收敛,则$\displaystyle\sum_{n=1}^{\infty}a_n$也收敛}
\par}\vspace{4ex}

\ding{173}极限比较判别法
{\par\centering
\framebox{若当$n\to\infty$时$a_n\backsim b_n$,且$a_n$和$b_n$均有限,则$\displaystyle\sum_{n=1}^{\infty}a_n$与$\displaystyle\sum_{n=1}^{\infty}b_n$同时收敛或发散}
\par}\vspace{4ex}

\ding{174}p判别法
{\par\centering
\framebox{$\displaystyle\sum_{n=a}^{\infty}\frac{1}{n^p}\left\{\begin{array}{l l}
\text{收敛} & \text{如果}p>1\\
\text{发散} & \text{如果}p\leqslant 1
\end{array}
\right.$}
\par}\vspace{4ex}

例1.\\
\phantom{例}$\displaystyle\sum_{n=1}^{\infty}\frac{2n^2+3n+7}{n^4+2n^3+1}$\\
推导过程:\\
当$n\to\infty$时,$\displaystyle\frac{2n^2+3n+7}{n^4+2n^3+1}\backsim\frac{2n^2}{n^4}=\frac{2}{n^2}$\\
由p判别法, 得:\\
$\displaystyle\sum_{n=1}^{\infty}\frac{2}{n^2}$收敛\\
由极限比较判别法, 得:\\
$\displaystyle\sum_{n=1}^{\infty}\frac{2n^2+3n+7}{n^4+2n^3+1}$收敛\\[1ex]

例2.\\
\phantom{例}$\displaystyle\sum_{n=1}^{\infty}2^{-n}n^{1000}$\\
推导过程:\\
对于所有的$n>0$,$\displaystyle 2^{-n}\leqslant\frac{C}{n^1002}$\\
$\displaystyle\sum_{n=1}^{\infty}2^{-n}n^{1000}\leqslant\sum_{n=1}^{\infty}\frac{C}{n^{1002}}\times n^{1000}=C\sum_{n=1}^{\infty}\frac{1}{n^2}$\\
由p判别法, 得:\\
$\displaystyle C\sum_{n=1}^{\infty}\frac{1}{n^2}$收敛\\
由比较判别法, 得:\\
$\displaystyle\sum_{n=1}^{\infty}2^{-n}n^{1000}$收敛\\[1ex]

例3.\\
\phantom{例}$\displaystyle\sum_{n=2}^\infty\frac{\ln(n)}{n^{1.001}}$\\
推导过程:\\
对于所有的$n>1$, $\ln(n)\leqslant Cn^{0.0005}$\\
$\displaystyle\sum_{n=2}^\infty\frac{\ln(n)}{n^{1.001}}\leqslant C\sum_{n=2}^\infty\frac{1}{n^{1.0005}}$\\
由p判别法, 得:\\
$\displaystyle C\sum_{n=2}^\infty\frac{1}{n^{1.0005}}$收敛\\
由比较判别法, 得:\\
$\displaystyle\sum_{n=2}^\infty\frac{\ln(n)}{n^{1.001}}$收敛\\[2ex]

8)可折叠级数
\begin{center}
\begin{boxedminipage}{\textwidth}
若数列$\{a_n\}$满足以下关系:\\
		\ding{172}$a_n=A_n-B_n$\\
		\ding{173}$B_n=A_{n+1}$\\
则级数
		\[\sum_{n=1}^\infty a_n=\lim_{n\to\infty}S_n=A_1-B_n\]
\end{boxedminipage}
\end{center}

例1.\\
\phantom{例}$\displaystyle\sum_{n=1}^\infty\frac{1}{n(n+1)}$\\
推导过程:\\
$\displaystyle\sum_{n=1}^\infty\frac{1}{n(n+1)}=\sum_{n=1}^\infty(\frac{1}{n}-\frac{1}{n+1})$\\
$\displaystyle S_n=(1-\frac{1}{2})+(\frac{1}{2}-\frac{1}{3})+\cdots+(\frac{1}{n}-\frac{1}{n+1})$\\
$\displaystyle\phantom{S_n}=1-\frac{1}{n+1}$\\
$\displaystyle\phantom{S_n}=\frac{n}{n+1}$\\
$\displaystyle\lim_{n\to\infty}S_n=\lim_{n\to\infty}\frac{n}{n+1}=1$\\
$\displaystyle\sum_{n=1}^\infty\frac{1}{n(n+1)}$收敛\\[1ex]

例2.\\
\phantom{例}$\displaystyle\sum_{n=1}^\infty\ln(\frac{n+1}{n+2})$\\
推导过程:\\
$\displaystyle\sum_{n=1}^\infty\ln(\frac{n+1}{n+2})=\sum_{n=1}^\infty[\ln(n+1)-\ln(n+2)]$\\
$\displaystyle S_n=(\ln 2-\ln 3)+(\ln 3-\ln 4)+\cdots+[\ln(n+1)-\ln(n+2)]$\\
$\displaystyle\phantom{S_n}=\ln 2-\ln(n+2)$\\
$\displaystyle\phantom{S_n}=\ln(\frac{2}{n+2})$\\
$\displaystyle\lim_{n\to\infty}S_n=\lim_{n\to\infty}\ln(\frac{2}{n+2})=-\infty$\\
$\displaystyle\sum_{n=1}^\infty\ln(\frac{n+1}{n+2})$发散

%最后编辑于: 2022-01-15
