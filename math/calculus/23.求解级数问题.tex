\chapter{求解级数问题}
1.级数的讨论\\
(1)是否为几何级数\\
(2)级数中的项是否趋于0 --- 第n项判别法\\
(3)级数中是否有阶乘 --- 比式判别法\\
(4)级数中的指数是否包含n --- 跟式判别法\\
(5)级数中是否含$\frac{1}{n}$或对数 --- 积分判别法\\
(6)级数中是否有负项 --- 第n项判别法/绝对收敛判别法/交错级数判别法\\
(7)上述皆不适用 --- 比较判别法/P判别法/极限比较判别法\\[2ex]

2.具体解决方案\\
(1)几何级数\\
\begin{center}
\framebox{若$-1<r<1$,无穷几何级数的和=$\displaystyle\frac{\text{首项}}{1-r}$}
\end{center}
例.\\
$\displaystyle\sum_{n=5}^{\infty}\frac{4}{3^n}$\\
推导过程:\\
$\displaystyle\because\frac{4}{3^n}=4(\frac{1}{3})^n$\\
$\displaystyle\phantom{\because}-1<r=\frac{1}{3}<1$\\
$\displaystyle\sum_{n=5}^{\infty}\frac{4}{3^n}=\sum_{n=5}^{\infty}4(\frac{1}{3})^n=\frac{4\times(\frac{1}{3})^5}{1-\frac{1}{3}}=\frac{2}{81}$\\[2ex]

(2)级数中的项是否趋于$0$ --- 第$n$项判别法\\
\begin{center}
\framebox{若$\displaystyle\lim_{n\to\infty}a_n\neq 0$或极限不存在,则级数$\displaystyle\sum_{n=1}^{\infty}a_n$发散}
\end{center}
该判别法不能用于级数收敛性的判定\\
例.\\
$\displaystyle\sum_{n=1}^{\infty}\frac{n^2-3n+7}{4n^2+2n+1}$\\[1ex]
推导过程:\\
$\displaystyle\because\lim_{n\to\infty}\frac{n^2-3n+7}{4n^2+2n+1}=\frac{1}{4}\neq 0$\\
$\therefore$由第n项判别法\\
$\displaystyle\phantom{\therefore}\frac{n^2-3n+7}{4n^2+2n+1}$发散\\[2ex]

(3)级数中是否有阶乘 --- 比式判别法\\
\begin{center}
\framebox{\begin{minipage}{\textwidth}
若$\displaystyle L=\lim_{n\to\infty}\Big|\frac{a_{n+1}}{a_n}\Big|$,则$\displaystyle n=\sum_{n=1}^{\infty}a_n$在$L<1$时绝对收敛,在$L>1$时发散;但当$L=1$或极限不存在时,比式判别法无效
\end{minipage}}
\end{center}
例.\\
$\displaystyle\sum_{n=1}^{\infty}\frac{n^{1000}}{2^n}$\\[1ex]
推导过程:\\
$\because\displaystyle\lim_{n\to\infty}\Big|\frac{a_{n+1}}{a_n}\Big|=\lim_{n\to\infty}\Big|\frac{\frac{(n+1)^{1000}}{2^{n+1}}}{\frac{n^{1000}}{2^n}}\Big|=\lim_{n\to\infty}\Big|\frac{(n+1)^{1000}}{n^{1000}}\times\frac{1}{2}\Big|=\frac{1}{2}\lim_{n\to\infty}(\frac{n+1}{n})^{1000}=\frac{1}{2}$\\
$\therefore$由比式判别法\\
$\displaystyle\phantom{\therefore}\sum_{n=1}^{\infty}\frac{n^{1000}}{2^n}$收敛\\[2ex]

(4)级数中的指数是否包含$n$ --- 根式判别法\\
\begin{center}
\framebox{\begin{minipage}{\textwidth}
若$\displaystyle L=\lim_{n\to\infty}|a_n|^{\frac{1}{n}}$,则$\displaystyle\sum_{n=1}^{\infty}a_n$在$L<1$时绝对收敛,在$L>1$时发散;但当$L=1$或极限不存在时,根式判别法无效
\end{minipage}}
\end{center}
例.\\
$\displaystyle\sum_{n=1}^{\infty}(1-\frac{2}{n})^{n^2}$\\[1ex]
推导过程:\\
$\because\displaystyle\lim_{n\to\infty}|a_n|^{\frac{1}{n}}=\lim_{n\to\infty}\Big|(1-\frac{2}{n})^{n^2}\Big|^{\frac{1}{n}}=\lim_{n\to\infty}(1-\frac{2}{n})^n=e^{-2}<1$\\
$\therefore$由根式判别法\\
$\phantom{\therefore}\displaystyle\sum_{n=1}^{\infty}(1-\frac{2}{n})^{n^2}$收敛性\\[2ex]

(5)级数中是否含$\displaystyle\frac{1}{n}$或对数 --- 积分判别法\\
\begin{center}
\framebox{若对连续递减函数$f$有$a_n=f(n)$,则$\displaystyle\sum_{n=N}^{\infty}a_n$与$\displaystyle\int_N^{\infty}f(x)\dif x$同时收敛或发散}
\end{center}
例.\\
$\displaystyle\sum_{n=2}^{\infty}\frac{1}{n\ln(n)}$\\[1ex]
推导过程:\\
级数的积分形式为:\\
$\displaystyle\int_2^{\infty}\frac{1}{x\ln(x)}\dif x$\\
设$t=\ln(x)$,则$\displaystyle\dif t=\frac{1}{x}\dif x$,得:\\
$\displaystyle\int_2^{\infty}\frac{1}{x\ln(x)}\dif x=\int_{\ln(2)}^{\infty}\frac{1}{t}\dif t$\\
$\because$由P判别法\\
$\phantom{\because}\displaystyle\int_{\ln(2)}^{\infty}\frac{1}{t}\dif t$发散\\
$\therefore\displaystyle\sum_{n=2}^{\infty}\frac{1}{n\ln(n)}$发散\\[2ex]

(6)级数中是否有负项 --- 第n项判别式/绝对收敛判别法/交错级数判别法\\
1)若所有项都为负,则在所有项前面添加负号来修改级数\\
例.\\
$\displaystyle\sum_{n=3}^{\infty}\ln(\frac{1}{n})\frac{1}{\sqrt{n}}$\\[1ex]
推导过程:\\
$\because$在$n\geqslant 3$时,$\ln(\frac{1}{n})<0$,$\ln(\frac{1}{n})\frac{1}{\sqrt{n}}<0$\\
$\therefore\displaystyle\sum_{n=3}^{\infty}-\ln(\frac{1}{n})\frac{1}{\sqrt{n}}=\sum_{n=3}^{\infty}\ln(n)\frac{1}{\sqrt{n}}$\\
$\because$当$n\in[3,\infty)$时,$\ln(n)\geqslant\ln(3)$\\
$\therefore\displaystyle\sum_{n=3}^{\infty}\ln(n)\frac{1}{\sqrt{n}}\geqslant\sum_{n=3}^{\infty}\ln(3)\frac{1}{\sqrt{n}}=\ln(3)\sum_{n=3}^{\infty}\frac{1}{\sqrt{n}}=\infty$\\
$\therefore\displaystyle\sum_{n=3}^{\infty}\ln(\frac{1}{n})\frac{1}{\sqrt{n}}$发散\\[2ex]

2)若有些项为正,有些项为负,当$n\to\infty$时通项不趋于$0$,用第n项判别法\\
例.\\
$\displaystyle\sum_{n=1}^{\infty}(-1)^nn^2$\\[1ex]
推导过程:\\
$\because\displaystyle\lim_{n\to\infty}(-1)^nn^2$极限不为0\\
$\therefore$由第n项判别法\\
$\phantom{\therefore}\displaystyle\sum_{n=1}^{\infty}(-1)^nn^2$发散\\[2ex]

3)若有些项为正,有些项为负,当$n\to\infty$时通项趋于0,用绝对收敛判别法\\
\begin{center}
\framebox{若$\displaystyle\sum_{n=1}^{\infty}|a_n|$收敛,则$\displaystyle\sum_{n=1}^{\infty}a_n$也收敛}
\end{center}
例.\\
$\displaystyle\sum_{n=1}^{\infty}\frac{\sin(n)}{n^2}$\\[1ex]
推导过程:\\
$\because\displaystyle\sum_{n=1}^{\infty}\frac{|\sin(n)|}{n^2}\leqslant\sum_{n=1}^{\infty}\frac{1}{n^2}<\infty$\\
$\therefore$由绝对收敛判别法\\
$\phantom{\therefore}\displaystyle\sum_{n=1}^{\infty}\frac{\sin(n)}{n^2}$收敛\\[2ex]

4)若有些项为正,有些项为负,并且级数不是绝对收敛,用交错级数判别法\\
\begin{center}
\framebox{若当$n\to\infty$时交错级数的通项的绝对值单调递减趋于0,则级数收敛}
\end{center}
例.\\
$\displaystyle\sum_{n=1}^{\infty}\frac{(-1)^n}{n}$\\[1ex]
推导过程:\\
$\because\displaystyle\lim_{n\to\infty}\Big|\frac{(-1)^n}{n}\Big|=\lim_{n\to\infty}\frac{1}{n}=0$,并且$\displaystyle\Big|\frac{(-1)^n}{n}\Big|=\frac{1}{n}$单调递减\\
$\therefore$由交错级数判别法\\
$\phantom{\therefore}\displaystyle\sum_{n=1}^{\infty}\frac{(-1)^n}{n}$收敛\\[2ex]

(7)上述皆不适用 --- 比式判别法/P判别法/极限比较判别法\\
1)比较判别法\\

