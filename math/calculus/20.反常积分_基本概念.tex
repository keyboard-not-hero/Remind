\chapter{反常积分:基本概念}
1.当$\int_a^bf(x)\dif x$出现以下情况, 称为反常积分:\\
(1)函数$f$在$[a,b]$内是无界的(垂直渐近线)\\
(2)$b=\infty$\\
(3)$a=-\infty$\\[2ex]

(1)函数在$[a,b]$内无界\\
破裂点: 当函数$f$在$x=a$处有垂直渐近线时,$x=a$为其\textbf{破裂点},也称为\textbf{瑕点}.\\[2ex]
\framebox{\begin{minipage}{\textwidth}
如果仅仅在$x$接近于$a$点该函数$f(x)$是无界的, 则定义
\[\int_a^bf(x)\dif x=\lim_{\varepsilon\to 0^+}\int_{a+\varepsilon}^bf(x)\dif x\]
\end{minipage}}\\
如果上述极限存在, 则积分收敛, 否则积分发散\\
例1.\\
$\displaystyle\int_0^1\frac{1}{x}\dif x$\\
推导过程:\\
$\begin{array}{>{\displaystyle}r >{\displaystyle}l}
\int_0^1\frac{1}{x}\dif x & =\lim_{\varepsilon\to 0^+}\int_{\varepsilon}^1\frac{1}{x}\dif x\\
& =\lim_{\varepsilon\to 0^+}\ln|x|\Big|_{\varepsilon}^1\\
& =\lim_{\varepsilon\to 0^+}(\ln(1)-\ln(\varepsilon))\\
& =\infty
\end{array}$\\
所以, 反常积分$\displaystyle\int_0^1\frac{1}{x}\dif x$发散\\[2ex]

例2.\\
$\displaystyle\int_0^1\frac{1}{\sqrt{x}}\dif x$\\
推导过程:\\
$\begin{array}{>{\displaystyle}r >{\displaystyle}l}
\int_0^1\frac{1}{\sqrt{x}}\dif x & =\lim_{\varepsilon\to 0^+}\int_{\varepsilon}^1\frac{1}{\sqrt{x}}\dif x\\
& =\lim_{\varepsilon\to 0^+}2\sqrt{x}\Big|_{\varepsilon}^1=\lim_{\varepsilon\to 0^+}(2-2\sqrt{\varepsilon})=2
\end{array}$\\
所以, 反常积分$\displaystyle\int_0^1\frac{1}{\sqrt{x}}\dif x$收敛\\

\framebox{\begin{minipage}{\textwidth}
如果函数仅仅在$x$接近于$b$点是无界的, 则定义
\[\int_a^bf(x)\dif x=\lim_{\varepsilon\to 0^+}\int_a^{b-\varepsilon}f(x)\dif x\]
\end{minipage}}\\
如果上述极限存在, 则积分收敛, 否则积分发散\\

\framebox{\begin{minipage}{\textwidth}
如果函数在区间$[a,b]$内有破裂点$c$, 则定义
\[\int_a^bf(x)\dif x=\lim_{\varepsilon\to 0^+}\int_a^{c-\varepsilon}f(x)\dif x+\lim_{\varepsilon\to 0^+}\int_{c+\varepsilon}^bf(x)\dif x\]
\end{minipage}}
如果上述公式两部分的积分都收敛时,总积分收敛,否则发散\\

(2)关于$\infty$区间上的积分\\
\begin{center}
\framebox{$\displaystyle\int_a^{\infty}f(x)\dif x=\lim_{N\to\infty}\int_a^Nf(x)\dif x$}
\end{center}
如果上述极限存在, 则积分收敛, 否则积分发散
例\\
$\displaystyle\int_1^{\infty}\frac{1}{x}\dif x$\\
推导过程:\\
$\begin{array}{>{\displaystyle}r >{\displaystyle}l}
\int_1^{\infty}\frac{1}{x}\dif x & =\lim_{N\to\infty}\int_1^N\frac{1}{x}\dif x\\
& =\lim_{N\to\infty}\ln|x|\Big|_1^N\\
& =\lim_{N\to\infty}(\ln(N)-\ln(1))\\
& =\infty
\end{array}$\\
所以,反常积分$\displaystyle\int_1^{\infty}\frac{1}{x}\dif x$发散\\

(3)关于$-\infty$区间上的积分\\
\begin{center}
\framebox{$\displaystyle\inf_{-\infty}^bf(x)\dif x=\lim_{N\to\infty}\int_{-N}^bf(x)\dif x$}
\end{center}
如果上述极限存在,则积分收敛,否则积分发散\\[4ex]

2.比较判别法(理论)\\
\framebox{\begin{minipage}{\textwidth}
如果在区间$(a,b)$内,函数$f(x)\geqslant g(x)\geqslant 0$,且积分$\displaystyle\int_a^bg(x)\dif x$是发散的,那么积分$\displaystyle\int_a^bf(x)\dif x$也是发散的. 即
\[\int_a^bf(x)\dif x\geqslant\int_a^bg(x)\dif x=\infty\]
\end{minipage}}
\framebox{\begin{minipage}{\textwidth}
如果在区间$(a,b)$内,函数$0\leqslant f(x)\leqslant g(x)$,且积分$\displaystyle\int_a^bg(x)\dif x$是收敛的,那么积分$\displaystyle\int_a^bf(x)\dif x$也一定是收敛的. 即
\[\int_a^bf(x)\dif x\leqslant\int_a^bg(x)\dif x<\infty\]
\end{minipage}}\vspace{4ex}

3.极限比较判别法(理论)\\
\begin{center}
\framebox{当$x\to a$时,$f(x)\backsim g(x)$同$\displaystyle\lim_{x\to a}\frac{f(x)}{g(x)}=1$有着同样的意义}
\end{center}

\framebox{\begin{minipage}{\textwidth}
如果当$x\to a$时$f(x)\backsim g(x)$,且这两个函数在区间$[a,b]$上仅有$a$一个破裂点,那么积分$\displaystyle\int_a^bf(x)\dif x$和$\displaystyle\int_a^bg(x)\dif x$是同时收敛或同时发散的,这称为\textbf{极限比较判别法}
\end{minipage}}
例\\
$\displaystyle\int_0^1\frac{1}{\sin(\sqrt{x})}\dif x$\\
推导过程:\\
$\because x\to 0$时,$\sin(x)\backsim x$\\
$\therefore x\to 0^+$时,$\sin(\sqrt{x})\backsim\sqrt{x}$\\
\phantom{$\therefore$}两边取倒数, 得:\\
$\displaystyle\phantom{\therefore}\frac{1}{\sin(\sqrt{x})\backsim\frac{1}{\sqrt{x}}}$\\
$\because[0,1]$区间上,仅有$0$为破裂点\\
$\displaystyle\therefore\int_0^1\frac{1}{\sqrt{x}}\dif x=\lim_{\varepsilon\to 0^+}\int_{\varepsilon}^1\frac{1}{\sqrt{x}}\dif x=\lim_{\varepsilon\to 0^+}2\sqrt{x}\Big|_{\varepsilon}^1=\lim_{\varepsilon\to 0^+}(2-2\sqrt{\varepsilon})=2$\\
$\displaystyle\phantom{\therefore}\int_0^1\frac{1}{\sqrt{x}}\dif x$在区间$[0,1]$上收敛\\
同理,$\displaystyle\int_0^1\frac{1}{\sqrt{x}}\dif x$在区间$[0,1]$上收敛\\[4ex]

4.P判别法(理论)\\
$\displaystyle\cdot\int^{\infty}$的情况:对于任意有限值$a>0$,积分
\[\int_a^{\infty}\frac{1}{x^p}\dif x\]
\phantom{$\cdot$}在$p>1$时是收敛的,在$p\leqslant 1$时是发散的\\
$\displaystyle\cdot\int_0$的情况:对于任意有限值$a>0$,积分
\[\int_0^a\frac{1}{x^p}\dif x\]
\phantom{$\cdot$}在$p<1$时是收敛的,在$p\geqslant 1$时是发散的\\[4ex]

5.绝对收敛判别法(理论)\\
\begin{center}
\framebox{如果$\displaystyle\int_a^b|f(x)|\dif x$是收敛的,那么$\displaystyle\int_a^bf(x)\dif x$也是收敛的}
\end{center}
例\\
$\displaystyle\int_1^{\infty}\frac{\sin(x)}{x^2}\dif x$\\
推导过程:\\
$\displaystyle\because|\frac{\sin(x)}{x^2}|\leqslant\frac{1}{x^2}$\\
$\therefore$根据比较判别法:\\
$\displaystyle\int_1^{\infty}\frac{|\sin(x)|}{x^2}\dif x\leqslant\int_1^{\infty}\frac{1}{x^2}\dif x$\\
$\because$根据P判别法\\
$\displaystyle\phantom{\because}\int_1^{\infty}\frac{1}{x^2}\dif x$收敛\\[1ex]
$\displaystyle\therefore\int_1^{\infty}\frac{|\sin(x)|}{x^2}\dif x$收敛\\
$\therefore$根据绝对收敛判别法\\
$\displaystyle\phantom{\therefore}\int_1^{\infty}\frac{\sin(x)}{x^2}$收敛
