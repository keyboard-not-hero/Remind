\chapter{极限导论}
极限: 描述函数的自变量接近于某一个值时, 相对应的函数值变化的趋势. 表示为:
\[\lim_{x\to a}f(x)=L\]\\[1em]
左极限: 描述函数的自变量从左边接近于某一个值时, 相对应的函数变化的趋势. 表示为
\[\lim_{x\to a^-}f(x)=L\]\\[1em]
右极限: 描述函数的自变量从右边接近于某一个值时, 相对应的函数变化的趋势. 表示为
\[\lim_{x\to a^+}f(x)=L\]\\[1em]
当左极限与右极限不相等时, 不存在双侧极限.\\[1em]
\begin{center}
\framebox{
\begin{minipage}{10cm}
``$f$在$x=a$处有一条垂直渐近线''说的是, $\displaystyle\lim_{x\to a^+}f(x)$和$\displaystyle\lim_{x\to a^-}f(x)$, 其中至少有一个极限是$\infty$或-$\infty$
\end{minipage}}
\end{center}\vspace{2em}
\begin{center}
\framebox{
\begin{minipage}{11cm}
``$f$在$y=L$处有一条右侧水平渐近线''意味着$\displaystyle\lim_{x\to\infty}f(x)=L$.\\
``$f$在$y=M$处有一条左侧水平渐近线''意味着$\displaystyle\lim_{x\to -\infty}f(x)=M$.
\end{minipage}}
\end{center}\vspace{2em}
三明治定理(夹逼定理):\\
\begin{center}
\framebox{
\begin{minipage}{10cm}
如果对于所有在$a$附近的$x$都有$g(x)\leqslant f(x)\leqslant h(x)$, 且$\displaystyle\lim_{x\to a}g(x)=\lim_{x\to a}h(x)=L$, 则$\displaystyle\lim_{x\to a}f(x)=L$.
\end{minipage}}
\end{center}
