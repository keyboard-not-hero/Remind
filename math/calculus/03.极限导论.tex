\chapter{极限导论}
\phantom{空格}函数$g$的定义域是所有实数, 并且$g(x)$可以被定义为如下的分段函数:\\
\[
	g(x)=\left\{
	\begin{array}{l l}
		x-1 & if\ x\neq2\\
		3 & if\ x=2
	\end{array}
	\right.
\]
$\displaystyle\lim_{x\to2}g(x)$与$g(2)$的值是不相关的, 关键在于$x$接近于$2$时, $g(x)$的值, 结果为: $\displaystyle\lim_{x\to2}g(x)=1$.\\
\phantom{空格}左极限: 描述函数的自变量从左边接近于某一个值时, 相对应的函数变化的趋势. 表示为
\[\lim_{x\to a^-}f(x)=L\]\\
\phantom{空格}右极限: 描述函数的自变量从右边接近于某一个值时, 相对应的函数变化的趋势. 表示为
\[\lim_{x\to a^+}f(x)=L\]\\
仅当左极限和右极限在$x=a$处都存在且相等, 双侧极限在$x=a$处存在. 用数学语言描述为:\\
\[\lim_{x\to a^-}f(x)=L\quad\text{且}\quad\lim_{x\to a^+}f(x)=L\]
等价于
\[\lim_{x\to a}f(x)=L\]

垂直渐近线的定义:
{\par\centering
\framebox{\begin{minipage}{10cm}
``$f$在$x=a$处有一条垂直渐近线''说的是, $\displaystyle\lim_{x\to a^+}f(x)$和$\displaystyle\lim_{x\to a^-}f(x)$, 其中至少有一个极限是$\infty$或-$\infty$
\end{minipage}}
\par}\vspace{4ex}

水平渐近线的定义:
{\par\centering
\framebox{\begin{minipage}{11cm}
``$f$在$y=L$处有一条右侧水平渐近线''意味着$\displaystyle\lim_{x\to\infty}f(x)=L$.\\
``$f$在$y=M$处有一条左侧水平渐近线''意味着$\displaystyle\lim_{x\to -\infty}f(x)=M$.
\end{minipage}}
\par}\vspace{4ex}

对于大的数和小的数的非正式定义:
\begin{itemize}
		\item 如果一个数的绝对值是非常大的数, 则这个数是大的
		\item 如果一个数非常接近于0(但不是真的等于0), 则这个数是小的
\end{itemize}\vspace{4ex}

一个函数可以有不同的右侧和左侧水平渐近线, 但最多只能有两条水平渐近线(一条在右侧, 一条在左侧), 也有可能一条都没有, 或者只有一条.\\

一个函数可以有很多条垂直渐近线.\\\vspace{4ex}

三明治定理(夹逼定理):
{\par\centering
\framebox{
\begin{minipage}{10cm}
如果对于所有在$a$附近的$x$都有$g(x)\leqslant f(x)\leqslant h(x)$, 且$\displaystyle\lim_{x\to a}g(x)=\lim_{x\to a}h(x)=L$, 则$\displaystyle\lim_{x\to a}f(x)=L$.
\end{minipage}}
\par}

%最后编辑于: 2021-12-24
