\documentclass[UTF8, fontset=ubuntu, fleqn]{ctexart}
\usepackage{parskip, tikz}
\renewcommand{\theequation}{\roman{equation}}
\setlength{\mathindent}{0ex}
\begin{document}
1.隐函数求导\\
例.\\
$x^2+y^2=4$\\[2ex]
推导过程:\\
设$u=y^2$, 则$\frac{du}{dx}=\frac{du}{dy}\frac{dy}{dx}=2y\frac{dy}{dx}$\\
原函数两边对$x$求导:\\[1ex]
$\displaystyle\frac{d}{dx}(x^2)+\frac{d}{dx}(y^2)=0\quad\Rightarrow\quad 2x+2y\frac{dy}{dx}=0\Rightarrow\frac{dy}{dx}=-\frac{x}{y}$\\[2ex]

2.隐函数求二阶导数\\
例.\\
$2y+\sin(y)=\frac{x^2}{\pi}+1$\\[2ex]
推导过程:\\
设$u=\sin(y)$, 则$\frac{du}{dx}=\frac{du}{dy}\frac{dy}{dx}=\cos(y)\frac{dy}{dx}$\\
原函数两边对$x$求导:\\[1ex]
$\displaystyle 2\frac{dy}{dx}+\cos(y)\frac{dy}{dx}=\frac{2x}{\pi}\quad\Rightarrow\quad\frac{dy}{dx}=\frac{2x}{\pi(2+\cos(y))}$\\[1ex]
两边再次对$x$求导:\\[1ex]
$\displaystyle 2\frac{d^2y}{dx^2}+\frac{d}{dx}(\cos(y)\frac{dy}{dx})=\frac{2}{\pi}\quad\Rightarrow\quad 2\frac{d^2y}{dx^2}-\sin(y)(\frac{dy}{dx})^2+\cos(y)\frac{d^2y}{dx^2}=\frac{2}{\pi}\quad\Rightarrow$\\
$\displaystyle (2+\cos(y))\frac{d^2y}{dx^2}-\sin(y)(\frac{dy}{dx})^2=\frac{2}{\pi}$\\[1ex]
将$\frac{dy}{dx}=\frac{2x}{\pi(2+\cos(y))}$带入结果:\\[1ex]
$\displaystyle (2+\cos(y))\frac{d^2y}{dx^2}=\frac{2}{\pi}+\sin(y)(\frac{2x}{\pi(2+\cos(y))})^2\quad\Rightarrow\quad\frac{d^2y}{dx^2}=\frac{2}{\pi(2+\cos(y))}+\sin(y)\cdot\frac{4x^2}{\pi^2(2+\cos(y))^3}$\\[2ex]

3.相关变化率\\[-3ex]
\begin{center}
\framebox{如果$Q$是某个量, 那么$Q$的变化率是$\displaystyle\frac{dQ}{dt}$}\\
\end{center}\vspace{-1ex}
\begin{center}
\framebox{
\begin{minipage}[c]{10cm}
设$x=f(t)$和$y=g(t)$为两个变量的变化率, 由于$x$和$y$都是关于时间$t$的函数, 所以$x$与$y$必定存在某种关系, 这种关系称为\textbf{相对变化率}
\end{minipage}
}
\end{center}
求解相关变化率的方法:\\
(1)识别出哪一个量需要求相关变化率;\\
(2)写出一个关联所有量的方程;\\
(3)对方程关于时间t做隐函数求导;\\
(4)将已知值带入方程中做替换.\\

例1.\\
用打气筒给一个完美球体的气球充气. 空气以常数速率$12\pi$立方英寸每秒进入气球.\\
(1)当气球的半径达到$2$英寸时, 气球的半径的变化率是多少?\\
(2)从外, 当气球的体积达到$36\pi$立方英寸时, 气球的半径的变化率又是多少?

解:\\
球体体积公式:\\[-2ex]
\begin{center}
\framebox{$\displaystyle V=\frac{4}{3}\pi r^3$}
\end{center}
方程对时间t进行隐式求导:
\begin{equation}
\displaystyle\frac{dV}{dt}=4\pi r^2\frac{dr}{dx}
\end{equation}
(1)将$r=2$和$\frac{dV}{dt}=12\pi$代入公式(i), 得:\\[1ex]
\phantom{(1)}$\displaystyle 4\pi\times 2^2\frac{dr}{dx}=12\pi\quad\Rightarrow\quad\frac{dr}{dx}=\frac{3}{4}$\\[1ex]
(2)根据球体体积公式, 得:\\[1ex]
\phantom{(2)}$\displaystyle\frac{4}{3}\pi r^3=36\pi\quad\Rightarrow\quad r=3$\\[1ex]
\phantom{(2)}将$r=3$和$\frac{dv}{dt}=12\pi$代入公式(i), 得:\\[1ex]
\phantom{(2)}$\displaystyle 4\pi\times 3^2\frac{dr}{dx}=12\pi\quad\Rightarrow\quad\frac{dr}{dx}=\frac{1}{3}$\\[2ex]

例2.\\
假设有两辆汽车A和B. 汽车A在一条路上径直向北行驶远离你家, 而汽车B在另一条路上径直向西行驶接近你家. 汽车A以55英里/小时的速度行驶, 而汽车B以45英里/小时的速度行驶. 当A到达你家北面21英里, 而B到达你家东面28英里时, 两辆汽车间的距离的变化率是多少?

解:\\
如图.\\
\begin{tikzpicture}
\draw[color=white] (0,0) -- (5,0);
\draw[color=white] (0,0) -- (0,5);
\begin{scope}[xshift=3cm]
% 三角形
\draw[auto,swap] (0,0) node[below left=2mm]{H} -- node{b} (4,0) node[below=2mm]{B} -- node{c} (0,3) node[above left=2mm]{A} -- node{a} cycle;
% 小屋子
\begin{scope}[xshift=-3.5mm, yshift=1mm, scale=0.5]
\draw (0,0) -- (45:1);
\draw (45:2 |- 0,0) -- +(135:1);
\draw (45:0.3) -- (45:0.3 |- 0,-0.7);
\draw (45:2 |- 0,0) +(135:0.3) -- (45:1.7 |- 0,-0.7);
\draw (45:0.3 |- 0,-0.7) -- (45:1.7 |- 0,-0.7);
\draw (45:0.8 |- 0,-0.3) rectangle (45:1.2 |- 0,0);
\draw (45:2 |- 0,0) +(135:0.6) -- +(135:0.4) -- (45:1.6 |- 0,0.5) -- (45:1.4 |- 0,0.5) -- cycle;
\end{scope}
% 小货车
\begin{scope}[rotate=-90, xshift=-3cm, yshift=-1mm, scale=0.5]
\draw[rounded corners=3pt] (-1,0) rectangle (1,0.46);
\filldraw[fill=white] (-0.5,0.04) circle (0.19);
\filldraw[fill=white] (0.6,0.04) circle (0.19);
\draw[rounded corners=2pt] (-0.7,0.46) -- (-0.44,0.76) -- (-0.03,0.76) -- (-0.03,0.46);
\end{scope}
% 小汽车
\begin{scope}[xshift=4cm, yshift=-1.5mm, scale=0.5]
\draw[rounded corners=4pt] (-1,0) rectangle (1,0.4);
\filldraw[fill=white] (-0.6,0.05) circle (0.15);
\filldraw[fill=white] (0.6,0.05) circle (0.15);
\draw (-0.5,0.4) -- (-0.3,0.64) -- (0.3,0.64) -- (0.6,0.4);
\draw (0,0.4) -- (0,0.64);
\draw (0.25,0.64) -- (0.36,0.4);
\end{scope}
\end{scope}
\end{tikzpicture}
由图可知:\\
\[a^2+b^2=c^2\]
对时间t作隐函数求导:\\
\begin{equation}
2a\frac{da}{dt}+2b\frac{db}{dt}=2c\frac{dc}{dt}
\end{equation}
由于A在远离H, 所以距离随着时间增加:\\[1ex]
$\displaystyle\frac{da}{dt}=55$\\[1ex]
而B在靠近H, 所以距离随着时间减少:\\[1ex]
$\displaystyle\frac{db}{dt}=-45$\\[1ex]
将结果带入公式(ii), 得:\\[1ex]
$\displaystyle\frac{dc}{dt}=-3$\\[2ex]

例3.\\

\end{document}
