\chapter{数列和级数:基本概念}\
1.数列的收敛和发散\\
数列: 一组有序的数, 可能为有限项, 也可能为无穷项. 如:$\{a_n\}=1,3,5,7,9$\\
无穷序列: 有无穷项的数列. 如:$\{b_n\}=1,3,5,\cdots$\\
数列的项:数列中的第$n$项. 如:$a_n$
\begin{center}
\begin{boxedminipage}{\textwidth}
当满足下列条件:
\[\lim_{x\to\infty}a_n=L\]
则称数列$\{a_n\}$收敛
\end{boxedminipage}
\end{center}\vspace{2ex}

数列的三明治定理(夹逼定理)
\begin{center}
\framebox{若$c_n\leqslant a_n\leqslant b_n$, 且当$n\to\infty$时, $b_n\to L$, $c_n\to L$, 则$a_n\to L$}
\end{center}\vspace{2ex}

洛必达法则的应用
\begin{center}
\framebox{利用洛必达法则求解$\displaystyle\lim_{x\to\infty}f(x)$, 从而得到$\displaystyle\lim_{n\to\infty}a_n$}
\end{center}

例.\\
\phantom{例}$\displaystyle\lim_{n\to\infty}\frac{\ln(n)}{\sqrt{n}}$\\
推导过程:\\
令$\displaystyle f(x)=\frac{\ln x}{\sqrt{n}}$\\
$\displaystyle\lim_{x\to\infty}\frac{\ln x}{\sqrt{x}}=\lim_{x\to\infty}\frac{\frac{1}{x}}{\frac{1}{2\sqrt{x}}}=\lim_{x\to\infty}\frac{2}{\sqrt{x}}=0$\\
由于$a_n$的极限为0\\
$\therefore$数列$\{a_n\}$收敛\\[2ex]

数列的收敛与发散规则
\begin{center}
$\displaystyle\lim_{n\to\infty}r^n\left\{
\begin{array}{l l}
=0 & \text{如果} -1<r<1\\
=1 & \text{如果} r=1\\
=\infty & \text{如果} r>1\\
\text{不存在} & \text{如果} r\leqslant -1
\end{array}
\right.$
\end{center}\vspace{6ex}

2.级数的收敛和发散\\
级数: 将数列${a_n}$的所有项加起来. 表示为:
\[\sum_{n=1}^{\infty}a_n=\lim_{N\to\infty}\sum_{n=1}^Na_n=a_1+a_2+\cdots+a_{n-1}+a_n\]\vspace{4ex}

3.几何级数\\
由等比数列构成的无穷的级数, 称为几何级数. 表示为:
\[\sum_{n=1}^{\infty}r^n=1+r+r^2+\cdots+r^n=\frac{1-r^{n+1}}{1-r}\]

规律总结:
\begin{center}
\begin{boxedminipage}{6cm}
如果$-1<r<1$,$\displaystyle\sum_{n=0}^{\infty}r^n=\frac{1}{1-r}$\\
如果$r\geqslant 1$或$r\leqslant -1$,级数发散
\end{boxedminipage}
\end{center}\vspace{4ex}

\begin{center}
\begin{boxedminipage}{6cm}
如果$-1<r<1$,$\displaystyle\sum_{n=0}^{\infty}ar^n=\frac{a}{1-r}$\\
如果$r\geqslant 1$或$r\leqslant -1$,级数发散
\end{boxedminipage}
\end{center}\vspace{6ex}

4.第$n$项判别法(理论)
{\par\centering
\framebox{若$\displaystyle\lim_{n\to\infty}a_n\neq 0$,或极限不存在,则级数$\displaystyle\sum_{n=1}^{\infty}a_n$发散}
\par}
注解: 以上判别法不能用于级数收敛性的判断\\[4ex]

5.无穷级数和反常积分的性质\\
1)比较判别法(理论)
{\par\centering
\framebox{若对所有$n$,有$0\leqslant b_n\leqslant a_n$,且$\displaystyle\sum_{n==1}^{\infty}b_n$发散,则$\displaystyle\sum_{n=1}^{\infty}a_n$也发散}\\[2ex]
\framebox{若对所有$n$,有$b_n\geqslant a_n\geqslant 0$,且$\displaystyle\sum_{n=1}^{\infty}b_n$收敛,则$\displaystyle\sum_{n=1}^{\infty}a_n$也收敛}
\par}\vspace{4ex}

2)极限比较判别法(理论)
{\par\centering
\framebox{若当$n\to\infty$时$a_n\backsim b_n$,且$a_n$和$b_n$均有限,则$\displaystyle\sum_{n=1}^{\infty}a_n$与$\displaystyle\sum_{n=1}^{\infty}b_n$同时收敛或发散}
\par}\vspace{4ex}

3)p判别法{理论}
{\par\centering
\framebox{$\displaystyle\sum_{n=a}^{\infty}\frac{1}{n^p}\left\{\begin{array}{l l}
\text{收敛} & \text{如果}p>1\\
\text{发散} & \text{如果}p\leqslant 1
\end{array}
\right.$}
\par}\vspace{4ex}

4)绝对收敛判别法(理论)
{\par\centering
\framebox{如果级数$\displaystyle\sum_{n=m}^{\infty}|a_n|$收敛,则级数$\displaystyle\sum_{n=m}^{\infty}a_n$收敛}
\par}\vspace{6ex}

6.级数的新判别法\\
1)比式判别法(理论)
{\par\centering
\begin{boxedminipage}{\textwidth}
若$\displaystyle L=\lim_{n\to\infty}|\frac{a_{n+1}}{a_n}|$,则$\displaystyle\sum_{n=1}^{\infty}a_n$在$L<1$时绝对收敛,在$L>1$时发散;但当$L=1$或极限不存在时,比式判别法无效
\end{boxedminipage}
\par}\vspace{4ex}

2)根式判别法(理论)
{\par\centering
\begin{boxedminipage}{\textwidth}
若$\displaystyle L=\lim_{n\to\infty}|a_n|^{\frac{1}{n}}$,则$\displaystyle\sum_{n=1}^{\infty}a_n$在$L<1$时绝对收敛,在$L>1$时发散;但当$L=1$或极限不存在时,根式判别法无效
\end{boxedminipage}
\par}\vspace{4ex}

3)积分判别法(理论)
{\par\centering
\begin{boxedminipage}{\textwidth}
若对连续递减函数$f$有$a_n=f(n)$,则$\displaystyle\sum_{n=N}^{\infty}a_n$与$\displaystyle\int_N^{\infty}f(x)\dif x$同时收敛或发散
\end{boxedminipage}
\par}\vspace{4ex}

4)交错级数判别法(理论)\\
当一个级数收敛而其绝对值形式发散,该级数\textbf{条件收敛}
{\par\centering
\begin{boxedminipage}{\textwidth}
若级数$\displaystyle\sum_{n=1}^{\infty}a_n$是交错的,且各项的绝对值递减趋于$0$,则级数收敛\\
收敛条件列表:\\
(1)\,$a_n$正负交错. 如:$(-1)^n$\\
(2)\,$|a_n|$递减\\
(3)\,$\displaystyle\lim_{n\to\infty}|a_n|=0$
\end{boxedminipage}
\par}

%最后编辑于: 2022-01-15
