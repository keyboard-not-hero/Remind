\chapter{数列和级数:基本概念}\
1.数列的收敛和发散\\
数列:一组有序的数. 如:$\{a_n\}=1,3,5,7,9$\\
数列的项:数列中的第$n$项. 如:$a_n$\\
\begin{center}
\framebox{\begin{minipage}{\textwidth}
当满足下列条件:
\[\lim_{x\to\infty}a_n=L\]
则称数列$\{a_n\}$收敛
\end{minipage}}
\end{center}

三明治定理和洛必达法则同样适用于数列\\
\begin{center}
\framebox{$\displaystyle\lim_{n\to\infty}r^n\left\{\begin{array}{l l}
=0 & \text{如果} -1<r<1\\
=1 & \text{如果} r=1\\
=\infty & \text{如果} r>1\\
\text{不存在} & \text{如果} r\leqslant -1
\end{array}\right.$}\\[2ex]
\framebox{$\displaystyle\lim_{n\to\infty}(1+\frac{k}{n})^n=e^k$}
\end{center}\vspace{4ex}

2.级数的收敛和发散\\
级数: 将数列${a_n}$的所有项加起来. 表示为:\\
\[\sum_{n=1}^{\infty}a_n=\lim_{N\to\infty}\sum_{n=1}^Na_n=a_1+a_2+\cdots+a_{n-1}+a_n\]
几何级数: 等比数列的级数. 表示为:\\
\[\sum_{n=1}^{\infty}r^n=1+r+r^2+\cdots+r^n\]
\begin{center}
\framebox{\begin{minipage}{\textwidth}
如果$-1<r<1$,$\displaystyle\sum_{n=0}^{\infty}r^n=\frac{1}{1-r}$\\
如果$r\geqslant 1$或$r\leqslant -1$,级数发散
\end{minipage}}
\end{center}\vspace{4ex}

3.第$n$项判别法(理论)\\
\begin{center}
\framebox{若$\displaystyle\lim_{n\to\infty}a_n\neq 0$,或极限不存在,则级数$\displaystyle\sum_{n=1}^{\infty}a_n$发散}
\end{center}
以上判别法不能用于级数收敛性的判断\\[4ex]

4.无穷级数和反常积分的性质\\
(1)比较判别法(理论)\\
\begin{center}
\framebox{若对所有$n$,有$0\leqslant b_n\leqslant a_n$,且$\displaystyle\sum_{n==1}^{\infty}b_n$发散,则$\displaystyle\sum_{n=1}^{\infty}a_n$也发散}\\[2ex]
\framebox{若对所有$n$,有$b_n\geqslant a_n\geqslant 0$,且$\displaystyle\sum_{n=1}^{\infty}b_n$收敛,则$\displaystyle\sum_{n=1}^{\infty}a_n$也收敛}
\end{center}\vspace{2ex}

(2)极限比较判别法(理论)\\
\begin{center}
\framebox{若当$n\to\infty$时$a_n\backsim b_n$,且$a_n$和$b_n$均有限,则$\displaystyle\sum_{n=1}^{\infty}a_n$与$\displaystyle\sum_{n=1}^{\infty}b_n$同时收敛或发散}
\end{center}\vspace{2ex}

(3)P判别法{理论}\\
\begin{center}
\framebox{$\displaystyle\sum_{n=a}^{\infty}\frac{1}{n^p}\left\{\begin{array}{l l}
\text{收敛} & \text{如果}p>1\\
\text{发散} & \text{如果}p\leqslant 1
\end{array}
\right.$}
\end{center}\vspace{2ex}

(4)绝对收敛判别法(理论)\\
\begin{center}
\framebox{如果级数$\displaystyle\sum_{n=m}^{\infty}|a_n|$收敛,则级数$\displaystyle\sum_{n=m}^{\infty}a_n$收敛}
\end{center}\vspace{2ex}

5.级数的新判别法\\
(1)比式判别法(理论)\\
\begin{center}
\framebox{若$\displaystyle L=\lim_{n\to\infty}|\frac{a_{n+1}}{a_n}|$,则$\displaystyle\sum_{n=1}^{\infty}a_n$在$L<1$时绝对收敛,在$L>1$时发散;但当$L=1$或极限不存在时,比式判别法无效}
\end{center}\vspace{2ex}

(2)根式判别法(理论)\\
\begin{center}
\framebox{\begin{minipage}{\textwidth}
若$\displaystyle L=\lim_{n\to\infty}|a_n|^{\frac{1}{n}}$,则$\displaystyle\sum_{n=1}^{\infty}a_n$在$L<1$时绝对收敛,在$L>1$时发散;但当$L=1$或极限不存在时,根式判别法无效
\end{minipage}}
\end{center}\vspace{2ex}

(3)积分判别法(理论)\\
\begin{center}
\framebox{\begin{minipage}{\textwidth}
若对连续递减函数$f$有$a_n=f(n)$,则$\displaystyle\sum_{n=N}^{\infty}a_n$与$\displaystyle\int_N^{\infty}f(x)\dif x$同时收敛或发散
\end{minipage}}
\end{center}\vspace{2ex}

(4)交错级数判别法(理论)\\
\begin{center}
\framebox{若当$n\to\infty$时交错级数的通项的绝对值单调递减趋于$0$,则级数收敛}
\end{center}
