\chapter{泰勒级数和幂级数:如何解题}
{\par\centering
\framebox{几何级数$\displaystyle\sum_{n=0}^{\infty}x^n$在$-1<x<1$时收敛,其他情况均发散}
\par}

{\par\centering
\framebox{级数$\displaystyle\sum_{n=0}^{\infty}\frac{x^n}{n!}$对任意$x$值收敛,因为$\displaystyle\lim_{x\to\infty}\frac{e^x}{x!}=0$}
\par}

1.幂级数收敛情况\\
(1)以$a$为中心,$R>0$为收敛半径
\begin{itemize}
\item 区域$|x-a|<R$内收敛
\item 区域$|x-a|>R$发散
\item 在$|x-a|=R$端点上有绝对收敛/条件收敛/发散的情况
\end{itemize}
(2)对所有$x$均绝对收敛,收敛半径为$\infty$\\
(3)只在$x=a$处收敛,收敛半径为$0$\\[2ex]

2.求收敛半径\\
(1)使用比式判别法或根式判别法
\[\lim_{n\to\infty}|\frac{a_{n+1}(x-a)^{n+1}}{a_n(x-a)^n}|=\lim_{n\to\infty}|\frac{a_{n+1}}{a_n}||x-a|\]
\[\lim_{n\to\infty}|a_n(x-a)^n|^{\frac{1}{n}}=\lim_{n\to\infty}|a_n|^{\frac{1}{n}}|x-a|\]
(2)算出极限$L|x-a|$\\
(3)若$L$为正数,收敛半径为$\displaystyle\frac{1}{L}$\\
(4)若$L$为$0$,收敛半径为$\infty$\\
(5)若$L$为$\infty$,收敛半径为$0$\\
例.\\
$\displaystyle\sum_{n=2}^{\infty}\frac{x^n}{n\ln(n)}$\\
推导过程:\\[1ex]
$\displaystyle\lim_{n\to\infty}\bigg|\frac{\frac{x^{n+1}}{(n+1)\ln(n+1)}}{\frac{x^n}{n\ln(n)}}\bigg|=\lim_{n\to\infty}\Big|\frac{x^{n+1}}{x^n}\times\frac{n}{n+1}\times\frac{\ln(n)}{\ln(n+1)}\Big|=|x|$\\[1ex]
$\because$收敛半径为1, 所以分为如下三种情况:\\
1)当$|x|<1$时,幂级数收敛. 此时$x\in(-1,1)$\\
2)当$|x|>1$时,幂级数发散. 此时$x\in(-\infty,-1)\cup(1,\infty)$\\
3)当$|x|=1$时,再次分为如下两种情况:\\[1ex]
\phantom{\qquad}$x=1$,得到$\displaystyle\sum_{n=2}^{\infty}\frac{1}{n\ln(n)}$,由积分判别法可知,级数发散\\[1ex]
\phantom{\qquad}$x=-1$,得到$\displaystyle\sum_{n=2}^{\infty}\frac{(-1)^n}{n\ln(n)}$,由交错级数判别法可知,级数收敛\\[2ex]

3.合成新泰勒级数\\
(1)六个常用的麦克劳林级数\\
\phantom{\quad}1)\quad$\displaystyle e^x=\sum_{n=0}^{\infty}\frac{x^n}{n!}=1+x+\frac{x^2}{2!}+\frac{x^3}{3!}+\cdots$\\[1ex]
\phantom{\quad}2)\quad$\displaystyle\sin(x)=\sum_{n=0}^{\infty}\frac{(-1)^nx^{2n+1}}{(2n+1)!}=x-\frac{x^3}{3!}+\frac{x^5}{5!}-\frac{x^7}{7!}+\cdots$\\[1ex]
\phantom{\quad}3)\quad$\displaystyle\cos(x)=\sum_{n=0}^{\infty}\frac{(-1)^nx^{2n}}{(2n)!}=1-\frac{x^2}{2!}+\frac{x^4}{4!}-\frac{x^6}{6!}+\cdots$\\[1ex]
\phantom{\quad}4)\quad$\displaystyle\frac{1}{1-x}=\sum_{n=0}^{\infty}x^n=1+x+x^2+x^3+\cdots$\\[1ex]
\phantom{\quad}5)\quad$\displaystyle\ln(1+x)=\sum_{n=1}^{\infty}\frac{(-1)^{n+1}x^n}{n}=x-\frac{x^2}{2}+\frac{x^3}{3}-\frac{x^4}{4}+\cdots$\\[1ex]
\phantom{\quad}6)\quad$\displaystyle\ln(1-x)=\sum_{n=1}^{\infty}-\frac{x^n}{n}=-x-\frac{x^2}{2}-\frac{x^3}{3}-\frac{x^4}{4}+\cdots$\\[2ex]

(2)常用麦克劳林级数的推导\\
1)\quad$\displaystyle e^x=\sum_{n=0}^{\infty}\frac{x^n}{n!}=1+x+\frac{x^2}{2!}+\frac{x^3}{3!}+\cdots$\\[1ex]
推导过程:\\
$e^x$关于$x=0$的导数表\\[1ex]
$\begin{array}{c|c|c}
\hline
n & f^{(n)}(x) & f^{(n)}(a)\\
\hline
0 & e^x & 1\\
1 & e^x & 1\\
2 & e^x & 1\\
\vdots & \vdots & \vdots\\
\hline
\end{array}$\\[1ex]
$e^x$的麦克劳林级数\\
$\begin{array}{>{\displaystyle}r >{\displaystyle}l}
e^x & =f(0)+f'(0)x+\frac{f''(0)}{2!}x^2+\frac{f^{(3)}}{3!}x^3+\cdots\\
& =1+x+\frac{x^2}{2!}+\frac{x^3}{3!}+\cdots\\
& =\sum_{n=0}^{\infty}\frac{x^n}{n!}
\end{array}$\\[1ex]

2)\quad$\displaystyle\sin(x)=\sum_{n=0}^{\infty}\frac{(-1)^nx^{2n+1}}{(2n+1)!}=x-\frac{x^3}{3!}+\frac{x^5}{5!}-\frac{x^7}{7!}+\cdots$\\[1ex]
推导过程:\\
$\sin(x)$关于$x=0$的导数表\\[1ex]
$\begin{array}{c|c|c}
\hline
n & f^{(n)}(x) & f^{(n)}(a)\\
\hline
0 & \sin(x) & 0\\
1 & \cos(x) & 1\\
2 & -\sin(x) & 0\\
3 & -\cos(x) & -1\\
\vdots & \vdots & \vdots\\
\hline
\end{array}$\\[1ex]
$\sin(x)$的麦克劳林级数\\
$\begin{array}{>{\displaystyle}r >{\displaystyle}l}
\sin(x) & =f(0)+f'(0)x+\frac{f''(0)}{2!}x^2+\frac{f^{(3)}}{3!}x^3+\cdots\\
& =x-\frac{x^3}{3!}+\frac{x^5}{5!}-\frac{x^7}{7!}+\cdots\\
& =\sum_{n=1}^{\infty}\frac{(-1)^nx^{2n+1}}{(2n+1)!}
\end{array}$\\[1ex]

3)\quad$\displaystyle\cos(x)=\sum_{n=0}^{\infty}\frac{(-1)^nx^{2n}}{(2n)!}=1-\frac{x^2}{2!}+\frac{x^4}{4!}-\frac{x^6}{6!}+\cdots$\\[1ex]
推导过程:\\
$\cos(x)$关于$x=0$的导数表\\[1ex]
$\begin{array}{c|c|c}
\hline
n & f^{(n)}(x) & f^{(n)}(a)\\
\hline
0 & \cos(x) & 1\\
1 & -\sin(x) & 0\\
2 & -\cos(x) & -1\\
3 & \sin(x) & 0\\
\vdots & \vdots & \vdots\\
\hline
\end{array}$\\[1ex]
$\cos(x)$的麦克劳林级数\\
$\begin{array}{>{\displaystyle}r >{\displaystyle}l}
\cos(x) & =f(0)+f'(0)x+\frac{f''(0)}{2!}x^2+\frac{f^{(3)}}{3!}x^3+\cdots\\
& =1-\frac{x^2}{2!}+\frac{x^4}{4!}-\frac{x^6}{6!}+\cdots\\
& =\sum_{n=0}^{\infty}\frac{(-1)^nx^{2n}}{(2n)!}
\end{array}$\\[1ex]

4)\quad$\displaystyle\frac{1}{1-x}=\sum_{n=0}^{\infty}x^n=1+x+x^2+x^3+\cdots$\\[1ex]
推导过程:\\
$\displaystyle\frac{1}{1-x}$关于$x=0$的导数表\\[1ex]
$\begin{array}{c|c|c}
\hline
n & f^{(n)}(x) & f^{(n)}(a)\\
\hline
0 & \frac{1}{1-x} & 1\\
1 & \frac{1}{(1-x)^2} & 1\\
2 & \frac{2!}{(1-x)^3} & 2!\\
3 & \frac{3!}{(1-x)^4} & 3!\\
\vdots & \vdots & \vdots\\
\hline
\end{array}$\\[1ex]
$\displaystyle\frac{1}{1-x}$的麦克劳林级数\\
$\begin{array}{>{\displaystyle}r >{\displaystyle}l}
\frac{1}{1-x} & =f(0)+f'(0)x+\frac{f''(0)}{2!}x^2+\frac{f^{(3)}}{3!}x^3+\cdots\\
& =1+x+x^2+x^3+\cdots\\
& =\sum_{n=0}^{\infty}x^n
\end{array}$\\[1ex]

5)\quad$\displaystyle\ln(1+x)=\sum_{n=1}^{\infty}\frac{(-1)^{n+1}x^n}{n}=x-\frac{x^2}{2}+\frac{x^3}{3}-\frac{x^4}{4}+\cdots$\\[1ex]
推导过程:\\
$\ln(1+x)$关于$x=0$的导数表\\[1ex]
$\begin{array}{c|c|c}
\hline
n & f^{(n)}(x) & f^{(n)}(a)\\
\hline
0 & \ln(1+x) & 0\\
1 & \frac{1}{1+x} & 1\\
2 & -\frac{1}{(1+x)^2} & -1\\
3 & \frac{2!}{(1+x)^3} & 2\\
\vdots & \vdots & \vdots\\
\hline
\end{array}$\\[1ex]
$\ln(1+x)$的麦克劳林级数\\
$\begin{array}{>{\displaystyle}r >{\displaystyle}l}
\ln(1+x) & =f(0)+f'(0)x+\frac{f''(0)}{2!}x^2+\frac{f^{(3)}}{3!}x^3+\cdots\\
& =x-\frac{x^2}{2}+\frac{x^3}{3}-\frac{x^4}{4}+\cdots\\
& =\sum_{n=1}^{\infty}\frac{(-1)^{n+1}x^n}{n}
\end{array}$\\[1ex]

6)\quad$\displaystyle\ln(1-x)=\sum_{n=1}^{\infty}-\frac{x^n}{n}=-x-\frac{x^2}{2}-\frac{x^3}{3}-\frac{x^4}{4}+\cdots$\\[2ex]
推导过程:\\
$\ln(1-x)$关于$x=0$的导数表\\[1ex]
$\begin{array}{c|c|c}
\hline
n & f^{(n)}(x) & f^{(n)}(a)\\
\hline
0 & \ln(1-x) & 0\\
1 & -\frac{1}{1-x} & -1\\
2 & -\frac{1}{(1-x)^2} & -1\\
3 & -\frac{2}{(1-x)^3} & -2\\
\vdots & \vdots & \vdots\\
\hline
\end{array}$\\[1ex]
$\ln(1-x)$的麦克劳林级数\\
$\begin{array}{>{\displaystyle}r >{\displaystyle}l}
\ln(1-x) & =f(0)+f'(0)x+\frac{f''(0)}{2!}x^2+\frac{f^{(3)}}{3!}x^3+\cdots\\
& =-x-\frac{x^2}{2}-\frac{x^3}{3}-\frac{x^4}{4}-\cdots\\
& =\sum_{n=1}^{\infty}-\frac{x^n}{n}
\end{array}$\\[2ex]

(3)常用麦克劳林级数的置换\\
例1.\\
\quad$\displaystyle e^{x^2}$的麦克劳林级数和收敛区间\\
推导过程:\\
$e^x$的麦克劳林级数\\
$\displaystyle e^x=\sum_{n=0}^{\infty}\frac{x^n}{n!}=1+x+\frac{x^2}{2!}+\frac{x^3}{3!}+\cdots$\\
将$x$替换为$x^2$,得:\\
$\displaystyle e^{x^2}=\sum_{n=0}^{\infty}\frac{(x^2)^n}{n!}=1+x^2+\frac{(x^2)^2}{2!}+\frac{(x^2)^3}{3!}+\cdots$\\
$\phantom{\displaystyle e^{x^2}=\sum_{n=0}^{\infty}\frac{(x^2)^n}{n!}}\displaystyle =1+x^2+\frac{x^4}{2!}+\frac{x^6}{3!}+\cdots$\\
$\because\displaystyle\lim_{n\to\infty}|\frac{\frac{x^{n+1}}{(n+1)!}}{\frac{x^n}{n!}}|=\lim_{n\to\infty}\left|\frac{\frac{1}{(n+1)!}}{\frac{1}{n!}}\right||x|=\lim_{n\to\infty}\frac{1}{n+1}|x|$\\
$\phantom{\because}\displaystyle L=\lim_{n\to\infty}\frac{1}{n+1}=0$\\
$\therefore e^x$级数的收敛半径为$\infty$,即$x\in(-\infty,\infty)$\\
$\phantom{\therefore}e^{x^2}$中对于$x^2\in(-\infty,\infty)$也成立\\
$\because x^2\in(-\infty,\infty)\Rightarrow x\in(-\infty,\infty)$\\
$\therefore e^{x^2}$在$x\in(-\infty,\infty)$上收敛\\[2ex]

例2.\\
\quad$\displaystyle\frac{1}{1+x^2}$的麦克劳林级数和收敛区间\\
推导过程:\\
$\displaystyle\frac{1}{1-x}$的麦克劳林级数\\[1ex]
$\displaystyle\frac{1}{1-x}=\sum_{n=0}^{\infty}x^n=1+x+x^2+x^3+\cdots$\\[1ex]
将$x$替换为$-x^2$,得:\\
$\displaystyle\frac{1}{1+x^2}=\sum_{n=0}^{\infty}(-x^2)^n=\sum_{n=0}^{\infty}(-1)^nx^{2n}=1-x^2+x^4-x^6+\cdots$\\
$\because\displaystyle\lim_{n\to\infty}|\frac{x^{n+1}}{x^n}|=|x|$\\
$\phantom{\because}L=1$\\
$\therefore\displaystyle\frac{1}{1-x}$的收敛半径为$1$,即$x\in(-1,1)$\\
$\phantom{\therefore}\displaystyle\frac{1}{1+x^2}$对应$-x^2\in(-1,1)$\\
$\because -x^2\in(-1,1)\Rightarrow x\in(-1,1)$\\
$\therefore\displaystyle\frac{1}{1+x^2}$在$x\in(-1,1)$上收敛
