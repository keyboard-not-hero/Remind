\chapter{连续性和可导性}
1.连续性\\
1)在一点处连续
\begin{center}
\framebox{如果$\displaystyle\lim_{x\to a}f(x)=f(a)$, 函数$f$在点$x=a$处连续}
\end{center}\vspace{6ex}

2)在区间上连续\\
函数$f$在区间$[a,b]$上连续, 如果\\
(1)函数$f$在(a,b)中的每一点都连续;\\
(2)函数$f$在点$x=a$处右连续; 即$\displaystyle\lim_{x\to a^+}f(x)$存在(且有限), $f(a)$存在, 并且这两个量相等;\\
(3)函数$f$在点$x=b$处左连续; 即$\displaystyle\lim_{x\to a^-}f(x)$存在(且有限), $f(b)$存在, 并且这两个量相等.\\[2ex]

\begin{boxedminipage}{12cm}
	\textbf{介值定理:}如果$f$在$[a,b]$上连续, 并且$f(a)<0$且$f(b)>0$, 那么在区间$(a,b)$上至少有一点$c$, 使得$f(c)=0$.
\end{boxedminipage}\vspace{4ex}

\begin{boxedminipage}{12cm}
	\textbf{最大值与最小值定理:}如果$f$在$[a,b]$上连续, 那么$f$在$[a,b]$上至少有一个最大值和一个最小值.
\end{boxedminipage}\vspace{6ex}

2.可导性\\
通过$(x,f(x))$的切线的斜率是$x$的一个函数, 称为$f$的\textbf{导数}. 表示为
\[f'(x)=\lim_{h\to 0}\frac{f(x+h)-f(x)}{h}\]

导数的导数称为\textbf{二阶导}

二阶及多阶导数:\\
\begin{math}
\begin{array}{l}
	\displaystyle f''(x)=f^{(2)}(x)=\frac{\dif^2y}{\dif x^2}\\[2ex]
	\displaystyle f'''(x)=f^{(3)}(x)=\frac{\dif^3y}{\dif x^3}\\[2ex]
	\cdots
\end{array}
\end{math}\\[2ex]

右导数:\\
$\displaystyle\phantom{\qquad}\lim_{h\to 0^+}\frac{f(x+h)-f(x)}{h}$\\

左导数:\\
$\displaystyle\phantom{\qquad}\lim_{h\to 0^-}\frac{f(x+h)-f(x)}{h}$\vspace{6ex}

\begin{center}
\framebox{如果一个函数$f$在$x$上可导, 那么它在$x$上连续}
\end{center}

%最后编辑于: 2022-01-19
