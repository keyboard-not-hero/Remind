\documentclass[UFT8, fontset=ubuntu]{ctexart}
\usepackage{parskip}
\begin{document}
1.在一点处连续\\[1ex]
\framebox{如果$\displaystyle\lim_{x\to a}f(x)=f(a)$, 函数$f$在点$x=a$处连续}\\[4ex]
2.在区间上连续\\
在区间$(a,b)$上连续 - 函数在区间范围内的所有点都连续(不包括$a,b$)\\
在区间$[a,b]$上连续 - (1)函数在(a,b)上连续; (2)函数在$x=a$处右连续(即$\displaystyle\lim_{x\to a^+}=f(a)$); (3)函数在$x=b$处左连续(即$\displaystyle\lim_{x\to a^-}=f(b)$)\\[4ex]
\framebox{
\begin{minipage}[c]{12cm}
	\textbf{介值定理:}如果$f$在$[a,b]$上连续, 并且$f(a)<0$且$f(b)>0$, 那么在区间$(a,b)$上至少有一点$c$, 使得$f(c)=0$. 代之以$f(a)>0$且$f(b)<0$, 同样成立.
\end{minipage}}\\[4ex]
\framebox{
\begin{minipage}[c]{12cm}
	\textbf{最大值与最小值定理:}如果$f$在$[a,b]$上连续, 那么$f$在$[a,b]$上至少有一个最大值和一个最小值.
\end{minipage}}\\[4ex]
求导公式:
\[f'(x)=f^{(1)}(x)=\frac{dy}{dx}=\lim_{h\to 0}\frac{f(x+h)-f(x)}{h}\]
二阶及多阶导数:\\
\begin{math}
\begin{array}{l}
	\displaystyle f''(x)=f^{(2)}(x)=\frac{d^2y}{dx^2}\\[2ex]
	\displaystyle f'''(x)=f^{(3)}(x)=\frac{d^3y}{fx^3}\\[2ex]
	\cdots
\end{array}
\end{math}\\[2ex]
\framebox{如果一个函数$f$在$x$上可导, 那么它在$x$上连续}
\end{document}
