\documentclass[UTF8,fontset=ubuntu]{ctexart}
\usepackage{ntheorem}
\begin{document}
	\parindent=0pt
	\newtheorem{definition}{定义}
	1.函数的常数倍求导\par
	\begin{definition}
			如果$f(x)=nh(x)$,那么$f'(x)=nh'(x)$
	\end{definition}
	2.函数和与函数差求导\par
	\begin{definition}
			如果$f(x)=g(x)+h(x)$,那么$f'(x)=g'(x)+h'(x)$
	\end{definition}
	3.函数的积求导\par
	\newtheorem*{multiple}[definition]{乘积法则}
	\begin{multiple}[两个变量]
			如果$f(x)=g(x)h(x)$,那么$f'(x)=g'(x)h(x)+g(x)h'(x)$
	\end{multiple}
	\begin{multiple}[三个变量]
			如果$f(x)=g(x)h(x)u(x)$,那么$f'(x)=g'(x)h(x)u(x)+g(x)h'(x)u(x)+g(x)h(x)u'(x)$
	\end{multiple}
	4.函数的商求导\par
	\newtheorem*{divide_theory}[definition]{商法则}
	\begin{divide_theory}
		如果$h(x)=\frac{f(x)}{g(x)}$,那么$h'(x)=\frac{f'(x)g(x)-f(x)g'(x)}{g^2(x)}$
	\end{divide_theory}
	5.链式求导法则\par
	\newtheorem*{chain}[definition]{链式求导法则}
	\begin{chain}
		如果$h(x)=f(g(x))$,那么$h'(x)=f'(g(x))g'(x)$.
	\end{chain}
\end{document}
