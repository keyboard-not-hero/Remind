\documentclass[UTF8,fontset=ubuntu]{ctexart}
\usepackage{ntheorem}
\begin{document}
	\theoremstyle{nonumberplain}
	\newtheorem{theorem}{罗尔定理}
	\begin{theorem}
		假设函数$f$在闭区间$[a,b]$内连续,在开区间$(a,b)$内可导.如果$f(a)=f(b)$,那么在开区间$(a,b)$内至少存在一点$c$,使得$f'(x)=0$.
	\end{theorem}
	\newtheorem{theorem1}{中值定理}
	\begin{theorem1}
		假设函数$f$在闭区间$[a,b]$内连续,在开区间$(a,b)$内可导,那么在开区间$(a,b)$内至少有一点$c$使得
		\[
			f'(c)=\frac{f(b)-f(a)}{b-a}
		\]
	\end{theorem1}
	\newtheorem{definition}{定义}
	\begin{definition}
		假设$f'(c)=0$,这时有:
		\begin{itemize}
		\itemsep=0pt
		\parskip=0pt
		\item 如果从左往右通过$c$点,$f'(x)$的符号由正变负,那么$c$点为局部最大值;
		\item 如果从左往右通过$c$点,$f'(x)$的符号由负变正,那么$c$点为局部最小值;
		\item 如果从左往右通过$c$点,$f'(x)$的符号不发生变化,那么$c$点为水平拐点.
		\end{itemize}
	\end{definition}
\end{document}
