\documentclass[UTF8,fontset=ubuntu]{ctexart}
\usepackage{ntheorem}
\usepackage{amsmath}
\begin{document}
\parindent=0pt
	1.指数公式
	\begin{flalign}
		&b^0=1&\\
		&b^1=b&\\
		&b^xb^y=b^{x+y}&\\
		&\frac{b^x}{b^y}=b^{x-y}&\\
		&(b^x)^y=b^{xy}&
	\end{flalign}
	2.对数公式
	\begin{flalign}
		&\log_b(1)=0&\\
		&\log_b(b)=1&\\
		&\log_b(xy)=\log_b(x)+\log_b(y)&\\
		&\log_b(\frac{x}{y})=\log_b(x)-\log_b(y)&\\
		&\log_b(x^y)=y\log_b(x)&
	\end{flalign}
	\newtheorem*{theorem1}{换底法则}
	\begin{theorem1}
		对于任意的底数$b>1$和$c>1$及任意的数$x>0$,
		\begin{equation}
			\boxed{\log_b(x)=\frac{\log_c(x)}{\log_c(b)}}
		\end{equation}
	\end{theorem1}
	3.自然对数\\
	$e=2.718 281 828 459 045 23\ldots$
	\begin{flalign}
		&e^{\ln(x)}=x&\\
		&\ln(e^x)=x&\\
		&\ln(1)=0&\\
		&\ln(e)=1&\\
		&\ln(xy)=ln(x)+ln(y)&\\
		&\ln(\frac{x}{y})=\ln(x)-\ln(y)&\\
		&\ln(x^y)=y\ln(x)&
	\end{flalign}
	4.常用对数求导\par
	\begin{flalign}
		&\lim_{n\to\infty}(1+\frac{x}{n})^n=e^x&\\
		&\lim_{n\to\infty}(1+\frac{1}{n})^n=e&\\
		&\frac{d}{dx}\log_b(x)=\frac{1}{x\ln(b)}&
	\end{flalign}
	\begin{flalign*}
		&\text{证明过程:}&\\
		&\text{假设}g(x)=\log_bx\text{,则}&
	\end{flalign*}
	\begin{align*}
		g'(x)&=\lim_{\Delta x\to0}\frac{\log_b(x+\Delta x)-\log_b x}{\Delta x}\\
			 &=\lim_{\Delta x\to0}\frac{1}{\Delta x}\log_b{x+\Delta x}{x}\\
			 &=\lim_{\Delta x\to0}\log_b(1+\frac{\Delta x}{x})^{\frac{1}{\Delta x}}\\
			 &=\log_be^{\frac{1}{x}}\\
			 &=\frac{1}{x}\log_be
	\end{align*}
	\begin{flalign*}
		&\text{由换底法则,得}&
	\end{flalign*}
	\begin{align*}
		\log_be&=\frac{\ln e}{\ln b}\\
			   &=\frac{1}{\ln b}
	\end{align*}
	\begin{flalign*}
		&\text{所以,得出结论}&
	\end{flalign*}
	\begin{align*}
		g'(x)&=\frac{1}{x}\log_be\\
			 &=\frac{1}{x\ln b}
	\end{align*}
	\begin{flalign}
		&\frac{d}{dx}\ln(x)=\frac{1}{x}&\\
		&\frac{d}{dx}b^x=b^x\ln(b)&\\
		&\frac{d}{dx}e^x=e^x&
	\end{flalign}
\end{document}
