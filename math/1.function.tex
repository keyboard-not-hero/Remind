\documentclass[UTF8,fontset=ubuntu]{ctexart}
\begin{document}
函数 - 将一个对象转化为另一个对象的规则,并且一个有效输入只能指定唯一的输出。表示为$y=f(x)$

开区间 - 包含区域内的数字,但不包含边界数字本身,如10

闭区间 - 

垂线检验 - 在定义域内,如果有任何垂直线与图像相交多余一次,那该图像就不是函数。反之视为函数

反函数 - 从输出y出发,如果有且仅有一个输入满足$f(x)=y$,则x与y为逆运算为反函数。函数与其反函数关于$y=x$成对称。表示为$x=f^{-1}(y)$

水平线检验 - 在定义域内,如果有任何水平线与图像相交对于一次,那该图像不能进行反函数逆运算。反之可以进行逆运算

复合函数 - 函数$y=g(x)$与$u=h(v)$,当y在h函数的定义域内时,满足$u=h(y)=h(g(x))$,该函数即为复合函数。简写为$h \circ g$

偶函数 - 对于定义域内所有x满足$f(-x)=f(x)$,则该函数为偶函数。函数图像关于y轴对称

奇函数 - 对于定义域内所有x满足$f(-x)=-f(x)$,则该函数为奇函数。函数图像关于原点对称


\end{document}
