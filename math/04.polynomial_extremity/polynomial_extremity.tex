\documentclass[UTF8,fontset=ubuntu]{ctexart}
\usepackage{amsmath}
\begin{document}
\parindent=0pt
1.$x\to a$时的有理函数\par
\begin{equation*}
	\lim_{x\to 2}\frac{x^2-3x+2}{x-2}=\lim_{x\to 2}\frac{(x-2)(x-1)}{x-2}=\lim_{x\to 2}(x-1)=1
\end{equation*}
2.$x\to a$时的平方根的极限\par
\begin{align*}
	\lim_{x\to 5}\frac{\sqrt{x^2-9}-4}{x-5}&=\lim_{x\to 5}\frac{\sqrt{x^2-9}-4}{x-5}\times\frac{\sqrt{x^2-9}+4}{\sqrt{x^2-9}+4}\\
	&=\lim_{x\to 5}\frac{x^2-25}{(x-5)(\sqrt(x^2-9)+4)}\\
	&=\lim_{x\to 5}\frac{(x+5)(x-5)}{(x-5)(\sqrt{x^2-9}+4)}\\
	&=\lim_{x\to 5}\frac{x+5}{\sqrt{x^2-9}+4}\\
	&=\frac{5}{4}
\end{align*}
**$\sqrt{x^2-9}-4$与$\sqrt{x^2-9}+4$互为共轭根式\par
3.$x\to\infty$时的有理函数的极限\par
\begin{align*}
	\lim_{x\to\infty}\frac{x-8x^4}{7x^4+5x^3+2000x^2-6}&=\lim_{x\to\infty}\frac{\frac{x-8x^4}{-8x^4}}{\frac{7x^4+5x^3+2000x^2-6}{7x^4}}\times\frac{-8x^4}{7x^4}\\
													   &=\lim_{x\to\infty}\frac{-\frac{1}{8x^3}+1}{1+\frac{5}{7x}+\frac{2000}{7x^2}-\frac{6}{7x^4}}\times\frac{-8x^4}{7x^4}\\
													   &=\lim_{x\to\infty}\frac{-8x^4}{7x^4}\\
													   &=-\frac{8}{7}
\end{align*}
4.$x\to\infty$时的多项式型函数的极限\par
\begin{align*}
	\lim_{x\to\infty}\frac{\sqrt{16x^4+8}+3x}{2x^2+6x+1}&=\lim_{x\to\infty}\frac{\frac{\sqrt{16x^4+8}+3x}{4x^2}}{\frac{2x^2+6x+1}{2x^2}}\times\frac{4x^2}{2x^2}\\
														&=\lim_{x\to\infty}\frac{\sqrt{1+\frac{8}{16x^4}}+\frac{3}{4x}}{1+\frac{6}{2x}+\frac{1}{2x^2}}\times\frac{4}{2}\\
														&=\lim_{x\to\infty}\frac{\sqrt{1+0}+0}{1+0+0}\times2\\
														&=2
\end{align*}
5.$x\to-\infty$时的有理函数的极限\par
\begin{align*}
	\lim_{x\to-\infty}\frac{\sqrt{4x^6+8}}{2x^3+6x+1}&=\lim_{x\to-\infty}\frac{\frac{\sqrt{4x^6+8}}{\sqrt{4x^6}}}{\frac{2x^3+6x+1}{2x^3}}\times\frac{\sqrt{4x^6}}{2x^3}\\
													 &=\lim_{x\to-\infty}\frac{\sqrt{1+\frac{8}{4x^6}}}{1+\frac{6}{2x^2}+\frac{1}{2x^3}}\times\frac{-2x^3}{2x^3}\\
													 &=\lim_{x\to-\infty}\frac{\sqrt{1+0}}{1+0+0}\times(-1)\\
													 &=-1
\end{align*}
\fbox{$\text{如果}x<0\text{,并且想写}\sqrt[n]{x^\text{某次幂}}=x^m\text{,那么需要在}x^m\text{之前加一个负号的唯一情形是:n是偶数且m是奇数}$}\par
6.包含绝对值的函数的极限\par
\begin{align*}
	&\lim_{x\to0^-}\frac{|x|}{x}=\lim_{x\to0^-}\frac{-x}{x}=-1\\
	&\lim_{x\to0^+}\frac{|x|}{x}=\lim_{x\to0^+}\frac{x}{x}=1\\
\end{align*}
常用因式分解:
\begin{gather}
	a^3-b^3=(a-b)(a^2+ab+b^2)
\end{gather}
\end{document}
