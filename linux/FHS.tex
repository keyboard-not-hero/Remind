\documentclass[UTF8, fontset=ubuntu]{ctexart}
\usepackage{longtable}
\title{Filesystem Hierarchy Standard}
\author{Steven Liu}
\date{Oct 30 2019}
\begin{document}
\maketitle
\vspace{40em}
\begin{longtable}{l l}
目录结构:\\
一级目录 \\
/&根目录(为了性能和安全性考虑,建议挂载较小的空间分区) \\
/bin  &  普通用户指令(系统启动相关,无子目录) \\
/boot  &  系统引导所需文件(系统核心-vmlinuz和引导程序-grub) \\
/dev   & 设备文件(硬盘等) \\
/etc   & 程序的全局配置文件(1.限存放ASCII纯文本文件;2.不同程序配置文件存放在不同子目录下) \\
/home   &  用户家目录,程序的用户级配置文件(1.单个.开头的隐藏配置文件;2.以.开头的隐藏目录,目录包含多个配置文件) \\
/lib   & 共享库和核心模块(1.以lib*.so格式的C动态链接库;2*.modules目录下的可加载核心模块) \\
/lib64  &  类似于/lib,和/lib合用,用于同时32bit和64bit二进制格式 \\
/media   & 可移除媒介挂载点 \\
/mnt   & 临时挂载点 \\
/opt  &  拓展包安装目录(程序存放在不同子目录下) \\
/root  &  root用户家目录,结构类似于/home \\
/run   & 系统运行相关目录(老版本使用/var/run,为向后兼容,现已链接到/run) \\
/sbin   & 管理员指令(系统启动相关,无子目录) \\
/srv   & 开放站点提供的资料目录 \\
/tmp   & 临时文件目录(系统启动后删除目录内容) \\
/usr   & 用户软件资源目录(User Software Resource) \\
/var   & 动态生成数据文档 \vspace{2em} \\

二级目录-/usr \\
/usr/bin   & 普通用户指令(包管理器安装,无子目录) \\
/usr/lib   & 共享库(用户安装) \\
/usr/local  &  local层级目录(源码安装目录) \\
/usr/sbin   & 管理员指令(包管理器安装,无子目录) \\
/usr/share   & 非硬件依赖的层级数据 \\
可选 \\
/usr/games   & 游戏类和教育类指令 \\
/usr/include  &  C类头文件 \\
/usr/libexec   & 被其他指令调用的二进制文件 \\
/usr/lib64   & 可选共享库 \\
/usr/src   & 源代码 \vspace{2em} \\

三级目录-/usr/local \\
/usr/local/bin  &  普通用户指令(本地安装,如源码安装) \\
/usr/local/etc   & 本地安装的配置文件 \\
/usr/local/games  &  本地安装的游戏类 \\
/usr/local/include &   本地安装头文件 \\
/usr/local/lib    &本地安装库 \\
/usr/local/man   & 本地在线manual文档 \\
/usr/local/sbin   & 管理员指令(本地安装) \\
/usr/local/share   & 本地非硬件依赖的层级数据 \\
/usr/local/src   & 本地源代码 \vspace{2em} \\

二级目录-/var \\
/var/cache   & 程序缓存内容 \\
/var/lib   & 实时状态信息 \\
/var/local  &  /usr/local相关的动态数据 \\
/var/lock   & 锁相关文件 \\
/var/log   & log文件或目录(一般包含dmesg/lastlog/messages/wtmp目录) \\
/var/opt   & /opt相关动态数据 \\
/var/run   & 当前进程相关信息 \\
/var/spool  &  同步队列(执行后自动删除,如mail/cron/at) \\
/var/tmp   & 重启后依然保留的临时文件 \\
/var/mail   & 用户邮件 \\
\end{longtable}
\begin{flushleft}
\begin{tabular}{l|l|l}
\hline
 & shareable & unshareable \\
\hline
static & /usr /opt & /etc /boot \\
\hline
variable & /var/mail /var/spool/news & /var/run /var/lock \\
\hline
\end{tabular} \\
\end{flushleft}
**分区建议:/ /lib /bin /sbin /etc /dev分成一个区 \\
**在CentOS 7中,已将/lib /bin /sbin合并到/usr,所以在/usr/local进行另外分区 \\
\end{document}

